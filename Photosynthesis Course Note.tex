% The entire content of this work (including the source code
% for TeX files and the generated PDF documents) by 
% Hongxiang Chen (nicknamed we.taper, or just Taper) is
% licensed under a 
% Creative Commons Attribution-NonCommercial-ShareAlike 4.0 
% International License (Link to the complete license text:
% http://creativecommons.org/licenses/by-nc-sa/4.0/).
\documentclass{article}

\usepackage{float}  % For H in figures
\usepackage{amsmath, amssymb} % For math
\usepackage{mathtools} % dcases*, see https://en.wikibooks.org/wiki/LaTeX/Advanced_Mathematics#The_cases_environment
\numberwithin{equation}{subsection} % have the enumeration go to the subsection level.
									% See:https://en.wikibooks.org/wiki/LaTeX/Advanced_Mathematics
\usepackage{graphicx}   % need for figures
\usepackage{cite} % For bibligraphy
\usepackage{fancyref} % For lazy reference \fref
\usepackage{hyperref} % For hyperlink everything.
\usepackage{CJKutf8} % For Chinese characters
%\usepackage{ dsfont } % For double struck fonts
\usepackage{braket} 
\usepackage[T1]{fontenc}
\usepackage{listings}

\usepackage{amsthm}
\newtheorem{defi}{Definition}[section]
\newtheorem{thm}{Theorem}[section]
\newtheorem{lemma}{Lemma}[section]
\newtheorem{remark}{Remark}[section]
\newtheorem{prop}{Proposition}[section]
\newtheorem{coro}{Corollary}[section]
\theoremstyle{definition}
\newtheorem{ex}{Example}[section]

\title{Photosynthesis Course Note}
\date{\today}
\author{we.taper}
\begin{document}
\maketitle

\tableofcontents

\abstract{
    This is a note of the course \textit{An Introduction
    to Light Harvesting in Bacteria and Plants}. The course is more
    of a research than of a pedagodic taste: ideas and
    methods are only mentioned, with reference to the published paper
    available in the corresponding slides.
    Therefore, this note aims to provide a guidance to read those slides.
    Note that the page numbers in this note are the page numbers of
    each individual PDF, not necessarily the page numbers displayed
    on the bottom-right corner inside the PDFs.
}
\section{June 29th}
\label{sec:June_29th}

In this lecture, the teacher presents backgroud for light-harvesting
mechanism in biology systems:
\begin{itemize}
    \item \textbf{pp.1 to 9 of PPT} 
        General backgroud to Biophysics.
    \item \textbf{pp.10 to 18 of PPT}
        Review of several mechanisms of light harvesting in different 
        species.
    \item \textbf{pp.19 to 20 of PPT}
        The central topics of this course
        \begin{itemize}
            \item The natrual of exciton in biological ligth harvesting 
                system.
                Is it of Frenkel (short-ranged) of Wannier (long-ranged) type?
            \item The transfer mechanism. (pp.20) It was mentioned that
                the transfer is much more efficient than predicted
                by a classical diffusion model.
            \item How is active self-regulation archieved. (pp.20)
        \end{itemize}
\end{itemize}
\textbf{pp.21} displays a observed long-range transport of excitation
    energy in a biomimetic light-harvesting system.
Then, after a review of Quantum Mechanics (\textbf{pp. 22 to 32} )\footnote{
    Interestingly, the teacher mentioned Bohmian Mechanics, an alternative
    model of Quantum Mechanics that features determinism and nonlocality.
    A comprehensive discussion of this model could be found here:
    \url{http://plato.stanford.edu/entries/qm-bohm/}
}, the teacher illustrated a model to calculate
the excitation energy of a dimer.(\textbf{pp.33 to 36})
Note that here it is expected that only one of the
dimer is excited, becuase this energetically eaiser to achieve.
Note also that the
symble "$\dagger$" denotes neither conjugation nor creation operator,
but simply a label for an excited state's wavefunction. 

The calculated excited state energys on page 35 are splitted into
two levels. the difference between them, denoted by $\mathcal{E}$,
is expressed as a function of the two
transition dipole moments and the speration length of two molecules.
Details about the interaction potential $V_{uv}$ can be found on
the next lecture.

\section{July 1th}
\label{sec:July_1th}
Begin with a review of two modes of excitons in organic system
\textbf{pp.1}, this
lecture proceeds to explain in detail the dipole interaction potential
(\textbf{pp.2 to 3})
$V_{uv}$ mentioned in previous lecture. \textbf{Page 4} digresses on
the difference between the transition dipole moment and the
permanent dipole moment. The picture shows the transition dipole
moment for a Bacteriacholorophyll (the photosynthetic pigments that 
occur in various phototrophic bacteria). The right side plot illustrates
the absorption spectra for Bacteriacholorophyll.
\textbf{Page 6 and 7} also digresses on eletronic momenta and oscillator
strength. 

The following slide comes back to the excitation energy. It states two
forms of dimer: the J aggregation ( \textbf{pp.9 to 10}, and the H
aggregation ( \textbf{pp.11}). The accessible energy level for absorption
is determined by conservation of moment. As the result, the J aggregated
dimer absorbs less energy than the monomer case, and a red shift will be
observed when comparing the absorption spectra. The case of H aggregation
is similar. \textbf{Page 12 to 13 } mentions the transition dipole in
more complicated cases. \footnote{Reference could be found in \textit{
J. Phys. Chem. B, Vol. 107, No. 35, 2003}.}

\section{July 6th}
\label{sec:July_6th}
This lecture begins with a note on the unbelievable efficiency of biological
system. Then it proceeds to review some concepts in quantum statistical 
mechanics ( \textbf{page 2 to 21}). Since we are quite familar with this
subject, it will be omitted from this note. The last page mentioned an
important equation for the evolvement of density matrix in an open system,
called the "master equation".
% TODO reference required!

\section{July 8th}
\label{sec:July_8th}
In the previous lecture, the \textit{master equation} for density matrix
\footnote{Reference not found.}
could lead to problematic result. When the density matrix propagates
as described by the master equation, the eigenvalues for it might
evolve into comprising negative values. This is because the added terms
are not rigorously derived, but just an approximation. On \textbf{page 2} the
teacher mentioned a specific case.\footnote{
    Digression: in the article PRE 65 056120, we can find
    something about the entanglement between two oscillators.
}
Therefore, one introduces the \textbf{Lindblad equations}
(\textbf{pp.3}). This equation ensures that the 
density matrix $\rho \geq 0$ (i.e. having all positive
eigenvalues). However, this operator sometimes breaks down
the symmetries of the system. That is, a system started with
translation symmetry at $t_0$ might not have translation
symmetry at $t > t_0$.

The teacher wrote an example for the $V_m$ term on ppt on blackboard:

Considering the case of a harmonic oscillator.
$$ H = w_0 a^\dagger a $$
If we add an interaction between the oscillator and other excitons:
$$ A^\dagger A (a^\dagger + a) = \text{extra} \otimes \text{bath}$$
Then $A^\dagger A$ is $V_m$ in ppt.

Next, the author provided a specific model of energy transfer.
\footnote{
\textbf{Note}: This section's material could be found in
\textit{Excitonic energy transfer in light-harvesting complexes 
in purple bacteria. JChemPhys\_136\_245104.pdf}
}

\paragraph{pp.5} shows diagramatically the exciton transport process.
External influence to the system includes trapping, decaying and disspation.
Here $k_t$ characterizes the trapping rate. And $k_d$ characterizes
the decay rate.

\paragraph{pp.6 of PPT} explains some notations.
Here $\tau_n$ characterize the life time 
of exciton in on state $n$. \footnote{ Here
$\rho_n \equiv \rho_{nn}$.} We hope that the total life time
$\braket{t} = \sum \tau_n$
is small, because the shorter an exciton is fixed on a state,
the more random the system is and the more likely that the exciton
is trapped.

Following are a series of toycases that has been calculated.

\paragraph{pp.8 to 9} are the $\rho(t)$ for these simple cases.
% TODO try to calculate the example.

\paragraph{pp.10 of PPT}
\footnote{Could be found on \textit{Efficient energy transfer in 
light-harvesting systems. Cao and Silbey et al New J Phys (2010)}}
This ring with 16 sites shows that there is always 
a peek Max population in the opposite site(See upper-right plot). This is
because there is always
two channels of equal length for excitons to get to the opposite site.

\paragraph{pp.11 of PPT} is 
a two ring case, with 8 sites on each ring. The initial configuration
is that excitons are evenly excited in the left ring. 
The right ring comprised of trapping sites.

\paragraph{pp.12 of PPT} is 
another two ring case, with a change in the right ring's number
of trapping sites. It shows that a asymmetric design would somehow
imporve the efficiency, since the plot on the right has a maximum region
while the left one has none. The next slide shows similar result.

\textbf{Caution:} I am quite confused by the following content,
from pp.16 to the end of PPT. Erogo the
following notes are note well organized. It is advised that one should
look at the original literature instead.

\paragraph{pp.15 to 16} shows a more advanced calculation of excited
states.

\paragraph{pp.16 of PPT} Calculating the absorption spectrum, 
showing that only one level is acceptable.

\paragraph{pp.17 of PPT}
Changing the configuration of laser could result (in generally) more
acceptable levels. However, these laser configurations are not practically
achievable.

\paragraph{pp.19 of PPT}
Using Frankel-Exciton Model and find very good fit, balabala.


\section{License}
The entire content of this work (including the source code
for TeX files and the generated PDF documents) by 
Hongxiang Chen (nicknamed we.taper, or just Taper) is
licensed under a 
\href{http://creativecommons.org/licenses/by-nc-sa/4.0/}{Creative 
Commons Attribution-NonCommercial-ShareAlike 4.0 International 
License}. Permissions beyond the scope of this 
license may be available at \url{mailto:we.taper[at]gmail[dot]com}.

\end{document}

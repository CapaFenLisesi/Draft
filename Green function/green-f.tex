% The entire content of this work (including the source code
% for TeX files and the generated PDF documents) by 
% Hongxiang Chen (nicknamed we.taper, or just Taper) is
% licensed under a 
% Creative Commons Attribution-NonCommercial-ShareAlike 4.0 
% International License (Link to the complete license text:
% http://creativecommons.org/licenses/by-nc-sa/4.0/).
\documentclass{article}

\usepackage{float}  % For H in figures
\usepackage{amsmath} % For math
\usepackage{amssymb}
\usepackage{mathrsfs}
\numberwithin{equation}{subsection} % have the enumeration go to the subsection level.
                                    % See:https://en.wikibooks.org/wiki/LaTeX/Advanced_Mathematics
\usepackage{graphicx}   % need for figures
\usepackage{cite} % need for bibligraphy.
\usepackage[unicode]{hyperref}  % make every cite a link
\usepackage{CJKutf8} % For Chinese characters
\usepackage{fancyref} % For easy adding figure,equation etc in reference. Use \fref or \Fref instead of \ref
\usepackage{braket} %http://tex.stackexchange.com/questions/214728/braket-notation-in-latex

% Following is for the special character: differential "d".
\newcommand*\diff{\mathop{}\!\mathrm{d}}
\newcommand*\Diff[1]{\mathop{}\!\mathrm{d^#1}}
% Following is for theorems etc environments
% http://tex.stackexchange.com/questions/45817/theorem-definition-lemma-problem-numbering && https://en.wikibooks.org/wiki/LaTeX/Theorems
\usepackage{amsthm}
\newtheorem{defi}{Definition}[section]
\newtheorem{thm}{Theorem}[section]
\newtheorem{lemma}{Lemma}[section]
\newtheorem{remark}{Remark}[section]
\newtheorem{prop}{Proposition}[section]
\newtheorem{coro}{Corollary}[section]
\theoremstyle{definition}
\newtheorem{ex}{Example}[section]

\usepackage{xcolor} %For colourful math:http://tex.stackexchange.com/questions/21598/how-to-color-math-symbols

% To include PDF in higher version format.
\pdfoptionpdfminorversion=6

% A list of nomenclatures.
\usepackage{nomencl}
\makenomenclature

\title{(Tentative) Calculating Green Function}
% TODO give it a better name.
\date{\today}
\author{Taper}


\begin{document}


\maketitle
\abstract{
    This is a note for reading the paper \cite{dissertation}, and for 
    understanding the code produced from that file.
}
\tableofcontents

\section{Chapter 1 - Introductions}
\label{sec:Chapter_1_-_Introductions}
This chapter is really a nice introduction to the current fields of 
mesoscopic physics. The writing is clear and it traces the develpment 
of this field. It gives me a lucid and holistic historical account of 
both the important discoveries and motives behind them. I should 
find those marked regions on pdf inside this part very useful.

\section{Chapter 2 - 2 Landauer-Büttiker formalism}
\label{sec:Chapter_2_-_2_Landauer-Buttiker_formalism}

This chapter introduces the Landauer-Büttiker formalism for calculating the
transport properties. The typical setup is illustrated below:

\begin{figure}[H]
    \centering
    \includegraphics[width=0.8\linewidth]{pics/{ch2.setup_for_L-B_formalism}.png}
    \caption{Setup for the Landauer-Büttiker formalism}
    %\label{fig:}
\end{figure}

In that formalism, the currents following through the leads have the following
expression:
\begin{align}
    \label{eq:ch1.Landauer-Buttiker_formula}
    I_p = \frac{-e}{h} \sum_q \int T_{qp}(E)\left( f_p(E)-f_q(E) \right)\diff E
\end{align}
where $T_{pq}$ is the transmission coefficients for electrons to go from
lead $q$ to lead $p$. This formula can be simplified/linearized into:
\begin{align}
    I_p = \frac{e^2}{h}\sum_q T_{pq}(E_F)(V_p-V_q)
\end{align}
An obvious advantage of Landauer-Büttiker formalism is that it makes the
dependence of $I_p$ on experimental setup explicit in the formula.

This chapter continue to discuss some time reversal symmetry (TR) properties 
of this formula, centring/centering around the coefficient $T_{pq}$.
But I am perplexed by that he, while discussing TR, mentions the
magnetic field $B$ and formulae like:
\begin{align}
    T_{12}(+B)= T_{12}(-B)
\end{align}
\nomenclature{confusion: TR and $B$}{\nomrefpage}

\section{Chapter 3 - Tight-binding model}
\label{sec:Chapter_3_Tight-binding_model}

Here in this chapter the author presents the fundamental Hamiltonian of
the system under consideration.
\marginpar{
    Given the formula mentioned in previous chapter, the Hamiltonian
    presented here seemed seemingly extraneous. Read the introduction
    here in chapter 4 to understand the underlying logic.
}

The process to obtain the Hamiltonian is discretization the Hamiltonian
in continuous case, quite the reverse of the first few chapters of
A. Zee's QFT in a Nutshell. It should be noted that "a site may represent 
a region containing many atoms", although "this region should be small 
compared to physically relevant quantities such as the Fermi wavelength".

    \subsection{3.1 Spin-degenerate system}
    \label{sec:3.1_Spin-degenerate_system}
    Here the general Hamiltonian, using tight-binding model, is
    mentioned:
    \begin{align}
        H = &\sum_{n,m} 
            \left( t^x_{nm}\ket{n+1,m}\bra{n,m}
                +t^y_{nm}\ket{n,m+1}\bra{n,m}+h.c.\right)\nonumber\\
            &+\sum_{n,m} \epsilon_{nm}\ket{n,m}\bra{n,m}
    \end{align}

    The important thing is to determine the coefficient $t$.
    In the absence of magnetic field, $t$ is give by:
    \begin{align}
        t^x_{nm}=t^y_{nm}=-t=-\frac{\hbar^2}{2m^* a^2}
    \end{align}
    When magnetic field is present, we effect a so called Peierls
    substitution to get $t$.
    \marginpar{
        The author points to a paper, saying that it contains a
        "lucid discussion on the physics" of Peierls substitution.
        This might be something worth reading.
    }
    The result is, using Landau gauge in one dimensional under a
    homogeneous magnetic field.
    \begin{align}
        t^x_{nm} =& -t e^{i2\pi (m-1) \Phi/\Phi_0} \\
        t^y_{nm} =& -t
    \end{align}

    When the magnetic field is inhomogeneous, it is generally difficult
    to choose a gauge to calculate analytically. The author uses a
    very intuitive discretization process to approach this problem.
    The process is best illustrated by just the following picture:
    \begin{figure}[H]
        \centering
        \includegraphics[width=0.8\linewidth]{pics/{ch3.not-uniform_magnetic_field}.png}
        \caption{Not uniform magnetic field}
        \label{fig:ch3-uniform_magnetic_field.png}
    \end{figure}

    Therefore, the result is something like:
    \begin{align}
        t^x_{nm}=-t e^{i2\pi\sum_{m'<m}\Phi_{nm'}/\Phi_0}
    \end{align}
\printnomenclature
    \subsection{3.2 Including spin degrees of freedom}
    \label{sec:3.2_Including_spin_degrees_of_freedom}
    
    When the spin is taken into consideration, the formulation should
    be modified accordingly.

    If Zeeman/exchange splitting is considered, then:
    \begin{align}
        H_S = -\frac{1}{2}g^* \mu_B
            \sum_{nm}\ket{n,m}\bra{n,m}\otimes (B^{eff}_{nm}\cdot\sigma)
    \end{align}
    Not that the magnetic field strength here is only "effective".

    If Spin-orbit coupling is taken into consideration,
    \footnote{
        In my opinion, this is essentially all about Spin-Magnetic 
        field coupling, not just Spin-Orbital coupling.
    }
    the following Hamiltonian should be considered:
    \begin{align}
        H_{SO}= \lambda P\cdot (\bigtriangledown V \times \sigma)
        \label{eq:ch3.SO_coupling}
    \end{align}
    Here $P$ is the mechanical momentum operator, $\sigma$ is the
    three pauli spin matrix $(\sigma_x, \sigma_y, \sigma_z)$.
    
    The tight-binding version of this is too complicated to be presented
    here, it is on page 16, equation (3.20).

    One additional consideration is the Rashba spin-orbit coupling.
    This is the peculiar result of electrons trapped in a two
    dimensional surface. In the $Z$ direction, with $Z$ perpendicular
    to the 2 dimensional plane, the potential looks like something
    below, called a triangular potential:
    \begin{figure}[H]
        \centering
        \includegraphics[width=0.6\linewidth]{pics/{ch3.rashba_spin-orbit_effect}.png}
        \caption{Conduction band at the interface of a semiconductor 
            heterostructure. Band bending creates a potential well 
            $V(z)$ confining the electrons to the XY plane. The asymmetry 
            of this well leads to Rashba spin-orbit coupling.}
        \label{fig:ch3-orbit_effect.png}
    \end{figure}
    It is pretty obvious, by equation \ref{eq:ch3.SO_coupling}, that
    this potential is going into our Hamiltonian.
    The result in tight-binding model is yet another complicated
    Hamiltonian, not to be presented here, numbered equation (3.23)
    on page 18.

\section{Chapter 4 - Green’s function formalism}
\label{sec:Chapter_4-Greens_function_formalism}

In this section we finally circled back to the Landauer-Büttiker
formalism. Here a breif summary of chapter 3 and 4 is opportune.
The purpose of chapter 3 and 4 as a whole, is to calculate the 
transmission coefficient $T_{qp}$ in equation
\ref{eq:ch1.Landauer-Buttiker_formula}. Here, chapter 3 establish the
Hamiltonian, and chapter 4 uses the Green function to calculate
transmission coefficient out of the Hamiltonian.

    \subsection{4.1 Green’s functions: The basics}
    \label{sec:4.1_Greens_functions_The_basics}
    
    The author takes the definition of Green's function as inverse
    of Hamiltonian, more specifically:
    \begin{align}
        [E-\hat{H}]\hat{G}(E) = 1
    \end{align}
    or, in the position-spin representation:
    \begin{align}
        [E-H(\vec{x})]G(\vec{x},\vec{x'},E)=\delta(\vec{x}-\vec{x'})
    \end{align}
    where $\vec{x}=(\vec{r},\sigma)$, containing in addition to the
    usual spatial part, the spin part.

    The physical meaning is also explained, although I learnt this
    better in A. Zee's QFT in a Nutshell. However, he notes that
    to disguish between the source and sink, i.e. to tell whether
    the Green's function calculated represents a wavefunction
    resulted from a unit excitation (the sink), or a source for such
    an excitation, we should incorporate boundary conditions. This is
    done by adding "an infinitesimal imaginary variable into the energy",
    "leading to the following definitions":
    \begin{align}
        G^{\pm}(\vec{x},\vec{x'},E)\equiv \lim_{\eta\to 0^{+}}
            G(\vec{x},\vec{x'},E\pm i\eta)
    \end{align}
    and $G^{\pm}$ satisfies:
    \begin{align}
        [E\pm i\eta - H(\vec{x})] G^{\pm}(\vec{x},\vec{x'},E)=
            \sigma(\vec{x}-\vec{x'})
    \end{align}
    The functions $G^+$ and $G^-$ are called respectively the
    \textbf{retarded and advanced Green's function}. The author
    mentions that, "when Fourier transforming the functions $G^\pm$ to
    the time domain using a closed contour integration in the complex 
    plane, they would correspond to causal and anticausal solutions".

    Though seemingly extraneous, the operator definition for above
    Green functions is also mentioned:
    \begin{align}
        \hat{G}^{\pm}(E)\equiv \lim_{\eta\to 0^+}
            \frac{1}{E\pm i\eta-\hat{H}}
    \end{align}
    \subsection{4.2 Transmission coefficients and the Green’s function}
    \label{sec:4.2_Transmission_coefficients_and_the_Greens_function}
    
    Here it is mentioned than the transmission coefficient is related
    Green's funciton in the following way:
    \begin{align}
        T_{pq} = \text{Tr}\left[\Gamma_p G_{pq}\Gamma_q G^{\dagger}_{pq}
                \right]
    \end{align}
    Here $G_{pq}$ is, to borrow the language of matrix mechanics,
    a submatrix of the Green's function. $\Gamma_p$ is a abbreviation
    for the following counting:
    \marginpar{I am not sure about this peculiar thing.}
    \begin{align}
        \label{eq:}
        \Gamma_p \equiv i(\sum_p-\sum^\dagger_p)
    \end{align}
    Here $\sum_p$ is named "self-energy of the lead", something I
    don't quite understand.
\section{Anchor}

\begin{thebibliography}{1}
    \bibitem{dissertation} 
        \href{https://sundoc.bibliothek.uni-halle.de/diss-online/07/07H039/prom.pdf}{Electronic Transport in Mesoscopic Systems}, 
        by von Georgo Metalidis. (Link found via Google)
\end{thebibliography}
\section{License}
The entire content of this work (including the source code
for TeX files and the generated PDF documents) by 
Hongxiang Chen (nicknamed we.taper, or just Taper) is
licensed under a 
\href{http://creativecommons.org/licenses/by-nc-sa/4.0/}{Creative 
Commons Attribution-NonCommercial-ShareAlike 4.0 International 
License}. Permissions beyond the scope of this 
license may be available at \url{mailto:we.taper[at]gmail[dot]com}.
\end{document}

% The entire content of this work (including the source code
% for TeX files and the generated PDF documents) by 
% Hongxiang Chen (nicknamed we.taper, or just Taper) is
% licensed under a 
% Creative Commons Attribution-NonCommercial-ShareAlike 4.0 
% International License (Link to the complete license text:
% http://creativecommons.org/licenses/by-nc-sa/4.0/).
\documentclass{article}

\usepackage{float}  % For H in figures
\usepackage{amsmath} % For math
\usepackage{amssymb}
\usepackage{bbm} % for numbers within mathbb
\usepackage{mathrsfs} % For \mathscr{ABC}
% Followings are for the special character: differential "d".
\newcommand*\diff{\mathop{}\!\mathrm{d}}
\newcommand*\Diff[1]{\mathop{}\!\mathrm{d^#1}}
\numberwithin{equation}{subsection} % have the enumeration go to the subsection level.
                                    % See:https://en.wikibooks.org/wiki/LaTeX/Advanced_Mathematics
\usepackage{graphicx}   % need for figures
\usepackage{cite} % need for bibligraphy.
\usepackage[unicode]{hyperref}  % make every cite a link
\usepackage{CJKutf8} % For Chinese characters
\usepackage{fancyref} % For easy adding figure,equation etc in reference. Use \fref or \Fref instead of \ref
\usepackage{braket} %http://tex.stackexchange.com/questions/214728/braket-notation-in-latex

% Following is for theorems etc environments
% http://tex.stackexchange.com/questions/45817/theorem-definition-lemma-problem-numbering && https://en.wikibooks.org/wiki/LaTeX/Theorems
\usepackage{amsthm}
\newtheorem{defi}{Definition}[section]
\newtheorem{thm}{Theorem}[section]
\newtheorem{lemma}{Lemma}[section]
\newtheorem{remark}{Remark}[section]
\newtheorem{prop}{Proposition}[section]
\newtheorem{coro}{Corollary}[section]
\theoremstyle{definition}
\newtheorem{ex}{Example}[section]

% A list of nomenclatures.
\usepackage{nomencl}
\makenomenclature

\title{Temp}
\date{\today}
\author{Taper}


\begin{document}


\maketitle
\abstract{
    (Unknown)
}
\tableofcontents
\section{Classification of Topological states by Hamiltonian}
\label{sec:Classification-of-Topological-states-by-Hamiltonian}

\begin{table}[H]
\centering
\caption{$d=1$}
\label{my-label}
\begin{tabular}{|l|l|l|}
\hline
Class A                        & $\mathcal{H}=$                                                        & Commments \\ \hline
$\mathbb{Z}$ class $A$ III     & $(m+\cos{k_x})\sigma_z+\sin{k_x} \sigma_y$                            & $S=\sigma_x$\\ \hline
$0$ Class A                    & $(m+\cos{k_x}\sigma_z+\sin{k_x}\sigma_y$(SPEMT: $M\sigma_x$           & No symmetries \\ \hline
$\mathbb{Z}_2$ class $D$       & $(m+\cos{k_x})\sigma_z + \sin{k_x}\sigma_y$                           & $C=\sigma_x K$ (Particle hole degree of freedom) \\ \hline
$\mathbb{Z}_2$ class $D$III    & $(m+\cos{k_x})S_0 \sigma_z+\sin{k_x}S_0\sigma_y$                      & $C=\sigma_x K$(PH)$T=S_y K$, Spin $1/2$\\ \hline
$0$ class AII                  & $(m+\cos{k_x})S_0 \sigma_z+\sin{k_x}S_0\sigma_y$ SPEMT:$MS_x\sigma_x$ & $T=S_yk$ Spin $1/2$ \\ \hline
$\mathbb{Z}$ class CII         & $(m+\cos{k_x})S_0 \sigma_z+\sin{k_x}S_0\sigma_y$                      & $T=S_yK, C=S_y\sigma_xK$ \\ \hline
%                             &                                                                       & \\ \hline
%                              &                                                                       & \\ \hline
%                              &                                                                       & \\ \hline
%                              &                                                                       & \\ \hline
%                              &                                                                       & \\ \hline
%                              &                                                                       & \\ \hline
%                              &                                                                       & \\ \hline
\end{tabular}
\end{table}

\section{Anchor}
\label{sec:Anchor}

\begin{thebibliography}{1}
    %\bibitem{book} 
\end{thebibliography}
\printnomenclature
\section{License}
The entire content of this work (including the source code
for TeX files and the generated PDF documents) by 
Hongxiang Chen (nicknamed we.taper, or just Taper) is
licensed under a 
\href{http://creativecommons.org/licenses/by-nc-sa/4.0/}{Creative 
Commons Attribution-NonCommercial-ShareAlike 4.0 International 
License}. Permissions beyond the scope of this 
license may be available at \url{mailto:we.taper[at]gmail[dot]com}.
\end{document}

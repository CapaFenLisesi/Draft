% The entire content of this work (including the source code
% for TeX files and the generated PDF documents) by 
% Hongxiang Chen (nicknamed we.taper, or just Taper) is
% licensed under a 
% Creative Commons Attribution-NonCommercial-ShareAlike 4.0 
% International License (Link to the complete license text:
% http://creativecommons.org/licenses/by-nc-sa/4.0/).
\documentclass{book}

\usepackage{float}  % For H in figures
\usepackage{amsmath, amssymb} % For math
\usepackage{mathtools} % dcases*, see https://en.wikibooks.org/wiki/LaTeX/Advanced_Mathematics#The_cases_environment
\numberwithin{equation}{subsection} % have the enumeration go to the subsection level.
% See:https://en.wikibooks.org/wiki/LaTeX/Advanced_Mathematics
\usepackage{graphicx}   % need for figures
\usepackage{cite} % For bibligraphy
\usepackage{fancyref} % For lazy reference \fref
\usepackage[unicode]{hyperref} % For hyperlink everything.
\usepackage{CJKutf8} % For Chinese characters
%\usepackage{ dsfont } % For double struck fonts
\usepackage{braket} 
\usepackage[T1]{fontenc}
\usepackage{listings}

\usepackage{amsthm}
\newtheorem{defi}{Definition}[section]
\newtheorem{thm}{Theorem}[section]
\newtheorem{lemma}{Lemma}[section]
\newtheorem{remark}{Remark}[section]
\newtheorem{prop}{Proposition}[section]
\newtheorem{coro}{Corollary}[section]
\theoremstyle{definition}
\newtheorem{ex}{Example}[section]

\usepackage{tensor}  % For tensor indices
\usepackage[all]{xy} % For drawing category diagrams
\usepackage{mathrsfs}

\title{Notes of Quantum Field Theory in a Nutshell}
\date{\today}
\author{we.taper}
\begin{document}
	\maketitle
	\tableofcontents

\chapter{Part I: Motivation and Foundations}
\label{sec:QFT_in_a_Nutshell.Part_I}
\section{I:2 Path Integral Formulation of Quantum Physics}
\label{sec:QFT_in_a_Nutshell_Part_I.2}
Here the path integral formulation of quantum mechanics is introduced.
The intuition comes from a limiting case of the traditional double 
slit electron interference experiment. (\textbf{pp.7 to 10}) Then 
it calculates the transition probability $\braket{q_F|e^{-iHT}|q_I}$
by divide it into a infinite of steps:
\begin{align}
    \braket{q_F|e^{-iHT}|q_I} = \lim_{N\to \infty}
    \braket{q_F|e^{iH\delta t}e^{iH\delta t}\cdots e^{iH\delta t}|q_I}
    (\text{with }N\delta t = T)
\end{align}

For illustration, it calculates this value when 
$H=\frac{\hat{p}^2}{2m}$. The result is that:
\begin{align*}
    \braket{q_F|e^{-iHT}|q_I} = \int Dq(t) 
        e^{i\int_0^T dt \frac{1}{2}m \dot{q}^2 }
\end{align*}

where:
\begin{align}
    \int Dq(t) \equiv \lim_{N\to \infty} 
    \left( \frac{-im}{2\pi \delta t}\right)^{N/2}
    \left( \prod_{k=1}^{N-1} \int dq_k \right)
\end{align}
It notes that when $H=\hat{p}^2/2m + V(\hat{q})$, the final result
would have been:
\begin{align}
    \label{eq:QFT_in_a_Nutshell_Part_I.2.TransProbability}
    \braket{q_F|e^{-iHT}|q_I} &= \int Dq(t) 
        e^{i\int_0^T dt \frac{1}{2}m \dot{q}^2 - V(q)}
        \nonumber \\
        &=\int Dq(t) e^{i \int_0^T dt L(\dot{q}, q)}
\end{align}
where $L$ is the Lagrangian of the system.

Often, the value we need to calculate is $\braket{F|e^{-iHT}|I}$.
Using $\int \ket{q}\bra{q} dq=1$, we have:
\begin{align}
    \braket{F|e^{-iHT}|I} = \int dq_F \int dq_I
        \braket{F|q_F} \braket{q_F|e^{-iHT}|q_I}\braket{q_I|I}
\end{align}

The value \(\braket{0|e^{-iHT}|0}\) is denoted $Z$. This part
mentions that one often effect a change of coordinate
$t \to -it$, called \textit{Wick rotation}, to obtain:
\begin{align}
    Z = \int Dq(t) e^{-\int_0^T dt H(\dot{q},q)}
\end{align}
where $H$ is the Hamiltonian of the system.
The mathematical rigorous aspect is often ignored.

It also discuss how this formulation could explain the classical
limit of quantum mechanics, i.e. classical mechanics, in a
very direct manner. This is related to the saddle point
approximation to the integral 
\ref{eq:QFT_in_a_Nutshell_Part_I.2.TransProbability}.

\textbf{Unclear point}
Why is $\int dq \ket{q}\bra{q} = 1$ while $\int \frac{p}{2\pi}
\ket{p}\bra{p} = 1$. What does it mean by saying
"to see that the normalization is correct" (pp. 10 and 11).
Why is effecting the Wick rotation is "somewhat rigorous"?

\textbf{Corresponding pages in draft} pp. 1 to 3.

    \subsection{Appendix 1 - Dirac Delta function and
        \texorpdfstring{$\varepsilon$}{} as infinitesimal
        small value}

    Here the Dirac Delta function is defined as the limit of another
    function $d_K(x)$. Since:
    \begin{align}
        d_K(x) \equiv \int_{-K/2}^{K/2} \frac{dk}{2\pi} e^{ikx} = 
            \frac{1}{\pi x}\sin\frac{Kx}{2}\\
        \int_{-\infty}^{\infty} dx\text{ } d_K(x) = 1
    \end{align}
    Hence we de fine $\delta(x) = \lim_{K\to \infty}d_K(x)$.
    Other important formula include:
    \begin{align}
        \delta(x) = \int_{-\infty}^{\infty} \frac{dk}{2\pi}e^{ikx} \\
        \frac{1}{x+i\varepsilon} = \mathcal{P}\frac{1}{x}-i\pi \delta(x)\\
        \delta(x) = \frac{1}{\pi}\frac{\varepsilon}{x^2+\varepsilon^2}
    \end{align}
    Here $\varepsilon$ is a infinitesimal value. $\mathcal{P}$ denotes the
    principal value integral, defined by:
    \begin{align}
        \int dx \mathcal{P}\frac{1}{x}f(x) = \lim_{\varepsilon\to 0}
            \int dx \frac{x}{x^2+\varepsilon^2}f(x)
    \end{align}
    \subsection{Appendix 2 - Wick theorem in Gaussian 
    Integral}

    This part introduces some very important formulae, listed 
    below:

    (It is very important that $A$ is a real symmetric matrix.)
    
    \begin{align}
        &\int_{-\infty}^{+\infty} dx
            e^{-\frac{1}{2}ax^2+Jx} =
            \left(\frac{2\pi}{a}\right)^{\frac{1}{2}}
            e^{\frac{J^2}{2a}} 
        \\
        &\int_{-\infty}^{+\infty} \int_{-\infty}^{+\infty} \cdots
        \int_{-\infty}^{+\infty} dx_1 dx_2 \cdots dx_N
            e^{-\frac{1}{2} x^T A x + J^T x} =
            \left( \frac{(2\pi)^N}{\text{det}A} \right)^{\frac{1}{2}}
            e^{\frac{1}{2}J^T A^{-1}J}
        \\
        &\braket{x_i x_j \cdots x_k x_l}
        \equiv
            \frac{
                \int_{-\infty}^{+\infty}\int_{-\infty}^{+\infty}\cdots
                \int_{-\infty}^{+\infty} dx_1 dx_2\cdots dx_N
                e^{-\frac{1}{2}x^T Ax} x_i x_j \cdots x_k x_l}
                {
                \int_{-\infty}^{+\infty}\int_{-\infty}^{+\infty}\cdots
                \int_{-\infty}^{+\infty} dx_1 dx_2\cdots dx_N
                e^{-\frac{1}{2}x^T Ax} }
            \nonumber \\
        &= \sum_{\text{Wick}} 
            (A^{-1})_{ab}\cdots(A^{-1})_{cd}
    \end{align}
    For exmaple:
    \begin{align*}
        &\braket{x^{2n}} \equiv 
            \frac{\int_{-\infty}^{+\infty}dx
            e^{-\frac{1}{2}ax^2} x^{2n}}
            {\int_{-\infty}^{\infty} dx e^{-\frac{1}{2}ax^2}
            }
        = \frac{1}{a^n}(2n-1)!! \\
        &\braket{x_ix_jx_kx_l} = 
        (A^{-1})_{ij}(A^{-1})_{kl}+(A^{-1})_{ik}(A^{-1})_{jl}
        +(A^{-1})_{il}(A^{-1})_{jk}
    \end{align*}
    \textbf{Corresponding pages in draft} pp. 4 to 6.
    
\begin{thebibliography}{1}
	\bibitem{zee} A. Zee. Quantum Field Theory in a Nutshell
		2ed. PUP.
\end{thebibliography}

\section{License}
The entire content of this work (including the source code
for TeX files and the generated PDF documents) by 
Hongxiang Chen (nicknamed we.taper, or just Taper) is
licensed under a 
\href{http://creativecommons.org/licenses/by-nc-sa/4.0/}{Creative 
	Commons Attribution-NonCommercial-ShareAlike 4.0 International 
	License}. Permissions beyond the scope of this 
license may be available at \url{mailto:we.taper[at]gmail[dot]com}.

\end{document}

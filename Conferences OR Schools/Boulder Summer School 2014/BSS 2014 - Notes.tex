% The entire content of this work (including the source code
% for TeX files and the generated PDF documents) by 
% Hongxiang Chen (nicknamed we.taper, or just Taper) is
% licensed under a 
% Creative Commons Attribution-NonCommercial-ShareAlike 4.0 
% International License (Link to the complete license text:
% http://creativecommons.org/licenses/by-nc-sa/4.0/).
\documentclass{article}

% My own physics package
% The following line load the package xparse with additional option to
% prevent the annoying warnings, which are caused by the package
% "physics" loaded in package "physicist-taper".
\usepackage[log-declarations=false]{xparse}
\usepackage{physicist-taper}
\makenomenclature % For an index of symbols.

\title{BSS 2014 - Notes}
\date{\today}
\author{Taper}


\begin{document}


\maketitle
\abstract{
    This is a note for an guide to the video recordings of Boulder
    Summer School (BSS) 2014,
    whose major topic is: \textbf{Modern Aspects of
    Superconductivity}. However, after I worked out some outlines of
    these lectures, I found the Yale's website about BSS, I found
    \href{http://boulderschool.yale.edu/2014/boulder-school-2014-lecture-notes}{here}
    their list of topics. I found nothing relavent to my current
    study. So I quit this note. The Yale's website is a better guide
    (in my opinion) to these videos.
}
\tableofcontents

\section{Introduction to BSS 2014 - Topic: Superconductivity}
\label{sec:intro}

Link:\href{https://www.youtube.com/watch?v=umVzk0r0tPo&list=PL8mMEmoXNBfajJ15HO5G-PZW_0yWGWBdh&index=1}{Introduction in Youtube}.

From 0' ~ 14', introduction about this school, its \textbf{success},
its rules, its people (TAs, etc.), Tips, \textbf{No Alcohol}.

From 14' ~ end' , introduction to content of this school. balabala.

\section{Dessau - ARPES}
\label{sec:Dessau}

    \subsection{Dessau 1}
    \label{sec:Dessau 1}

    Link:
    \href{https://www.youtube.com/watch?v=g-qlJ9S-BKY&list=PL8mMEmoXNBfajJ15HO5G-PZW_0yWGWBdh&index=2}{Dessau 1 in Youtube}, Skipped.

    Dessau (University of Colorado, Boulder. Experimentalist). 

    3': \textbf{ARPES for the studies of superconductivity}, this lecture
    will tells us about: what it can measure, and its comparison with
    other spectroscopies.

    5'30'': \textbf{Plan}
    \begin{itemize}
        \item Discussion of technique
        \item Case studies, mostly from p-type cuprates
        \item Focus where ARPES made largest initial impacts.
        \item Early studies using Fermi Liquid theory (failed), Recent ...
        \item Topics: Evolutin of electronic structure from the parent
            Mott state, Fermi surfaces and Fermi arcs, superconducting
            gaps, pseudogaps, self-energy effects.
    \end{itemize}

    \subsection{Dessau 2}
    \label{sec:Dessau 2}
    Link:\href{https://www.youtube.com/watch?v=z0_ncvUxxrg&list=PL8mMEmoXNBfajJ15HO5G-PZW_0yWGWBdh&index=3}{Dessau 2 in Youtube}, Skipped.

    \textbf{Plan}:
    \begin{itemize}
        \item Overview of 2D electronic detection
        \item Studies of gaps (superconducting gaps, pseudogaps)
        \item Mode coupling (dispersion kinks)
    \end{itemize}

    \subsection{Dessau 3}
    \label{sec:Dessau 3}
    Link:\href{https://www.youtube.com/watch?v=DuQLBEBK7SM&list=PL8mMEmoXNBfajJ15HO5G-PZW_0yWGWBdh&index=4}{Dessau 3 in Youtube}, Skipped.

    \textbf{Plan}
    \begin{itemize}
        \item ARPES on n-type cuprate superconductors, esp. "hot spots".
        \item Pseudogaps and SC gaps in p-type cuprates.
        \item Competition between pairing and pair-breaking (electron
            self energies)
            \todo{Maybe I should watch this.}
        \item Linearity, deviation from linearity, ARPES scattering rates,
            and transport.
    \end{itemize}
    \bibliography{cite}{}
    \bibliographystyle{alphaurl}

    \printnomenclature

\section{Martinis - Desig of Superconducting Quantum Computer}
\label{sec:Martinis - Desig of Superconducting Quantum Computer}

    \subsection{Martinis 1}
    \label{sec:Martinis 1}
    Link:\href{https://www.youtube.com/watch?v=CyeIeciAMDU&index=5&list=PL8mMEmoXNBfajJ15HO5G-PZW_0yWGWBdh}{Martinis 1 in Youtube}, Skipped.
    
    John Martinis, from UC Santa Barbara (UCSB).

    \textbf{How Superconducting gives us a very unique, reasonably
    feasible, way to build a Quantum Computer}.

    \textbf{Outline}
    \begin{itemize}
        \item Exponential computing power (why we build)
        \item Hardware Requirements
        \item Physics of Qubits, review
        \item Error correction, the need for fault-tolerant
            computation.
        \item Surface code theory (error-correction and architecture)
        \item Xmon superconducting qubits, integrated circuits for
            scaling above fidelity threshold.
    \end{itemize}

    \subsection{Martinis 2}
    \label{sec:Martinis 2}
    Link:\href{https://www.youtube.com/watch?v=nNeXqsVegUk&list=PL8mMEmoXNBfajJ15HO5G-PZW_0yWGWBdh&index=6}{Martinis 2 in Youtube}, Skipped.
    
    (It seems that there is not outline/plan available.)

\section{Q \& A Week 1}
\label{sec:QA Week 1}
Link:\href{https://www.youtube.com/watch?v=hsPTifCGgoE&list=PL8mMEmoXNBfajJ15HO5G-PZW_0yWGWBdh&index=7}{Youtube}, Skipped.

\section{Randeria - Phenomenology of Superconducting}
\label{sec:Randeria}
Links:
\href{https://www.youtube.com/watch?v=bQ8mB9kTcMY&index=8&list=PL8mMEmoXNBfajJ15HO5G-PZW_0yWGWBdh}{Randeria 1},
\href{https://www.youtube.com/watch?v=YTJE6-XZkKY&list=PL8mMEmoXNBfajJ15HO5G-PZW_0yWGWBdh&index=9}{Randeria 2},
\href{https://www.youtube.com/watch?v=i_nlB9_ZFMQ&list=PL8mMEmoXNBfajJ15HO5G-PZW_0yWGWBdh&index=10}{Randeria 3},
\href{https://www.youtube.com/watch?v=6xSWAL54ZhM&list=PL8mMEmoXNBfajJ15HO5G-PZW_0yWGWBdh&index=11}{Randeria 4}.

Skipped.

\section{Sigris - Symmetry Aspects of Unconventional Superconductivity}
\label{sec:Sigris}
\url{https://www.youtube.com/watch?v=YtEF9sm03rg&list=PL8mMEmoXNBfajJ15HO5G-PZW_0yWGWBdh&index=12}

\section{Keimer - Neutron and x-ray scattering studies of superconductors}
\label{sec:Keimer}
\url{https://www.youtube.com/watch?v=V_Ijobdl-DY&list=PL8mMEmoXNBfajJ15HO5G-PZW_0yWGWBdh&index=16}


\section{Kvelson - Simples models of Interacting Electrons}
\label{sec:Kvelson}
\url{https://www.youtube.com/watch?v=k5qZILzmIDA&list=PL8mMEmoXNBfajJ15HO5G-PZW_0yWGWBdh&index=19}
\todo{I should watch this later too!}

\section{Paramekanti - }
\label{sec:Paramekanti}
\url{https://www.youtube.com/watch?v=8g1YAISQhPM&list=PL8mMEmoXNBfajJ15HO5G-PZW_0yWGWBdh&index=24}

\textbf{Plan}
\begin{itemize}
    \item Phase diagram and broad questions
    \item Some basic microscopics
    \item Mott transition - entropy arguments
    \item Mott transport - renormalized mean field theory
    \item Superconductivity
    \item Parton (Slave Boson) description (Lec.2)
    \item Variational Monte Carlo method (Lec.2)
\end{itemize}

\section{Public Lecture - Steven Kivelson from Stanford}
\label{sec:Public Lecture}
\url{https://www.youtube.com/watch?v=bO1_rs12gpY&index=26&list=PL8mMEmoXNBfajJ15HO5G-PZW_0yWGWBdh}

Outreach effort for junior people. 

Topic: Emmergence, Macroscopics Quantum Phenomenon and High-T Superconductivity.
\todo{watch this later!}

\section{Q \& A Week 2}
\label{sec:QA Week 2}
\url{https://www.youtube.com/watch?v=j1NgPXq_6QM&list=PL8mMEmoXNBfajJ15HO5G-PZW_0yWGWBdh&index=27}

\section{Todadri - Superconductivity, its friends and its enemies,
near Mott transition}
\label{sec:Todadri}

T. Senthil from MIT.



% \section{License}
% The entire content of this work (including the source code
% for TeX files and the generated PDF documents) by 
% Hongxiang Chen (nicknamed we.taper, or just Taper) is
% licensed under a 
% \href{http://creativecommons.org/licenses/by-nc-sa/4.0/}{Creative 
% Commons Attribution-NonCommercial-ShareAlike 4.0 International 
% License}. Permissions beyond the scope of this 
% license may be available at \url{mailto:we.taper[at]gmail[dot]com}.
\end{document}

% The entire content of this work (including the source code
% for TeX files and the generated PDF documents) by 
% Hongxiang Chen (nicknamed we.taper, or just Taper) is
% licensed under a 
% Creative Commons Attribution-NonCommercial-ShareAlike 4.0 
% International License (Link to the complete license text:
% http://creativecommons.org/licenses/by-nc-sa/4.0/).
\documentclass{article}

\usepackage{float}  % For H in figures
\usepackage{amsmath, amssymb} % For math
\usepackage{mathtools} % dcases*, see https://en.wikibooks.org/wiki/LaTeX/Advanced_Mathematics#The_cases_environment
\numberwithin{equation}{subsection} % have the enumeration go to the subsection level.
									% See:https://en.wikibooks.org/wiki/LaTeX/Advanced_Mathematics
\usepackage{graphicx}   % need for figures
\usepackage{cite} % For bibligraphy
\usepackage{fancyref} % For lazy reference \fref
\usepackage[unicode]{hyperref} % For hyperlink everything.
\usepackage{braket} 
\usepackage[T1]{fontenc}
\usepackage{listings}

\usepackage{amsthm}
\newtheorem{defi}{Definition}[section]
\newtheorem{thm}{Theorem}[section]
\newtheorem{lemma}{Lemma}[section]
\newtheorem{remark}{Remark}[section]
\newtheorem{prop}{Proposition}[section]
\newtheorem{coro}{Corollary}[section]
\theoremstyle{definition}
\newtheorem{ex}{Example}[section]

\usepackage[all]{xy} % For drawing category diagrams
\usepackage{mathrsfs}
\usepackage{nomencl} % For a index of symbols.
\makenomenclature % For a index of symbols.
\title{Noetherian Ring}
\date{\today}
\author{we.taper}
\begin{document}
\maketitle
\abstract{
    A note about Noetherian Ring, majorly from the book \cite{lang}.
}
\tableofcontents

\section{Module}
\label{sec:Module}
(\textbf{pp.117 to 118 of \cite{lang}})
\begin{defi}[Module]
    \nomenclature{$M$ Module}{A left module\nomrefpage.}
Let $A$: ring. $M$ is a left $A$-module if and only if 
\begin{enumerate}
    \item $M$ is an abelian group, usually written additively.
    \item there exists an operation of $A$ on $M$, written as a
        multiplicative monoid, such that, for any $a,b\in A$, any
        $x,y \in M$, we have:
        \begin{align}
            (a+b) x = ax + bx\\
            a(x+y) = ax + ay
        \end{align}
\end{enumerate}
\end{defi}

By definition of an operation, we have $1x = x$. Also, it can be easily
derived that $a(-x) = -ax$, and $0x=0$.

\begin{ex}{Examples of modules}
    \begin{enumerate}
        \item $A$ is a module over itself.
        \item Any commutative group is a $\mathbb{Z}$-module.
        \item Any left ideal of $A$ is a module over $A$, i.e. a left
            $A$-module.
        \item A vector space $V$ over $K$, is basically a $K$-module,
            with the additional structure of $K$ being a field.
        \item Let $V$ be a vector space. 
            Let $R$ be the ring of all linear maps of $V$ into itself.
            Then $V$ is also a module over $R$.
    \end{enumerate}
\end{ex}

\begin{defi}[Submodule]
\nomenclature{Submodule}{\nomrefpage.}
    A submodule $M$ is an additive subgroup such that 
    $AN\subset N$
\end{defi}

\begin{defi}[factor module]
\nomenclature{factor module}{\nomrefpage.}
    Let $M$ be an $A$-module, and $N$ a submodule. 
    A factor module $M/N$ is the factor group $M/N$ (for the
    additive group structure) equipped with a module structure.
    The action of $A$ on $M/N$ is defined by $a(x+N) = ax+N$.
    This is well defined, since if $y$ is in the same coset as
    $x$, then $ay$ is in the same coset as $ax$.
\end{defi}
(\textbf{pp.119 of \cite{lang}})

\section{Noetherian}
\label{sec:Noetherian}
\begin{defi}[Noetherian Module]
    \nomenclature{Noetherian Module}{\nomrefpage.}
    Let $A$: a ring. $M$: a left $A$-module. $M$ is called Noetherian
    if $M$ satisfies any of the following conditions:
    \begin{enumerate}
        \item Every submodule of $M$ is finitely generated.
        \item Every ascending sequences of submodules of $M$
            \[ M_1 \subset M_2 \subset \cdots \]
            such that $M_i \neq M_{i+1}$, is fininte.
        \item Every non-empty set $S$ of submodules of $M$ has a maximal
            element.
    \end{enumerate}
\end{defi}

The equivalence of the above conditions are proved in page 413 to 414 of 
\cite{lang}.

\begin{defi}[Noetherian Ring]
\nomenclature{Noetherian Ring}{\nomrefpage.}
    A ring $A$ is Noetherian if and only if it is Noetherian when
    viewed as a left module over itself.
\end{defi}
(\textbf{pp.415 of \cite{lang}})

\paragraph{Some theorems} 

Here are some theorems mentioned in Chapter X,
section 2 of \cite{lang}.

The propositions \ref{prop:Noetherian.1.1},
\ref{prop:Noetherian.1.2}, \ref{coro:Noetherian.1.3} expresses the
"Noetherian relation" between $M$ and its submodules. The proposition
\ref{prop:Noetherian.1.4} relates a Noetherian ring $A$ and the
$A$-modules. The proposition \ref{prop:Noetherian.1.5} relates two
rings. The proposition \ref{prop:Noetherian.1.6} relates a commutative
Noetherian ring and its multiplicative subset. The following diagram
summarized these relations.
$$ \xymatrix{
S^{-1}A & A
            \ar[d]^{\text{prop.} \ref{prop:Noetherian.1.5}}
            \ar[l]^{\text{prop.} \ref{prop:Noetherian.1.6}}
            \ar[r]^{\text{prop.} \ref{prop:Noetherian.1.4}}
        & M\ar[d]^{\text{prop.}:
            \ref{prop:Noetherian.1.1},
            \ref{prop:Noetherian.1.2},
            \text{coro.} \ref{coro:Noetherian.1.3}} \\
            & B       & \text{submodules}\ar[u]
} $$
($A,B$: ring, $S$: $A$'s multiplicative subset. $M$: a $A$-module.)


The structure of being Noetherian is consistent between a module
and its submodules, factor modules, in the sense of the 
following two propositions.

\begin{prop}
    \label{prop:Noetherian.1.1}
    Let $M$ be a Noetherian $A$-module, then every submodule and
    every factor module of $M$ is Noetherian.
\end{prop}
(\textbf{pp.414 of \cite{lang}})

\begin{prop}
    \label{prop:Noetherian.1.2}
    Let $M$ be a module, $N$ be a submodule. If $N$ and $M/N$ are
    Noetherian, then $M$ is Noetherian.
\end{prop}
(\textbf{pp.414 of \cite{lang}})

The above statements could be summarized by saying that, given an
exact sequence:
\begin{align*}
    \xymatrix{
    0\ar[r] & M'\ar[r]^f & M\ar[r]^g & M''\ar[r] & 0
    }
\end{align*}
$M$ is Noetherian if and only if $M'$ and $M''$ are Noetherian.
This could be seen by two immediate fact of an exact sequence:
\[
    M'\cong \text{Im}f, \text{ } M/\text{Ker}g \cong M''
\]

\begin{coro}
    \label{coro:Noetherian.1.3}
    A finite direct sum of Noetherian modules is Noetherian. Specifically,
    let $M$ be a module, let $N,N'$ be two submodules. If $M=N+N'$ and if
    both $N,N'$ are Noetherian, then $M$ is Noetherian.
\end{coro}
(\textbf{pp.415 of \cite{lang}})
\begin{prop}
    \label{prop:Noetherian.1.4}
    Let $A$ be a Noetherian ring, and let $M$ be a finitely generated
    $A$-module. Then $M$ is Noetherian.
\end{prop}
(\textbf{pp.415 of \cite{lang}})
\begin{prop}
    \label{prop:Noetherian.1.5}
    Let $A$ be a ring which is Noetherian, and let $\phi:A\to B$ be
    a surjective ring-homomorphism. Then $B$ is Noetherian.
\end{prop}
(\textbf{pp.415 of \cite{lang}})
In colloquial term, a surjective homomorphism induces a Noetherian ring.
\begin{prop}
    \label{prop:Noetherian.1.6}
    Let $A$ be a commutative Noetherian ring, and let $S$ be a
    multiplicative subset of $A$. Then $S^{-1}A$ is Noetherian.
\end{prop}
(\textbf{pp.415 of \cite{lang}})
\section{Examples}
\label{sec:Examples}
\begin{itemize}
    \item The polynomial Ring
    \item The ring of power series
    \item The ring of formal power series is \textbf{NOT} Noetherian.
        See this \href{http://math.stackexchange.com/questions/281597/an-example-of-a-non-noetherian-ufd}{post}
        \begin{quote}
        The essential point is that the polynomial ring in infinitely many variables is the ascending union of subrings $K[x_1,\ldots,x_n]$, since any polynomial can involve only finitely-many indeterminates. Each of these rings is a UFD, and it is easy to see that a polynomial in which $x_N$ does not appear has only factorizations in which $x_N$ does not appear, again because everything takes place inside some polynomial ring in finitely-many variables. But the ring is not Noetherian, because the ideal generated by all the indeterminates is certainly not finitely-generated.
        \end{quote}
\end{itemize}
\section{Anchor}
% This is just an anchor for seperating thebibliography from above contents.
\printnomenclature
\begin{thebibliography}{1}
    \bibitem{lang} S. Lang. Algebra. 3rd. Springer.
\end{thebibliography}
\section{License}
The entire content of this work (including the source code
for TeX files and the generated PDF documents) by 
Hongxiang Chen (nicknamed we.taper, or just Taper) is
licensed under a 
\href{http://creativecommons.org/licenses/by-nc-sa/4.0/}{Creative 
Commons Attribution-NonCommercial-ShareAlike 4.0 International 
License}. Permissions beyond the scope of this 
license may be available at \url{mailto:we.taper[at]gmail[dot]com}.

\end{document}

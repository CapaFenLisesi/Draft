% The entire content of this work (including the source code
% for TeX files and the generated PDF documents) by 
% Hongxiang Chen (nicknamed we.taper, or just Taper) is
% licensed under a 
% Creative Commons Attribution-NonCommercial-ShareAlike 4.0 
% International License (Link to the complete license text:
% http://creativecommons.org/licenses/by-nc-sa/4.0/).
\documentclass{article}

% My own physics package
% The following line load the package xparse with additional option to
% prevent the annoying warnings, which are caused by the package
% "physics" loaded in package "physicist-taper".
\usepackage[log-declarations=false]{xparse}
\usepackage{physicist-taper}
\makenomenclature % For an index of symbols.

\title{Condensed Matter Field Theory notes}
\date{\today}
\author{Taper}


\begin{document}


\maketitle
\abstract{
Notes of book \cite{Altland2010a}.
}
\tableofcontents

\section{pp.33 eq.1.43}
\label{sec:Table}

In page 33 of \cite{Altland2010a}, the author derives a difference of action, when we
have a symmetry transformation paraterized by $\omega_a$:

\begin{align}
    x_\mu \to x'_\mu &= x_\mu + \frac{\partial x_\mu}{\omega_a}|_{\omega=0} \omega_a(x) \\
    \phi^i(x) \to \phi'(x') &= \phi^i(x) + \omega_a(x)F^i_a[\phi]
\end{align}

We have:
\begin{align}
    \mathcal{L} &= \mathcal{L}(\phi^i(x),\partial_{x_\mu}\phi^i(x)) \\
    \mathcal{L}' &= \mathcal{L'}(\phi'^i(x'),\partial_{x'_\mu}\phi'^i(x')) \\
    &= \mathcal{L}\left(\phi^i+F^i_a\omega_a, 
                    \left(\delta_{\mu\nu}-\partial_{x_\mu}(\omega_a
                    \partial_{\omega_a}x_\mu)\right)
                    \partial_{x_\nu}(\phi^i+F^i_a\omega_a)
                  \right)
\end{align}
And 

\begin{align}
    \Delta S &= \int \dd[m]{x'} \mathcal{L}' - \int \dd[m]{x} \mathcal{L} 
    \label{eq:dS-integrand}\\
    &=
    \int \dd[m]{x}
    \left(1+\partial_{x_\mu}\left(\omega_a \partial_{\omega_a}x_\mu\right) \right) 
    \nonumber\\
    &\times \mathcal{L}\left(
                    \phi^i+F^i_a\omega_a, 
                    \left(\delta_{\mu\nu}-\partial_{x_\mu}\left(\omega_a
                        \partial_{\omega_a}x_\mu\right)\right)
                        \partial_{x_\nu}(\phi^i+F^i_a\omega_a)
                    \right) \nonumber\\
    &- \int \dd[m]{x}  \mathcal{L}(\phi^i(x),\partial_{x_\mu}\phi^i(x))
\end{align}

Then he argues that, "for constant parameters $\omega_a$ the action difference
$\Delta a$ vanishes". Therefore "the leading contribution to the action
difference of a symmetry transformation must be linear in the derivative
$\partial_{x_\mu}\omega_a$".

Then he writes that "A straightforward expansion of the formula above
for $\Delta S$ shows that these terms are given by"

\begin{equation}
    \Delta S = -\int \dd[m]{x} j^a_\mu (x) \partial_{x_\mu}\omega_a
\end{equation}
where $j^a_\mu$ is:
\begin{equation}
    j^a_\mu = \left(
        \frac{\partial \mathcal{L}}{\partial(\partial_{x_\mu}\phi^i)}
            \partial_{x_\nu}\phi^i
        -\mathcal{L}\delta_{\mu\nu} \right) \frac{\partial x_\nu}{\partial\omega_a}
    - \frac{\partial\mathcal{L}}{\partial(\partial_{x_\mu}\phi^i)} F^i_a
\end{equation}

I am partically confused about how to do the "straightforward expansion". I
guess I should do $\frac{\partial}{\partial (\partial_{x_\mu}\omega_a)}$ to the
integrand inside expression for $\Delta S$, though I don't really understand the
reason. Even so, the integrand contains terms like
$\partial_{x_\mu}\partial_{\omega_a}x_\mu$, which I don't know how to deal with.

\textbf{Solution}. The reality is a bit more complicated. We first do a first
order expasion to get the infinitesimal difference:

\begin{align}
    &\mathcal{L}'-\mathcal{L} \\
    \approx &
    \frac{\partial \mathcal{L}}{\partial\phi^i} F^i_a\omega_a
    +\frac{\partial \mathcal{L}}{\partial(\partial_{x_\mu}\phi^i)}
        \left[
        \partial_\mu\left(F^i_a\omega_a\right) 
            - \partial_\mu\left(\omega_a\frac{\partial x_\nu}{\partial\omega_a}\right)
                \partial_\nu\left(\phi^i+F^i_a\omega_a\right)
        \right]
    \nonumber\\
    =&\quad
    \omega_a
    \left[
        \frac{\partial \mathcal{L}}{\partial\phi^i}F^i_a 
        +
        \frac{\partial \mathcal{L}}{\partial(\partial_{x_\mu}\phi^i)}
        \left(
            \partial_\mu F^i_a 
            - \partial_\mu(\frac{\partial x_\nu}{\partial \omega_a})
            \partial_\nu(\phi^i+F^i_a\omega_a)
        \right)
    \right]
    \label{eq:l-l-omega}
    \\
    +&
    \partial_\mu\omega_a
    \left[
        \frac{\partial \mathcal{L}}{\partial(\partial_{x_\mu}\phi^i)}
        \left(
            F^i_a-\frac{\partial x_\nu}{\partial \omega_a}
            \partial_\nu(\phi^i+F^i_a\omega_a)
        \right)
    \right]
    \label{eq:l-l-pmu-omega}
\end{align}

We also discover the integrand in Eq.\ref{eq:dS-integrand} to be
\begin{align}
    &\left(1+\partial_\mu(
        \omega_a\frac{\partial x_\mu}{\partial \omega_a})
        \right)\mathcal{L}'-\mathcal{L} 
    \\
    =& 
    \left(
    1+\partial_\mu(\omega_a\frac{\partial x_\mu}{\partial \omega_a})
    \right)
    (\mathcal{L}'-\mathcal{L})
    +
    \left(\partial_\mu(\omega_a\frac{\partial x_\mu}{\partial \omega_a})\right)
    \mathcal{L}
    \label{eq:integrand-l-density}
\end{align}
For the first term 
$\left(
    1+\partial_\mu(\omega_a\frac{\partial x_\mu}{\partial \omega_a})
\right) (\mathcal{L}'-\mathcal{L}) $, the $(\mathcal{L}'-\mathcal{L})$ already
has terms of first order of $\omega_a$ and of first order of
$\partial_\nu\omega_a$. For our purpose, the second order terms
($\partial_\nu(F^i_a\omega_a)$) from item \ref{eq:l-l-omega} and item
\ref{eq:l-l-pmu-omega} can be ignored. Also, the item
$(\partial_\mu(\omega_a\frac{\partial x_\mu}{\partial
\omega_a}))(\mathcal{L}'-\mathcal{L})$ in eq.\ref{eq:integrand-l-density} can
also be ignored.

Therefore the integrand in Eq.$\ref{eq:dS-integrand}$ becomes

\begin{align}
    & (\mathcal{L}'-\mathcal{L})
    +
    \left(\partial_\mu(\omega_a\frac{\partial x_\mu}{\partial \omega_a})\right)
    \mathcal{L} \\
    = & 
    \omega_a
    \left[
        \frac{\partial \mathcal{L}}{\partial\phi^i}F^i_a 
        +
        \frac{\partial \mathcal{L}}{\partial(\partial_{x_\mu}\phi^i)}
        \left(
            \partial_\mu F^i_a 
            - (\partial_\mu\frac{\partial x_\nu}{\partial \omega_a})
            \partial_\nu(\phi^i+F^i_a\omega_a)
        \right)
        +
        (\partial_\nu \frac{\partial x_\mu}{\partial \omega_a})\mathcal{L}
    \right]
    \\
    +&
    \partial_\mu\omega_a
    \left[
        \frac{\partial \mathcal{L}}{\partial(\partial_{x_\mu}\phi^i)}
        \left(
            F^i_a-\frac{\partial x_\nu}{\partial \omega_a}
            \partial_\nu(\phi^i+F^i_a\omega_a)
        \right)
        +
        \frac{\partial x_\mu}{\partial \omega_a}\mathcal{L}
    \right]
\end{align}

Therefore, the term we seek, i.e. the coefficient of $\partial_\mu\omega_a$ is
\begin{align}
  & \frac{\partial \mathcal{L}}{\partial(\partial_{x_\mu}\phi^i)}
    \left(
        F^i_a-\frac{\partial x_\nu}{\partial \omega_a}
        \partial_\nu(\phi^i+F^i_a\omega_a)
    \right)
    +
    \frac{\partial x_\mu}{\partial \omega_a}\mathcal{L} \\
    =&
    \left(
        \mathcal{L}\delta_{\mu\nu}
        -\frac{\partial \mathcal{L}}{\partial(\partial_{x_\mu}\phi^i)}
            \partial_\nu \phi^i
    \right) \frac{\partial x_\nu}{\partial \omega_a}
    +
    \frac{\partial \mathcal{L}}{\partial(\partial_{x_\mu}\phi^i)}F^i_a
\end{align}
which is what we expect in equation 1.43 of \cite{Altland2010a}.

\textbf{Question}: as for why we should ignore the term with $\omega_a$, there
are two posts (
    \href{http://physics.stackexchange.com/questions/122965/derivation-of-the-noether-current}{[1]},
    \href{http://physics.stackexchange.com/questions/99853/on-a-trick-to-derive-the-noether-current}{[2]} )
might be useful for a thought.

\todo{confusion}

I had great doubt about this problem. Though I have posted an answer on
\href{http://physics.stackexchange.com/questions/122965/derivation-of-the-noether-current}{[1]},
I don't think that answer is satisfactory.

\section{Eq. 3.5}

It is not so obvious to get Eq.3.5 in pp.99 of \cite{Altland2010a}. Here is my
notes.

According to the book, Eq.3.3 is turned into (I set $\hbar=1$ occasionally,
though sometimes I forgot that I have set $\hbar=1$, orz):
\begin{align}
    \bra{q_f} 
    &\int \dd{q_N}\dd{p_N}\ket{q_N}\braket{q_N|p_N}\bra{p_N}
        e^{-i\hat{T}\Delta t}e^{-i\hat{V}\Delta t}\times \nonumber\\
    &   \int \dd{q_{N-1}}\dd{p_{N-1}}\ket{q_{N-1}}\braket{q_{N-1}|p_{N-1}}\bra{p_{N-1}}
        e^{-i\hat{T}\Delta t}e^{-i\hat{V}\Delta t} \times
        \dots \nonumber\\
    &   \int \dd{q_1}\dd{p_1}\ket{q_1}\braket{q_1|p_1}\bra{p_1}
        e^{-i\hat{T}\Delta t}e^{-i\hat{V}\Delta t} \ket{q_i}
\end{align}
Notice that
\todo{T has only p, V has only q}
\begin{align}
    \braket{q|p} &= \frac{\exp(iqp/\hbar)}{\sqrt{2\pi\hbar}} \\
    \bra{p_N}e^{-i\hat{T}\Delta t} &= \bra{p_N}e^{-iT(p_N)\Delta t} \\
    e^{-i\hat{V}\Delta t}\ket{q_{N-1}} &= e^{-iV(q_{N-1})\Delta t}\ket{q_{N-1}} \\
\end{align}
Also, 
\begin{align}
    &\braket{q_N|p_N}\bra{p_N}e^{-i\hat{T}\Delta t}e^{-i\hat{V}\Delta t}\ket{q_{N-1}} 
    = \frac{e^{iq_N p_N/\hbar}}{\sqrt{2\pi\hbar}}
        \bra{p_N}e^{-iT(p_N)\Delta t} e^{-iV(q_{N-1})\Delta t}\ket{q_{N-1}}
    \nonumber\\
    =& \frac{e^{iq_N p_N/\hbar}}{\sqrt{2\pi\hbar}}
        \braket{p_N | q_{N-1}}e^{-iT(p_N)\Delta t} e^{-iV(q_{N-1})\Delta t}
    = \frac{e^{ip_N (q_N-q_{N-1})/\hbar}}{2\pi\hbar}
        e^{-i[T(p_N)+V(q_{N-1})]\Delta t} 
\end{align}
etc. Now we have to pay special attentiont to the start and end. For the start,
we have a
$$ \int \dd{q_N} \braket{q_f | q_N} = \int\dd{q_N}\delta(q_N-q_f) $$
So every $q_N$ is replaced by $q_f$. For the end, we have
$$ \braket{q_1|p_1}\bra{p_1}e^{-i\hat{T}\Delta t}e^{-i\hat{V}\Delta t} \ket{q_i}
= e^{-i[T(p_1)+V(q_i)]}\frac{e^{ip_1(q_1-q_i)}}{2\pi\hbar}
$$

Together we have the whole thing into:

\begin{align}
    \int & \dd{q_1}\cdots\dd{q_{N-1}}\dd{p_1}\dd{p_N} \frac{1}{(2\pi\hbar)^N}
    \times \nonumber\\
  & e^{i\left[p_1(q_1-q_i)+\cdots p_N(q_N-q_{N-1})\right]} \times\nonumber\\
  & e^{-i\left[T(p_1)+\cdots+T(p_N)+V(q_i)+V(q_1)+\cdots+V(q_{N-1})\right]}
\end{align}
which is exactly eq.(3.5) in book.
\section{Eq 9.4}

The Hamiltonian for particle on a ring is claimed to be (Eq. 9.1 of
\cite{Altland2010a}, pp. 498):
\begin{equation}
    H = \frac{1}{2}(-i\partial_\phi -A)^2 = \frac{1}{2}(p-A)^2
\end{equation}

The book \cite{Altland2010a} claims that 
\begin{equation}
    L = \frac{1}{2}\dot{\phi}^2 - iA \dot{\phi}
\end{equation}
I am quite confused, especially about the appearance of $\dot{\phi}$. Can any explain
a bit?

How I tried: Since the inverse of a Legendre transformation is Legendre
transformation itself, 
\begin{align}
    \text{Denote }x &\equiv \frac{\partial H}{\partial p} = p-A,\text{ so,} \\
    p &= x + A,\quad H = \frac{1}{2}x^2 ,\text{ so,}\\
    L = x p - H &= x(x+A) - \frac{1}{2}x^2 = \frac{1}{2}x^2 + x A 
\end{align}
So my calculation found that the Lagrangian of above Hamiltonian is:
\begin{equation}
    L = \frac{1}{2}x^2 + x A
\end{equation}
where
\begin{equation}
    x = \frac{\partial H}{\partial p}
\end{equation}


\bibliography{cite}{}
\bibliographystyle{alpha}

\printnomenclature
\section{License}
The entire content of this work (including the source code
for TeX files and the generated PDF documents) by 
Hongxiang Chen (nicknamed we.taper, or just Taper) is
licensed under a 
\href{http://creativecommons.org/licenses/by-nc-sa/4.0/}{Creative 
Commons Attribution-NonCommercial-ShareAlike 4.0 International 
License}. Permissions beyond the scope of this 
license may be available at \url{mailto:we.taper[at]gmail[dot]com}.
\end{document}

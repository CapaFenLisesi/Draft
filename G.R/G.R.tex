% The entire content of this work (including the source code
% for TeX files and the generated PDF documents) by 
% Hongxiang Chen (nicknamed we.taper, or just Taper) is
% licensed under a 
% Creative Commons Attribution-NonCommercial-ShareAlike 4.0 
% International License (Link to the complete license text:
% http://creativecommons.org/licenses/by-nc-sa/4.0/).
\documentclass{article}

\usepackage{float}  % For H in figures
\usepackage{amsmath} % For math
\usepackage{amssymb}
\usepackage{mathrsfs}
% Followings are for the special character: differential "d".
\newcommand*\diff{\mathop{}\!\mathrm{d}}
\newcommand*\Diff[1]{\mathop{}\!\mathrm{d^#1}}
\numberwithin{equation}{subsection} % have the enumeration go to the subsection level.
                                    % See:https://en.wikibooks.org/wiki/LaTeX/Advanced_Mathematics
\usepackage{graphicx}   % need for figures
\usepackage{cite} % need for bibligraphy.
\usepackage[unicode]{hyperref}  % make every cite a link
\usepackage{CJKutf8} % For Chinese characters
\usepackage{fancyref} % For easy adding figure,equation etc in reference. Use \fref or \Fref instead of \ref
\usepackage{braket} %http://tex.stackexchange.com/questions/214728/braket-notation-in-latex

% Following is for theorems etc environments
% http://tex.stackexchange.com/questions/45817/theorem-definition-lemma-problem-numbering && https://en.wikibooks.org/wiki/LaTeX/Theorems
\usepackage{amsthm}
\newtheorem{defi}{Definition}[section]
\newtheorem{thm}{Theorem}[section]
\newtheorem{lemma}{Lemma}[section]
\newtheorem{remark}{Remark}[section]
\newtheorem{prop}{Proposition}[section]
\newtheorem{coro}{Corollary}[section]
\theoremstyle{definition}
\newtheorem{ex}{Example}[section]

% A list of nomenclatures.
\usepackage{nomencl}
\makenomenclature

\title{Notes for General Relativity}
\date{\today}
\author{Taper}


\begin{document}


\maketitle
\abstract{
    (None)
}
\tableofcontents
\section{The Principle of Relativity}
\label{sec:Principle_of_Relativity}
Landau's book \cite{landau} describes the speed of light, as a speed of the
maximum velocity of propagation of interaction. He perceives velocity as
describing the propagation of interaction.

In another aspect, I think that time does not exist. I perceive time as
a agent of the environmental effect, communicating information and 
coordinating movement between the system under consideration and its 
envrionment. Therefore, a single existance is eternal by nature. Perhaps 
before the universe originates, time is a meaningless concept. In my 
view, the light is the only reliable tool to serve function of
communication and coordination.

Fram either perspective, the speed of light is inherently constent.
However, from my view, it is not clear why this speed should be a maximum.

    \subsection{Invariance of Interval}
    \label{sec:Invariance of Interval}

    To do dynamics, we necessarily need a measure of distance. 
    Here introduces the distance in spacetime - interval. It is:
    \begin{align}
        \diff s^2 = \diff t^2 - \diff x^2 - \diff y^2 - \diff z^2
    \end{align}
    \textbf{Note} that from now on, I will try to use natural units
    as often as possible.

    Such a measure should be invariant in different Lorentz frames.
    Landau proves it by postulating a priori that:
    \begin{align}
        \diff s^2 = a \diff s'^2
    \end{align}
    This is suggested by $\diff s=0$ is invariant (invariance of the
    speed of light), and $\diff s$ and $\diff s'$ should be 
    infinitesimals of the same order. From this postulation, it is
    straightforward to argue that $a$ should be a constant and is
    equal to $1$.

    Using this property, one can classify interval between events
    as being \textit{timelike, spacelike,} and \textit{lightlike},
    by whether $ds>0$, $ds<0$ or $ds=0$. A mnemonic tip is that
    for timelike intervals, the "time difference" is dominant,
    and for spacelike intervals, the spatial difference is
    dominant.

    The following figure gives an indication of how the timelike,
    spacelike, lightlike classification is related to causality:
    \begin{figure}[H]
        \centering
        \includegraphics[width=0.6\linewidth]{pics/{ch1.spacetime_regions}.png}
        \caption{Spacetime regions}
    \end{figure}

    Only those events in the absolute future and in the absolute past
    regions can have a causal link with the people at $O$, due to the 
    limit of speed of propagation. And the future and past combined
    together as the timelike region.

    Another difference of timelike and spacelike regions lies in the
    prefix "absoluate". All events in timelike region cannot be
    simultaneous with $O$ in any reference frame, since the interval
    must have a nonzero time component. Similarly all events in
    spacelike region must happen in different place with the $O$.
    An additional requirement, which comes naturally from law of
    causality, is that all events in the future must remain absolutely
    in the future in any reference frame. Similarly we have an
    absolute past.

    Interestingly, the concept of being simultaneous with $O$, 
    being before or after $O$ in time, are relative for events in
    spacelike region. However, since there can be no causal link
    between $O$ and those events, this relativity does not pose
    a challenge to causality.

    Lastly, the cone formed by all events with $ds=0$ is called the
    \textit{light cone}. Events in it is very special that it deserve
    to devote a separate section to discuss it, which will be done
    later in this note.

    \subsection{Proper Time}
    \label{sec:Proper_Time}
    
    By
    \begin{align*}
        dt'^2 = dt^2-dx^2 = dt^2-(v\diff t)^2
    \end{align*}
    we have 
    \begin{align}
        \label{eq:}
        \Delta t' = \int_{t_1}^{t_2} \diff t \sqrt{1-v^2}
    \end{align}

    This shows the time dilation effect of a moving clock.
    Also, by this we can always calculate the time experienced by a clock
    by $t=\frac{1}{c}\int\diff s$ (SI unit).

    The Landau's book \cite{landau} explains a classical paradox about
    time dilation, which is omitted here.

    Interestingly, we have the property that, for all lines between two
    events, the longest one (i.e. the path that has the longest interval),
    is the straight line, contrary to the classical case. To see this,
    we note that any two events (assumed to be causally linked) can be
    connected by a flying clock with uniform speed.

    \subsection{Light's life}
    \label{sec:Lights_life}
    The life of a proton must be miserable.
    
    Possible sources: \href{http://physics.stackexchange.com/questions/16018/does-a-photon-in-vacuum-have-a-rest-frame}{1},
    \href{http://physics.stackexchange.com/questions/29082/would-time-freeze-if-you-could-travel-at-the-speed-of-light}{2}.
    \href{https://www.quora.com/What-does-the-frame-of-reference-of-a-photon-look-like}{3}.
\begin{thebibliography}{1}
    \bibitem{landau} The Classical Theory of Fields
\end{thebibliography}
\printnomenclature
\section{License}
The entire content of this work (including the source code
for TeX files and the generated PDF documents) by 
Hongxiang Chen (nicknamed we.taper, or just Taper) is
licensed under a 
\href{http://creativecommons.org/licenses/by-nc-sa/4.0/}{Creative 
Commons Attribution-NonCommercial-ShareAlike 4.0 International 
License}. Permissions beyond the scope of this 
license may be available at \url{mailto:we.taper[at]gmail[dot]com}.
\end{document}

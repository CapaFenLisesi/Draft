% The entire content of this work (including the source code
% for TeX files and the generated PDF documents) by 
% Hongxiang Chen (nicknamed we.taper, or just Taper) is
% licensed under a 
% Creative Commons Attribution-NonCommercial-ShareAlike 4.0 
% International License (Link to the complete license text:
% http://creativecommons.org/licenses/by-nc-sa/4.0/).
\documentclass{article}

% My own physics package
% The following line load the package xparse with additional option to
% prevent the annoying warnings, which are caused by the package
% "physics" loaded in package "physicist-taper".
\usepackage[log-declarations=false]{xparse}
\usepackage{physicist-taper}

\makenomenclature

\title{Examples}
\date{\today}
\author{Taper}


\begin{document}

\maketitle
\abstract{
Nothing here.
}
\tableofcontents
\section{Nomenclature}

\nomenclature{aa}{aaa}

\section{Diagram}

\begin{center}
    Bad diagrams
\end{center}
$$
    \begin{tikzcd}[column sep=small]
        E \ar[dr,"f"]\ar[rr,"\rho"]\ar[drr, phantom, "\#", very near
        start] & & F\ar[dl,"\rho'"] \\
        & G &  \,
    \end{tikzcd}
    \begin{tikzcd}[column sep=small]
        E \ar[dr,"f"]\ar[rr,"\rho"]\ar[drr, phantom, "\#"] & & F\ar[dl,"\rho'"] \\
        & G &  \,
    \end{tikzcd}
$$
\begin{center}
    Good diagrams
\end{center}
$$
    \begin{tikzcd}[column sep=small]
        E \ar[dr,"f"]\ar[rr,"\rho"] &
        \ar[d,phantom, "\#", near start] & F\ar[dl,"\rho'"] \\
        & G &  \,
    \end{tikzcd}
$$
$$ \begin{tikzcd}[column sep=small]
& G &  \, \\
E \ar[ur,"f"]\ar[rr,"\rho" below=3] &
\ar[u,phantom, "\#"] 
& F\ar[ul,"\rho'" right=3]
\end{tikzcd} $$

\section{Table}

\begin{table}[ht]
	\centering
	\caption{Classification}
	\label{tab:classification}
	\vspace{+0.5pt}
	\rowcolors{1}{white}{lightgrey}
	\tabulinesep=1.2mm
	\begin{tabu}{  X[l] X[l] X[l] X[l] }
		\textit{Symmetry} & \textit{Spatial Dimension} &
		\textit{Result} & \textit{Other Keywords} \\
		\hline
		T & 0 & An intger: the number of particle-occupied Kramers
		doublet states & \\
		T & 1 & None & \\
		T & 2 & $\mathbb{Z}_2$ & \\
		T & 3 & $\mathbb{Z}_2$ & $3D$ crystals have additional
		$3\mathbb{Z}_2$ invariant$ \Rightarrow$ "weak topological
		insulators \\
		Q(?) & 2 & Characterized by $\mu$ in units of $e^2/h$ & TKNN \\
		Q(?) & even  d & Topological invariant ($k$-th Chern number) \\
		Q(?) & 0 &  number of single-particle states with negative
		energy ($E< E_F = 0$), which are filled with electrons. \\
		T\& Q \\
		No T \& No Q & 0 & $\mathbb{Z}_2$ \\
		No T \& No Q & 1 &$\mathbb{Z}_2$ &"majorana chain"  \\
		No T \& No Q & 2 & Topological number is integer. & Even-odd
		effects. \\
		\bottomrule
	\end{tabu}
\end{table}
\textbf{Tentative Schedule:}
\begin{table}[H]
	\centering
	\caption{caption}
	\label{tab:label}
	%   \begin{tabu}{ *4c }
	\begin{tabu}{ p{15 pt} X[l] X[l] X[r] }
		$\#$ & & & Due date \\
		\hline
		\multicolumn{3}{p{\getTabuwidth}}{1. Summarise the review paper} & December, 2016 \\
		\multicolumn{3}{p{\getTabuwidth}}{2. Learn related mathematical tools (homotopy thoery, group
			cohomology, etc.)} & April, 2017 \\
		\multicolumn3{p{\getTabuwidth}}{3. Play with toy models such
			as the $1D$ quantum walk model} &
		Faburary, 2017 \\
		\multicolumn3{p{\getTabuwidth}}{4. Possible research topics:} & July, 2017 \\
		4.1 &  \multicolumn2{p{\getTabuwidth}}{classifying topological materials in new symmetry
			groups, such as the space groups;} & \multirow{4}{*}{ As above }\\
		4.2 &  \multicolumn2{p{\getTabuwidth}}{finding new ways to classify in the non-interacting picture;} & \\
		4.3 &  \multicolumn2{p{\getTabuwidth}}{experiment about the
			effectiveness of existing;} & \\
		4.4 &  \multicolumn2{p{\getTabuwidth}}{explorer approaches to the
			classification in interacting.} & \\
		\bottomrule
	\end{tabu}
\end{table}
\section{Anchor}
\label{sec:Anchor}

\begin{thebibliography}{1}
    \bibitem{book}  s
\end{thebibliography}
\printnomenclature
\section{License}
The entire content of this work (including the source code
for TeX files and the generated PDF documents) by 
Hongxiang Chen (nicknamed we.taper, or just Taper) is
licensed under a 
\href{http://creativecommons.org/licenses/by-nc-sa/4.0/}{Creative 
Commons Attribution-NonCommercial-ShareAlike 4.0 International 
License}. Permissions beyond the scope of this 
license may be available at \url{mailto:we.taper[at]gmail[dot]com}.
\end{document}

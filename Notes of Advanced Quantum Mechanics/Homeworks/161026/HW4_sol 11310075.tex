% The entire content of this work (including the source code
% for TeX files and the generated PDF documents) by 
% Hongxiang Chen (nicknamed we.taper, or just Taper) is
% licensed under a 
% Creative Commons Attribution-NonCommercial-ShareAlike 4.0 
% International License (Link to the complete license text:
% http://creativecommons.org/licenses/by-nc-sa/4.0/).
\documentclass{article}

\usepackage{float}  % For H in figures
\usepackage{amsmath} % For math
\usepackage{amssymb}
\usepackage{bbm} % for numbers within mathbb
\usepackage{mathrsfs} % For \mathscr{ABC}
% Followings are for the special character: differential "d".
\newcommand*\diff{\mathop{}\!\mathrm{d}}
\newcommand*\Diff[1]{\mathop{}\!\mathrm{d^#1}}
\numberwithin{equation}{subsection} % have the enumeration go to the subsection level.
                                    % See:https://en.wikibooks.org/wiki/LaTeX/Advanced_Mathematics
\usepackage{graphicx}   % need for figures
\usepackage{cite} % need for bibligraphy.
\usepackage[unicode]{hyperref}  % make every cite a link
\usepackage{CJKutf8} % For Chinese characters
\usepackage{fancyref} % For easy adding figure,equation etc in reference. Use \fref or \Fref instead of \ref
\usepackage{braket} %http://tex.stackexchange.com/questions/214728/braket-notation-in-latex

% Following is for theorems etc environments
% http://tex.stackexchange.com/questions/45817/theorem-definition-lemma-problem-numbering && https://en.wikibooks.org/wiki/LaTeX/Theorems
\usepackage{amsthm}
\newtheorem{defi}{Definition}[section]
\newtheorem{thm}{Theorem}[section]
\newtheorem{lemma}{Lemma}[section]
\newtheorem{remark}{Remark}[section]
\newtheorem{prop}{Proposition}[section]
\newtheorem{coro}{Corollary}[section]
\theoremstyle{definition}
\newtheorem{ex}{Example}[section]


\usepackage{CJKutf8}

% A list of nomenclatures.
\usepackage{nomencl}
\makenomenclature

% For drawing diagrams with arrows
\usepackage[all]{xy}

\title{Solution for HW3 20161019}
\date{\today}
\author{Taper}

\begin{document}

\maketitle
\abstract{
\begin{CJK}{UTF8}{gbsn}陈鸿翔(11310075)\end{CJK}
}
%\tableofcontents
\section{Explain the measurement of quantum states}
A quantum system in state $\ket{\phi}$ upon measurement $\mathcal{A}$,
will collpase to one of the eigenstates (for example, $\ket{a}$) of the
corresponding operator $A$. The probability of collapsing one this
eigenstate if determined by $|\braket{\phi|a}|^2$.

\section{Solve the eigenvalues and eigenfunctions of
\texorpdfstring{$\sigma_z$}{}}

Physically we have a spin up eigenstate and a spin down eigenstae. So
the answer can be guessed. They are just:
$\alpha \left( \begin{array}{c}
 1 \\
 0 \\
\end{array} \right)$, with eigenvalue $1$, and
$\beta \left( \begin{array}{c}
 0 \\
 1 \\
\end{array} \right)$, with eigenvalue $-1$. Here $\alpha,\beta$ are
arbitrary nonzero complex constants.

\section{Prove some commutation relationships}
\begin{proof}
    With Einstein summation convention, we can write
\begin{align}
    L_i = \epsilon_{ijk} r_j p_k,\quad [r_j,p_k]= i\hbar\delta_{jk}
\end{align}
    Then
\begin{align}
    [L_i,r_l] &= \epsilon_{ijk} [r_j p_k,r_l] = \epsilon_{ijk}
    r_j[p_k,r_l] = -i\hbar r_j \epsilon_{ijk} \delta_{kl} \nonumber\\
    &= -i\hbar r_j \epsilon_{ijl} = i\hbar\epsilon_{ilj} r_j
\end{align}
Therefore, $[L_i,r_i] = 0$ since $\epsilon_{iij}\equiv 0$.
\end{proof}


\end{document}

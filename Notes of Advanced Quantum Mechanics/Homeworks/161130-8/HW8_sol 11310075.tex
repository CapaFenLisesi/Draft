% The entire content of this work (including the source code
% for TeX files and the generated PDF documents) by 
% Hongxiang Chen (nicknamed we.taper, or just Taper) is
% licensed under a 
% Creative Commons Attribution-NonCommercial-ShareAlike 4.0 
% International License (Link to the complete license text:
% http://creativecommons.org/licenses/by-nc-sa/4.0/).
\documentclass{article}

% My own physics package
% The following line load the package xparse with additional option to
% prevent the annoying warnings, which are caused by the package
% "physics" loaded in package "physicist-taper".
\usepackage[log-declarations=false]{xparse}
\usepackage{physicist-taper}

\title{Solution for HW8}
\date{\today}
\author{Taper}

\begin{document}

\maketitle
\abstract{
\begin{CJK}{UTF8}{gbsn}陈鸿翔(11310075)\end{CJK}
}
%\tableofcontents
\section*{Problem 1}
Observe that the matrix
$$ \begin{pmatrix}
    1 & 1 \\
    1 & -1 
\end{pmatrix}$$
Has eigenvalues $\{-\sqrt{2},\sqrt{2}\}$, and eigenvectors
$$
\begin{pmatrix}
    1-\sqrt{2}\\ 1
\end{pmatrix},\quad \begin{pmatrix}
                        1+\sqrt{2}\\ 1
                    \end{pmatrix}
$$
So the eigenvalues and eigenvectors of $H$ are
\begin{align}
    & E_+=-\sqrt{2}a,\quad
    &\ket{E_+}=\frac{1}{\sqrt{4-2\sqrt{2}}}\left[
        (1-\sqrt{2})\ket{1} + \ket{2} \right] \\
    & E_-=\sqrt{2}a,\quad
    &\ket{E_-}=\frac{1}{\sqrt{4+2\sqrt{2}}}\left[
        (1+\sqrt{2})\ket{1} + \ket{2} \right]
\end{align}

\section*{Problem 2}
\begin{proof}

Write them using Pauli matrices:
$$
S_x = \frac{\hbar}{2} \sigma_x,\, S_y =\frac{\hbar}{2} \sigma_y,
\, S_y = \frac{\hbar}{2}\sigma_z $$

So we have
$$[S_i,S_j] = \frac{\hbar^2}{4}[\sigma_i,\sigma_j]$$
$$\{S_i,S_j\} = \frac{\hbar^2}{4}\{\sigma_i,\sigma_j\}$$
Now let's calculate:
\begin{table}[H]
    \centering
    \begin{tabu}{c c c c}
        $i$ & $j$ & $[\sigma_i,\sigma_j]$ & $\{\sigma_i,\sigma_j\}$ \\
        \hline
        $x$ & $x$ & 0 & $ \begin{pmatrix}
                            2 & 0 \\ 0 & 2
                            \end{pmatrix}=2\id $ \\
        $y$ & $y$ & 0 & $ \begin{pmatrix}
                            2 & 0 \\ 0 & 2
                            \end{pmatrix}=2\id $ \\
        $z$ & $z$ & 0 & $ \begin{pmatrix}
                            2 & 0 \\ 0 & 2
                            \end{pmatrix}=2\id $ \\
        $x$ & $y$ & $ \begin{pmatrix}
                            2i & 0 \\ 0 & -2i
                      \end{pmatrix}=2i\sigma_z$ 
            & $ \begin{pmatrix}
                0 & 0 \\ 0 & 0
            \end{pmatrix} $ \\
        $z$ & $x$ & $ \begin{pmatrix}
                            0 & 2 \\ -2 & 0
                      \end{pmatrix}=2i\sigma_y$ 
            & $ \begin{pmatrix}
                0 & 0 \\ 0 & 0
            \end{pmatrix} $ \\
        $y$ & $x$ & $ \begin{pmatrix}
                            0 & 2i \\ 2i & 0
                      \end{pmatrix}=2i\sigma_z$ 
            & $ \begin{pmatrix}
                0 & 0 \\ 0 & 0
            \end{pmatrix} $ \\
            \hline
    \end{tabu}
\end{table}
% note: tedious calculation performed by the following Mathematica Code:
%   \begin{lstlisting}[language=matlab]
%       i := 2
%       j := 3
%       PauliMatrix[i] // MatrixForm 
%       PauliMatrix[j] // MatrixForm
%       PauliMatrix[i].PauliMatrix[j] - 
%       PauliMatrix[j].PauliMatrix[i] // MatrixForm
%       PauliMatrix[i].PauliMatrix[j] + 
%       PauliMatrix[j].PauliMatrix[i] // MatrixForm
%   \end{lstlisting}
The rest can be calculated using simple rules like $[A,B]=-[B,A]$ and
$\{A,B\}=\{B,A\}$. The above table tells us the relation
$[\sigma_i,\sigma_j]=2i\varepsilon_{ijk}\sigma_k$,
$\{\sigma_i,\sigma_j\}=2\delta_{ij}$. Hence
$$[S_i,S_j]=\frac{\hbar^2}{2}i\varepsilon_{ijk}\sigma_k=
i\hbar\varepsilon_{ijk}S_k$$
$$\{S_i,S_j\}=\frac{\hbar^2}{2}\delta_{ij}=i\hbar\delta_{ij}$$
\end{proof}

\section*{Problem 3}
The Heisenberg equation
\begin{equation*}
    i\hbar \frac{\dd }{\dd t}A(t) = [A(t),H(t)]
\end{equation*}
tells us that
\begin{align*}
    &i\hbar \frac{\dd}{\dd t} S_x(t) = [S_x(t),\omega S_z]
    = -\omega i\hbar S_y \\
    &i\hbar \frac{\dd}{\dd t} S_y(t) = [S_y(t),\omega S_z]
    = \omega i\hbar S_x \\
    &i\hbar \frac{\dd}{\dd t} S_z(t) = [S_z(t),\omega S_z] = 0 \\
\end{align*}
Next
\begin{align*}
    &\frac{\dd^2}{\dd t^2}S_x(t) = -\omega (\omega S_x) \\
    &\frac{\dd^2}{\dd t^2}S_y(t) = \omega (-\omega S_y)
\end{align*}
One see that the solution of $S_x$ and $S_y$ is inextricably related.
Then we fix $S_x$ first.
\begin{align}
    &S_x = A e^{i\omega t} + Be^{-i\omega t} \\
    &S_x(0) = A+B \nonumber\\
    &\frac{\dd}{\dd t}S_x(0) = -\omega S_y(0) = Ai\omega- Bi\omega
    \nonumber\\
    \Rightarrow
    &A = \frac{1}{2}(S_x(0)+iS_y(0)),\quad B=\frac{1}{2}(S_x(0)-iS_y(0))
\end{align}
Similary, we have
\begin{align}
    &S_y = C e^{i\omega t} + De^{-i\omega t} \nonumber\\
    &C = \frac{1}{2}(S_y(0)-iS_x(0)),\quad B=\frac{1}{2}(S_y(0)+iS_x(0))
\end{align}
The rest is 
\begin{equation}
S_z(t)=S_z(0)
\end{equation}
\end{document}

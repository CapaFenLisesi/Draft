% The entire content of this work (including the source code
% for TeX files and the generated PDF documents) by 
% Hongxiang Chen (nicknamed we.taper, or just Taper) is
% licensed under a 
% Creative Commons Attribution-NonCommercial-ShareAlike 4.0 
% International License (Link to the complete license text:
% http://creativecommons.org/licenses/by-nc-sa/4.0/).
\documentclass{article}

% My own physics package
% The following line load the package xparse with additional option to
% prevent the annoying warnings, which are caused by the package
% "physics" loaded in package "physicist-taper".
\usepackage[log-declarations=false]{xparse}
\usepackage{physicist-taper}

\title{Solution for HW5 20161102}
\date{\today}
\author{Taper}

\begin{document}

\maketitle
\abstract{
\begin{CJK}{UTF8}{gbsn}陈鸿翔(11310075)\end{CJK}
}
%\tableofcontents
\section{Uncertainty of electron velocity}
% TODO email the T.A. about my mistake in the last homework.

By uncertainty relationship for $x$ and $p$ we have:
\begin{equation}
    \Delta p \approx \frac{\hbar}{2\Delta x} \approx 5.27\times
    10^{-25} \text{kg}\cdot\text{m/s}
\end{equation}
Then
\begin{equation}
    \Delta v = \frac{\Delta p }{m_e} \approx = 5.8\times 10^5
    \text{m/s}
\end{equation}
\section{Proof}
\begin{proof}
Assuming an orthonormal basis labeled by $n$: $\ket{n}$. Using
Einstein Summation convention, one finds:
\begin{align*}
    \Tr(AB) &= \braket{n|AB|n} = \braket{n|A|m}\braket{m|B|n}
      = \braket{m|B|n}\braket{n|A|m} \\
      &= \braket{m|BA|m} = \Tr(BA)
\end{align*}
Hence
\begin{align*}
    \Tr(XYZ) = \Tr( (XY) Z) = \Tr(Z (XY)) = \Tr( (ZX) Y) = \Tr(YZX)
\end{align*}
\end{proof}

\section{Proof}
\begin{proof}
\begin{align*}
    \braket{[A,B]} = \braket{ AB-BA} = \braket{AB} - \braket{BA} =
    \braket{AB}-\braket{AB}^* = 2i \Im(\braket{AB})
\end{align*}
Hence it is imaginary or zero. Similar, by replacing the $-$ sign
above with $+$ sign, one easily finds:
\begin{align*}
    \braket{\{A,B\}} = 2\Re(\braket{AB})
\end{align*}
So it is real.
\end{proof}

\section{Diagonalization}
$A$ is real and symmetric, hence it is diagonalizable:
\begin{align*}
    &\det\left( \begin{array}{ccc}
     1-\lambda & 1 & 3 \\
     1 & 5-\lambda & 1 \\
     3 & 1 & 1-\lambda \\
    \end{array} \right) = 
    \det\left( \begin{array}{ccc}
     1-\lambda & 1 & 3 \\
     1 & 5-\lambda & 1 \\
     0 & -14+3\lambda & -\lambda-2 \\
        \end{array} \right)\\
     &=(1-\lambda)\left[(5-\lambda)(-\lambda-2)+14-3\lambda\right]
        -(-10\lambda+4)
        = -\lambda^3+7\lambda^2 -36
\end{align*}
The roots are $\lambda_1 = -2$, $\lambda_2=3$, $\lambda_3=6$.
For $\lambda=-2$, we have
$$
\left( \begin{array}{ccc}
 1 & 1 & 3 \\
 1 & 5 & 1 \\
 3 & 1 & 1 \\
\end{array} \right)\left( \begin{array}{c}
 x \\
 y \\
 z \\
\end{array} \right) = 
-2\left( \begin{array}{c}
 x \\
 y \\
 z \\
\end{array} \right)
$$
Or
$$
\left( \begin{array}{ccc}
 3 & 1 & 3 \\
 1 & 7 & 1 \\
 3 & 1 & 3 \\
\end{array} \right)\left( \begin{array}{c}
 x \\
 y \\
 z \\
\end{array} \right) = 0
$$
Or
$$
\left( \begin{array}{ccc}
 0 & -20 & 0 \\
 1 & 7 & 1 \\
 0 & 0 & 0 \\
\end{array} \right)\left( \begin{array}{c}
 x \\
 y \\
 z \\
\end{array} \right) = 0
$$
Obviously the corresponding eigenvector is $\alpha\left( \begin{array}{c}
     1 \\
     0 \\
     -1 \\
\end{array} \right)$, where $\alpha$ is any nonzero complex number.
The case for $\lambda=3$ and $\lambda=6$ can be similar solved by
examing the following two equations:
$$
\left( \begin{array}{ccc}
 -2 & 1 & 3 \\
 1 & 2 & 1 \\
 3 & 1 & -2 \\
\end{array} \right)\left( \begin{array}{c}
 x \\
 y \\
 z \\
\end{array} \right) = 0, \quad
\left( \begin{array}{ccc}
 -5 & 1 & 3 \\
 1 & -1 & 1 \\
 3 & 1 & -5 \\
\end{array} \right)\left( \begin{array}{c}
 x \\
 y \\
 z \\
\end{array} \right) = 0
$$
and the result is summarized as
\begin{table}[H]
    \centering
    \begin{tabular}{c c}
        \textit{Eigenvalue} & \textit{Eigenvector} \\
        \hline
        $-2$ & $\alpha \left( \begin{array}{c}
                     1 \\
                     0 \\
                     -1 \\
                \end{array} \right)$ \\
        $3$ & $\beta \left( \begin{array}{c}
                     1 \\
                     -1 \\
                     1 \\
                \end{array} \right)$ \\
        $6$ & $\gamma \left( \begin{array}{c}
                     1 \\
                     2 \\
                     1 \\
                \end{array} \right)$
    \end{tabular}
\end{table}
where $\alpha,\beta,\gamma$ are arbitrary nonzero complex numbers.
Therefore, take the three eigenvector as basis (to get an orthonormal
    basis, we can let $\alpha=\frac{1}{\sqrt{2}}$,
    $\beta=\frac{1}{\sqrt{3}}$, $\gamma=\frac{1}{\sqrt{6}}$.), we
have:
$$
X^{-1}AX = \left(
\begin{array}{ccc}
 -2 & 0 & 0 \\
 0 & 3 & 0 \\
 0 & 0 & 6 \\
\end{array}
\right) 
$$
where
$$
X=\left( \begin{array}{ccc}
 \alpha & \beta & \gamma \\
 0 & -\beta & 2\gamma \\
 \alpha & \beta & \gamma \\
\end{array} \right)
$$
\end{document}

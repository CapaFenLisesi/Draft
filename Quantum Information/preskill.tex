% The entire content of this work (including the source code
% for TeX files and the generated PDF documents) by 
% Hongxiang Chen (nicknamed we.taper, or just Taper) is
% licensed under a 
% Creative Commons Attribution-NonCommercial-ShareAlike 4.0 
% International License (Link to the complete license text:
% http://creativecommons.org/licenses/by-nc-sa/4.0/).
\documentclass{article}

\usepackage{float}  % For H in figures
\usepackage{amsmath} % For math
\usepackage{amssymb}
\usepackage{mathrsfs}
% Followings are for the special character: differential "d".
\newcommand*\diff{\mathop{}\!\mathrm{d}}
\newcommand*\Diff[1]{\mathop{}\!\mathrm{d^#1}}
\numberwithin{equation}{subsection} % have the enumeration go to the subsection level.
                                    % See:https://en.wikibooks.org/wiki/LaTeX/Advanced_Mathematics
\usepackage{graphicx}   % need for figures
\usepackage{cite} % need for bibligraphy.
\usepackage[unicode]{hyperref}  % make every cite a link
\usepackage{CJKutf8} % For Chinese characters
\usepackage{fancyref} % For easy adding figure,equation etc in reference. Use \fref or \Fref instead of \ref
\usepackage{braket} %http://tex.stackexchange.com/questions/214728/braket-notation-in-latex

% Following is for theorems etc environments
% http://tex.stackexchange.com/questions/45817/theorem-definition-lemma-problem-numbering && https://en.wikibooks.org/wiki/LaTeX/Theorems
\usepackage{amsthm}
\newtheorem{defi}{Definition}[section]
\newtheorem{thm}{Theorem}[section]
\newtheorem{lemma}{Lemma}[section]
\newtheorem{remark}{Remark}[section]
\newtheorem{prop}{Proposition}[section]
\newtheorem{coro}{Corollary}[section]
\theoremstyle{definition}
\newtheorem{ex}{Example}[section]

% A list of nomenclatures.
\usepackage{nomencl}
\makenomenclature

\title{Notes of Preskill's Quantum Information Theory}
\date{\today}
\author{Taper}


\begin{document}


\maketitle
\abstract{
This is a note taken when I read the course notes of Preskill's
course on quantum computation.
}
\tableofcontents
\section{Chapter 2 Foundations I: States and Ensembles}
\label{sec:Chapter_2_Foundations_I_States_and_Ensembles}

\subsection{2.2 The Qubit}
\label{sec:2.2_The_Qubit}
Here the concept of qubit is defined:
\begin{defi}[Qubit]
\nomenclature{Qubit}{\nomrefpage.}
    A qubit is a quantum system described by a two-dimensional 
    Hilbert space.
\end{defi}
However, since measurement collapse the superposition state of quantum
mechanics, there seems to be not difference between a qubit and a 
classical bit whose initial state is only known probabilistically.
More important distinction should be and will be pointed out in the
next few sections.

    \subsubsection{Spin-\texorpdfstring{$\frac{1}{2}$}{}} 
    In this part, the author in fact discusses the symmetry and its role
    in quantum mechanis, albeit roughly.

    \paragraph{Symmetry}
    \begin{defi}[Symmetry]
    \nomenclature{Symmetry}{\nomrefpage.}
    A symmetry is a \textit{transformation} that acts on a state of a
    system, yet leaves all observable properties of the system unchanged.
    \end{defi}
    Hence a symmetry must leaves any $|\braket{\psi|\phi}|^2$ 
    (probability) unchanged. This leads Wigner to derive a famous theorem
    stating that:
    \begin{thm}
        Any symmetry transformation, by adopting suitable phase conventions,
        can always chosen to be either unitary or antiunitary.
    \end{thm}

    He mentions that symmetry transformations formes a group. (Omitted here)

    Here the author focuses on unitary ones, since "The antiunitary
    alternative, while important for discrete symmetries, can be 
    excluded for continuous symmetries".

    Also, he demands that symmetry "respect the dynamical evolution of 
    the system", thus deducing:
    \begin{align}
        [\hat{U}(R),\hat{H}] = 0 \\
        [\hat{Q},\hat{H}] = 0,\text{ }\hat{Q} = \hat{Q}^\dagger \\
        \hat{U} = \hat{I} - i\varepsilon \hat{Q}
    \end{align}
    Here $\hat{U}$ is the symmetry transformation corresponding to the
    symmetry $R$. $Q$ is an observable (since $\hat{Q}$ is hermitian)
    related to this symmetry by above formula, and $\varepsilon$ as
    expected is an infinitesimal number. We also finds that this
    observable is conserved by above formula.

    $\hat{Q}$ is said to be the generator
    \nomenclature{$\hat{Q}$, generator}{\nomrefpage}
    of the symmetry by the following formula:
    \begin{align}
        \label{eq:}
        \hat{U}(R) = 
          \lim_{N\to \infty}(\hat{I}+i\frac{\theta}{N}\hat{Q})^N
          = e^{i\theta \hat{Q}}
    \end{align}

\begin{thebibliography}{1}
    \bibitem{book} Preskill's notes are available through his Caltech
    course website, Phy219.
\end{thebibliography}
\printnomenclature
\section{License}
The entire content of this work (including the source code
for TeX files and the generated PDF documents) by 
Hongxiang Chen (nicknamed we.taper, or just Taper) is
licensed under a 
\href{http://creativecommons.org/licenses/by-nc-sa/4.0/}{Creative 
Commons Attribution-NonCommercial-ShareAlike 4.0 International 
License}. Permissions beyond the scope of this 
license may be available at \url{mailto:we.taper[at]gmail[dot]com}.
\end{document}

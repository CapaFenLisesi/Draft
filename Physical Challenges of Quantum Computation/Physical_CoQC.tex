% The entire content of this work (including the source code
% for TeX files and the generated PDF documents) by 
% Hongxiang Chen (nicknamed we.taper, or just Taper) is
% licensed under a 
% Creative Commons Attribution-NonCommercial-ShareAlike 4.0 
% International License (Link to the complete license text:
% http://creativecommons.org/licenses/by-nc-sa/4.0/).
\documentclass{article}

\usepackage{float}  % For H in figures
\usepackage{amsmath} % For math
\usepackage{amssymb}
\usepackage{mathrsfs}
% Followings are for the special character: differential "d".
\newcommand*\diff{\mathop{}\!\mathrm{d}}
\newcommand*\Diff[1]{\mathop{}\!\mathrm{d^#1}}
\numberwithin{equation}{subsection} % have the enumeration go to the subsection level.
                                    % See:https://en.wikibooks.org/wiki/LaTeX/Advanced_Mathematics
\usepackage{graphicx}   % need for figures
\usepackage{cite} % need for bibligraphy.
\usepackage[unicode]{hyperref}  % make every cite a link
\usepackage{CJKutf8} % For Chinese characters
\usepackage{fancyref} % For easy adding figure,equation etc in reference. Use \fref or \Fref instead of \ref
\usepackage{braket} %http://tex.stackexchange.com/questions/214728/braket-notation-in-latex

% Following is for theorems etc environments
% http://tex.stackexchange.com/questions/45817/theorem-definition-lemma-problem-numbering && https://en.wikibooks.org/wiki/LaTeX/Theorems
\usepackage{amsthm}
\newtheorem{defi}{Definition}[section]
\newtheorem{thm}{Theorem}[section]
\newtheorem{lemma}{Lemma}[section]
\newtheorem{remark}{Remark}[section]
\newtheorem{prop}{Proposition}[section]
\newtheorem{coro}{Corollary}[section]
\theoremstyle{definition}
\newtheorem{ex}{Example}[section]

% A list of nomenclatures.
\usepackage{nomencl}
\makenomenclature

\title{Notes of Physical Challenges of Quantum Computation}
\date{\today}
\author{Taper}


\begin{document}


\maketitle
\abstract{
    This is a note to the dissertation \cite{dissertation} by professor MH Wong in SUSTC.
}
\tableofcontents
\section{Chapter 1 Overview}
\label{sec:Chapter_1_Overview}

    \subsection{1.1 Introduction to quantum computing}
    \label{sec:1.1_Introduction_to_quantum_computing}
    Here he presents some differences between the classical and quantum
    computers. Keywords: \textbf{quantum parallelism}, \textbf{reversible
    computation process}, \textbf{simulation of quantum dynamics}.

    \subsection{1.2~1.3 Physical implementations and challenges 
    of quantum computing}
    
    These two parts do as the title suggests. The methods are summarized
    in the following table:

    \begin{table}[H]
    \centering
    \caption{Different Quantum Computing Approaches}
    %\label{my-label}
    
    \begin{tabular}
        {|p{\dimexpr0.333\textwidth-2\tabcolsep-\arrayrulewidth\relax}|
         p{\dimexpr0.333\textwidth-2\tabcolsep-\arrayrulewidth\relax}|
         p{\dimexpr0.333\textwidth-2\tabcolsep-\arrayrulewidth\relax}|
        }
        %{lll}
    \multicolumn{1}{c}{Name} & \multicolumn{1}{c}{Method} & \multicolumn{1}{c}{Error Prevention} \\ \hline
        gate model 
            & $U\ket{\text{Input}}=\ket{\text{Output}}$ 
            & quantum error correction, 
              analogous to its classical counterparts. \\ \hline
        adiabatic model 
            & Keep the quantum states of qbits in ground state. 
            & Prevent thermalization. \\ \hline
        one-way quantum computing/measurement-based quantum computing 
            & Initialized in cluster state. Computation achived via 
              a series of adaptive measurements. 
            & High quality cluster state. Precision in measurement.
        \\ \hline
    \end{tabular}
    
    \end{table}

    It should be noted that the above classification is not exclusive,
    there are certainly overlap between the three approaches. 
\begin{thebibliography}{1}
    \bibitem{dissertation} From UIUC:
    \url{https://www.ideals.illinois.edu/bitstream/handle/2142/14565/yung_manhong.pdf?sequence=1&isAllowed=y}
\end{thebibliography}
\printnomenclature
\section{License}
The entire content of this work (including the source code
for TeX files and the generated PDF documents) by 
Hongxiang Chen (nicknamed we.taper, or just Taper) is
licensed under a 
\href{http://creativecommons.org/licenses/by-nc-sa/4.0/}{Creative 
Commons Attribution-NonCommercial-ShareAlike 4.0 International 
License}. Permissions beyond the scope of this 
license may be available at \url{mailto:we.taper[at]gmail[dot]com}.
\end{document}

% The entire content of this work (including the source code
% for TeX files and the generated PDF documents) by 
% Hongxiang Chen (nicknamed we.taper, or just Taper) is
% licensed under a 
% Creative Commons Attribution-NonCommercial-ShareAlike 4.0 
% International License (Link to the complete license text:
% http://creativecommons.org/licenses/by-nc-sa/4.0/).
\documentclass{article}

% My own physics package
% The following line load the package xparse with additional option to
% prevent the annoying warnings, which are caused by the package
% "physics" loaded in package "physicist-taper".
\usepackage[log-declarations=false]{xparse}
\usepackage{physicist-taper}

\makenomenclature % For an index of symbols.

\title{Quantum Field Theory in Condensed Matter Physics}
\date{\today}
\author{Taper}


\begin{document}

\maketitle
\abstract{
This is my study notes of various books, listed in the reference.
\nocite{Nagaosa1999}
}
\tableofcontents

\section{Second Quantization}
\label{sec:Second-Quantization}

The \cite{Nagaosa1999} introduces second quantization in a heuristic, non-rigorous
way. It starts from conjecturing that

\begin{itemize}
    \item $N_n\approx \hat{N}_n$, 
        
        $N_n$ represents $N_n$ times repreatance of the experiments done in on a
        single particle. $\hat{N}_n$ represents the results from one experiments
        done on a system of $\hat{N}_n$ non-interacting particles. 
        
        This equality means that we expect these two kinds of experiments to be
        approximately equivalent.
\end{itemize}

With such spirit, we first examine the single-particle state and tries to
promote something to many-body picture. In single particle state, we have
\begin{equation}
    \psi(\vb{r},t) = \sum_n a_n(t) \phi_n(\vb{r})
\end{equation}
as decomposing any wave function into orthonormal basis and concentrate the
dynamics property on the coefficients $a_n(t)$. And we could found
\footnote{Eq (1.2.5) to eq(1.2.10) of \cite{Nagaosa1999}}
\begin{align}
    \frac{\dd a_n(t)}{\dd t} &= 
        \frac{\partial \braket{\hat{H}}}{\partial (i\hbar a^*_n)} \\
    \frac{\dd (i\hbar a_n^*(t))}{\dd t} &= 
        -\frac{\partial \braket{\hat{H}}}{\partial a_n}
\end{align}
where $\braket{H}=\braket{\psi|H|\psi}$. 
These equations are analogous to Hamilton's canonical equations with $a_n
\Leftrightarrow x$, and $i\hbar a^*_n\Leftrightarrow p$. Then propose that we
can promote $a_n$ and $a_n^*$ as operators. The promotion is not analysed but we
can give a quick analogy as:

\begin{itemize}
    \item $N_n\equiv N\abs{a_n}^2$ is promoted to $\hat{N_n}$, which is an
        observable that counts the total number of particles in state $n$.
    \item $\sqrt{N}a_n \rightarrow \hat{A}_n$,
        $\sqrt{N}a^*_n\rightarrow\hat{A}_n^\dagger$. So that
        $\hat{N}_n=\hat{A}_n^\dagger\hat{A}_n$.
\end{itemize}

And 
\begin{equation}
    \begin{cases}
        [\hat{A}_n,\hat{A}_m^\dagger] = \delta_{n,m},
        [\hat{A}_n,\hat{A}_m]=[\hat{A}_n^\dagger,\hat{A}_n^\dagger]=0,& \text{for bosons} \\
        \{\hat{A}_n,\hat{A}_n^\dagger\} = \delta_{n,m},
        \{\hat{A}_n,\hat{A}_m\}=\{\hat{A}_n^\dagger,\hat{A}_n^\dagger\}=0,& \text{for fermions}
    \end{cases}
\end{equation}
and many other usual commutative/anti-commutative relations.

The \textit{wave picture} starts from promoting $\psi$ and $\psi^*$ into
\nomen{field operators}:
\begin{align}
    \hat{\psi}(\vb{r}) &\equiv \sum_n \hat{A}_n \phi_n (\vb{r}) \\
    \hat{\psi}^\dagger(\vb{r}) &\equiv \sum_n \hat{A}_n^\dagger \phi_n^* (\vb{r})
\end{align}
This definition is again analogous to the single particle picture.
Calculation\footnote{Page 15 of \cite{Nagaosa1999}} shows for bosons
\begin{align}
    [\hat{\psi}(\vb{r}),\hat{\psi}^\dagger(\vb{r}')] &= \delta(\vb{r}-\vb{r}') \\
    [\hat{\psi}(\vb{r}),\hat{\psi}(\vb{r}')] &=
        [\hat{\psi}^\dagger(\vb{r}),\hat{\psi}^\dagger(\vb{r}')] = 0
\end{align}
with commutators $[]$ replaced with anti-commutators $\{\}$ for fermions. The
\nomen{particle density operator} is defined as
$n(\vb{r})=\hat{\psi}^\dagger(\vb{r})\hat{\psi}(\vb{r})$. 
and the rest are old things.

The operator new to me is the \nomen{phase operator} $\hat{\theta}_n$ for bosons
\footnote{The phase for fermions is complicated and is delegated to the task of
chapter 5 of \cite{Nagaosa1999}.}. It has the property that:
\begin{align}
    \hat{A}_n^\dagger &= \sqrt{\hat{N}_n} e^{-i\hat{\theta}_n/\hbar} \\
    \hat{A}_n         &= \sqrt{\hat{N}_n} e^{i\hat{\theta}_n/\hbar} \\
    [\hat{N}_n,\hat{\theta}_n] &= i\hbar \label{eq:Ntheta}
\end{align}
It is shonw that equation~\ref{eq:Ntheta} leads to
$[\hat{A}_n,\hat{A}_n^\dagger]=1$. It is shown also that
\begin{align}
    \exp(\frac{i}{\hbar}\hat{\theta}_n) (\hat{N}_n)^m
        \exp(-\frac{i}{\hbar}\hat{\theta}_n) &= (\hat{N}_n+1)^m \\
    \exp(\frac{i}{\hbar}\hat{\theta}_n) \ket{N_n} &\propto \ket{N-1}
    \label{eq:eithetaN}
\end{align}
It is concluded, because of the above equation~\ref{eq:eithetaN} and that there
is not state $\ket{-1}$, the $\hat{\theta}_n$ \textit{is not Hermitian}.
\footnote{See page 17 of \cite{Nagaosa1999}}


\bibliography{cite}
\bibliographystyle{alpha}


\printnomenclature
\section{License}
The entire content of this work (including the source code
for TeX files and the generated PDF documents) by 
Hongxiang Chen (nicknamed we.taper, or just Taper) is
licensed under a 
\href{http://creativecommons.org/licenses/by-nc-sa/4.0/}{Creative 
Commons Attribution-NonCommercial-ShareAlike 4.0 International 
License}. Permissions beyond the scope of this 
license may be available at \url{mailto:we.taper[at]gmail[dot]com}.
\end{document}

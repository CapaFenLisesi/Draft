% The entire content of this work (including the source code
% for TeX files and the generated PDF documents) by 
% Hongxiang Chen (nicknamed we.taper, or just Taper) is
% licensed under a 
% Creative Commons Attribution-NonCommercial-ShareAlike 4.0 
% International License (Link to the complete license text:
% http://creativecommons.org/licenses/by-nc-sa/4.0/).
\documentclass{article}

\usepackage{float}  % For H in figures
\usepackage{amsmath} % For math
\usepackage{amssymb}
\usepackage{bbm} % for numbers within mathbb
\usepackage{mathrsfs} % For \mathscr{ABC}
\numberwithin{equation}{subsection} % have the enumeration go to the subsection level.
                                    % See:https://en.wikibooks.org/wiki/LaTeX/Advanced_Mathematics
\usepackage{graphicx}   % need for figures
\usepackage{cite} % need for bibligraphy.
\usepackage[unicode]{hyperref}  % make every cite a link
\usepackage{CJKutf8} % For Chinese characters
\usepackage{fancyref} % For easy adding figure,equation etc in reference. Use \fref or \Fref instead of \ref

% For highlighting
\usepackage{color,soul}

% Following is for theorems etc environments
% http://tex.stackexchange.com/questions/45817/theorem-definition-lemma-problem-numbering && https://en.wikibooks.org/wiki/LaTeX/Theorems
\usepackage{amsthm}
\newtheorem{defi}{Definition}[section]
\newtheorem{thm}{Theorem}[section]
\newtheorem{lemma}{Lemma}[section]
\newtheorem{remark}{Remark}[section]
\newtheorem{prop}{Proposition}[section]
\newtheorem{coro}{Corollary}[section]
\newtheorem{fact}{Fact}[section]
\theoremstyle{definition}
\newtheorem{ex}{Example}[section]
\newtheorem{argument}{Argument}[section]

% A list of nomenclatures.
\usepackage{nomencl}
\makenomenclature

% For drawing diagrams with arrows
\usepackage[all]{xy}

% -=---------------- My Own New commands ---------------
\usepackage[log-declarations=false]{xparse}
\usepackage{physics-taper}
% Referenced package: \usepackage{physics}

%  Vector notation ---------------------
% \DeclareDocumentCommand\vectorbold{ s m }{
%     \IfBooleanTF{#1}
%     {\boldsymbol{#2}}
%     {\mathbf{#2}}
% } % Vector bold [star for Greek and italic Roman]
% \DeclareDocumentCommand\vb{}{\vectorbold} % Shorthand for \vectorbold
% 
% \DeclareDocumentCommand\vectorunit{ s m }{
%     \IfBooleanTF{#1}
%     {\boldsymbol{\hat{#2}}}
%     {\mathbf{\hat{#2}}}
% } % Unit vector [star for Greek and italic Roman]
% \DeclareDocumentCommand\vu{}{\vectorunit} % Shorthand for \vectorunit
% 
% \DeclareDocumentCommand\vdot{}{\boldsymbol\cdot} % Vector dot product symbol
% \DeclareDocumentCommand\cross{}{\boldsymbol\times} % Vector cross product symbol
% \DeclareDocumentCommand\vnabla{}{\boldsymbol\nabla} % Vector bold \nabla symbol
% \DeclareDocumentCommand\grad{}{\vnabla} 
% \DeclareDocumentCommand\dive{}{\vnabla\vdot} 
% \DeclareDocumentCommand\curl{}{\vnabla\cross}
% \DeclareDocumentCommand\laplacian{}{\nabla^2} 
% 
% % Operators ----------------------------
% \let\imaginary\Im
% \RenewDocumentCommand\Im{g}{
%     \IfNoValueTF{#1}
%     {\operatorname{Im}}
%     {\operatorname{Im}\left\{ #1 \right\}}
% }
% \let\real\Re
% \DeclareDocumentCommand\Re{g}{
%     \IfNoValueTF{#1}
%     {\operatorname{Re}}
%     {\operatorname{Re} \left\{ #1 \right\} }
% }
\title{Notes of Chapter 2 of Bernevig's Book}
\date{\today}
\author{Taper}


\begin{document}


\maketitle
\abstract{
Since I have already made a written one, this note is only an outline
of the written script, in the hope of makeing it eaiser to read.
}
\tableofcontents

\section{Deriving Berry Phase}
\label{sec:Deriving-Berry-Phase}
From page 1 to page 2 (equation 2.7), one derives the expression for
the phase $e^{i\gamma_m}$ using instaneous energy eigenstate for
$E_m$:
\begin{align}
    & H(\vec{R})\ket{n(\vec{R})} = E_m(\vec{R})\ket{n(\vec{R})} \\
    & \gamma_m = i\int_0^t
    \braket{
        m\left(\vec{R}(t')\right)|\frac{\partial }{\partial t}
        m\left(\vec{R}(t')\right) \dd t' }
\end{align}
Then I writes about different ways to get the $\gamma_m$:
\begin{align}
    \gamma_m = i\int_{\text{curve}}\braket{m|\nabla_{\vec{R}}\, m }
    \dd \vec{R}
    = \int_{\text{curve}} \vec{A}_n \vdot\dd \vec{R}
\end{align}
(equation 2.8) 
where one defines
\begin{defi}[Berry Connection, Berry Vector Potential $\vec{A}$]
\nomenclature{Berry Connection, Berry Vector Potential
$\vec{A}$}{\nomrefpage.}
    \begin{align}
        \vec{A}_n \equiv i \braket{n|\nabla_{\vec{R}}\, n }
    \end{align}
\end{defi}
Then I proves several facts:
\begin{fact}
    \label{fact:gamma_is_real}
        $\gamma_n$ is real
\end{fact}
By virtue of fact \ref{fact:gamma_is_real}, we have
\begin{align}
    \gamma_n = -\Im \int_\text{curve} \braket{ n|\nabla_{\vec{R}}\,n}
    \vdot \dd \vec{R} 
\end{align}
\begin{fact}
    Berry connection $\vec{A}_n$ is gauge-dependent.
    The dependence is: If 
    $$\ket{n}\to \ket{n'}= e^{i\xi(\vec{R})}\ket{n}$$
    then
    \begin{align}
        \vec{A}_n \to \vec{A}_n - \nabla_{\vec{R}}\,\xi(\vec{R})
    \end{align}
\end{fact}
Therefore we have
\begin{fact}
    $\gamma_n$ is gauge-dependent, unless the path transverses a
    closed loop.
    Since 
    $$ \gamma_n \to \gamma_n - \left(\xi(\vec{R}(T))-
    \xi(\vec{R}(0)) \right) $$
    It is unchanged unless the integration curve is a closed loop. In
    which case
    $$\xi(\vec{R}(T)) - \xi(\vec{R}(0)) = 2\pi m
    \overset{\text{mod }2\pi}{=} 0 $$
\end{fact}
\begin{ex}
    There is a simple example on page 3 to show that the Berry phase
    can be actually detected.
\end{ex}

When the parameter space is $\mathbb{R}^3$, we have a simpler
expression for berry phase. It is derived on page 4 (equation 2.12)
that
\begin{align}
    \gamma_n = -\Im \oint \braket{\grad_{\vb{R}}
    n|\cross|\grad_{\vb{R}} n} \vdot \dd \vb{s}
    = -\Im \oint \mathcal{A}^n \vdot \dd\vb{s}
\end{align}
where we have defined:
\begin{defi}[Berry Curvature $\mathcal{A}^n$]
\nomenclature{Berry Curvature $\mathcal{A}^n$}{\nomrefpage.}
    \begin{align}
        \mathcal{A}^n = \braket{\grad_{\vb{R}} n|\cross|\grad_{\vb{R}} n}
    \end{align}
    In components (repeated index automatically summed):
    \begin{align}
        \mathcal{A}_i^n = \varepsilon_{ijk} \braket{\partial_j n |
        \partial_k n}
    \end{align}
\end{defi}
By analogy with the theory of electromagnetic fields, this is also
called Berry field, dentoed by $F_{jk}$ in Bernevig's book.
\nomenclature{Berry field $F_{jk}$}{\nomrefpage}. More specifically:
$F_{jk} = \braket{\partial_j n(\vb{R})|\partial_j n(\vb{R})} - (j
    \leftrightarrow k) $.

\section{Gauge-independent calculation of Berry Curvature}
\label{sec:Gauge-independent-calculation-of-Berry-Curvature}
For numerical considerations, we Bernevig gives a new way to calculate
the Berry curvature. I shows in page 5 to 6 that:
\begin{align}
    \gamma_n = -\Im \oint \dd\vb{s} \sum_{m\neq n}
    \frac{\bra{n(\vb{R})}\grad_{\vb{R}} H \ket{m(\vb{R})}
        \cross \bra{m(\vb{R})}\grad_{\vb{R}} H \ket{n(\vb{R})} }
        {(E_n-E_m)^2}
\end{align}
There are two advantages of this formula:
\begin{enumerate}
    \item It is intrinsically gauge-independent (see page 6 for
        explanation).
    \item It is nolonger necessary to pick $\ket{n}$ to be smooth \&
        single-valued. \marginpar{Don't know why}
\end{enumerate}
There are several remarks about this formula on page 7. One important
remark is that:
\begin{fact}
    \begin{align}
        \sum_n \gamma_n = 0
    \end{align}
\end{fact}

\section{Degeneracy and level-crossing}
\label{sec:Degeneracy-and-level-crossing}
Page 7 has a discussion about the serious problems posed by the
degenracy points. Then we turn our attention to two-level system.

    \subsection{Two-level system and Berry connection
    \texorpdfstring{$\vb{A}_n$}{}}
    \label{sec:Two-level-system-and-Berry-connection}
    Here Bernevig considers a system with Hamiltonian
    \begin{equation}
        H = \epsilon(\vb{R}) I + \vb{d}(\vb{R})\vdot\vb{\sigma}
    \end{equation}

    He first calculate in a general setting and concludes that the Berry
    field is
    \begin{equation}
        \vb{V}_{-} = \frac{1}{2}\frac{\vb{d}}{d^3}
    \end{equation}
    However, I don't think it is a good calculation, explained in page 8.
    On the other hand, Bernevig also gives an alternative calculation in
    the following.

        \subsubsection{Two-level system using Hamiltonian approach}
        \label{sec:Two-level-system-using-Hamiltonian-approach}
        On page 8 to 10, I gives the calculation of Berry phase of
        Spin in a varying magnetic environment. Consider the
        Hamiltonian:
        \begin{equation}
            H(\vb{B}) = \vb{B}\vdot \vb{S} 
        \end{equation}
        with
        \begin{equation}
            n = -s, -s+1,\cdots, s
        \end{equation}
        One finds:
        \begin{align}
            \gamma_n = - n\oint \frac{\vb{B}}{B^3} \dd\vb{s}
        \end{align}
        (equation 2.31)
        If the integration is taken on the surface of a sphere, then
        we have (noting that the solid angle $\dd
        \Omega=\sin(\theta)\dd\theta\dd\phi$. Use $\Omega$ to denote
        the solid angle of the integration area)
        \begin{equation}
            e^{i\gamma_n} = e^{-i n \Omega}
        \end{equation}

        There are several remarks on page 10. The important ones are
        \begin{remark}
            If we integrate in a whole sphere, and the sphere contains
            several "poles" - the degeneracy point, we have
            \begin{align}
                |\gamma_i| = 4\pi n \times \text{(number of poles
                inside)}
            \end{align}
            Similary, if the paremeter space in 2 dimensional, then
            \begin{align}
                |\gamma_i| = 2\pi n \times \text{(number of poles
                inside)}
            \end{align}
            Specifically, for electrons
            \begin{align}
                |\gamma_i| = \pi \times \text{(number of poles
                inside)}
            \end{align}
            Therefore, an electron transversing a circle will
            accumulate a phase of $e^{i\pi}=-1$.
        \end{remark}
\section{Anchor}
\label{sec:Anchor}

\begin{thebibliography}{1}
    \bibitem{book} Bernevig's Topological Insulators and
    Superconductors
\end{thebibliography}
\printnomenclature
\section{License}
The entire content of this work (including the source code
for TeX files and the generated PDF documents) by 
Hongxiang Chen (nicknamed we.taper, or just Taper) is
licensed under a 
\href{http://creativecommons.org/licenses/by-nc-sa/4.0/}{Creative 
Commons Attribution-NonCommercial-ShareAlike 4.0 International 
License}. Permissions beyond the scope of this 
license may be available at \url{mailto:we.taper[at]gmail[dot]com}.
\end{document}

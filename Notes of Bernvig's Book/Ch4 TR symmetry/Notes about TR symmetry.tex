% The entire content of this work (including the source code
% for TeX files and the generated PDF documents) by 
% Hongxiang Chen (nicknamed we.taper, or just Taper) is
% licensed under a 
% Creative Commons Attribution-NonCommercial-ShareAlike 4.0 
% International License (Link to the complete license text:
% http://creativecommons.org/licenses/by-nc-sa/4.0/).
\documentclass{article}

\usepackage{float}  % For H in figures
\usepackage{amsmath} % For math
\usepackage{amssymb}
\usepackage{bbm} % for numbers within mathbb
\usepackage{mathrsfs} % For \mathscr{ABC}
% Followings are for the special character: differential "d".
\newcommand*\diff{\mathop{}\!\mathrm{d}}
\newcommand*\Diff[1]{\mathop{}\!\mathrm{d^#1}}
\numberwithin{equation}{subsection} % have the enumeration go to the subsection level.
                                    % See:https://en.wikibooks.org/wiki/LaTeX/Advanced_Mathematics
\usepackage{graphicx}   % need for figures
\usepackage{cite} % need for bibligraphy.
\usepackage[unicode]{hyperref}  % make every cite a link
\usepackage{CJKutf8} % For Chinese characters
\usepackage{fancyref} % For easy adding figure,equation etc in reference. Use \fref or \Fref instead of \ref

% Following is for theorems etc environments
% http://tex.stackexchange.com/questions/45817/theorem-definition-lemma-problem-numbering && https://en.wikibooks.org/wiki/LaTeX/Theorems
\usepackage{amsthm}
\newtheorem{defi}{Definition}[section]
\newtheorem{thm}{Theorem}[section]
\newtheorem{lemma}{Lemma}[section]
\newtheorem{remark}{Remark}[section]
\newtheorem{prop}{Proposition}[section]
\newtheorem{coro}{Corollary}[section]
\newtheorem{fact}{Fact}[section]
\theoremstyle{definition}
\newtheorem{ex}{Example}[section]
\newtheorem{argument}{Argument}[section]

% A list of nomenclatures.
\usepackage{nomencl}
\makenomenclature

% For drawing diagrams with arrows
\usepackage[all]{xy}

% My own physics package
% The following line load the package xparse with additional option to
% prevent the annoying warnings, which are caused by the package
% "physics" loaded in package "physics-taper".
\usepackage[log-declarations=false]{xparse}
\usepackage{physics-taper}

\title{Notes about Spin and TR Symmetry}
\date{\today}
\author{Taper}


\begin{document}


\maketitle
\abstract{
This is a note about TR symmetry, with its application in the theory of
condensed matter physics. The two reference book are the classic by
Sakurai \cite{sakurai}, and the other by Bernevig \cite{bernevig}.
It contains a summary of treatment of Spin (actually, rotation) in
quantum mechanics, which is needed to discuss the TR symmetry operator.
}
\tableofcontents

\section{Review of Spin theory}
\label{sec:Review-of-Spin-theory}
Outline:
\begin{enumerate}
    \item The representation of infinitesimal operator.
    \item From traditional rotational operators to commutation
        relationship
    \item From commutators to their eigenvalues.
    \item Summarize some further results regarding the matrix
        representation of those operators.
\end{enumerate}

\subsection{The representation of transformation}
\label{sec:The-representation-of-transformation}

To find an operator for a physical transformation, we naturally encounters
the theory of representation of groups. Why groups? See the following
notes for an bird view. (The note is an excerpt from my notes about
Quantum Field Theory by Weinberg, which is still in its infancy because
the book is so hard to read)
\begin{quote}
    The set of all symmetries transformations obviously can have a group
    structure. By giving each symmetry transformation a unitary or
    antiunitary operator, a representation of such group is obtained.
    However, such a representation can be projective (projective as in
    projective geometry, or $\mathbb{CP}^n$), since operators acts on ket
    spaces, which is already projective (i.e. within a freedom of phase). 
\end{quote}

Here I outlined one general procedure to obtain such representation from
infinitesimal transformations. This procedure is discussed in chapter 1
and 2 of \cite{sakurai}.
% TODO update more specific location inside that book.
We first needs two thing to characterize a tranformation: a Hermitian
operator $G$ and an infinitesimal change $\varepsilon$. I am not sure about
how to choose $G$, but I think the choice for $\varepsilon$ should be
% TODO how to choose G, and what is G
clear. Here are three typical examples.

For displacement in $x$ direction:
\begin{align}
    G \rightarrow \frac{p_x}{\hbar},\, \varepsilon \rightarrow \diff x
\end{align}
For an infinitesimal time evolution:
\begin{align}
    G \rightarrow \frac{H}{\hbar},\, \varepsilon \rightarrow \diff t
\end{align}
For infinitesimal rotation around $k$ th axis:
\begin{align}
    G \rightarrow \frac{J_k}{\hbar},\, \varepsilon \rightarrow \diff \phi
    \label{eq:revSpin.repreTrans.angulalMomGen}
\end{align}
Or more general (let the rotation be characterized by $\mathbf{J}\cdot
\hat{\mathbf{n}}$, where $\hat{\mathbf{n}}$ is a unit vector), we have:
\begin{align}
    G \rightarrow \frac{\mathbf{J}\cdot \hat{\mathbf{n}} }{\hbar},\,
    \varepsilon \rightarrow \diff \phi
\end{align}

Here $\mathbf{J}=(J_x,J_y,J_z)$ is defined such that the substitution
\ref{eq:revSpin.repreTrans.angulalMomGen} indeed produces a rotation.
What do we mean by "indeed" here? This will be explained in section
\ref{sec:From-classical-mechanic-to-commutators}.
\label{page:question-def_of_J}

Then the operator corresponding to the infinitesimal transformation is
\begin{align}
    U_\varepsilon= 1-i\, G\varepsilon
\end{align}
And for a finite transformation (let $\Delta$ denotes the finite
transformation):
\begin{align}
    \lim_{N\to \infty}\left[ 1-iG\frac{\Delta}{N}\right]^N =
    \exp(-iG\Delta)
\end{align}
For example, in the case of rotation we have (using $\mathscr{D}$ to
denote the operator):
\begin{align}
    \mathscr{D}(\hat{\mathbf{n}},\diff \phi) =
    1-i\frac{\mathbf{J}\cdot\hat{\mathbf{n}}}{\hbar}\diff\phi
    \\
    \mathscr{D}(z,\phi)=\exp(\frac{-iJ_z \phi}{\hbar})
    \label{eq:revSpin.repreTrans.repreFiniteRot}
\end{align}

Note that, here $G$ is regarded as a generator of physical transformation:
\begin{table}[H]
    \centering
    \caption{Generators}
    \begin{tabular}{|c|c|}
    \hline
    Generator                     & Physically generated transformation \\
    \hline
    Momentum $p_x$                & Translation $\diff x$ \\
    \hline
    Energy $H$                    & Time evolution: $\diff t$ \\
    \hline
    Angular momentum $J_x$ & Rotation about $x$ axis \\
    \hline
    \end{tabular}
\end{table}
\subsection{From classical mechanics to commutators in quantum mechanics}
\label{sec:From-classical-mechanic-to-commutators}

This part is discussed in secion 3.1 of \cite{sakurai}.

The commutation relationship is obtained by considering infinitesimal
transformation. In general, classical rotation can be decomposed into
rotation about three axes. We can use linear transformation to express
them. For example, rotation about $z$ axis is:
\begin{align}
    R_z(\phi) = \left( \begin{array}{ccc}
         \cos (\phi ) & -\sin (\phi ) & 0 \\
         \sin (\phi ) & \cos (\phi ) & 0 \\
         0 & 0 & 1 \\ \end{array} \right)
\end{align}
Put it into infinitesimal amount, one has:
\begin{align} 
    R_z(\varepsilon)=\left( \begin{array}{ccc}
     1-\frac{\varepsilon ^2}{2} & -\varepsilon               & 0 \\
     \varepsilon                & 1-\frac{\varepsilon ^2}{2} & 0 \\
     0                          & 0                          & 1 \\
\end{array} \right) \end{align}
Similar procedure could be done for other axes (see equation 3.1.4 and
3.1.5 in \cite{sakurai}). By calculation one finds that
$$ R_x(\varepsilon)R_y(\varepsilon)=R_y(\varepsilon)R_x(\varepsilon)$$
if terms higher than $\varepsilon$ is ignored. This means that
infinitesimal rotation $\diff \mathbf{\omega}$ can be safely considered as
vectors.

However if only terms of order equal and lower than $\varepsilon^2$ is kept
(note especially that $\varepsilon^4$ is ignored), one finds:
\begin{align}
    \label{eq:revSpin.fromCM2QM.classicalRotComu}
    R_x(\varepsilon)R_y(\varepsilon)-R_y(\varepsilon)R_x(\varepsilon)=
    R_z(\varepsilon^2)-\mathbb{I}
\end{align}

Expand equation \ref{eq:revSpin.repreTrans.repreFiniteRot} in orders of
$\phi$ and keep only terms with order $leq \phi^2$, we have:
$$
\mathscr{D}(x,\varepsilon)=
1-\frac{iJ_x \varepsilon}{\hbar}-\frac{J_x^2 \varepsilon^2}{2\hbar^2}
$$
and so on for other axes.

Plug these results into the classical commutation relationship
\ref{eq:revSpin.fromCM2QM.classicalRotComu}, and with some calculation,
one will find:
\begin{align}
    \label{eq:revSpin.fromCM2QM.comuOfAngularM}
    [J_i,J_j]=i\hbar \varepsilon_{ijk}J_k
\end{align}
here $\varepsilon_{ijk}$ is the Levi-Civita notation in $3$ dimension.
This relation also ensures that the operator $\mathbf{J}$ we defined
indeed corresponds to a physical rotation, which answers the question is
mentioned in page \pageref{page:question-def_of_J}.

\subsection{From commutators to eigenvalues}
\label{sec:From-commutators-to-eigenvalues}
Now one may use the commutation relation
\ref{eq:revSpin.fromCM2QM.comuOfAngularM} to get the eigenvalues of these
operators. The procedure is quite standard and can be found in section 3.5
of \cite{sakurai}. There the author defines the ladder operators:
\begin{defi}[Ladder operator]
\nomenclature{Ladder operator}{\nomrefpage.}
    \begin{align}
        J_{\pm}\equiv J_x \pm i J_y
    \end{align}
\end{defi}
Then he derives the cummutation relationship:
\begin{align}
    [\mathbf{J}^2,J_k] &= 0\\
    [J_+,J_-] &=2\hbar J_z \\
    [J_z,J_\pm] &= \pm \hbar J_\pm \\
    [\mathbf{J}^2,J_\pm] &=0 \\
\end{align}
Chossing the $z$ direction as eigenstate. Denote the simultaneous
eigenstates of $\mathbf{J}^2$ and $J_z$ as $\ket{j,m}$, such that:
\begin{align}
    \mathbf{J}^2\ket{j,m} &= j(j+1)\hbar^2 \ket{j,m} \\
    J_z \ket{j,m} &= m \hbar\ket{j,m}
\end{align}
(such a strange choice is conventional in the case of spin).

As usual, one finds that
\begin{itemize}
    \item For $J_z$, $J_\pm$ will pull up/lower its eigenvalue for
        eigenstate in units of $\hbar$.
    \item For $\mathbf{J}^2$, $J_\pm$ will not alter its eigenvectors,
        since it commutes with $\mathbf{J}^2$.
\end{itemize} 
By above relationships, one can derive the following facts (for
convenience let $a$ be the eigenvalue for $\mathbf{J}^2$, and $b$ be the
eigenvalue for $J_z$.):
\begin{itemize}
    \item $a\geq b^2$, i.e $b$ is bounded by $a$. More specifically,
    \item $a=b_{\text{max}}(b_{\text{max}}+\hbar)$.
    \item Also, $-b_{\text{max}}\leq b \leq b_{\text{max}}$.
    \item Using $b_{\text{max}}=b_{\text{min}}+n\hbar$ ($n$ is some
        integer), one gets
    \item $b_{\text{max}}= \frac{n\hbar}{2}$. This, by identifying
        $j=\frac{n}{2}$, one has
\end{itemize}
\begin{align}
    a &= j(j+1)\hbar^2 \\
    b &= m\hbar\\
    (m &= -j, -j+1,\cdots, j-1,j;\, j=\frac{n}{2}) \nonumber
\end{align}
\subsection{Getting the matrix of rotation operators}
\label{sec:Using-Euler-angles-to-get-matrix-of-rotation-operators}

Simply by commutation relationships above, one can easy get:
\begin{align}
    \bra{j',m'}\mathbf{J}^2\ket{j,m} &= 
        j(j+1)\hbar^2 \delta_{j'j}\delta_{m'm} \\
    \bra{j',m'}J_z\ket{j,m} &= 
        m\hbar \delta_{j'j}\delta_{m'm} \\
    \bra{j',m'}J_\pm\ket{j,m} &= 
        \sqrt{(j\mp m)(j\pm m+1)} \hbar \delta_{j'j}\delta_{m',m\pm 1}
        \label{eq:revSpin.getMatRot.J_pm}
\end{align}

To get matrix element for
$\exp(\frac{-i\mathbf{J}\cdot\hat{\mathbf{n}}\phi}{\hbar})$, one needs to
use Euler angles.

    \subsubsection{Eulerian Angles}
    \label{sec:Eulerian-Angles}
    The Euler angles is discussed in section 3.3 of \cite{sakurai}.
    However, Sakurai's usage in his book \cite{sakurai} is a bit different
    from the usual one. But the following graph taken from that section
    should explains all:
    \begin{figure}[H]
        \centering
        \includegraphics[width=1.2\linewidth]{pics/{fig3.4in-Sakurai-Euler-angles}.png}
        \caption{Euler Angles (fig 3.4 in that book)}
        \label{fig:fig3.4-Sakurai-Euler-angles}
    \end{figure}

    One can find that a Euler rotaion characterized by
    $(\alpha,\beta,\gamma)$ could be decomposed into:
    \begin{align}
        R(\alpha,\beta,\gamma)=R_z(\alpha)R_y(\beta)R_z(\gamma)
    \end{align}
    
    \subsubsection{Using Euler angles to get matrix for rotation operator}
    \label{sec:Using Euler angles to get matrix for rotation operator}
    
    As the title suggests, in section 3.5 of \cite{sakurai}, we have
    $$
    \exp\left(\frac{-i\mathbf{J}\cdot\hat{\mathbf{n}}\phi}{\hbar}\right)
    =
    \exp\left(\frac{-i J_z\alpha}{\hbar}\right)
    \exp\left(\frac{-i J_y\beta}{\hbar}\right)
    \exp\left(\frac{-i J_z\gamma}{\hbar}\right)
    $$,
    define 
    \begin{align}
        \nomenclature{
            Wigner function $\mathscr{D}^{(j)_{m'm}}$}{\nomrefpage.}
        \mathscr{D}^{(j)}_{m'm}\equiv
        \bra{j,m'}
            \exp\left(\frac{-i J_z\alpha}{\hbar}\right)
            \exp\left(\frac{-i J_y\beta}{\hbar}\right)
            \exp\left(\frac{-i J_z\gamma}{\hbar}\right)
        \ket{j,m}
    \end{align}
    This $\mathscr{D}$ is sometimes called the \textbf{Wigner function}.
    Note that we have implicitly assumed that the rotation in 
    $\exp\left(\frac{-i\mathbf{J}\cdot\hat{\mathbf{n}}\phi}{\hbar}\right)$
    does not alter the eigenstate of length $\mathbf{J}^2$. This is
    physically satisfying, and can also be proved (equation 3.5.43 in
    \cite{sakurai}).

    % TODO I thin we should distinguish of two concepts of rotation:
    % J and exp(-iJ/h).

    Since we have diagonalize using the eigenstate of $J_z$, we have:
    \begin{align}
        \mathscr{D}^{(j)}_{m'm} =
            e^{-i(m'\alpha+m\gamma)}\bra{j,m'}
                \exp\left(\frac{-i J_y\beta}{\hbar}\right)
            \ket{j,m}
    \end{align}
    The remaining part can only be calculated case by case. Denote it as 
    $d^{(j)}_{m'm}$\nomenclature{ $d^{(j)}_{m'm}$}{\nomrefpage.}, i.e.:
    \begin{align}
        d^{(j)}_{m'm}\equiv \bra{j,m'}
                \exp\left(\frac{-i J_y\beta}{\hbar}\right)
            \ket{j,m}
    \end{align}

    Detailed calculation could start from equation
    \ref{eq:revSpin.getMatRot.J_pm}.
    In case of $j=\frac{1}{2}$, as is familiar for use, it is:
    \begin{align}
        d^{\frac{1}{2}}=\left( \begin{array}{cc}
            \cos \left(\frac{\beta }{2}\right) 
                & -\sin \left(\frac{\beta}{2}\right) \\
            \sin \left(\frac{\beta }{2}\right) 
                & \cos \left(\frac{\beta}{2}\right) \\
            \end{array} \right)
    \end{align}
    Footnote \footnote{This can also be obtained using the Pauli matrices,
    which can be found in my miscellaneous notes.}
    
    \subsubsection{Pauli matrices in half-spin: about \texorpdfstring{$e^{-i\pi S_y/\hbar}$}{}}
    \label{sec:About-exp(-i*pi-S_y/hbar)}

    Since everyone is familiar with the Pauli matrices in spin $1/2$
    system, I here only shows a useful result. This result is related to
    equation (4.4.65) in \cite{sakurai}.

    Since 
    $$S_y=\frac{\hbar}{2}\sigma_y=\frac{\hbar}{2}
        \left( \begin{array}{cc}
         0 & -i \\
         i & 0 \\
        \end{array} \right)$$
    It is easy to check that $\sigma_y^2 = \mathbb{I}$. So:
    \begin{align*}
        e^{-i\phi S_y/\hbar}=
        e^{-i\frac{\phi}{2}\sigma_y}&=\sum_{n=0}^{\infty}
         \frac{(-i\frac{\phi}{2})^n(\sigma_y)^n}{n!}\nonumber\\
         &=\sum_{\text{odd }n}-\frac{i(\frac{\phi}{2})^n}{n!}\sigma_y
           + \sum_{\text{even }n}\frac{(i\frac{\phi}{2})^n}{n!}\mathbb{I}
    \end{align*}
    Using the taylor expansions:
    \begin{align}
        \sinh{x}&= \sum_{n=0}^{\infty}\frac{x^{2n+1}}{(2n+1)!}
        \text{ (odd terms)}\\
        \cosh{x}&= \sum_{n=0}^{\infty}\frac{x^{2n}}{(2n)!}
        \text{ (even terms)}
    \end{align}
    We see that:
    \begin{align}
        e^{-i\phi S_y/\hbar}=e^{-i\frac{\phi}{2}\sigma_y}&=
         - \sigma_y\sinh{i\frac{\phi}{2}}+ \mathbb{I}\cosh{i\frac{\phi}{2}}
         \nonumber\\
         &=
         -i\sigma_y \sin{\frac{\phi}{2}}+\mathbb{I}\cos{\frac{\phi}{2}}
        \\
         &=-i\frac{2}{\hbar}S_y\sin{\frac{\phi}{2}}
            +\mathbb{I}\cos{\frac{\phi}{2}}
        \\
         &=\left( \begin{array}{cc}
            \cos(\frac{\phi }{2}) & -\sin(\frac{\phi }{2}) \\
            \sin (\frac{\phi }{2}) & \cos (\frac{\phi }{2}) \\
         \end{array}\right)
    \end{align}
    So
    $$e^{-i\pi S_y/\hbar} = -i\frac{2S_y}{\hbar}$$
    By exactly the same argument, or simply by the symmetry
    ($x\leftrightarrow y$), we
    have:
    \begin{align}
        e^{-i\phi S_x/\hbar}=e^{-i\frac{\phi}{2}\sigma_x}&=
         -i\sigma_x \sin{\frac{\phi}{2}}+\mathbb{I}\cos{\frac{\phi}{2}}
        \\
        &=-i\frac{2}{\hbar}S_x\sin{\frac{\phi}{2}}
            +\mathbb{I}\cos{\frac{\phi}{2}}
        \\
        &=\left(\begin{array}{cc}
            \cos(\frac{\phi }{2}) & -\sin(\frac{\phi }{2}) \\
            \sin (\frac{\phi }{2}) & \cos(\frac{\phi }{2}) \\
         \end{array}\right)
    \end{align}
    Also, one can easily find that:
    \begin{align}
        e^{-i\phi S_z/\hbar}=e^{-i\frac{\phi}{2}\sigma_z}&=
          \left(\begin{array}{cc}
              e^{-i\frac{\phi }{2}} & 0 \\
              0                     & e^{i\frac{\phi }{2}} \\
         \end{array}\right)
    \end{align}
    Also, we have a formula that encompass all the above relationship
    (proved in page 170, section 3.2 of \cite{sakurai}):
    \begin{align}
        \exp(\frac{-i\vec{\sigma}\cdot\hat{\vec{n}}\phi}{2})
        =
        \cos(\frac{\phi}{2})\mathbb{I}-i\vec\sigma\cdot\hat{\vec{n}}
        \sin(\frac{\phi}{2})
    \end{align}
    where $\vec\sigma=(\sigma_x,\sigma_y,\sigma_z)$, and 
    $\hat{\vec{n}}$ is any unit vector.

    \subsubsection{Eigenstate for \texorpdfstring{$\mathbf{S}\cdot
    \hat{\mathbf{n}}$}{}}
    \label{sec:Eigenstate_for_Sn}
    Here is a physical arguemnt to construct an eigenstate for the
    operator $\mathbf{S}\cdot \hat{\mathbf{n}}$. It mimics what was done
    in page 172 of \cite{sakurai}. This is certainly correct when
    $j=\frac{1}{2}$. However, I am not sure about its validity in case of
    higher spin.
    % TODO check this!

    Suppose we have a state of the highest eigenvalue for $J_z$, i.e.
    a state labeled by $\ket{j,j}$, then in principle any state can
    be obtained from this state after a rotatation and a scale. But since
    $\ket{\mathbf{n}}$ is unit vector, it can be characterized by two
    angle: the polar angle $\beta$ and the azimuthal angles $\alpha$.
    As shown in the following picture:
    \begin{figure}[H]
        \centering
        \includegraphics[width=0.8\linewidth]{pics/{fig3.3in-Sakurai-polar-n-azimuthal}.png}
        \caption{Polar and azimuthal angle (fig 3.3 in \cite{sakurai})}
    \end{figure}

    So I think the eigenstate should be:
    \begin{align}
        \label{eq:revSpin.getMatRot.Sn.Snket}
        \ket{\hat{\mathbf{n}};+}= e^{-iS_z \alpha/\hbar}
        e^{-iS_y \beta/\hbar} \ket{+}
    \end{align}
    here $\ket{+}$ denotes the state $\ket{j,j}$. The eigenvalue is of
    course $j\hbar$.

    This is confirmed when $j=\frac{1}{2}$ in page 171 of \cite{sakurai},
    but not in other cases.
    
\section{Review of Symmetry Operator}
\label{sec:Review-of-Symmetry-Operator}

This part is discussed in section 4.4 of \cite{sakurai}.

Outline:
\begin{enumerate}
    \item A note about antilinear operator, and we should treat it with
        care.
    \item The invariance of probability and Wigner's theorem
    \item The values of measurement before and after transformation
\end{enumerate}

\subsection{A note about antiunitary operator}
\label{sec:A-note-about-antiunitary-operator}
Please note that the Dirac bra-ket notation is only convenient when we are
dealing with unitary operators. When antiunitary operators, or arbitrary
operators are considered, the situation can get tricky. Here are two
examples.

    \subsubsection{Formula for base transformation}
    \label{sec:Formula-for-base-transformation}
    
    Suppose we transform the basis
    $\ket{e_i}$ by an operator $\mathcal{O}$, what is the formula for the
    corresponding basis for the dual space? The requirement of dual basis
    requires that:
    $$ \bra{\widetilde{e}_i} = \bra{\widetilde{e}_i}U^{-1}$$
    But by the "dual-correspondance" in Dirac notation, we expect:
    $$ \bra{\widetilde{e}_i} = \bra{\widetilde{e}_i}U^{\dagger}$$
    The two notion coincide only when $U$ is unitary.

    Similarly, consider the operator $\ket{e_i}\bra{e_i}$, if we perform a
    scale of the basis $\ket{e_i}\to \lambda\ket{e_i}$, then how will the
    above operator transform? Is it
    $$ \ket{\lambda e_i}\bra{\lambda e_i}$$
    or is it:
    $$ \ket{\lambda^{-1} e_i}\bra{\lambda e_i}$$
    Of course, when the transformation is unitary, the two notion
    coincide, because $\lambda^*=\lambda^{-1}$ implies
    $\lambda=\lambda^{-1}$.
    \subsubsection{The adjoint of antilinear operator}
    label{sec:The-adjoint-of-antilinear-operator}
    
    If we were to define adjoint of an antilinear operator, we might
    run into troubles with traditional definition. For example, suppose
    $A$ is antilinear, let us multiply it with $c\mathbbm{1}$, where $c$
    is just some nonzero constant.

    We have:
    \begin{align}
        \braket{\phi|cA \psi} = c\braket{\phi|A\psi}
    \end{align}
    on the other hand:
    \begin{align}
        &\braket{\phi|cA \psi} = \braket{cA\psi|\phi}^*
            = \braket{\psi|A^\dagger c^* \phi}^* 
            =(c\braket{\psi|A^\dagger \phi})^* \nonumber\\
            =& c^* \braket{A\psi|\phi}^* 
            = c^* \braket{\phi|A\psi}
    \end{align}
    A contracdiction. Then, it was proposed, in the first answer to
    this Math.SE post\cite{math.SE-adjoint-of-antilinear}, that we define
    the adjoint operator differently for antilinear operator $A$:
    \begin{align}
        \braket{A^\dagger v|w} = \braket{v| Aw}^*
    \end{align}
    Then the above discrepancy will not exist anymore (easily confirmed).

    Here I also include the complete answer for reference:
    \begin{center}\noindent\rule{8cm}{0.4pt}\end{center}
    \begin{quote}
        I) First of all, one should never use the [Dirac bra-ket notation](http://en.wikipedia.org/wiki/Bra-ket\_notation) (in its ultimate version where an operator acts to the right on kets and to the left on bras) to consider the definition of [adjointness](http://en.wikipedia.org/wiki/Adjoint\_operator), since the notation was designed to make the adjointness property look like a mathematical triviality, which it is not. See also [this](http://physics.stackexchange.com/q/43069/2451) Phys.SE post.

        II) OP's question(v1) about the existence of the adjoint of an [antilinear](http://en.wikipedia.org/wiki/Antilinear\_map) operator is an interesting mathematical question, which is rarely treated in textbooks, because they usually start by assuming that operators are $\mathbb{C}$-linear. 

        III) Let us next recall the mathematical definition of the adjoint of a linear operator. Let there be a [Hilbert space](http://en.wikipedia.org/wiki/Hilbert\_space) $H$ over a [field](http://en.wikipedia.org/wiki/Field\_\%28mathematics\%29) $\mathbb{F}$, which in principle could be either real or complex numbers, $\mathbb{F}=\mathbb{R}$ or $\mathbb{F}=\mathbb{C}$. Of course in quantum mechanics, $\mathbb{F}=\mathbb{C}$. In the complex case, we will use the standard physicist's convention that the [inner product/sequilinear form](http://en.wikipedia.org/wiki/Sesquilinear\_form) $\langle \cdot | \cdot \rangle$ is conjugated $\mathbb{C}$-linear in the first entry, and $\mathbb{C}$-linear in the second entry.

        Recall [Riesz' representation theorem](http://en.wikipedia.org/wiki/Riesz\_representation\_theorem): For each continuous $\mathbb{F}$-linear functional $f: H \to \mathbb{F}$ there exists a unique vector $u\in H$ such that
        $$\text{tag\{1\}} f(\cdot)~=~\langle u | \cdot \rangle.$$

        Let $A:H\to H$ be a continuous$^1$ $\mathbb{F}$-linear operator. Let $v\in H$ be a vector. Consider the continuous $\mathbb{F}$-linear functional 

        $$\text{tag\{2\}} f(\cdot)~=~\langle v | A(\cdot) \rangle.$$ 

        The value $A^{\dagger}v\in H$ of the adjoint operator $A^{\dagger}$ at the vector $v\in H$ is by definition the unique vector $u\in H$, guaranteed by Riesz' representation theorem, such that 
        $$\text{tag\{3\}} f(\cdot)~=~\langle u | \cdot \rangle.$$

        In other words, 
        $$\text{tag\{4\}} \langle A^{\dagger}v | w \rangle~=~\langle u | w \rangle~=~f(w)=\langle v | Aw \rangle. $$

        It is straightforward to check that the adjoint operator $A^{\dagger}:H\to H$ defined this way becomes an $\mathbb{F}$-linear operator as well. 

        IV) Finally, let us return to OP's question and consider the definition of the adjoint of an antilinear operator. The definition will rely on the complex version of Riesz' representation theorem. Let $H$ be given a complex Hilbert space, and let  $A:H\to H$ be an antilinear continuous operator. In this case, the above equations (2) and (4) should be replaced with

        $$\text{tag\{2'\}} f(\cdot)~=~\overline{\langle v | A(\cdot) \rangle},$$

        and

        $$\text{tag\{4'\}} \langle A^{\dagger}v | w \rangle~=~\langle u | w \rangle~=~f(w)=\overline{\langle v | Aw \rangle}, $$

        respectively. Note that $f$ is a $\mathbb{C}$-linear functional.

        It is straightforward to check that the adjoint operator $A^{\dagger}:H\to H$ defined this way becomes an antilinear operator as well.  

        --

        $^{1}$We will ignore subtleties with [discontinuous/unbounded operators](http://en.wikipedia.org/wiki/Unbounded\_operator), domains, [selfadjoint extensions](http://en.wikipedia.org/wiki/Extensions\_of\_symmetric\_operators), etc., in this answer.
    \end{quote}
    
    \begin{center}\noindent\rule{8cm}{0.4pt}\end{center}

    \subsubsection{Conclusion}
    \label{sec:Conclusion}
    For antilinear operators, we should always working with it acting on
    the kets. See page 289 in section 4.4 of \cite{sakurai}.
    
\subsection{Properties of a symmetry operator}
\label{sec:Property-of-symmetry-operator}
Consider a symmetry operation acting on states such that:
\begin{align}
    \ket{\alpha}\to \ket{\widetilde{\alpha}}\,\,
    \ket{\beta}\to \ket{\widetilde{\beta}}
\end{align}
We require that the probability, the most important thing in quantum
mechanics, is unchaged:
\begin{align}
    |\braket{\widetilde{\beta}|\widetilde{\alpha}}| =
        |\braket{\beta|\alpha}|
\end{align}
Wigner has proved that\footnote{
Though it was only mentioned in page 289 of \cite{sakurai}.},
all such symmetry transformations can be characterized either by an unitary
or an antiunitary operator. Also, any antiunitary operator $\theta$ can be
expressed as
\begin{align}
    \theta = U K
\end{align}
where $U$ is unitary, and $K$ is complex conjugation. Three facts about
$K$:
\begin{itemize}
    \item \textit{"acts only on any coefficient that multiplies a ket
        (and stands on the right of K)."} 
    \item $K$ also does not act on the base ket, i.e. $K\ket{e}=\ket{e}$.
    \item $K$'s action depends heavily the base ket. This is obvious from
        the above two points. Of course, this is not important when every
        physical state is only projective. However, I still doubt whether
        this will call additional problem.
        \footnote{Consider $K\ket{e}=\ket{e}$, but when
        $\ket{e}=i\ket{e'}$, then $K\ket{e}=-i\ket{e'}=-\ket{e}$.}
\end{itemize}

\subsection{States and Operators after transformation}
\label{sec:States-and-Operators-after-transformation}
Let $U$ be an unitary operator and $A$ be an antiunitary operator. Suppose
we have $\ket{\alpha}$ and $\ket{\beta}$, and regard both to be
transformed under $U$ and $A$. Let:
\begin{align}
    \ket{\alpha_u}=U\ket{\alpha},\, \ket{\alpha_a}=A\ket{\alpha} \\
    \ket{\beta_u}=U\ket{\beta},\, \ket{\beta_a}=A\ket{\beta}
\end{align}
Then by defintion of unitary and antiunitary, we have:
\begin{align}
    \braket{\beta_u|\alpha_u} &=\braket{\beta|\alpha} \\
    \braket{\beta_a|\alpha_a}^* &=\braket{\beta|\alpha}
\end{align}

For any operator $\mathcal{O}$, we have, after transformation:
\begin{align}
    \mathcal{O}_u &= U\mathcal{O}U^\dagger \\
    \mathcal{O}_a &= A \mathcal{O}^\dagger A^{-1}
\end{align}
The two equation are mean to expresses the following idea:
\begin{align}
    \bra{\alpha}\mathcal{O}\ket{\beta} &=
    \braket{\widetilde{\alpha}|\mathcal{O}_u |\widetilde{\beta}} \\
    \bra{\alpha}\mathcal{O}\ket{\beta} &=
    \braket{\widetilde{\beta}|\mathcal{O}_a |\widetilde{\alpha}}
\end{align}

Proof may be found on page 292 of \cite{sakurai}.
\section{TR operator}

Finally I can turn to the TR operator. But before the formal discussion of
time reversal, I should first note that the name \textbf{time reversal} is
actually a misnomer. As mentioned in page 284, section 4.4 in 
\cite{sakurai}:
\begin{quote}
    What we do in this section can be more appropriately characterized by
    the term \textbf{reversal of motion}. Indeed, that is the phrase used
    by E.  Wigner, who formulated time reversal in a very fundamental
    paper written in 1932.
\end{quote}
This section can be found in section 4.4 of \cite{sakurai}.

Outline:
\begin{enumerate}
    \item Deduce the property of TR symmetry operator from classical
        picture.
    \item The problem of degenracy and how a breaking of TR symmetry lifts
        this degenracy.
\end{enumerate}

    \subsection{From classical notion to properties of TR symmetry}
    \label{sec:From-classical-notion-to-property-of-TR-symmetry}
    To get the time reversal operator, we can first look at the following
    equality (let $T$ dentoes the time reversal operator):
    \begin{align}
        \left(1-\frac{iH}{\hbar}\delta t\right) T \ket{\alpha}
        = T\left(1-\frac{iH}{\hbar}(-\delta t)\right)\ket{\alpha}
        \label{eq:TRoper.fromClas2Quam.phyPic}
    \end{align}
    This follows from the following notion: a time reversed state is
    equivalent to the original state at an earlier time. Recall that
    $\left(1-\frac{iH}{\hbar}\delta t\right)$ produces an infinitesimal
    time translation.

    From equality \ref{eq:TRoper.fromClas2Quam.phyPic}, one finds:
    \begin{align}
        -iHT = T i H
    \end{align}

    One can show that \textbf{$T$ must be antilinear}. Because if one
    assumes that $T$ is linear, one will find that the energy
    spectrum for time reversal symmetric system will always has symmetric
    energy spectrum (i.e. with both and symmetric positive and negative
    energy), which is not true.

    So we have:
    \begin{itemize}
        \item $T$ is antiunitary.
        \item $T H=HT$ for a time reversal symmetric state.
    \end{itemize}
    
    Recall that $T= UK$, and $K$ is highly dependent on the basis
    chosen. So $U$ will depend on the basis too. To get $T$ in a
    basis $\ket{e_i}$, obviously we only need to consider the result of
    $K\ket{e_i}$. This however, is hard to determine. However we can argue
    that in case when the basis has some classical connection, then by the
    Correspondance Principle we have, for example:
    \begin{align}
        T \hat{p} T^{-1}&= -\hat{p} \\
        T \hat{x} T^{-1}&= \hat{x} \\
        T \mathbf{J} T^{-1}&= -\mathbf{J}
    \end{align}
    with these one can easily find:
    \begin{align}
        T \ket{p} &= \ket{-p} \\
        T \ket{x} &= \ket{x} \\
        [x_i, p_j]T \ket{} &= i\hbar \delta_{ij} T\ket{}
    \end{align}
    where $\ket{}$ stands for any ket. Then clearly, $T$ is just
    complex conjugation for wave function $\Psi(x)$. But $T$ acts on
    $\Psi(p)$ will produces $\Psi^*(-p)$. The case for eigenstate of
    spin is a little complicated, which will be discussed in section
    \ref{sec:Time-reversal-operator-in-spin-j}.

    \subsection{Properties of TR operator}
    \label{sec:Properties-of-TR-operator}
    
    Following is a useful theorem, which is easy to proof (proof can be
    found on page 294 of \cite{sakurai}):
    \begin{thm}
        \label{thm:TRoper.fromCM2QM.TR_nondegen_real}
        When Hamiltonian is invariant under time reversal and the energy
        eigenket is nondegenerate; then the corresponding energy
        eigenfunction is real (or, more generally, a real function times a
        phase factor independent of $x$)
    \end{thm}
    Therefore, a complex eigenfunction almost means a degenrate state, or
    the system is not invariant under time reversal.
    
    The next theorem is about $T^2$. Classically, since there is only
    two directions in time, we expect $T^2=1$. But quantum states are
    rays, so there is the freedom of phase. Then $T^2 = \phi$ for
    some $|\phi|^2=1$. But better, we have:
    
    \begin{lemma}
        $T^2 = \pm \mathbbm{1}$
    \end{lemma}
    \begin{proof}
        Proof can be found on page 34 of \cite{bernevig}. But the proof is
        so clever that I want to include it here. First observe that 
        $U U^\dagger = \mathbbm{1}$, implies that $U^* U^T = \mathbbm{1}$
        by transposing both sides. Next one use the expression $T=UK$ to
        expand $T^2$, only to find that $U=\phi U^T$, this means that
        $\phi^2=1$. Hence $\phi=\pm 1$.
    \end{proof}

    We then have the useful fact:
    \begin{fact}
        For any kets $\ket{\alpha},\ket{\beta}$, we have
        \begin{align}
            \pm \braket{\alpha|T\beta} = 
                \braket{\beta|T\alpha}
        \end{align}
        The sign is the same as in $T^2=\pm 1$. Specifically, one has
        \begin{align}
            \braket{\alpha |T \alpha} = 0
        \end{align}
        when $T^2= -\mathbbm{1}$.
    \end{fact}
    \begin{proof}
        This is simple by using 
        $\braket{T\alpha|T\beta} = \braket{\beta|\alpha}$, 
        and replacing $\ket{\alpha}$ with $\ket{T \alpha}$.
    \end{proof}

    
    \subsection{Time reversal operator in system with spin 
    \texorpdfstring{$j$}{}}
    \label{sec:Time-reversal-operator-in-spin-j}
    Using my guess \ref{eq:revSpin.getMatRot.Sn.Snket}:
    \begin{align*}
        \ket{\hat{\mathbf{n}};+}= e^{-iS_z \alpha/\hbar}
        e^{-iS_y \beta/\hbar} \ket{+}
    \end{align*}
    and by similar argument in that section \ref{sec:Eigenstate_for_Sn},
    we can argue that the state with opposite direction is
    \begin{align}
        \ket{\hat{\mathbf{n}};-}= e^{-iS_z \alpha/\hbar}
        e^{-iS_y (\beta+\pi)/\hbar} \ket{+}
    \end{align}
    Again, I am using notation $\ket{-}$ to denotes $\ket{j,-j}$.
    
    Since time reversal reverses the angular momentum, I expects:
    \begin{align}
        T \ket{\hat{\mathbf{n}};+} = \eta \ket{\hat{\mathbf{n}};-}
    \end{align}
    where $\eta$ is just some unnecessay factor.

    With above relationship, one can easily see that:
    \begin{align}
        T = \eta e^{-i\pi J_y/\hbar}K
    \end{align}
    The proof is trivial for $j=\frac{1}{2}$, but I cannot find a proof
    for other values of $j$.

    It is also interesting to note that in general:
    \begin{thm} \begin{align}
        \label{eq:TRoper.TRoper.square_minus}
        T^2 &= -1 \text{ if $j$ is a half-integer} \\
        T^2 &= 1 \text{ if $j$ is an integer}
        \label{eq:TRoper.TRoper.square_plus}
    \end{align} \end{thm}
    However, this can only be proved when $j=\frac{1}{2}$, in which case
    the proof is also straightforward if we use the formula mentioned in
    section \ref{sec:About-exp(-i*pi-S_y/hbar)}.
    
    However, if we imagine a spin $j$ system as composed all just of
    electrons of spin $\frac{1}{2}$, then we have a strong argument that
    the above formula is true.
    \subsection{Kramer's theorem and the lifting of degenracy}
    \label{sec:Kramers-theorem-and-the-lifting-of-degenracy}
    When one is concerned with the energy spectrum, one encounters the
    Kramer's theorem, which says:
    \begin{thm}[Kramer's theorem]
        For a system with halt-integer $j$ spin, the degree of degeneracy
        is at least two.
    \end{thm}
    \begin{proof}
        Let $\ket{n}$ be an energy eigenstate. If the system is time
        reversal symmetric, i.e. $[H,T]=0$, then $T \ket{n}$ is
        another energy eigenstate with the same energy. Supposing there is
        no degeneracy, then we have
        $$ \ket{n} = e^{i\eta}T\ket{n}$$
        for some $\eta\in \mathbb{R}$. Apply $T$ on both sides, then
        we have
        \begin{align*}
            e^{-i\eta} T^2 \ket{n} &= \pm e^{-i\eta}\ket{n} \\
            &= T \ket{n} = e^{i\eta} \ket{n}
        \end{align*}
        Clearly this is a contracdiction unless $T^2=1$, which is
        true only for interger spin system by equation
        \ref{eq:TRoper.TRoper.square_plus} and
        \ref{eq:TRoper.TRoper.square_minus}.
    \end{proof}
    Another way to show this is presented in page 37 of
    \cite{bernevig}. That method is too troublesome and basis dependent 
    \footnote{Borrowing the jargon of mathematicians, not canonical,
        geometric, or intrinsic.}
    to be shown here.
    But it shows an interesting fact that $T^2=-1$, actually
    implies that $U$, its unitary component mentioned in section
    \ref{sec:Property-of-symmetry-operator}, is antisymmetric. This is
    obvious when one juxtapose the following two facts:
    $$ U (U^*)^T = \mathbbm{1}$$
    $$ T^2 = UKUK = U U^* = -\mathbbm{1} $$

    By Kramer's theorem, one also says that for system with half-integer
    spin, one \textit{lift the degenracy by breaking the time reversal
    symmetry of the system.}
    
        \subsubsection{Argument against Bernevig's result}
        \label{sec:Argument again Bernevig's result}
        
        An additional point by Bernevig is about the transition probability
        $\bra{T \psi}H\ket{\psi}$ from a state to its time reversed
        state.

        Here I first mention two points:
        \begin{enumerate}
            \item When $T H=HT$, and $T=UK$, one can deduce
                $UH^* = HU$, or $U^*H=H^*U^*$, i.e.
                $(U^*)_{mp}H_{pn}=(H^*)_{mp}(U^*)_{pn}$.
            \item As mentioned before in section
                \ref{sec:Kramers-theorem-and-the-lifting-of-degenracy},
                $U$ is antisymmetric, so $U^*$ is antisymmetric too.
        \end{enumerate}

        Then begin Bernevig's wrong calculation (Summation implied when
        index is repeated twice):
        \begin{align}
            & \braket{T\psi|H|\psi} = (U_{mp}K\psi_p)^*H_{mn}\psi_n 
            \nonumber\\
            =& U_{mp}^* \psi_p H_{mn} \psi_n 
            \nonumber\\
            =& (U^\dagger)_{pm}\psi_p H_{mn}\psi_n 
            \label{eq:TRoper.Kramer.bernevig1}
            \\
            =& (U^\dagger)_{pm}\psi_p (THT^{-1})_{mn} \psi_n 
            \nonumber\\       
            =& (U^\dagger)_{pm}\psi_p \left(U_{mr}KH_{rq}(-U_{qn}K)\right)
                \psi_n
            \nonumber\\       
            =& - (U^\dagger)_{pm}\psi_p \left(U_{mr}H^*_{rq}U^*_{qn}\right)
                \psi_n
            \nonumber\\       
            =& - \psi_p \left((U^\dagger)_{pm}
                U_{mr}\right)H^*_{rq}U^*_{qn}\psi_n
            \nonumber\\       
            =& - \psi_p H^*_{pq}U^*_{qn}\psi_n
            \nonumber\\       
            \text{(by point 1)}=& - \psi_p U^*_{pq}H_{qn}\psi_n 
            \nonumber\\       
            \text{(by point 2)}=& U^*_{qp}\psi_p H_{qn} \psi_n 
            \nonumber\\       
            \text{(by the first line)}=& \braket{T\psi|H|\psi}
        \end{align}
        So I see that he just circles back to the origin...

        No I give an argument that statement in page 37 of
        \cite{bernevig} cannot be true.
        \begin{argument}
            Suppose we can prove that $\braket{T \psi|H|\psi}=0$.
            Then $\braket{\psi|HT |\psi}=0$, notice that $H=T
            HT^{-1}$, and $T^{2}=-1$, one can see that
            $\braket{\psi|H|\psi}=0$. Such is too strong an result. For
            example, if $\ket{\psi}$ is any eigenket of $H$, then our
            argument shows that its eigenvalue is zero. Does this means
            that $H$ is just the trivial operator $0$? Of course not!
            Therefore the Bernevig's result is wrong.
        \end{argument}
    
    \subsection{Time reversal symmetry in crystal}
    \label{sec:Time-reversal-symmetry-in-crystal}
    
        \subsubsection{Time reversal symmetry in spinless crystal}
        \label{sec:Time-reversal-symmetry-in-spinless-}
        Since there is not spin, time reversal symmetry does not alter a
        creation/annihilation of a particle. Then:
        \begin{align}
            Tc_j T^{-1} = c_j
        \end{align}
        Then, in the fourier expansion of $c_j$:
        \begin{align}
            c_j = \sum_k e^{ikR_j} c_k
        \end{align}
        Since $T$ acting on the left, will change $e^{ikR_j}$ into
        $e^{-ikR_j}$, then naturally:
        \begin{align}
            Tc_k T^{-1} = c_{-k}
        \end{align}
        Then, for a time reversal symmetric system, let 
        $H = \sum_k h(k)c_k^\dagger c_k$. Since $T H T^{-1}=H$,
        $T$ has to flip the sign of $k$ again, so:
        \begin{align}
            T h(k) T^{-1} = h(-k) \,\text{(For TR symmetric
            state)}
            \label{eq:TRoper.trInCry.trInSpinless.hk}
        \end{align}
        
        \begin{remark}[Implications for crystal systems]
            By equation \ref{eq:TRoper.trInCry.trInSpinless.hk}, one has
            $\phi(k)$ and $T\phi(k)$ corresponding to two eigenstate in
            $k$ and $-k$. But since $\bra{\alpha}\ket{T\alpha}$ is not
            necessarily $0$ for $T^2= \mathbbm{1}$, We cannot prove
            that we have a double generacy because the two candidate
            $h(k)$ and $h(-k)$ might be the same wave function.
        \end{remark}

        Another important implication is that
        \begin{thm}
            Hall Conductance vanishes for Spinless time
            symmetric system.  
        \end{thm}
        \begin{proof}
            Recall we have:
            \begin{align}
                \sigma_{ij} = \int \frac{\diff k_x\diff k_y}{(2\pi)^2}
                \sum_{a=1}^{m} (-i) F_{jk}^a(\vb{k})
                = \frac{1}{(2\pi)^2} \sum_{a=1}^m \gamma_a
            \end{align}
            i.e., The Hall conductivity is an integral over the filled
            bands of the Berry curvature. (See (3.79) of\cite{bernevig})

            Here 
            $$F_{ij}(\vb{k}) = -i \left( \braket{\partial_i
            u(k)|\partial_j u(k)} - (i\leftrightarrow j) \right)$$
            is the Berry field strength.
            For system with TR symmetry, we have
            $$ u(k) = T u(-k)$$
            (Note that $T$ flips the sign of $k$) Since $T$ is
            antiunitary, we have $ u(-k) = u(k)^*$. Therefore
            $\braket{\partial_i u(-k)|\partial_j u(-k)} = 
              \braket{\partial_j u(k)|\partial_i u(k)}$
            So:
            \begin{equation}
                F_{ij}(\vb{-k}) = - F_{ij}(\vb{k})
            \end{equation}
            Therefore $F_{ij}$ is a odd function. So its integration
            over the whole BZ is zero, i.e. $\sigma_{ij}$ is zero.
        \end{proof}
        
        
\section{Anchor}
\label{sec:Anchor}
\begin{thebibliography}{1}
    \bibitem{sakurai} Modern Quantum Mechanics. J.J. Sakurai.
    \bibitem{bernevig} Introduction to Topological Insulators. Bernevig.
    \bibitem{math.SE-adjoint-of-antilinear}\href{http://physics.stackexchange.com/questions/45227/existence-of-adjoint-of-an-antilinear-operator-time-reversal}{Existence of adjoint of an antilinear operator, time reversal}
\end{thebibliography}
\printnomenclature
\section{License}
The entire content of this work (including the source code
for TeX files and the generated PDF documents) by 
Hongxiang Chen (nicknamed we.taper, or just Taper) is
licensed under a 
\href{http://creativecommons.org/licenses/by-nc-sa/4.0/}{Creative 
Commons Attribution-NonCommercial-ShareAlike 4.0 International 
License}. Permissions beyond the scope of this 
license may be available at \url{mailto:we.taper[at]gmail[dot]com}.
\end{document}

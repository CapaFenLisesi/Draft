\subsection{Relativistic Quantum Mechanics}
\label{sec:Relatvistic_Quantum_Mechanics}

\subsubsection{2.1 Quantum Mechanics}
\label{sec:2.1_Quantum_Mechanics}
This part sumarize the axioms of quantum mechanics. (Omitted)
However, one class of important but seldom mentioned operator,
the antilinear operators and the antiunitary operators are not
discussed here but postponed to the next part.

\subsubsection{2.2 Symmetries}
\label{sec:2.2_Symmetries}
This part discribe some important theorems concerning the symmetries
in quantum mechanics.

Firstly, symmetries in quantum mechanics some times requires the use
of antilinear and antiunitary operators.

\paragraph{Antilinear and Antiunitary}
Antilinear operators: $U (\lambda A + \upsilon B)
                        = \lambda^* UA + \upsilon^* UB$.

Antiunitary: $ (U \phi, U \psi) = (\psi, \phi) = (\phi,\psi)^*$

The adjoint operator of an antilinear operator requires special
attention:
$$ (\phi, A^\dagger\psi) = (A\phi, \psi)^*$$
(The usual definition will become troublesome since the LHS will
be antilinear in $\phi$, whereas the RHS is linear.)

Thankfully, the criterion for unitarity or antiunitarity is the same:
$$U^\dagger = U^{-1}$$

An important example is those symmetries which can be infinitesimally
close to identity. In the vincinity of unity, it is:
$$ U =1+i\epsilon t $$
where $t$ is Hermitian and linear.
% TODO why is it Hermitian?

\paragraph{Symmetry Group and its Representation} The set of all
symmetries transformations obviously can have a group structure.
By giving each symmetry transformation a unitary or antiunitary
operator, a representation of such group is obtained. However,
such a representation can be projective (projective as in
projective geometry, or $\mathbb{CP}^n$), since operators acts
on ket spaces, which is already projective (i.e. within a freedom of
phase). 

If the phase is taken into consideration, it is found that this
phase is independent of the ket that the operator acts on:
\begin{align}
    \label{eq:2.2_symmetries_projective_phase}
    U(T_1 T_2) \Psi = e^{i\phi} U(T_1) U(T_2) \Psi
\end{align}
Here $U(T_1)$/$U(T_2)$ is the operator corresponding to the symmetry
transformation $T_1$/$T_2$. $\phi$ is the phase, depends only
on $T_1 T_2$.

Howver, there is one exception to this rule. It exists when the
state $\Psi$ is not the linear some of some other wave functions.
That is, we can not express $\Psi$ as $\sum_i \lambda_i \psi_i$.
This seems incomprehesible to me.
% TODO Understand this!
There are some link on page 53
between the structure of Lie group associated
with this symmetry and whether the representation can be projective.
\footnote{Though not state, it can be infered that a representation
can be made not-projective by a good choice of $U(T)$, so that
$\phi \equiv 0$ in \fref{eq:2.2_symmetries_projective_phase} }

\paragraph{Connected Lie group} is of special importance in physics.
However, the books description obfuscate the mathematical description
of this structure. I will update this note later to remedy such
% TODO update this part.
discussion. The message I understand is that the representation is
tightly bound to the Lie algebra. An important example is that when
$$ U(T(\theta)) \approx 1 + i \theta^a t_a $$
$$ [t_b,t_c] = 0$$
then
$$ U(T(\theta)) = \exp(it_a \theta^a) $$

\subsubsection{2.3 Quantum Lorentz Transformations}
\label{sec:2.3_Quantum_Lorentz_Transformations}
\textbf{Note}: although the title suggests "quantum", this part is
mostly classical.

Starting with the invariance of interval:
\begin{align}
    \label{eq:2.3_Quantum_Lorentz_Transformations invariance_of_interval}
    \eta_{\mu\nu}dx'^\mu dx'^\nu = \eta_{\mu\nu}dx^\mu dx^\nu
\end{align}
or equivalently % TODO why equivalently?
\begin{align}
    \label{eq:2.3_Quantum_Lorentz_Transformations invariance_of_interval 2}
    \eta_{\mu\nu} \frac{\partial x'^\mu}{\partial x^\rho}
        \frac{\partial x'^\nu}{\partial x^\sigma}
        =
    \eta_{\rho\sigma}
\end{align}
It claims that any coordinate transformation satisfying 
\ref{eq:2.3_Quantum_Lorentz_Transformations invariance_of_interval 2}
must be linear.

Such linear Lorentz transformations satisfy:
\begin{align}
    T(\bar{\Lambda},\bar{a})T(\Lambda,a) = 
        T(\bar{\Lambda}\Lambda, \bar{\Lambda}a+\bar{a})
\end{align}
where $\bar{\Lambda}+\bar{a}$ and $\Lambda+a$ are two such 
transformations.

The $\Lambda$ and $\Lambda^{-1}$ can be determined by several 
equations that relate it with $\eta_{\mu\nu}$. 

Note that:
\[ \Lambda\indices{^\rho_\nu} \neq \Lambda\indices{_\nu^\rho} \]
And:
\begin{equation}
    \label{eq:2.3_Quantum_Lorentz_Transformations transformation_matrix}
    (\Lambda^{-1})\indices{^\rho_\nu} = 
        \Lambda\indices{_\nu^\rho} = 
        \eta_{\nu\mu}\eta^{\rho\sigma} \Lambda\indices{^\mu_\sigma}
\end{equation}

The only part about quantum mechanics here, is that the operators
corresponding to the Lorentz transformations, have the property that:
\begin{equation}
    U(\bar\Lambda,\bar{a})U(\Lambda, a) = 
        U(\bar\Lambda\Lambda, \bar\Lambda a + \bar{a})
\end{equation}
where $a$ represents the translation.

The group of Lorentz transformations is very important. There is:
    %\text{sssssss} & B
\[
    \xymatrix{ 
    \text{Poincar\'e Group}  \\
    \text{Homogeneous Lorentz group} \ar@{^{(}->}[u] \\
    \text{proper orthochronous Lorentz group}\ar@{^{(}->}[u]
    }
\]
The important \textit{proper orthochronous Lorentz group} consists
of those $\Lambda$ with $\det \Lambda = 1$ and 
$\Lambda\indices{^0_0} \geq 1$.

We have also $ \mathscr{P}$, the space inversion matrix. And
\(\mathscr{T}\), the time-reversal matrix. The definition for
both can be easily written.

A important rule is that all homogeneous Lorentz transformations can
are generated by 
\(\{\text{proper orthochronous}, \mathscr{P}, \mathscr{T} \}\).
However this property is not proved

\paragraph{Digression about tensor notation} It seems that phsysicists
have not settled down on their notation for tensors. Weinberg using
the notation $\Lambda\indices{^\nu_\mu} \equiv \Lambda(e^\nu,e_\mu)$,
so it acts on $(v,w)$, where $v$ is a cotangent vector and $w$ is a
tangent vector. I see that this notation helps to distinguish the
type of arguments. This is unnecessary, since the upper and lower
indices already fulfill this function. Another benefit is that it
conveies the process of lowering of raising a index. 

However, this idea that each slot in tensor indices are distinct
and can be lowered and raised seperatly, is not even mentioned
in some modern mathematical physics book.

Helpful link:
\href{http://physics.stackexchange.com/questions/158309/convention-of-tensor-indices}{Convention of tensor indices in Phy.SE}, and 
\href{http://physics.stackexchange.com/questions/237270/working-with-indices-of-tensors-in-special-relativity?noredirect=1&lq=1}{Working with indices of tensors in special relativity in Phy.SE}.

\subsubsection{2.4 The Poincar\'{e} Algebra}
\label{sec:2.4_The_Poincare_Algebra}
\textbf{I am lost in this part.} The general idea seems to develop, in a 
infinitesimal sense, the property of a Lorentz transformation.
For a infinitesimal Lorentz transformation:
\begin{align}
    U(1+\omega, \epsilon) = 
        1+ \frac{1}{2} i\omega_{\rho\sigma}J^{\rho\sigma}
        -i\epsilon_\rho P^\rho + \cdots
\end{align}
where $J$ and $P$ are independent of the infinitesimal value
$\omega$ and $\epsilon$.

Later $P$ is identified as the energy-momentum operator (with
$P^0$ being the energy operator), and $J^{23}$, $J^{31}$,
\(J^{12}\) are identified with the angular momentum operator.
However, the reason for this identification is not provided. 
% TODO why P is energy momentum, and J is angular momentum

Later, it was eatablished that $J$ and $P$ satisfy:
\begin{align}
    i[J^{\mu\nu},J^{\rho\sigma}] &=
        \eta^{\nu\rho}J^{\mu\sigma} - \eta^{\mu\rho}J^{\nu\sigma}
        -\eta^{\sigma\mu}J^{\rho\nu} + \eta^{\sigma\nu}J^{\rho\mu}
    \\
    i[P^\mu,J^{\rho\sigma} &=
        \eta^{\mu\rho}P^\sigma - \eta^{\mu\sigma}P^{\rho}
    \\
    [P^\mu,P^\rho] &= 0
\end{align}
This is called the \textit{Lie Algebra} of the Poincar\'{e}
group.
The relation \([P^\mu,P^\rho] = 0\) is particular interesting.

With this, the finite translation and a rotation by angle
$\theta$ are expressed as:
\begin{align}
    U(1,a) = \exp(-iP^\mu a_\mu) \\
    U(\theta,0) = \exp(i \mathbf{J}\cdot \theta)
\end{align}

This part ends with a discussion on the low-velocity limit
of the Lie Algebra obtained above, i.e. the Galilean algebra.

\subsection{Note}
\label{sec:Note}
I will now change to another book: Q.F.T. in a Nutshell by A. Zee.

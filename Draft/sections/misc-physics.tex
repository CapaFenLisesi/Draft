\subsection{Super conductor}
Mean-field approach to deal with a four operator diagonalization.

Suppose we have: $D^*C^* CD$, then let $\delta = CD - \braket{CD} =
CD - avg$. Then if we assume $\braket{CD}\neq 0$, and $\delta \approx 0$. Then we have:

   	$$ \delta^2 \approx 0 $$
i.e.:

\begin{align}
   	( (CD)^* - avg ) ( CD - avg ) = 0\\
   	D^*C^*CD = avg*(CD+D^*C^*) - avg^2
\end{align}
Hence a four operator is reduced into a few of two operators.
Such method could be naturally extended to treat the operator
$\sum_{i,j} D^*_i C^*_i C_j D_j$.

A copper pair has the energy of:
$$\Delta = \braket{C_{k \uparrow}C_{-k \downarrow} }$$

To resist the flow of current carried by Copper Pair, is equivalent to destroying a pair of Copper Pair:
$$ \braket{C_{k \uparrow}C_{-k \downarrow} } \longrightarrow C_{k \uparrow}C_{-k \downarrow}$$

This will require an additional enegy of $2\Delta$.

The exact meaning of "equivalent to" is as follows:
\begin{align}
    & \text{break a copper pair} \longrightarrow 
    \text{scatter two electrons consecutively} 
    \nonumber\\ & \longrightarrow 
    \text{create two electron-hole mixed type quasi-particle} \longrightarrow 2\Delta \nonumber
\end{align}



\subsection{Preface of BSCS}
    \label{sec:Preface_of_Bosonization_and_Strongly 
    Correlated_Systems}
BSCS: see \cite{BSCS}.
Parallism between theories in condensed matter physics and those in
particle physics.
\begin{itemize}
        \item Anderson-Higgs Phenomenon (Paritcle), Meissner effect
                (C.M.P.)
        \item 'inflation' in Cosmology, first order phase transition
        \item 'cosmic strings', magnetic field vortex lines in type
                II superconductors
        \item Hadron-meson interaction, Ginzburg-Landau theory of
                superfluid $He^3$.
\end{itemize}
Same ideas on different space-time scales, different hierachical
'layers'.
Strong parallism: \textbf{strongly correlated low dimensional system}

E.g.:

The problem of formation and structure of heavy particles - hadrons and mesons. The corresponding fine structure constant $\alpha_G\approx 1$.

Approaches:
\begin{enumerate}
        \item Exact solutions
        \item Reformulate complicated interacting models in such a way
                that they become weekly interacting. -> Bosonization.
                
                Spin $1/2$ anisotropic Hisenberg chain $\approx$
                Model of interacting

                fermions.
                (Jordan and Wigner, 1928)
\end{enumerate}
Bosonization: transformation from fermions to a scalar massless bosonic
field.

\subsection{Appearance of Gauge Structure in Simple Dynamical Systems}

\begin{align}
    0=(\eta_b,\dot{\eta_a}) = (\eta_b,\dot{U}_{ac}\psi) +
        (\eta_b,U_{ac}\dot{\psi}_c
\end{align}
\subsection{Quantum Statistical Mechanics}
\label{sec:Quantum Statistical Mechanics}
\begin{defi}[Time Evolution Operator]
    The time evolution operator $U(t,t_0)$ is defined such that
    \begin{align}
        \label{eq:def_time_evo_op}
        \ket{\Psi(t)} = U(t,t_0) \ket{\Psi(t_0)}
    \end{align}
\end{defi}
It satisfy the relationship:
\begin{align}
    \label{eq:def_time_evo_op 2}
    i\hbar \partial_t U(t,t_0)= H U(t,t_0)
\end{align}
This is obvious when substituting $U(t,t_0)$ into the Schrodinger 
Equations.

\paragraph{Quantum Macrostates}
Macrostates of the system depend on only a few the thermodynamic
functions. We can form an ensemble of a large number $\mathcal{N}$ of
microstates $\{\psi_\alpha\}$, corresponding too a given macrostates.
The different microstates occur with probability $p_\alpha$.
When wen no longer have exact knowledge of the microstate of a
system the system is said to be in a \textit{mixed state}.
The ensemble average of the quantum mechanical expectation
value is given by:
\begin{align}
    \label{eq:quantum_macrostates_ensemble_avg}
    \bar{\braket{O}} &= 
        \sum_\alpha p_\alpha \braket{\psi_\alpha|O|\psi_\alpha}
        = \sum_{\alpha,m,n} p_\alpha
            \braket{\psi_\alpha|m}\braket{m|O|n}\braket{n|\psi_\alpha}
            \nonumber\\
        &= \sum_{m,n} \braket{n|\rho|m}\braket{m|O|n}
            = \text{tr}(\rho O)
\end{align}
where we have introduced the density matrix:
\begin{defi}[Density Matrix]
    The density matrix $\rho(t)$ is defined as
    \begin{align}
        \label{eq:density_matrix_def}
    \braket{n|\rho(t)|m} \equiv
    \sum_\alpha p_\alpha \braket{n|\psi_\alpha} \braket{\psi_\alpha|m}
    \end{align}
    or
    \begin{align}
        \label{eq:density_matrix_def_2}
        \rho(t) \equiv \sum_\alpha p_\alpha
            \ket{\psi_\alpha}\bra{\psi_\alpha}
    \end{align}
\end{defi}
Density matrix is denoted by $\rho(t)$ by analogy of the notation for
P.D.F, since $\rho$ often represents density.

Density matrix satisfies several good properties:
\begin{itemize}
    \item Normalized
    \item Hermiticity
    \item Positivity. For any $\Phi$, $\braket{\Phi|\rho|\Phi} \geq 0$.
\end{itemize}
The time evolution of density matrix, directly obtained from Schrodinger's
equation, is
\begin{align}
    \label{eq:quantum_macrostates:density_matrix:evolution}
    i\hbar \frac{\partial}{\partial t}\rho = [H,\rho]
\end{align}
\subsection{Special techniques in Electrostatistics}
\label{sec:Special-techniques-in-Electrostatistics}

For the book \cite{griffiths-EM}, pp. 120 of section 3.1.6. 



\subsection{Unitarity of Representation}
\label{sec:Unitarity-of-Representation}

\begin{thm}[Unitary Representation]
    For finite groups and for compact Lie groups, all representations
    are equivalent to a unitary representation.
\end{thm}
\begin{remark}
    Before the formal proof, I remarked that since any finite group
    can be embedded inside the symmetric group (Cayley's theorem), and
    the symmetric has clearly a unitary representation (perhaps the
    easiest representation one could ever construct besides the
    trivial representation), one could naturally guess whether it is
    possible that any representation could be turned into a unitary
    one.

    As for the compact Lie group case, it is just mentioned in
    \cite{Ludeling} and will not be proved here. It is mentioned in
    page 25 of \cite{Ludeling} that:
    \begin{quote}
        For compact Lie groups, however, there exists a (unique)
        translationally invariant measure, so we can replace
        $\sum_g\to \int \dd g$ and the integral is convergent because
        of compactness. Then the proof directly carries over.
    \end{quote}
\end{remark}
\begin{proof}
    For $D(g)$, we need to find a $X$ such that $\bar{D}(g)\equiv
    X^{-1}D(g)X$ is unitary.

    Since
    \begin{align*}
        1 = \bar{D}^\dagger \bar{D}
    \end{align*}
    One will find 
    $$ (XX^\dagger)^{-1} = D^\dagger (XX^\dagger)^{-1} D$$
    Then let
    \begin{equation}
        H\equiv \sum_{s\in G} D^\dagger(s) D(s)
    \end{equation}
    One can verify that
    \begin{equation}
        D^\dagger (g) H D(g) = H
    \end{equation}

    Now we construct $X$ from $H$. We have
    \begin{equation}
        H = (XX^\dagger)^{-1}
    \end{equation}

    Notice we have $H$ is Hermitian by above equation. Also, $H$ is
    positive definite (easily seen from the definition of $H$ and 
    $a^\dagger H a \geq 0$. Also $H$ is of full rank.).

    Then we have $UHU^{-1} = \mathrm{diag}\{\gamma_1,\gamma_2,\cdots\}$
    and $\gamma_i > 0$. The rest for constructing $X$ should be
    obvious.
\end{proof}
\begin{remark}[Examples of Non-unitary representation]
    It is remard in \cite{Ludeling} that for infinite and non-compact
    groups, the representation are not unitary in general. He gives
    two examples:
    \begin{itemize}
        \item The group $\mathbb{Z}^*$ acting on $\mathbb{C}$ by
            mulplication is certain not equivalent to any unitary
            representation.
        \item Arugment for a representation of non-compact Lie group
            cannot be made unitary by a similarity transformation:
            \begin{quote}
                The group of unitary operators on a finite dimensional
                vector space is isomorphic to $U(n)$ (for complex
                vector spaces, $O(n)$ otherwise), and is hence
                compact. But there cannot be a bijective continuous
                map between a compact and a non-compact space, so
                faithful finite-dimensional representations of
                non-compact groups will be non-unitary. 
            \end{quote}
            But then he mentions something about a perculiar case, the
            Lorentz group, which I do not understand. 
    \end{itemize}
\end{remark}
Anyway, he summarised in the following:
\begin{quote}
   To summarise, for finite groups and for compact Lie groups, all
   representations are equivalent to a unitary one (and we will
   usually take them to be unitary from now on).  For infinite groups
   and non-compact Lie groups, on the other hand, finite-dimensional
   faithful representations are never unitary. Finally, some
   non-compact groups may have representations which are unitary with
   respect to a non-definite scalar product, such as the Lorentz group 
\end{quote}

From class, Ye Fei proved that:
\begin{thm}
    For any two equivalent representation, there is always a unitary
    matrix to relate the two, i.e. exits $Y$ unitary, s.t.
    $$ \bar{D}(g) = Y^{-1} D(g) Y$$
\end{thm}
\begin{proof}
    Suppose we have two unitary representation $D(g)$ and
    $\bar{D}(g)$, related by
    $$ \bar{D}(g) = X^{-1} D(g) X$$
    where $X$ is not necessarily unitary.

    Let $H\equiv X^\dagger X$, it is easy to prove that $H$ is
    Hermitian and positive definite. Direct calculation shows that
    $$ \bar{D}^{-1}(g) H \bar{D}(g) = H$$
    That is $\bar{D}(g)$ and $H$ commute.
    Then we construct $Y$ from $H$. Let $V$ be such that
    \begin{equation}
        V^{-1} H V = \Gamma
    \end{equation}
    where $\Gamma$ is a diagonal matrix of $H$'s eigenvalues.
    Obviously $V$ has to be unitary. Define $\bar{Y}$ to be
    \begin{equation}
        \bar{Y}\equiv V \sqrt{\Gamma} V^{-1}
    \end{equation}
    By direct calculation, we have
    $$ (X\bar{Y})^\dagger (X\bar{Y}) = \id$$
    One can show that
    $$ \bar{Y}^{-1} \bar{D} \bar{Y} = \bar{D}$$
    with laborious calculation. Then one can easily verify that
    $Y\equiv \bar{Y}X$ is the required unitary transformation.
\end{proof}

\subsection{Reduciblility of Representations}
\label{sec:Reduciblility-of-Representations}

\begin{defi}[Reducible/Irreducible Representation]
    \nomenclature{Reducibility of Representation}{\nomrefpage.}
    A representation $D$ is called \textit{reducible} if $V$ contains an
    invariant subspace. Otherwise $D$ is called irreducible.

    A representation is called \textit{fully reducible} if $V$ can be
    written as the direct sum of irreducible invariant subspaces, i.e.
    $V=V_1\oplus \cdots \oplus V_p$, all the $V_i$ are invariant and the
    restriction of $D$ to each $V_i$ is irreducible.
\end{defi}

\begin{ex}
    The regular representation mentioned before is reducible. For
    example, the vector
    \begin{equation}
        V = \sum_{g\in G} g
    \end{equation}
    spans an invariant subspace of the operator $D_\text{reg}$.
\end{ex}


\begin{ex}$ $

    \begin{itemize}
        \item The representation of finite group is obviously fully
            reducible or irreducible (since a unitary matrix cannot
            have off diagonal blocks).
        \item The representation of translation group is not fully
            reducible. For example, for translation we have:
            $T_a T_b = T_{a+b}$,
            One can confirm that the following representation obeys
            the above relationship:
            $$T_a = \left( \begin{array}{cc}
                 1 & a \\
                 0 & 1 \\
            \end{array} \right)$$
            but this is obviously not fully reducible.
    \end{itemize}
\end{ex}

\begin{defi}[Interwiner]
\nomenclature{Interwiner}{\nomrefpage.}
    Given two representations $D_1$ and $D_2$ acting on $V_1$ and
    $V_2$, an intertwiner between $D_1$ and $D_2$ is a linear operator
    \begin{equation}
        F: V_1 \mapsto V_2
    \end{equation}
    which "commutes" with $G$ in the sense that
    \begin{equation}
        F D_1(g) = D_2(g) F
    \end{equation}
    for all $g\in G$.
\end{defi}

(From pp.30 of \cite{Ludeling})

The existence of an intertwiner has a number of consequences. First,
$D_1$ and $D_2$ are equivalent exactly if there exists an invertible
intertwiner. Second, \hl{the kernel and the image of $F$ are invariant
subspaces}: Assume $v\in \Ker{F}$, i.e. $Fv = 0$. Then
\begin{equation}
    FD_1 v = D_2 F v = D_2 0 = 0
\end{equation}
so $D_1 v\in \Ker{F}$. On the other hand, let $w_2 = Fw_1$ be an
arbitrary element of the image of $F$. Then from the definition we have
\begin{equation}
    D_2 w_2 = D_2 F w_1 = FD_1 w_1
\end{equation}
which is again in the image of $F$. Now if $D_1$ is irreducible, the only invariant subspaces,
hence the only possible kernels, are $\{0\}$ and $V_1$ itself, so $F$ is either injective or zero.
Similarly, if $D_2$ is irreducible, F is either surjective or zero. Taking these statements together, we arrive at Schur’s Lemma: 

\begin{lemma}[Schur's lemma I]
    An intertwiner between two irreducible representations is either
    an isomorphism, in which case the representations are equivalent,
    or zero
\end{lemma}

An important special case is the one where $D_1 = D_2$. In that case,
we see that $F$ is essentially unique. More precisely, we have the
following theorem, also often called Schur’s Lemma:
\begin{lemma}[Schur's lemma II]
    If $D$ is an irreducible finite-dimensional representation on a complex
vector space and there is an endomorphism $F$ of $V$ which satisfies 
\begin{equation}
    FD(g) = D(g) F
\end{equation}
for all $g\in G$, then F is a multiple of the identity, $F = \lambda
\id$
\end{lemma}
\begin{proof}
    Note that $F$ has at least one eigenvector $v$ with eigenvalue
    $\lambda$. (This is where we need $V$ to be a complex vector
    space: A real matrix might have complex eigenvalues, and hence no
    real eigenvectors.) Clearly, $F-\lambda\id$ is also an
    intertwiner, and it is not an isomorphism since it annihilates
    $v$. Hence, by Schur’s Lemma, it vanishes, thus $F = \lambda\id$.
\end{proof}

(From the book \cite{book})
\subsection{Orthogonality Relations and Counting Irreducible
    Representations}
\label{sec:Orthogonality-Relations-and-Counting-Irreducible-Representations}

\begin{thm}[Orthogonality Theorem for Representations]
    For finite group $G$, let $D^i(G)$ and $D^j(G)$ be its two
    irreducible representation. Then, as a vector in group algebra,
    they have the following orthogonal
    relationship:
    \begin{equation}
        \sum_{h\in G} D^{i}_{\mu\rho}(h^{-1})D^j_{\nu \lambda}(h) =
        \frac{N}{d_j} \delta_{ij} \delta_{\mu\nu}\delta_{\rho\lambda}
    \end{equation}
    $N$ is the order of the group, and $d_j$ is the
    dimension of representation $D^j(G)$. If in addition, the two
    representations are unitary, we have
    \begin{equation}
        \sum_{h\in G} D^{i*}_{\mu\rho}(h)D^j_{\nu \lambda}(h) =
        \frac{N}{d_j} \delta_{ij} \delta_{\mu\nu}\delta_{\rho\lambda}
    \end{equation}
\end{thm}
\begin{proof}
    \textbf{Note}: The following proof is clumsy and only applies for
    the unitary case. For a good proof, please refer to page 44 of
    \cite{Ludeling}.

    Let
    \begin{equation}
        Y_{\rho\lambda}^{\mu\nu} \equiv
        \delta_{\rho\lambda}\delta_{\mu\nu}
    \end{equation}
    Then let
    \begin{align*}
        X^{\mu\nu} = \sum_{h\in G} D^i(h^{-1}) Y^{\mu\nu} D^j(h)
    \end{align*}
    One can find by direct calculation
    \begin{align}
        X^{\mu\nu}_{\rho\lambda}= \sum_{h\in G}
        D^{i*}_{\mu\rho}(h)D^j_{\nu \lambda}(h)
    \end{align}
    And also through direct calculation, one finds
    \begin{align*}
        D^i (s) X^{\mu\nu} = X^{\mu\nu} D^j(s)
    \end{align*}
    for any $s\in G$. 
    Then $X^{\mu\nu}$ is a interwiner. So the case for $i\neq j$ is
    obvious. When $i=j$, we have:
    $$ X = \lambda \id$$
    Now we find the $\lambda$, i.e. the eigenvalue of $X^{\mu\nu}$.
    
    Now since
    \begin{align*}
        X^{\mu\nu}_{\rho\lambda} = \lambda^{\mu\nu}
        \delta_{\rho\lambda}
    \end{align*}
    One can find two fact by direct calculation:
    \begin{align*}
        &\sum_\rho X^{\mu\nu}_{\rho\lambda} = d_j \lambda^{\mu\nu} \\
        &\sum_\rho X^{\mu\nu}_{\rho\lambda} = N \delta^{\mu\nu}
    \end{align*}
    Hence $\lambda^{\mu\nu} = \frac{N}{d_j} \delta^{\mu\nu}$.
\end{proof}
The above relation gives us the first clue to the total number of
irreducible representations by the following corollary. 
\begin{coro}
    \begin{equation}
        \sum_{j} d_j^2 = N = \abs{G}
    \end{equation}
    where the index $j$ runs over all possible irreducible
    representations.
\end{coro}
\begin{proof}
    Define the following vector in group algebra
    \begin{equation}
        v^{i}_{\mu\nu} = \sqrt{\frac{d_j}{N}} \sum_{h\in G}
        \left(D^i(h)\right)_{\mu\nu} h
    \end{equation}
    Then by the orthogonality theorem, this vector is orthonomal in
    vector space/group algebra $\mathbb{C}[G]$. But the dimension of
    this vector space is $N$, and for each representation $D^i$ we
    have $d_i^2$ vectors like the one above, which are all orthogonal
    to each other. Note that orthogonal implies linear independence,
    hence we have:
        $$\sum_{j} d_j^2 \leq N $$
    To make the inequality an equality, it is suffice to prove that
    such $v^{i}_{\mu\nu}$ forms a basis of the group algebra
    $\mathbb{C}[G]$. In page 45 to 46 of \cite{Ludeling}, he shows
    that for any $g\in G\hookrightarrow \mathbb{C}[G]$, one has
    \begin{align}
        g = \sum_{h\in G} \left(D_\text{reg}(h)\right)_{ge} h
        = \sum_{h\in G} \left[ \sum_{j,\mu\nu} c^j_{\mu\nu}
        \left(D^j(h)\right)_{\mu\nu}\right] h
    \end{align}
    where $e$ is the unit in the group. $c$ are some hard to tell
    constants. To show this, \cite{Ludeling} uses that fact that
    $D_\text{reg}$ is unitary, so it is completely reducible into
    irreducible components, which have to be (some of) the $D^i$,
    i.e. $D_\text{reg}= U(D^{i_1}\oplus D^{i_2}\oplus
    \cdots)U^\dagger$. The detailed proof is not reproduced here.

    Then, $g$ is a linear combination with
    coefficients in $\left(D^j(h)\right)_{\mu\nu}$, hence a linear
    combination of $v^i_{\mu\nu}$. Since $g$ can be any basis in the
    group algebra, this shows that the $\{v^i_{\mu\nu}\}$ is complete.
\end{proof}
\begin{remark}
    The above proof shows in some sense that the regular
    representation contains all irreducible representation as
    components. Since all $v^i_{\mu\nu}$ appears in the regular
    representation.
\end{remark}

\subsection{Characters}
\label{sec:Characters}
\begin{defi}[Character of Representation]
\nomenclature{Character of Representation}{\nomrefpage.}
    Given a representation $D$ of $G$ over vector space on field $K$,
    the character $\chi: G\to K$ is
    defined as the trace
    \begin{equation}
        \chi(g) = \Tr D(g)
    \end{equation}
\end{defi}
\begin{remark}
    Since trace is invariant under similarity transformation (remember
    that $\Tr AB = \Tr BA$), trace can be used to distinguishes
    between non-equivalent representations. Hence it is called the
    character of that representation.
\end{remark}
Using the orthogonality relation in
Section~\ref{sec:Orthogonality-Relations-and-Counting-Irreducible-Representations}
it is esay to see we have
\begin{coro}
    \begin{equation}
        \sum_{h\in G} \chi^{i}(h^{-1})\chi^j(h) = N \delta_{ij}
    \end{equation}
    Or when the representations are unitary:
    \begin{equation}
        \sum_{h\in G} \chi^{i*}(h)\chi^j(h) = N \delta_{ij}
    \end{equation}
    Here again $N = \abs{G}$.
\end{coro}

Let the conjugacy classes be labeled by $K_a$, $a=1,\cdots,k$. $k$ is
the number of conjugacy classes. Let $n_a$ be the number of group
elements in each conjugacy class $K_a$. Consider one thing:
\begin{itemize}
    \item Characters is invariant within each conjugacy classes
\end{itemize}
One can derive (pp. 47 of \cite{Ludeling})
\begin{equation}
    \sum_a n_a \chi^{i,a}(\chi^{j,a})^* = N\delta^{ij}
    \label{eq:20161031-nxx-is-Ndelte}
\end{equation}
here $\chi^{i,a}$ menas the character in the $i$'s irreducible
representation of the group element in $a$'s conjugacy class.
Then the vectors (labeled by each irreducible representation $i$)
\begin{equation}
    \frac{1}{\sqrt{N}}
    (\sqrt{n_1}\chi^{i,1},\cdots,\sqrt{n_k}\chi^{i,k})
\end{equation}
are orthonomal. Notice that given a $k$-dimensional space, we in
effect show cases $r$ linear independence vectors. Therefore we must
have 
\begin{equation}
    r \leq k
\end{equation}

    \subsubsection{Finding Components of Representations}
    \label{sec:Finding-Components-of-Representations}
    \begin{remark}
        It's easy to see that if $D=aD^1\oplus bD^2$, one has $\chi =
        a\chi^1 + b\chi^2$.
    \end{remark}
    Now we consider the coefficients in the general case. Suppose a
    representation $D$ has:
    \begin{equation}
        D = \oplus a^i D^i
    \end{equation}
    where the sum is taken over all irreducible representations
    $D^i$. We have:
    \begin{equation}
        X = \sum_{i} a^i \chi^i
    \end{equation}
    It is calculated (page 47 to 48 of \cite{Ludeling}) using equation
    \ref{eq:20161031-nxx-is-Ndelte},
    that
    \begin{equation}
        \sum_a n_a \chi^{i,a*}X^a = N a^i
    \end{equation}
    so
    \begin{equation}
        a^i = \frac{1}{N} \sum_a n_a(\chi^{i,a})^* X^a
        \label{eq:20161031-components-formula}
    \end{equation}
    From now on, we always denote the class of identity by $K_1$, and
    the corresponding character in that representation by
    $\chi^{i,1}$. Clearly $\chi^{i,1}\equiv d_i$.
    \begin{ex}
        If $D$ is the regular representation. It is easy to see that
        \begin{equation}
            X_\text{reg}(g) = N \delta_{g,e}
        \end{equation}
        where $e$ is the unit element in the group.
        Then
        \begin{equation}
            a^i = \frac{1}{N}\sum_a n_a (\chi^{i,a})^* X^a_\text{reg}
            = \frac{1}{N} (\chi^{i,1})^* X^1_\text{reg} = 
            \frac{1}{N} d_i N = d_i
        \end{equation}
    \end{ex}
    \begin{remark}[Criterion for irreduciblility]
        The above result gives us one way to assess the level of
        irreduciblility of a representation. Direct calculation shows
        (with the help of equation
        \ref{eq:20161031-components-formula})
        \begin{equation}
            \sum_a n_a (X^a)^* X^a = N \sum_i (a^i)^2
        \end{equation}
        So for a irreducible representation $X$, we have $\sum_a n_a
        (X^a)^* X^a= N$. For other representations, we have $\sum_a
        n_a (X^a)^* X^a= N/2N/3N/\cdots$. In this way one somehow
        measures qualitatively how complex this representation is.
    \end{remark}

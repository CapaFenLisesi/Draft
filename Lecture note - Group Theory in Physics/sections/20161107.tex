\subsection{Finding Representation}
\label{sec:Finding-Representation}
% TODO Polish the arragement of this section.
One important work in group theory is to find all its irreducible
representations. The book \cite{book} suggests we first find the
characters of irreducible representations, and list them in a
character table, and then find the matrix of those irreducible
representations.

He also notes that, a unfaithful representation of a group is
equivalent to a faithful representation of a certain quotient group.
Since quotient group is less complex than the original group, finding
irreducible representation of normal subgroups is very useful for
analysing a group.

\begin{fact}
    Also, a the direct product of two irreducible representations is also
    a representation. If the dimension of one of the representation is
    $1$, then their product is still a irreducible representation.
\end{fact}
\begin{fact}
    \label{fact:character-of-identity}
    Another helpful point is that, for all irreducible
    representations, the identity element is always mapped to the
    identity matrix. So the character for this unit class is just the
    dimension of the representation.
\end{fact}
\subsection{Self-conjugate representation and Real Representation}
\label{sec:Self-conjugate-representation-and-Real-Representation}

It's easy to see that for any representation $D$, conjugating ever
matrix $D(g)^*$ will give another representation. We define
\begin{defi}[Conjugate Representation]
\nomenclature{Conjugate Representation}{\nomrefpage.}
    The conjugate representation of a given representation $D$ is obtained
    by complex conjugate all its matrices $D(g)^*,\,g\in G$.
\end{defi}
\begin{defi}[Self-conjugate Representation]
\nomenclature{Self-conjugate Representation}{\nomrefpage.}
    A representation $D$ is called self-conjugate if and only if it is
    equivalent (but not necessarily equal) to its conjugate
    representation $D^*$.
\end{defi}
Note that the character of a self-conjugate representation is a real
function. Since (assume $XD(g)X^{-1}=D^*(g)$):
$$
    \Tr(XD(g)X^{-1}) = \Tr(X^{-1}XD(g)) = \Tr(D(g)) = \Tr(D^*(g))
    = \Tr(D(g))^*
$$
In fact, we have
\begin{thm}
    A represents is self-conjugate if and only if its character is
    real-valued.
\end{thm}
\begin{proof}
    If it is self-conjugate, clearly it is real-valued. On the other
    hand, if the character is real-valued, then by
    \begin{equation}
        \chi_{D^*(g)} = \Tr(D^*(g)) = \Tr(D(g))^* = \chi^*_{D(g)}
        = \chi_{D(g)}
    \end{equation}
    The conjugate representation has the same character, hence is
    equivalent to original representation by theorem \ref{thm:20161031-character-determine-rep}
\end{proof}

Fun fact. Since all representation of finite group are equivalent to a
unitary one, if for en element $g\in G$ its character $\chi(g)$ is
complex, then its inverse $g^{-1}$ has character $\chi(g)^*$.
Therefore, for a representation which is not self-conjugate, its
character will has complex value existing in pairs, unless $g=g^{-1}$.

We have a similar concept of real representation:
\begin{defi}[Real representation]
\nomenclature{Real representation}{\nomrefpage.}
    A representation $D$ is called real if and only if it is
    equivalent to a representation $D^\text{real}$, such that for any
    $g\in G$, the matrix $D^\text{real}(g)$ is real.
\end{defi}

Observe that a real representation is obviously self-conjugate, but a
self-conjugate representation is not necessarily real. We have two
indicator for its exact classifications.

Firstly,
\begin{thm}
    Let $D$ be a unitary irreducible representation of a finite group.
    Let $D^*$ be its conjugate representation. Assume that $D$ is
    self-conjugate. Let $X$ be the matrix that relates the two. Then
    $X$ is either symmetric or anti-symmetric. $X$ is symmetric if and
    only if $D$ is real. $X$ is anti-symmetric if and only if $D$ is
    self-conjugate but is not real.
\end{thm}

Secondly, 
\begin{thm}
    for any irreducible representation $D$ of finite group:
    \begin{align}
        \frac{1}{|G|}\sum_{g\in G} \chi(g^2) = \begin{cases}
            1, & \text{$D$ is real} \\
            -1, & \text{$D$ is self-conjugate but is not real} \\
            0, & \text{$D$ is not self-conjugate}
        \end{cases}
    \end{align}
\end{thm}

The proof of these two facts are presented in page 68 of \cite{book}.
Note that there is more general version of the second fact, called
\textit{Frobenius–Schur indicator}, see
\href{https://en.wikipedia.org/wiki/Frobenius%E2%80%93Schur_indicator}
{Wikipedia page}.

\subsection{Subduced Representation}
\label{sec:Subduced-Representation}
\marginpar{To be finished.} %TODO
Let $G$ be a group. Let $\mathcal{C}_a$ denotes it conjugacy classes.
Let $D^j$ be its irreducible representations, with dimension $m_j$,
character $\chi^j_a$. Assume it has a subgroup $H=\{T_1=E,
T_2,\cdots,T_h\}$, where $h=|H|$. Let $n=|G|/h$ be its index. Denotes
its coset by $R_rH$. With $R_1:=E$, any element in $G$ can be denotes
as $R_r T_t$ (assume $R_r$ is chosen). Let $\bar{\mathcal{C}}_\beta$
denotes conjugacy classes of the subgroup $H$, with character
$\bar{\chi}^k_\beta$, dimension $\bar{m}_k$.

% \begin{thm}[Frobenius Theorem]
%     % TODO
% \end{thm}
% \begin{proof}
%     Assume that $D^j(T_t)$ can be decomposed into:
%     \begin{align}
%         X^{-1}D^j(T_t)X &= \oplus_k a_{jk}\bar{D}^k(T_t) \\
%         m_j = \sum_k a_{jk} \bar{m}_k
%     \end{align}
%     It is easy to prove that
%     \begin{equation}
%         a_{jk} = \frac{1}{|H|} \sum_{T_t\in H} \bar{\chi}^k(T_t)^*
%         \chi^j(Tt) 
%         = \frac{1}{|H|} \sum_\beta \bar{n}_\beta
%         (\bar{\chi}^k_\beta)^* \chi^j_\beta
%     \end{equation}
%     (Notice that, since character is invariant in conjugate class, we
%     can also label the conjugacy class of $\chi^j_\alpha$ with
%     $\beta$, where $\beta$ labels the conjugacy of subgroup.)
% \end{proof}




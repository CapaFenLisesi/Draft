
Now we consider the classification of wavefunctions using the
symmetric groups we have known. In general, we need a symmetry, its
symmetry group, and the (irreducible) representation matrices.
During the classification, the representation space is the space
spanned by the set of degenerate wave functions $\psi_\rho$
($\rho=1,\cdots,m$), $\mathcal{L}_E=\operatorname{span}\{\psi_\rho\}$.

assuming we have
\begin{equation}
    H\psi_\rho(x) = E\psi_\rho(x)
\end{equation}

The $\psi_\rho$ satisfies orthogonal relations.
Now we do a spatial transformation, and assume this transformation
satisfy the symmetry of the Hamiltonian ($HP_R = P_R H$). Thn we have:
\begin{equation}
    P_R(\psi_\rho(x)) = \psi_\rho(R^{-1} x) = \sum_\lambda \psi_\lambda
    D_{\lambda\rho}(x)
\end{equation}
In this way, we obtain a representations of our symmetry. Usually,
$P_R$ is unitary, and since our basis $\psi_\rho$ is orthonormal, the
matrix $D$ is unitary. Now we want to find those irreducible
representations, so we do:
\begin{align}
    X^{-1} D(R) X = \oplus_j a_j D^j(R)
\end{align}
where $R\in G$, $D^j$ labels those irreducible representations, $X$ is
some matrix that we do not know now.

$a_j$ can be calculated using previous orthogonal relationships. More
specifically, we have
\begin{equation}
    a_j = \frac{1}{|G|} \sum_{R\in G} (\chi^j(R))^* \chi(R)
\end{equation}

Now we are interested in geting $X$. Notice that
\begin{align*}
    X^{-1} D(R) X = \oplus_j a_j D^j(R)
\end{align*}
tells us something about $X$. Let $X=X_{a,b}$. Here $1\leq a \leq m$.
$b$ can be labeled by $j,\mu,\gamma$, i.e. the irreducible
representation it is inside ($j$), did this irreducible representation
repeats more than one time ($\gamma$), which element is it ($\mu$). So
we have the new basis $\Psi$ related to the odd basis $\psi$ as
\begin{align}
    P_R(\Psi^j_{\mu\gamma}(x)) &= \sum_\nu
    \Psi^j_{\nu\gamma}(x)D^j_{\nu\mu}(R) \\
    \Psi^j_{\mu\gamma} &= \sum_{\rho} \psi_\rho X_{\rho,j\mu\gamma} \\
    \psi_\rho &= \sum_{j\mu\gamma} \psi^{j}_{\mu\gamma}
    (X^{-1})_{j\mu\gamma,\rho}
\end{align}

we have the theorem
\begin{thm}[W-E]
    \begin{itemize}
        \item Those belong to different irreducible representations
            are orthogonal
        \item Those belong to the same irreducible representation, but
            in different row are orthogonal
        \item Those belong to the same irreducible representation, the
            same row, has inner product independent of the index
    \end{itemize}
\end{thm}

Let $\psi^j_\mu$ and $\phi^k_\rho$ belong to $D^j_\mu$ and $D^k_\rho$.
Then 
$P_R \psi^j_\mu= \sum_\nu \psi^j_\nu D^j_{\nu\mu}(R)$,
$P_R \phi^k_\rho= \sum_\lambda \psi^k_\lambda D^j_{\lambda\nu}(R)$

Define $X^{jk}_{\rho\mu} := \braket{\phi^k_\rho|\psi^j_\mu}$.

\begin{align}
    \braket{\phi^k_\rho|P_R\psi^j_\mu} = \sum_\nu
    \braket{\phi^k_\rho|\psi^j_\nu} D^j_{\nu\mu}(R) = (X^{jk}
    D^j(R))_{\rho\mu} \\
    \braket{P^{-1}\phi^k_\rho|\psi^j_\mu} = \braket{
    \sum_\lambda\phi^k_\lambda D^k_{\lambda\rho}(R^{-})|\psi^j_\nu}
    \\
    = \sum_\lambda [D^k_{\lambda\rho}(R^{-1})]^*
    \braket{\phi^k_\lambda|\psi^j_\mu} 
    = \sum_\lambda D^k_{\rho\lambda}
    \braket{\phi^k_\lambda|\psi^j_\mu}
    = \sum_\lambda X^{kj}_{j\mu} = (D^k(R)X^{kj})_{\rho\mu}
\end{align}

Comparing two sides, we have
\begin{equation}
    D^k(R) X^{kj} = X^{kj}D^j(R)
\end{equation}
So by Schur lemma, we have when $k\neq j$
\begin{equation}
    X^{kj} = 0
\end{equation}
when $k=j$
\begin{equation}
    X^{jk} = \lambda\id
\end{equation}
the constant $\lambda$ is denoted as $\bra{k}\ket{j}$
so
\begin{equation}
    X^{jk} = \bra{k} \ket{j} \id
\end{equation}
In all
\begin{equation}
    \braket{\phi^k_\rho|\psi^j_\mu} = \delta_{jk} \delta_{\rho\mu}
    \bra{k} \ket{j}
\end{equation}

\subsection{Canonical Degeneracy}
\label{sec:Canonical-Degeneracy}
\begin{defi}[Canonical/Accidential Degeneracy]
\nomenclature{Canonical/Accidential Degeneracy}{\nomrefpage.}
    Assume $H$ has $M$-fold degeneracy, wit corresponding
    representations $D(G)$. If the representation is irreducible, then
    the degeneracy is called canonical, otherwise it is called
    accidential.
\end{defi}

Assume we have $[P_R,H_0]=0$, $[P_R,H_1]=0$. 
\begin{align}
    P_R \psi^j_\mu = \sum_\nu \psi^j_\nu D^j_{\nu\mu}(R)\\
    H_0 \psi^j_\mu = E^j \psi^j_\mu \\
    P_R H_1 \psi^j_\mu = H_1 P_R \psi^j_\mu = H_1 \sum_\nu \psi^j_\nu
    D^j_{\nu\mu}(R) = \sum_\nu (H_1\psi^j_\nu) D^j_{\nu\mu}(R) \\
\end{align}
By theorem
\begin{align}
    \braket{\psi^j_\nu|H_1 \psi^j_\nu} = \bra{j}\ket{j}\id = \Delta
    E^j \delta_{\nu\mu}
\end{align}
So the first order correction does not remove the degeneracy. For the
higher order correction, the case will be the same. Here only an
argument is given. We consider the process $H_0 + \lambda H_1$, from
$\lambda=0$ to $\lambda=1$, in this process, if $\lambda$ varies
adiabatically, then the degenerate function will keep their degeneracy
in the gradually process if the Hamiltonian is good enough. 

For \textbf{accidental degeneracy}, for $j\neq i$
\begin{align}
    \braket{\psi^i_\nu|H_1 \psi^i_\mu} = \delta_{\nu\mu} \Delta E^i \\
    \braket{\psi^j_\nu|H_1 \psi^j_\mu} = \delta_{\nu\mu} \Delta E^j \\
\end{align}
In general, $\Delta E^i$ and $\Delta E^j$ is not the same. If they are
the same, then there may be additional symmetry that we have not yet
discovered.

For $j=i$, 
\begin{equation}
P_R \psi^j_{\nu a} = \sum_\nu \psi^j_{\nu a} D^j_{\nu\mu}(R)
\end{equation}
\begin{align}
    \braket{\psi^j_{\mu a} | H_1 |\psi^j_{\nu b}}  = 
\end{align}

Example

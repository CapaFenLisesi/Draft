
\subsection{Common Concepts in Group}
\label{sec:Common-Concepts-in-Group}

Introducing to various groups:$S_4, V_4$, $D_3$, all omited.
(\textbf{pp.22-23 of \cite{book}})

$D_n$ group. See pp. 25-26 of the book \cite{book}.
Note that here the $n$ refers to the $n$-polygon, not that the group is
of order $n$. For the mathematicians, they might be comfortable with
$D_n$ means the dihedral group of order $n$, but is actually the group of
symmetries of $n/2$-polygon.

\begin{defi}[Subgroup]
\nomenclature{Subgroup}{\nomrefpage.}
    Omitted.
\end{defi}
\begin{fact}
    One only has to check the closeness for determining a subgroup, if it
    is of finite order.

    However, for group of infinite order, one has to check the existance
    of unit and inverse elements.
\end{fact}
Examples of subgroup (\textbf{pp.26 of \cite{book}})

Noteworth:$C_6$ has three copies of $D_2$, this can be intuitively 
guessed by the fact that a hexago has three rectanle in it.

\begin{defi}[Coset]
\nomenclature{Coset}{\nomrefpage.}
    Omitted.
\end{defi}
Properties of coset (omitted).

\begin{defi}[Index of subgroup]
\nomenclature{Index of subgroup}{\nomrefpage.}
    Omitted.
\end{defi}

\begin{prop}
    Two elements $R$ and $T$ belongs to the same coset $kH$, if and only
    if $R^{-1}T\in H$.
\end{prop}
\begin{proof}
    Omitted.
\end{proof}
\begin{defi}[Normal/Invariant subgroup]
\nomenclature{Normal/Invariant subgroup}{\nomrefpage.}
    A subgroupNormal/Invariant subgrouproup(also invariant), if and only
    if for any $x\in G$, we have $xH = Hx$.
\end{defi}
\begin{fact}
    If $H$ has index 2, then it must be normal/invariant. This is
    obvious.
\end{fact}
\begin{defi}[Quotient]
\nomenclature{Quotient}{\nomrefpage.}
    Omitted.
\end{defi}
Note that quotient group (a.k.a. factor group) is only defined for a normal subgroup.

\begin{ex}
    $D_3$ (Using the multiplication table). Omitted because it is too
    complex to be typed down here.
\end{ex}

\subsection{Conjugacy classes}
\label{sec:Conjugacy-classes}

\begin{defi}[Conjugate]
\nomenclature{Conjugate}{\nomrefpage.}
    If exists $S\in G$, s.t. $R' = S^{-1}RS$, then we say $R'$ is
    conjugate to $R$.
\end{defi}
see (\textbf{pp.28-30 of \cite{book}})
This is clearly an equivalence relationship. By this we can define
conjugate class, denoted by $C=\{R_1,\cdots\}$, then we have the
characterization $C = \{ s^{-1} R_i s|\forall s\in G\}$, for any
$R_i$. We then have the following facts (all are obvious):
\begin{fact}$ $

    \begin{enumerate}
        \item The unit class formed just by the unit element.
        \item The inverse class formed by just all the inverse element.
            $C^{-1} = \{ R_i^{-1}\}$. If $C=C^{-1}$, then $C$ is called
            self-inverse.
        \item The order of elements in a class is just the same.
        \item For $\forall T,S\in G$, $TS$ and $ST$ are conjugate to each
            other. This means that all elements symmetric on the
            multiplication table is conjugate to each other.
        \item For two elements $R$, $R'$ conjugate to each other, both
            can be expressed by the products of two elements in two
            different way. (Isn't this too obvious to mention.)
        \item Let $G$ be a rotation group. Suppose it has an axis of the
            order of $n$, with its operation denoted as $R$,, we can get
            a new axis by the following steps. (Supose we have another
            rotation $S$),

            \begin{enumerate}
                \item  $S^{-1}$, rotate $m$ back to $n$.
                \item  $R$, rotate about n around $2\pi/n$,
                \item  $S$ rotate $n$ to $m$
            \end{enumerate}

            Result: $S^{-1}RS$ rotate around a new axis $m$ about
            $2\pi/n$. So $R'=S^{-1}RS$ and $R$ is calle the equivalent
            axis. 
            
            Also, if $m=-n$, then they are called polar axis to each
            other.

        \item $C_n$, which is an abel group, every element form a
            conjugate class by itself. Specifically, $e$ and $R^{n/2}$
            are self inverse, if $n$ is even.
    \end{enumerate}
\end{fact}
\begin{prop}
    \label{prop:20161010.conjugate_subgroup}
    For an invariant subgroup, then every conjugate element is also
    inside the same invariant subgroup. This shows that an invariant
    subgroup can be decomposed into a series of sum of conjugate classes.
\end{prop}
\begin{proof}
    If $R\in H$, then we show that $S^{-1}RS\in H$, this is obviously
    since it belongs to $S^{-1}HS$.
\end{proof}

\begin{ex}
    For $D_3: E,D,F,A,B,C$, their orders are respetively
    $1,3,3,2,2,2$. We have the following conjugate classes:
    \begin{enumerate}
        \item $\{ E\}$. Is self inverse.
        \item $\{ D,F\}$. $D$ is conjugate to $F$. We can see
            this physically by looking at rotation from the front
            or the below. This class is also self-inverse.
        \item $\{A,B,C\}$, is clearly a self-inverse conjugate
            class.
    \end{enumerate}
\end{ex}
\begin{ex}[$D_6$]
    Ramiliarize one with the formulae for $D_n$. Hint: use the order of
    elements to find classes of conjugate. Then use the proposition
    \ref{prop:20161010.conjugate_subgroup} to find the subgroups.
    % TODO maybe I can try this when I am free.
\end{ex}






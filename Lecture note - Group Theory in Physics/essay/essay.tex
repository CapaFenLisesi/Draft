% The entire content of this work (including the source code
% for TeX files and the generated PDF documents) by 
% Hongxiang Chen (nicknamed we.taper, or just Taper) is
% licensed under a 
% Creative Commons Attribution-NonCommercial-ShareAlike 4.0 
% International License (Link to the complete license text:
% http://creativecommons.org/licenses/by-nc-sa/4.0/).
\documentclass{article}

% My own physics package
% The following line load the package xparse with additional option to
% prevent the annoying warnings, which are caused by the package
% "physics" loaded in package "physicist-taper".
\usepackage[log-declarations=false]{xparse}
\usepackage{physicist-taper}
\usepackage[
    disable
]{todonotes}
\makenomenclature % For an index of symbols.

% Show the keys for labels, replace options with "final" when done
% with editing.
\usepackage[
    %draft,
    final,
    notref
]{showkeys}
\usepackage{CJKutf8}
\title{Wigner theorem and Time reversal}
\date{\today}
\author{Taper}


\begin{document}
\maketitle

\abstract{
    In this article, we use the Wigner theorem to argue the general behavior of
    a symmetry operator, based on which we will discuss how the time-reversal
    operator should act. And finally, we present the time-reversal operator's
    exact form in spins using results from representations of $SU(2)$.
}
\tableofcontents

% Representative operators and quantum mechanical selection rules 3.6
% Todo: this is my original plan. Finish it if I have time.
% Let $G$ be a group. We discuss how it act on quantum mechanical states and how
% this might determine the structure of quantum mechanics.

\section{Introduction}
\label{sec:Introduction}

We note that quantum mechanical states lives in the Hilbert space
$\mathscr{H}$.  Therefore, a
symmetry will transform the vectors in Hilbert space. But a quantum mechanical
state $\ket{\phi}$ is distinguished from a vector $\phi\in\mathscr{H}$, in that
any nonzero multiple of it $\lambda\phi (\forall \lambda\in\C^*)$, the states it
represent is the same $\ket{\phi}=\ket{\lambda\phi}$. Even when we require that
the vector should be unit, we still have a phase freedom $\lambda=e^{i\theta}$.
Therefore, the action
(i.e. representation) of a symmetry on $\mathscr{H}$ is not so easily obtained.
It should respect this freedom of $\lambda$.

To all reasonable symmetries on the Hamiltonian space $\mathscr{H}$, there is a
theorem discussed below. In a simple word the theorems says that any 
representation of a symmetry group on $\mathscr{H}$ is either unitary or
anti-unitary. The operator $U$ obtained from representation is uniquely
determined if we have fixed some $\phi'=U\phi$, where $\phi'$ is the vector after
a symmetry operation has been performed on $\phi$.

\section{Wigner's Theorem on Symmetry}
\label{sec:Wigner-Theorem}

This theorem is proved on appendix D of \cite{sternberg1995group}. Here I will only
present the theorem and explain how we may interpret and use it physically.

\paragraph{Quantum Mechanical State}
In quantum mechanics, we speak of a state as a vector $\phi$ and often ignore
its length. But if we want to formulate quantum mechanics precisely we will need
to specify a state precisely. Therefore we define a a \nomen{quantum mechanical
state} as a projection $P$ onto a one-dimensional subspace of $\mathscr{H}$.
This coincide with physical description because two vectors in $\mathscr{H}$
represent the same state when they are in the same one-dimensional subspace.
Therefore, we have one-to-one relationships between states, one-dimensional
subspaces, and projections. Explicitly, in Dirac notation, $P=\ket{p}\bra{p}$.
Then we have,
\begin{equation}
    \Tr{PQ} = \sum_n \braket{n|p}\braket{p|q}\braket{q|n} = \braket{p|q}
    \sum_n \braket{q|n}\braket{n|p} = \abs{\braket{p|q}}^2
\end{equation}
In this way, the \nomen{transition probability} of two states $P,Q$, will then
be $\Tr(PQ)$, and is denoted by \nomen{$P\cdot Q$}.

\paragraph{Quantum Mechanical Map}
A quantum mechanical map, $T$, is roughly speaking, a map that does not alter
the physical feature of the system. It is the name that the book
\cite{sternberg1995group} gave to symmetry transformations. In quantum mechanics, the
only thing we can measure is of the form $\abs{\braket{p|q}}^2$, i.e. the
probabilities.  Therefore, a quantum mechanical map is, more precisely, a map
from the states of a Hilbert space $\mathscr{H}$ to the states of a Hilbert
space $\mathscr{H}'$.  such that for any two states $P,Q$, the transition
probability $\Tr{PQ}$ is unchanged:
\begin{equation}
    (TP)\cdot(TQ) = P\cdot Q
\end{equation}

\begin{thm}[Wigner's Theorem]
    Given a quantum mechanical map $T$, there exists a map,
    $U$ from vectors of $\mathscr{H}$ to vectors of $\mathscr{H}'$, such that
    \begin{align}
        P_{U\phi} &= TP_{\phi} \\
        U(\xi+\eta)&=U\xi + U\eta,\text{ for any }\xi,\eta\in\mathscr{H} \\
        \braket{U\xi,U\eta} &= \kappa(\braket{\xi,\eta}) \label{eq:conserv-prob}
    \end{align}
    where $\kappa$ is a function that is either identity
    $\kappa(\lambda)=\lambda$ for $\lambda\in\C$, or is complex conjugate
    $\kappa(\lambda)=\bar{\lambda}$ for $\lambda\in\C$. Also, 
    \begin{equation}
        U(\lambda\xi) = \kappa(\lambda) U\xi,\quad \forall\lambda\in\C
    \end{equation}
\end{thm}

\begin{remark}
    When the quantum mechanical map $T$ is bijective and onto, then the map $U$
    is unitary if $\kappa(\lambda)=\lambda$, or anti-unitary if
    $\kappa(\lambda)=\bar{\lambda}$. For a symmetry $T$, this means that we can
    always find a way to represent it in the form of $U$, an unitary or
    anti-unitary operator.
\end{remark}

But of course the theorem tells us nothing about how to construct this operator
$U$. The proof of it in book\cite{sternberg1995group} did give a construction. Some of it
is explained below.

\textbf{Determine} $\mathbf{\kappa}$:

Assume we have three states $P_1,P_2,P_3$, and so we have three vectors
$\phi_1,\phi_2,\phi_3$ of norm $1$ in $\mathscr{H}$, for each state. Define a
function:
\begin{equation}
    \Delta(P_1,P_2,P_3) \equiv 
        \braket{\phi_1|\phi_2}\braket{\phi_2|\phi_3}\braket{\phi_3|\phi_1}
\end{equation}
This function is obviously independent of the choice of the unit vectors
$\phi_i$, i.e. it is independent of the phase of $\phi_i$.

If $\dim\mathscr{H}=1$, then $\kappa$ is not determined. But there is
really nothing interesting to be considered here.

If $\dim\mathscr{H}>1$, then $\kappa$ is determined by the following equality:
\begin{align}
    \kappa(\Delta(P_1,P_2,P_3)) &= 
    \kappa(
        \braket{\phi_1|\phi_2}\braket{\phi_2|\phi_3}\braket{\phi_3|\phi_1}
    ) \\
    &= \braket{U\phi_1|U\phi_2}\braket{U\phi_2|U\phi_3}\braket{U\phi_3|U\phi_1}
\end{align}
where the second line is determined by the action of $T$.

Note that we can pick $P_1,P_2,P_3$ such that
$\Delta(P_1,P_2,P_3)$ is not real. For example, when $\dim\mathscr{H}=2$, then
$\Delta(P_1,P_2,P_3)$ can be made into the form 
$(\sum_{i=1}^2 x_i^*y_i)(\sum_{i=1}^2 y_i^*z_i)(\sum_{i=1}^2 z_i^*x_i)$. It
is simple to make this not real. In general, we choose two orthogonal unit
vectors $\phi,\psi$ (since $\dim\mathscr{H}>1$), and let $\phi_1=\phi$,
$\phi_2=(\phi-\psi)/\sqrt{2}$, $\phi_3=(\phi+(1-i)\psi)/\sqrt{3}$. Then
\begin{align*}
    \Delta(P_1,P_2,P_3) &=
    \braket{\phi|\frac{\phi-\psi}{\sqrt{2}}}
    \braket{\frac{\phi-\psi}{\sqrt{2}}|\frac{\phi+(1-i)\psi}{\sqrt{3}}}
    \braket{\frac{\phi+(1-i)\psi}{\sqrt{3}}|\phi} \\
    &= \frac{1}{\sqrt{2}}\cdot \frac{1}{\sqrt{6}}(1-1+i)\cdot\frac{1}{\sqrt{3}}
    =\frac{i}{6}
\end{align*}

\textbf{Determine} $\mathbf{U}$:

Initially, $U$ has a phase freedom, i.e., for the same map $T$, $U$ is only
determined up to multiplication of a complex scalar of norm $1$. This can be
seen from equation~\ref{eq:conserv-prob}. But we can fix it by choosing a gauge.

Given any unit vector $\phi\in\mathscr{H}$, and any unit vector
$\phi'\in\mathscr{H}$ with
\begin{equation}
    P_{\phi'}=TP_\phi
\end{equation}
we can always choose $U$ such that
\begin{equation}
    U\phi = \phi'
\end{equation}
And having made such a choice, $U$ will be completely determined. 

However, the method given in the proof (in \cite{sternberg1995group}) is in most cases
useless. It proceeds in the mathematician's aspect that we can readily say how the
map $T$ acts on Hilbert space $\mathscr{H}$. But in reality, the method is too
mathematical to be useful. The important thing we can get from above discussion
is that $U$ can be uniquely found when we fixed a gauge. Then, we usually fixed
a basis, and argue physically how the $U\phi$ will be. Below is an example.

\section{Time reversal operator}
\label{sec:Time-reversal-operator}

To get the time reversal operator, we can first look at the following
equality (let $T$ denotes the time reversal operator):
\begin{align}
    \left(1-\frac{iH}{\hbar}\delta t\right) T \ket{\alpha}
    = T\left(1-\frac{iH}{\hbar}(-\delta t)\right)\ket{\alpha}
    \label{eq:TRoper.fromClas2Quam.phyPic}
\end{align}
This follows from the following picture: the time evolution of a time reversed
state is equivalent to the original state at an earlier time. Recall that
$\left(1-\frac{iH}{\hbar}\delta t\right)$ produces an infinitesimal time
translation.

From equality \ref{eq:TRoper.fromClas2Quam.phyPic}, one finds:
\begin{align}
    -iHT = T i H
\end{align}

Wigner's theorem shows that $T$ could either unitary or anti-unitary.
Now we show that \textbf{$T$ must be anti-linear}, hence anti-unitary. Because if
one assumes that $T$ is linear, then we have
$$ TH = -HT $$
Then one will find that the energy spectrum for time
reversal symmetric system will always has symmetric energy spectrum (i.e. with
both positive and negative energy), which is not true.

So we have:
\begin{itemize}
    \item $T$ is anti-unitary
    \item $T H=HT$ for a time reversal symmetric state.
\end{itemize}

Now we write $T= UK$, where $K$ is the complex conjugation operator that
acts only on coefficient and have $K\lambda\ket{\phi} =
\bar{\lambda}\ket{\phi}$.
Be careful that it is highly dependent on the basis
chosen. So $U$ will depend on the basis too. To get $T$ in a
basis $\ket{e_i}$, obviously we only need to consider the result of
$U\ket{e_i}$. This however, is hard to determine. However we can argue
that in case when the basis has some classical connection, then by the
Correspondence Principle we have, for example:
\begin{align}
    T \hat{p} T^{-1}&= -\hat{p} \\
    T \hat{x} T^{-1}&= \hat{x} \\
    T \mathbf{J} T^{-1}&= -\mathbf{J}
\end{align}
with these one can easily find:
\begin{align}
    T \ket{p} &= \ket{-p} \\
    T \ket{x} &= \ket{x} \\
    [x_i, p_j]T \ket{} &= i\hbar \delta_{ij} T\ket{}
\end{align}
where $\ket{}$ stands for any ket. Then clearly, $T$ is just complex conjugation
for wave function $\Psi(x)$. But $T$ acts on $\Psi(p)$ will produces
$\Psi^*(-p)$. The case for eigenstate of spin is a little complicated, which
will be discussed in next section.

\section{Time reversal operator in system with spin 
\texorpdfstring{$j$}{}}
\label{sec:Time-reversal-operator-in-spin-j}

Now label a spin state $\ket{j,m}$, where $j$ is the spin quantum number, and
$m$ is the secondary spin quantum number, where $m=-j,-j+1,\cdot,j$.
Since the commutation relation of Pauli matrices $\sigma$ are the same as the
spin operator $J$, we postulate that representations of $SU(2)$ acts on spin produces
rotations in spins. Also, physically we require that a time-reversal is equal to
a $\pi$ rotation around $y$ axis. Therefore, we have
\begin{align}
    T = \eta D(\vb{e_2},\pi)K
\end{align}
where $\eta$ accounts for the undetermined phase of $T$, $D$ is some
representation of $SU(2)$ and $D(\vb{e_2},\pi)$ is the operator of rotation by $\pi$ in
the $y$ axis. $K$ accounts for the complex conjugation.

If the spin is $j$, when we have, by representation theory of $SU(2)$:
\begin{align}
    T_{\nu\mu} = \eta d^j_{\nu\mu} K
\end{align}
where $d^j_{\nu\mu}(\omega)\equiv D^j_{\nu\mu}(\vb{e_2},\omega)$.

Using this we can prove a very useful formula:
\begin{thm} \begin{align}
    \label{eq:TRoper.TRoper.square_minus}
    T^2=
    \begin{cases}
        -1 & \text{ if $j$ is a half-integer} \\
        1 & \text{ if $j$ is an integer}
    \end{cases}
\end{align} \end{thm}
\begin{proof}
    Using group representation theory, this is almost straightforward to prove.
    Notice that rotation among $y$ axis is determined by the the matrix
    $d^j_{\nu\mu}(\omega)\equiv D^j_{\nu\mu}(\vb{e_2},\omega)$. This matrix is
    real and orthogonal. So
    $$T^2=\eta D(\vb{e_2},\pi) K \eta D(\vb{e_2},\pi) K = \eta\eta^* d^j(2\pi)K^2 = d^j(2\pi)$$
    Note also that $d^j_{\nu\mu}(2\pi) = (-1)^{2j}\delta_{\nu\mu}$. Therefore,
    when $j$ is a half integer, $T^2=-1$, and when $j$ is an integer, $T^2=1$.
\end{proof}

This theorem has a important consequence, as will be shown below:

\begin{thm}[Kramer's theorem]
    For a system with halt-integer $j$ spin, the degree of degeneracy for every
    eigenstate is at least two.
\end{thm}
\begin{proof}
    Let $\ket{n}$ be an energy eigenstate. If the system is time
    reversal symmetric, i.e. $[H,T]=0$, then $T \ket{n}$ is
    another energy eigenstate with the same energy. Supposing there is
    no degeneracy, then we have
    $$ \ket{n} = e^{i\eta}T\ket{n}$$
    for some arbitrary $\eta\in \mathbb{R}$. Apply $T$ on both sides, then
    we have
    \begin{align*}
        e^{-i\eta} T^2 \ket{n} &= \pm e^{-i\eta}\ket{n} \\
        &= T \ket{n} = e^{-i\eta} \ket{n}
    \end{align*}
    Clearly this is a contradiction unless $T^2=1$. But we have shown that
    $T^2=-1$ for half-integer spin. Therefore, 
    $$ \ket{n} \neq e^{i\eta}T\ket{n}$$
\end{proof}

\section{Concluding Remark}
\label{sec:Concluding-Remark}

The proof above is amazingly short, embodying the amazingly powerful application
of group representation theory in quantum mechanics. However, it took some trial
and effort to establish the Wigner theorem and the irreducible representations
of $SU(2)$. Hope we can learn more about group representation in the future.
\nocite{ma2007group}

\todo{improve reference in this doc}
\bibliography{cite}
\bibliographystyle{alpha}


\section{License}
The entire content of this work (including the source code
for TeX files and the generated PDF documents) by 
Hongxiang Chen (nicknamed we.taper, or just Taper) is
licensed under a 
\href{http://creativecommons.org/licenses/by-nc-sa/4.0/}{Creative 
Commons Attribution-NonCommercial-ShareAlike 4.0 International 
License}. Permissions beyond the scope of this 
license may be available at \url{mailto:we.taper[at]gmail[dot]com}.
\end{document}

% The entire content of this work (including the source code
% for TeX files and the generated PDF documents) by 
% Hongxiang Chen (nicknamed we.taper, or just Taper) is
% licensed under a 
% Creative Commons Attribution-NonCommercial-ShareAlike 4.0 
% International License (Link to the complete license text:
% http://creativecommons.org/licenses/by-nc-sa/4.0/).
\documentclass{article}

% My own physics package
% The following line load the package xparse with additional option to
% prevent the annoying warnings, which are caused by the package
% "physics" loaded in package "physicist-taper".
\usepackage[log-declarations=false]{xparse}
\usepackage{physicist-taper}

\makenomenclature

% Show the keys for labels, replace options with "final" when done
% with editing.
\usepackage[draft,notref]{showkeys}

\title{Representation theory of Finite Groups}
\date{\today}
\author{Taper}


\begin{document}


\maketitle
\abstract{
A note for correspoding chapter in S. Lang's book.
}
\tableofcontents

\section{Representationas and Semisimplicity}
\label{sec:Representationas-and-Semisimplicity}

Now I digress to read chapter 18 of the book \cite{lang-algebra}.

Let $R$ be a commutative ring and $G$ a group. Let $E$ be a
$R$-module.

For convenience, given a representation $\rho:G\to \Aut_R(E)$, let
$\sigma\ni G$, $x\in E$, we write $\sigma x$ instead of
$\rho(\sigma)x$.

An $R$-module $E$, together with a representation $\rho$, will be
called a \textbf{$G$-module}\nomenclature{$G$-module}{\nomrefpage} or
\textbf{$G$-space}\nomenclature{$G$-space}{\nomrefpage}, or also a
\textbf{$(G,R)$-module}\nomenclature{$(G,R)$-module}{\nomrefpage}.

If $E$, $F$ are $G$-modules, we define a
\textbf{$G$-homomorphism}\nomenclature{$G$-homomorphism}{\nomrefpage}
$f:E\to F$ as a $R$-linear map such that $f(\sigma x) = \sigma f(x)$
for all $x\in G$ and $\sigma\in G$.

\begin{equation}
    \begin{tikzcd}[column sep=small]
        E \ar[rr,"f"] &
        \ar[d,phantom, "\#", near start] & F \\
        & G\ar[ul,"\rho"]\ar[ur,"\rho'"] &  \,
    \end{tikzcd}
\end{equation}

We make $R$ into a $G$-module by making $G$ act trivially on $R$, i.e.
we have a trivial representation on $R$, itself regarded as a
$R$-module.

\begin{key}
Now the we discuss a systematically way to construct a new
representation from a given one. More specifically, we want to make
$\Hom_R(E,F)$ into a $G$-module, i.e. construct a representation from
$G$ to $\Hom_R(E,F)$.

First, we define the action of $G$ on $\Hom_R(E,F)$ by (let $f\in
\Hom_R(E,F)$, $f: E\to F$).
\begin{equation}
    ([\sigma] f) (x) = \sigma (f(\sigma^{-1}x))
\end{equation}
We verify that this is an operation/action of $G$ on $\Hom_R(E,F)$:
\begin{align*}
    ([e] f) (x) &= e (f(e^{-1}x)) = f(x) \\
    ([a] [b] f )(x) &= [a] bf(b^{-1}x) = ab f(b^{-1}a^{1}x) =
    abf((ab)^{-1}x) \\
    &= ([ab] f)(x)
\end{align*}
\end{key}

\begin{fact}
\label{fact:g-hom-sg-g}
For convenience we write $([\sigma]f)(x)$ simply as $[\sigma]f(x)$.
Also, 
\begin{align*}
    f(\sigma x)=\sigma f(x) \Leftrightarrow [\sigma]f(x) = \sigma
    f(\sigma^{-1}x) = \sigma\sigma^{-1}f(x) = f(x)
\end{align*}
So $f$ is a $G$-homomorphism if and only if $[\sigma]f=f$ for all
$\sigma\in G$.
\end{fact}
\begin{remark}
    This is the first time I understand why $\sigma^{-1}$ is so
    important. Without it, we don't have the beautiful fact above.
\end{remark}
\begin{defi}[Dual representation $\rho^\vee$]
\nomenclature{Dual representation $\rho^\vee$}{\nomrefpage.}
When $F=R$ (so with trivial action), then $\Hom_R(E,F)=E^\vee$ is the
dual module. If we have $\rho:G\to \Aut_R(E)$, a representation of $G$
on $E$, then the action we just defined gives the dual
representation (also called \nomen{contragredient} in the literature).
\end{defi}
More specifically, $\sigma$ acts trivially on $R$, so
\begin{align}
    \rho^\vee: & G\to \Aut_R(E^\vee)  \\
               & \sigma \mapsto [\sigma] f = f(\sigma^{-1}x) \nonumber
\end{align}

Suppose now that the modules $E$,$F$ are free and finite dimensional
over $R$ (recall that a module is free when it has a basis). Let
$\rho$ be a representation of $G$ on $E$, $M$ be the matrix of
$\rho(\sigma)$ with respect to a basis, $M^\vee$ be the matrix of
$\rho^\vee (\sigma)$ with respect to the dual basis. Then
\begin{align*}
    [\sigma] f: x &\mapsto f(\sigma^{-1} x) \\
    x^i &\mapsto \sum_{ij} f_i(M^{-1})^i_j x^j \\
    \Longrightarrow [\sigma]: f &\mapsto f\sigma^{-1} \\
    f_j &\mapsto \sum_i f_i(M^{-1})^i_j \\
    \left(\begin{array}{c}
        f_1 \\
        f_2 \\
        \cdots 
\end{array}\right) &\mapsto (M^{-1})^T  \left(\begin{array}{c}
        f_1 \\
        f_2 \\
        \cdots 
\end{array}\right) 
\end{align*}
Hence
\begin{equation}
    M^\vee = (M^{-1})^T
\end{equation}

Next we discuss the tensor product instead of Hom. Let $E$,$E'$ be
$(G,R)$-modules. We form their tensor product $E\otimes E'$, taken over
$R$. then we define an action of $G$ on $E\otimes E'$ by (let
$\sigma\in G$):
\begin{equation}
    \sigma (x\otimes x') := \sigma x \otimes \sigma x'
\end{equation}

\begin{fact}
\label{fact:Ev-F--Hom-G}
Suppose that $E$, $F$ are free and finite dimensional over $R$. Then
the $R$-isomorphism
\begin{align}
    E^\vee \otimes F &\overset{g}{\cong} \Hom_R(E,F) \\
    f_i \otimes x^j &\overset{g}{\cong} A^j_i = (f_i x^j)
    \label{eq:fx-g-aji}
\end{align}
is a $G$-isomorphism.
\end{fact}
\begin{proof} 
In this proof, we will use Einstein summation
convention. We see that
\begin{align}
    \sigma (f_i\otimes x^j) = (\sigma^\vee f_i )\otimes (\sigma x^j) =
    f_k (M_E^{-1})^k_i \otimes (M_F)^j_l x^l 
\end{align}
So
\begin{align}
    g(\sigma (f_i\otimes x^j)) = g(f_k (M_E^{-1})^k_i \otimes 
    (M_F)^j_l x^l) = (M^{-1}_E)^k_i A_k^l (M_F)^j_l
\end{align}
Meanwhile, for $h\in \Hom_R(E,F)$,
\begin{align}
    [\sigma] h: x &\mapsto \sigma h(\sigma^{-1} x) \\
    x^i &\mapsto \sigma h^i_k (M_E^{-1})^k_j x^j \\
    &= (M_F)^i_l h^l_k (M_E^{-1})^k_j x^j
\end{align}
So $G$ acts on $h\in \Hom_R(E,F)$ is
\begin{align}
    [\sigma]: h^i_j &\mapsto  (M_F)^i_l h^l_k (M_E^{-1})^k_j
\end{align}
So, consider $g(f_i\otimes x^j)\in \Hom_R(E,F)$, we have
\begin{align*}
    [\sigma] [g (f_i\otimes x^j)] = [\sigma] A^j_i = (M_F)^j_l A^l_k
    (M_E^{-1})^k_i
\end{align*}
Hence
\begin{equation}
    \label{eq:g-iso-sggs}
    g(\sigma (f_i\otimes x^j )) = [\sigma] [g(f_i\otimes x^j)]
\end{equation}
\end{proof}

Whether $E$ is free or not, we define the \nomen{$G$-invariant
submodule of $E$} to be \nomen{$\operatorname{inv}_G(E)$}$=
R$-submodule of elements $x\in E$ such that $\sigma x=x$ for all
$\sigma\in G$. If $E,F$ are free, we have an $R$-isomorphism
\begin{equation}
    \operatorname{inv}_G (E^\vee\otimes F) \cong \Hom_G(E,F)
\end{equation}
\begin{proof}
    I give only a finite dimensional proof (for no proof is provided
    in Lang's book \cite{lang-algebra}. In finite dimensional case,
    this fact is closely related to the fact \ref{fact:Ev-F--Hom-G}:
    \begin{equation*}
        E^\vee \otimes F \overset{g}{\cong} \Hom_R(E,F)
    \end{equation*}

    Let me use the notation used before. The isomorphism is actually
    the $g$ defnied for equation \ref{eq:fx-g-aji}.

    On one hand, for $f_i\otimes x^j \in \operatorname{inv}_G
    (E^\vee\otimes F)$, denote $g(f_i\otimes x^j)$ by $A^j_i$. Then
    $g(\sigma (f_i\otimes x^j)) = g(f_i\otimes x^j) = A^j_i$ for any
    $\sigma\in G$. By equation \ref{eq:g-iso-sggs}, we have
    $[\sigma]A^j_i = A^j_i$, hence by fact \ref{fact:g-hom-sg-g}
    (notice that $A^j_i\in \Hom_R(E,F)$), $A^j_i\in \Hom_G(E,F)$.  The
    converse is similar.

    I think that for infinite dimensional cases, we might replace
    $\sum$ with $\int$ and use the same logic here.
\end{proof}

\begin{defi}[Sum $\rho\oplus \rho'$]
\nomenclature{Sum $\rho\oplus \rho'$}{\nomrefpage.}
If $\rho: G\to \Aut_R(E)$, $\rho':G\to \Aut_R(E')$ are two
representation of $G$, we define their sum $\rho\oplus \rho'$ to be
the representation on the direct sum $E\oplus E'$, with $\sigma\in G$
acting componentwise.
\end{defi}

Give $G$, observe the $G$-isomorphism classes of representations have
an additive monoid structure under the above direct sum, and also have
an associative multiplicative structure under the tensor product. With
the notation of representations, we denote this product by
\nomen{$\rho\otimes \rho'$}. This product is distributive with respect
to the addition (i.e. direct sum), by their definition.

\begin{defi}[Trace $\Tr_G$]
\nomenclature{Trace $\Tr_G$}{\nomrefpage.}
    If $G$ is a finite group, and $E$ is a $G$-module, then we can
    define the trace $\Tr_G:E\to E$ which is an $R$-homomorphism,
    namely
    \begin{equation}
        \Tr_G(x) = \sum_{\sigma\in G} \sigma x
    \end{equation}
\end{defi}
\begin{fact}
$\Tr_G$ belongs to $ \operatorname{inv}_G (E)$, i.e. is
fixed under the operation of any $\sigma\in G$:
\begin{align*}
    \tau \Tr_G(x) = \sum_{\sigma\in G} \tau \sigma x =
    \sum_{\sigma'\in G} \sigma' x
\end{align*}
\end{fact}
Let $E,F$ be twn $G$-modules, and $f:E\to F$ is an $R$-homomorphism of
$G$-modules, then we can easily extend this definition to define
\nomen{$\Tr_G(f)$} by
\begin{align}
    \Tr_G(f)&:E\to F \\
    \Tr_G(f) &= \sum_{\sigma\in G} [\sigma] f \nonumber
\end{align}
Also, $\Tr_G(f)$ is invariant under any $\sigma\in G$, so it is a
$G$-homomorphism, i.e.  $\Tr_G(f)\in \Hom_G(E,F)$.

\begin{prop}
    Let $G$ be a finite group and let $E,E',F,F'$ be $G$-modules, let
    \begin{equation*}
        E'\overset{\phi}{\to} E\overset{f}{\to} 
            F \overset{\psi}{\to} F'
    \end{equation*}
    be $R$-homomorphism, and asume that $\phi,\psi$ are
    $G$-homomorphisms. Then 
    \begin{equation}
        \Tr_G(\psi\circ f\circ\phi) = \psi\circ\Tr_G(f)\circ\phi
    \end{equation}
\end{prop}
\begin{proof}
    Proof is provided on the book \cite{lang-algebra}. The essential
    point is that
    \begin{equation}
        [\sigma] (\psi\circ f\circ\phi) = \sigma \psi \sigma^{-1}
        \sigma f \sigma^{-1} \sigma \phi \sigma^{-1} =
        [\sigma]\psi\circ [\sigma]f\circ [\sigma]\phi 
    \end{equation}
\end{proof}

\begin{thm}[Maschke]
    Let $G$ be a finite group of order $n$, and let $k$ be a field
    whose characteristic does not divide $n$. then the group ring
    $k[G]$ is semisimple.
\end{thm}
Recall the an object is called \nomen{semisimple} if it is the direct
sum of simple objects. A \nomen{simple} object is some ireducible
building blocks, without being too simple to be simple. \footnote{
    As remarked in nLab
    (\href{https://ncatlab.org/nlab/show/too+simple+to+be+simple}{link}),
    There is a general principle in mathematics that "A trivial object
    is too simple to be simple". For example, $1$ is not a prime
number.} A \nomen{simple ring} is a nonzero ring who has no two-sided ideals
other than the zero ideal and itself.\footnote{ I did not find the
    definition for a simple ring on Lang's book \cite{lang-algebra}.
Rather, I found it online.}

The proof of this theorem is provided in page 666, section 18.1 of
\cite{lang-algebra}. I do not understand it. So I do not put it down
here.

\section{Anchor}
\label{sec:Anchor}

\begin{thebibliography}{1}
    \bibitem{book} Zhongqi Ma, Group Theory in Physics
    \bibitem{Ludeling} Lecture Notes for physics751: Group Theory (for
    Physicists), by C Ludeling.
    \href{http://www.th.physik.uni-bonn.de/nilles/people/luedeling/grouptheory/data/grouptheorynotes.pdf}{Link}
    \bibitem{lang-algebra} Serge Lang. Algebra. Revised 3rd. Springer.
\end{thebibliography}
\printnomenclature
\section{License}
The entire content of this work (including the source code
for TeX files and the generated PDF documents) by 
Hongxiang Chen (nicknamed we.taper, or just Taper) is
licensed under a 
\href{http://creativecommons.org/licenses/by-nc-sa/4.0/}{Creative 
Commons Attribution-NonCommercial-ShareAlike 4.0 International 
License}. Permissions beyond the scope of this 
license may be available at \url{mailto:we.taper[at]gmail[dot]com}.
\end{document}

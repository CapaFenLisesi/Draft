% The entire content of this work (including the source code
% for TeX files and the generated PDF documents) by 
% Hongxiang Chen (nicknamed we.taper, or just Taper) is
% licensed under a 
% Creative Commons Attribution-NonCommercial-ShareAlike 4.0 
% International License (Link to the complete license text:
% http://creativecommons.org/licenses/by-nc-sa/4.0/).
\documentclass{book}

\usepackage{float}  % For H in figures
\usepackage{amsmath} % For math
\usepackage{amssymb}
\usepackage{mathrsfs}
\numberwithin{equation}{subsection} % have the enumeration go to the subsection level.
									% See:https://en.wikibooks.org/wiki/LaTeX/Advanced_Mathematics
\usepackage{graphicx}   % need for figures
\usepackage{cite} % need for bibligraphy.
\usepackage[unicode]{hyperref}  % make every cite a link
\usepackage{CJKutf8} % For Chinese characters
\usepackage{fancyref} % For easy adding figure,equation etc in reference. Use \fref or \Fref instead of \ref
\usepackage{braket} %http://tex.stackexchange.com/questions/214728/braket-notation-in-latex

% Following is for theorems etc environments
% http://tex.stackexchange.com/questions/45817/theorem-definition-lemma-problem-numbering && https://en.wikibooks.org/wiki/LaTeX/Theorems
\usepackage{amsthm}
\newtheorem{defi}{Definition}[section]
\newtheorem{thm}{Theorem}[section]
\newtheorem{lemma}{Lemma}[section]
\newtheorem{remark}{Remark}[section]
\newtheorem{prop}{Proposition}[section]
\newtheorem{coro}{Corollary}[section]
\theoremstyle{definition}
\newtheorem{ex}{Example}[section]

\usepackage{xcolor} %For colourful math:http://tex.stackexchange.com/questions/21598/how-to-color-math-symbols

% To include PDF in higher version format.
\pdfoptionpdfminorversion=6

\title{Complex Geometry - Index of Notations and ideas}
\date{\today}
\author{Taper}


\begin{document}


\maketitle
\tableofcontents

\section{Note}
\label{sec:Note}
This notes aims to provide an index of symbols, definitions of the book
\cite{book}. It is very useful, especially when there are so many
wildly different concepts introduced!

\part{Indices of Notation}


\chapter{Book}
\section{1. Local Theory}
    \subsection{1.1 Holomorphic Functions of Several Variables}
    \label{sec:1.1_book}
    \textbf{Note}: the content covered by this seciton is geared for accompanying my personal notes of lecture 1.
    
    \textcolor{blue}{holomorphic}: pp.1. pp4. Def 1.1.1. pp.10(Def.1.1.8).
    
    \textcolor{blue}{Cauchy-Riemann equations}: pp.2
    
    \textcolor{blue}{$\frac{\partial}{\partial z}$,$\frac{\partial}{\partial \bar{z}}$}:
    $\frac{\partial}{\partial z}:= \frac{1}{2}
            (\frac{\partial}{\partial x} 
            -i \frac{\partial}{\partial y})
            $
    ,
    $\frac{\partial}{\partial \bar{z}}:= \frac{1}{2}
    (\frac{\partial}{\partial x} 
    + i \frac{\partial}{\partial y})
    $
    
    
    \textcolor{blue}{Maximum principle}: pp.3.
    
    \textcolor{blue}{Identity theorem}: pp.3.
    
    \textcolor{blue}{Riemann extension theorem}: pp.3. pp.9 (Prop. 1.1.7).
    
    \textcolor{blue}{Riemann mapping theorem}: pp.3.
    
    \textcolor{blue}{Liouville theorem}: pp.4.
    
    \textcolor{blue}{Residue theorem}: pp.4.
    
    \textcolor{blue}{polydiscs $B_\epsilon(\omega)$}: $\{z| |z_i-\omega_i| <\epsilon\} $.  pp.4.
    
    \textcolor{blue}{Hartogs' theorem}: Prop. 1.1.4. pp.6.
    
    \textcolor{blue}{Weierstrass preparation theorem (WPT)}:
    Prop. 1.1.6. pp.8.
    
    \textcolor{blue}{Weierstrass polynomial}: Def. 1.1.5. pp.7.
    
    \textcolor{blue}{$Z(f)$}: zero set of $f$. pp.9.
    
    \textcolor{blue}{biholomorphic}: pp.10.
    
    \textcolor{blue}{(complex) Jacobian, regular, regular value}: Def. 1.1.9. pp.10.
    
    \textcolor{blue}{IFT. Inverse function theorem}: Prop 1.1.10 pp.11.
    
    \textcolor{blue}{IFT. Implicit function theorem}: Prop 1.1.11. pp.10.
    
    \textcolor{blue}{$\mathcal{O}_{\mathbb{C}^n}$}:
    sheaf of holomorphic functions on $\mathbb{C}^n$. Def. 1.1.14. pp.14.
    
    \textcolor{blue}{$\mathcal{O}_{\mathbb{C}^n,z}$}: Def. 1.1.14. pp.14.
    
    \textcolor{blue}{$\mathcal{O}^*_{\mathbb{C}^n,0}$}:
    units of $\mathcal{O}_{\mathbb{C}^n,0}$. pp.14.
    
    \textcolor{blue}{UFD, unique factorization domain, irreducible}: 
    Def. 1.1.16. pp.14.
    
    \textcolor{blue}{Gauss Lemma}: pp.14.
    
    \textcolor{blue}{Weierstrass division theorem}: Prop. 1.1.17. pp.15.
    
    \textcolor{blue}{germ of set}: (\textbf{pp.18})

    \textcolor{blue}{$Z(f)$}: germ of zero set of $f$. (\textbf{pp.18})

    \textcolor{blue}{analytic germ}: $Z(f_1,\cdots,f_k)$. (\textbf{pp.18})

    \textcolor{blue}{analytic subset}: locally are zero sets. (\textbf{pp.18})

    \textcolor{blue}{$I(X)$}: the set of all $f\in
    \mathcal{O}_{\mathbb{C}^n,0}$ with $X\subset Z(f)$. (\textbf{pp.18})
    % TODO bookmark
    \subsection{1.2 Complex and Hermitian Structures}
    \textcolor{blue}{almost complex structure, $I$}: $I^2=-\text{id}$.
    (\textbf{pp.25})

    \textcolor{blue}{$V^{1,0}$ and $V^{0,1}$}: the $\pm i$ eigenspaces 
    of $I$.
    (\textbf{pp.25})

    \textcolor{blue}{$\bigwedge^{p,q} V$}:=
    $\bigwedge^p V^{1,0}\oplus_{\mathbb{C}}\bigwedge^q V^{0,1}$.
    (\textbf{pp.27})

    \textcolor{blue}{$\prod^k,\prod^{p,q}$}: natural projections.
    (\textbf{pp.28})
    
    \textcolor{blue}{$\mathbf{I}$}:=$\sum_{p,q}i^{p-q}\cdot\prod^{q,p}$.
    (\textbf{pp.28})

    \textcolor{blue}{compatible}:an almost complex structure $I$ is
    compatible with the scalar product $<,>$, if $<I(v),I(w)>=<v,w>$.
    (\textbf{pp.28})

    \textcolor{blue}{Conformal equivalence}(between scalar product):
    (\textbf{pp.29})

    \textcolor{blue}{fundamental form, $\omega$}:=$<I\text{( )},()>.$
    (\textbf{pp.29})
    $$\omega=\frac{i}{2}\sum_i z^i\wedge\bar{z}^i=\sum_i x^i\wedge y^i.$$
    \quad(Local calculation could be found on \textbf{pp.31})

    \textcolor{blue}{hermitian form $(,)$}:=$<,>-i\cdot\omega$.
    (\textbf{pp.30})

    \textcolor{blue}{Lefschetz operator $L$}:
        $\bigwedge^* V^*_{\mathbb{C}}\to
    \wedge^* V_{\mathbb{C}}^*$, given by $\alpha\to\omega\wedge\alpha$.
    (\textbf{pp.31})

    \textcolor{blue}{Hodge star $*$-operator}: $\alpha\wedge *\beta=
        \langle\alpha,\beta\rangle \cdot \text{vol}$.
    (\textbf{pp.33})

    \textcolor{blue}{dual Lefschetz operator $\Lambda$}:
        $\langle\Lambda\alpha,\beta\rangle =\langle\alpha,L\beta\rangle$,
        degree $-2$, bidegree $(-1,-1)$, $\Lambda=*^{-1}\circ L\circ *$.
    (\textbf{pp.33 to 34})

    \textcolor{blue}{Counting operator $H$}:
        $H=\sum_{k=0}^{2n}(k-n)\cdot\prod^k$, 
        where $\text{dim}_{\mathbb{R}}=2n$.
    (\textbf{pp.34})

    \textcolor{blue}{commutator [A,B]}:$=A\circ B-B\circ A$.
    (\textbf{pp.34})


\paragraph{Commutators}
$$[H,L]=2L,\text{ } [H,\Lambda]=-2\Lambda,\text{ } [L,\Lambda]=H.$$
(\textbf{pp.34})

$$[L^i,\Lambda](\alpha)= i(k-n+i-1)L^{i-1}(\alpha),\text{for all}
    \alpha\in\bigwedge^k V^*$$
(\textbf{pp.35})

    \textcolor{blue}{primitive element} in $\bigwedge^kV^*$:
        $\alpha$ is primitive if and only if $\Lambda\alpha=0$.
    (\textbf{pp.36})

    \textcolor{blue}{$P^k$} $\subset\bigwedge^kV^*$: is the subspace
    of all primitive elements.
    (\textbf{pp.36})

    \textcolor{blue}{Hodge-Riemann pairing $Q$}:
    $$\bigwedge^kV^*\times\bigwedge^kV^*\to \mathbb{R}\text{, }
     (\alpha,\beta)\mapsto (-1)^{k(k-1)}{2}\alpha\wedge\beta\wedge w^{n-k}$$

    Note: here we identify $\bigwedge^{2n}V^*$ with $\mathbb{R}$ 
    by the volumn form $\text{vol}$. Also the $\mathbb{C}$-linear 
    extension of this is still denoted $Q$.

\section{2.Complex Manifolds}

    \subsection{2.1 Complex and Hermitian Structures}
    \textcolor{blue}{almost complex structure}: (\textbf{pp.25})
	\subsection{2.2 Holomorphic Vector Bundles}

    \textcolor{blue}{$\tau_X$}: holomorphic tangent bundle of a complex manifold $X$ (Def 2.2.14 at pp. 71).
    
    \textcolor{blue}{$\varOmega_X$, $\varOmega^p_X$}: holomorphic cotangent bundle and holomorphic $p$-forms. (Def 2.2.14 at pp. 71)
    
    \textcolor{blue}{$K_X$}:=det($\varOmega_X$) = $\varOmega_X^n$, the canonical bundle of $X$. (Def 2.2.14 at pp. 71)

    \subsection{2.6 Differential Calculus on Complex Manifolds}

    \textcolor{blue}{$\wedge^k_{\mathbb{C}}X$}:=$\wedge^k(T_{\mathbb{C}}X)^*$. (Def 2.6.7 at pp. 105)
    
    \textcolor{blue}{$\wedge^{p,q}X$}:=$\wedge^p(T^{1,0}X)^*
                        \bigoplus_{\mathbb{C}}\wedge^q(T^{0,1}X)^*$. (Def 2.6.7 at pp. 105)

    \textcolor{blue}{$\mathcal{A}^k_{X,\mathbb{C}}$,$\mathcal{A}^{p,q}_X$}: sheaves of section of the above correspond items. (Def 2.6.7 at pp. 105)
    
    \textcolor{blue}{$\mathcal{A}^{p,q}(E)$}: the sheaf of $p,q$-forms with values in $E$, a complex vector bundle. (Def 2.6.22 at pp.109). Note that in particular, $\mathcal{A}^0(E)$ is the sheaf of sections of $E$.

	\subsection{Appendix B: Sheaf Cohomology}
	
	\begin{itemize}
		\item
		\textcolor{blue}{pre-sheaf}: Def B.0.19, pp. 287.
		
		\item
		\textcolor{blue}{$\mathcal{C^0_M}$}: the pre-sheaf of continuous functions on $M$. Example B.0.20, pp. 287.
		
		\item
		\textcolor{blue}{sheaf}: Def B.0.21, at pp.288.
		
		\item
		\textcolor{blue}{\underline{$\mathbb{R}$},\underline{$\mathbb{Z}$}}: constant sheaves, Sometimes written simply as \textcolor{blue}{$\mathbb{R}$}, \textcolor{blue}{$\mathbb{Z}$} respectively. pp. 288.
		\item
		\textcolor{blue}{$\mathcal{E}$}: actually a $\mathcal{C}^0_M$-modules. Sometimes identified as $E$. pp.288.
		\item
		\textcolor{blue}{(pre)-sheaf homomorphism}: Def B.0.23.  pp.288.
		\item
		\textcolor{blue}{Ker($\phi$),Im($\phi$),Coker($\phi$)}: as pre-sheaves in pp.288. sheaves in pp.289, Def B.0.26.
		\item
		\textcolor{blue}{injective, surjective of sheaf-homomorphism}:pp.289.
		\item
		\textcolor{blue}{complex, exact complex}: Def B.0.27. pp.289
		\item
		\textcolor{blue}{text}:
		\item
		\textcolor{blue}{text}:
		\item
		\textcolor{blue}{text}:
		\item
		\textcolor{blue}{text}:
		\item
		\textcolor{blue}{text}:
		\item
		\textcolor{blue}{text}:
		\item
		\textcolor{blue}{text}:
		\item
		\textcolor{blue}{text}:
		\item
		\textcolor{blue}{text}:
		\item
		\textcolor{blue}{text}:
		\item
		\textcolor{blue}{text}:
		\item
		\textcolor{blue}{text}:
		\item
		\textcolor{blue}{text}:
		\item
		\textcolor{blue}{text}:
		\item
		\textcolor{blue}{text}:
		\item
		\textcolor{blue}{text}:
		\item
		\textcolor{blue}{text}:	
	\end{itemize}


\chapter{My lecture Notes}

\section{Lecture 2016 Lecture 1}
The first few lectures are not well noted, hence I delegate the task of recording the theorems and notations to the book's correspoding section:\fref{sec:1.1_book}.

\section{Lecture 4 (20160307) Complex Manifold}

\textbf{Note}: we use abbrevation \textit{mnfd} for \textit{manifold}.

pp. A:
    \begin{itemize}
        \item \textcolor{blue}{Holomorphic Atlas}
        \item \textcolor{blue}{Holmorphic chart}
        \item \textcolor{blue}{Complex mnfd}
    \end{itemize}

pp. B:
    \begin{itemize}
        \item \textcolor{blue}{Holomorphic function}
        \item \textcolor{blue}{$\mathcal{O}_X$}: sheaf of holomorhic functions on a complex mnfd $X$.
    \end{itemize}
pp. C:
    \begin{itemize}
        \item \textcolor{blue}{Hartdogs' theorem}: on complex mnfd.
        \item \textcolor{blue}{Holomorphic functions on complex mnfd}:
    \end{itemize}
pp. D:
    \begin{itemize}
        \item \textcolor{blue}{Complex Lie group}
        \item \textcolor{blue}{Complex Projective Space, $\mathbb{CP}^n$, or just $\mathbb{P}^n$}.
    \end{itemize}
pp. E:
    \begin{itemize}
        \item \textcolor{blue}{Topology in $\mathbb{P}^n$}
        \item \textcolor{blue}{Mnfd structure on $\mathbb{P}^n$, atlas, and the \textbf{canonical covering}}
    \end{itemize}
pp. F
    \begin{itemize}
        \item Grassmannian mnfd.
    \end{itemize}

\section{Lecture 5 Submanifolds (20160308)}

pp. A:
    \begin{itemize}
        \item Affine Hypersurface (actually this is not quite different from the usual $\mathbb{C}^n$.)
    \end{itemize}
	
Part 2. \textbf{Sheaf Theory}

pp. A:
	\begin{itemize}
		\item pre-sheaf
		\item $\mathcal{O}_X(U)$
		\item $\mathcal{O}^*_X(U)$
	\end{itemize}
pp. B:
	\begin{itemize}
		\item $C^{\infty}$
		\item $\mathbb{\underline{Z}}$, sometimes simply denoted as $\mathbb{Z}$: sheaf of localy constant $\mathbb{Z}$-valued functions.
		\item Sheaf
	\end{itemize}
pp. D:
	\begin{itemize}
		\item sheaf-morphisms
	\end{itemize}
pp. E:
	\begin{itemize}
		\item Section
		\item Ker($\phi$) - sheaf of kernals.
	\end{itemize}
pp. F:
	\begin{itemize}
		\item Im($\phi$) is a presheaf, but not a sheaf.
		\item Im($\phi$): the sheafification of Im($\phi$) above. Note that we use the same notation to denote both.
	\end{itemize}
	
\section{Lecture 6 Sheaf \& Cohomology (20160315)} pp. A:
    \begin{itemize}
        \item Stalk $\mathcal{F}_x$.
        \item germ
        \item Directed partial order set
        \item Directed System
    \end{itemize}
pp. B:
    \begin{itemize}
        \item Directed limit
    \end{itemize}
pp. C:
\begin{itemize}
    \item Exact Complex/ Exact Sequence.
    \item Exponential sequence \textit{(mentioned under the definition of exact sequence)}.
    \item
\end{itemize}
pp. D:
\begin{itemize}
    \item Čech cohomology
\end{itemize}
pp. E:
\begin{itemize}
    \item q-cochain
    \item coboundary operator. $\delta$.
    \item $Z^p(U,\mathcal{F})$ = Ker.
    \item $B^p(U,\mathcal{F})$ = Im.
    \item $\check{H}^p(U,\mathcal{F})$ = $\frac{Ker}{Im}$.
\end{itemize}

    \subsection{Notes of \v{C}ech Cohomology with Coeficients in a Sheaf}
    pp.1:
    \begin{itemize}
        \item q-simplex $\sigma$.
        \item support $|\sigma|$.
        \item q-cocain
        \item $C^q(U,\mathcal{F})$
        \item Coboundary Operator $\delta$.
    \end{itemize}
    pp.2,3,4:
    \begin{itemize}
        \item Cochain Complex
        \item Čech cohomology
        \item cocycle
        \item cochain
        \item $\check{H}^p(U,\mathcal{F}),Z^p(U,\mathcal{F}),B^p(U,\mathbb{F})$.
        \item $\check{H}^0(\{u_i\},\mathcal{F})$ = $\mathcal{F}(X)$.
    \end{itemize}
	
\section{Lecture 7 Vector Bundle (20160321)}
pp.1,2:
\begin{itemize}
    \item Vector Bundle
    \item Trivializing covering, $\{(U_i,\tau_i)\}$.
    \item trivializing maps, trivializes.
    \item VB-equivalent of trivializing maps.
    \item E: total space, X: base space.
\end{itemize}
pp. 3,5:
\begin{itemize}
    \item transition maps.
    \item fibre.
    \item $\mathcal{O}(-1)$
    \item cocycle condition.
    \item $\mathcal{T}_X$, Holomorphic tangent bundle.
\end{itemize}
pp. 8:
\begin{itemize}
    \item s: section of a holomorphic vector bundle.
    \item $\mathcal{E}$: sheaf of sections of holomorphic vector bundle. $\mathcal{E}(U)$.
\end{itemize}
	
\section{Lecture 8 Almost Complex Structures (20160322)}
pp. 1,2:
\begin{itemize}
    \item $I$: Almost Complex Structure. $I^2=-1$. Sometime $J$ is used in place of $I$.
    \item $V_{\mathbb{C}}$ := $V\otimes \mathbb{C}$.
    \item $I_\mathbb{C}$: $I$ extending to $V_{\mathbb{C}}$. Usually abbreviated simply as $I$.
    \item $V^{1,0}$:= ker$(I+i)$.
    \item $V^{0,1}$:= ker$(I-i)$.
\end{itemize}
	
\section{Lecture 9 Exterior Algebra on Complex Manifold (20160329)}

pp.1,2:
\begin{itemize}
    \item $V^*$: dual of $V$.
    \item $\{dx^i,dy^i\}$.
    \item $J^*$: $J$ extending to dual space.
    \item $dz^i,d\bar{z}^i$.
\end{itemize}
pp. 3:
\begin{itemize}
    \item $S^k(V)$, $\Lambda^k(V)$.
    \item $s$ and $a$, symmetrization and anti-symmetrization of a tensor.
    \item $\Lambda^* V$.
\end{itemize}
pp. 4:
\begin{itemize}
    \item $\Lambda^n T^*_{\mathcal{C}}X$.
    \item $\Lambda^* T^*_{\mathcal{C}}X$.
    \item $\Lambda^{p,q} T^*_{\mathcal{C}}X$.
\end{itemize}
pp. 5,6:
\begin{itemize}
    \item $\mathcal{A}$: sheaf of section of cotangent bundle.
    \item $\mathcal{A}^n(U)$, $\mathcal{A}^{p,q}(U)$.
    \item $\Lambda$ on $\mathcal{A}$.
    \item $d$: de Rham differential.
    \item $\partial, \bar{\partial}$.
\end{itemize}
	
\section{Lecture 10 Debeault Cohomology (20160406)}

pp. 1:
\begin{itemize}
    \item $\mathcal{H}^{p,q}(X)$.
    \item $f^*$: pull-back. Various defintion from pp.1 to pp.4.
\end{itemize}
pp. 5,6,7:
\begin{itemize}
    \item $\mathcal{A}^{p,q}(U,E):=\Gamma(U,\Lambda^{p,q}T_{\mathbb{C}}^* X \otimes E)$.
    \item $\bar{\partial}_E$
    \item $\mathcal{H}^{p,q}(X,E)$.
    \item $\bar{\partial}$-Poincaré lemma in one variable.
\end{itemize}
	
\section{Lecture 11 (20160412)}
pp.1,2,3: 
\begin{itemize}
    \item $\bar{\partial}$-Poincaré lemma in n-dimension
    \item $\Omega_X^p$: holomorphic p-forms. On pp.2.
    \item $\check{H}^q(X,\Omega^p)\text{(Čech)}\cong \mathcal{H}_{\bar{\partial}}^{p,q}(X)$(Dolbeault). On pp.3.
\end{itemize}
pp. 6,7:
\begin{itemize}
    \item Analytic Subvarity.
    \item Analytic Hybersurface.
    \item Cousin's Problem.
\end{itemize}
	
\section{Lecture 12 Hermitian Structure on Manifold  Manifold (20160418)}
pp. 1,2,3:
\begin{itemize}
    \item $I$ compatible with $<-,->$.
    \item $\omega$: Fundamental form associated with $<,>$ and $I$. $\omega(v,w):= <I(v),w>$.
    \item Conformal Equivalence.
    \item $<,>$: Hermitian Inner Product.
\end{itemize}
pp. 4:
\begin{itemize}
    \item $( , )$: s.t. $(v,w):=<v,w>-i\omega(v,w) = <v,w> - i <I(v),w>$
\end{itemize}
pp. 5:
\begin{itemize}
    \item $<,>_{\mathbb{C}}$ be s.t.$<v\otimes \alpha, w\otimes \beta>:= \alpha \bar{\beta} <v,w>$.
\end{itemize}
pp. 6:
\begin{itemize}
    \item $\frac{1}{2}(,) = <,>_{\mathbb{C}} \arrowvert_{V^{1,0}}$
\end{itemize}
pp. 7,8:
\begin{itemize}
    \item Local computations: $z_i$,$h_{ij}$,
    \item $\omega = (...dx^i...dy^i)$
    \item $\omega$, Fundamental form on Riemannian Mnfd.
    \item Kähler mnfd: $d\omega \equiv 0$.
\end{itemize}
	
\section{Lecture 13 Kähler Manifold (20160419)}
pp.1:
\begin{itemize}
    \item Local computation: $\omega = (...dz^i...d\bar{z}^i)$
\end{itemize}
pp.4:
\begin{itemize}
    \item Fubini-Study Metric on $\mathbb{CP}^n$.
\end{itemize}

\section{Lecture 14 Hodge Theory (20160425)}
pp.1:
\begin{itemize}
    \item $<,>$ on $\Lambda^k V$
    \item vol: volumn element.
    \item $*$: Hodge Star Operator.
\end{itemize}
pp.4:
\begin{itemize}
    \item $L$: Lefschetz Operator
    \item $\Lambda$: adjoint of $L$. $\Lambda = *^{-1}\circ L\circ *$.
\end{itemize}
pp.5 :
\begin{itemize}
    \item $*,L,\Lambda$ on Kähler mnfd.
    \item $d^*:= (-1)^{m*(k+1)+1}*\circ d\circ*$, adjoint of $d$. On a Kähler mnfd, $d^*= -*\circ d\circ*$
    \item $\Delta:=d^*\circ d + d\circ d^*$.
\end{itemize}
pp. 6:
\begin{itemize}
    \item $\bar{\partial}^*, \partial^*$: Similar to the above for $d$.
    \item $\Delta_\partial, \Delta_{\bar{\partial}}$: Similar to the above for $d$.
\end{itemize}
	
\section{Lecture 15 Hodge Theory on Manifold (20160426)}
pp.1:
\begin{itemize}
    \item $(,)$ on $\mathcal{A}^*(X)$. $(\alpha,\beta):=\int_X g_\mathbb{C}(\alpha,\beta) vol$
\end{itemize}
pp.3:
\begin{itemize}
    \item $\mathcal{H}^k(X,g)$: d-harmonic forms. Sometimes we replace $\mathcal{H}$ with $\mathscr{H}$ for harmonic forms, so is for symbols below.
    \item $\mathcal{H}^k_{\bar{\partial}}(X,g)$: $\bar{\partial}$-harmonic forms. 
    (Be careful to distinguish this with Dolbeault Cohomology groups).
    \item $\mathcal{H}^k_{\partial}(X,g)$: $\partial$-harmonic forms.
\end{itemize}
pp. 5:
\begin{itemize}
    \item $\mathcal{H}^k_d(X,g) \cong \mathcal{H}^{2n-k}_d(X,g)$, Poincar\'{e} duality
    \item $\mathcal{H}^{p,q}_{\bar{\partial}}(X,g) \cong 
    \left( \mathcal{H}^{n-p,n-q}_{\bar{\partial}}(X,g)\right)^*$, both are harmonic forms, called Serre Duality. 
\end{itemize}
pp. 6,7:
\begin{itemize}
    \item 
    $\mathcal{A}^{p,q} = 
    \bar{\partial}\mathcal{A}^{p,q-1}(X)\oplus
    \bar{\partial}^* \mathcal{A}^{p,q+1}(X) \oplus
    \mathcal{H}^{p,q}_{\bar{\partial}}(X,g)$:
    
     Hodge decomposition

    \item $\mathcal{H}^{p,q}_{\bar{\partial}}
        \text{(harmonic forms)}
        \cong
        \mathcal{H}^{p,q}_{\bar{\partial}}(X)
        \text{(Dolbeault Cohomology group)}$
    \item $\mathcal{H}^{p,q}_{d}
        \text{(harmonic forms)}
        \cong
        \mathcal{H}^{p,q}_{dR}(X)
        \text{(de Rham Cohomology group)}$
\end{itemize}
pp. 8:
\begin{itemize}
    \item A lot of isomorphisms between \textit{de Rham}, \textit{Dolbeault} and \textit{harmonic forms}.
\end{itemize}

\section{Lecture 16 Harmonic forms on K\"{a}hler Manifold (20160503)}

pp.1:
\begin{itemize}
    \item $\Delta_\partial = \Delta_{\bar{\partial}} = 
        \frac{1}{2}\Delta_d$, for K\"{a}hler mnfd.
\end{itemize}

\section{Lecture 17 Hermitian Vector Bundle (20160510)}
pp.1,3:
\begin{itemize}
    \item Hermitian Vector Bundle. pp.1
    \item Antilinear map. pp.3.
    \item Hermitian Inner Product on $\mathcal{A}^{p,q}(X,E)$. pp.4
    \item $\bar{*}_E$ Hodge Operator on Hermitian vector bundle. pp.5.
    \item $\bar{\partial}_E^*$
\end{itemize}
pp. 8:
\begin{itemize}
    \item Kadaira-Serre Duality.
\end{itemize}

\section{Lecture 18 Connection (20160516)}
\begin{itemize}
    \item $\bigtriangledown$: connection. pp.1.
    \item Trivial connections. pp.2
    \item $\mathcal{A}^1(M,End(E)):=\Gamma(M,\Lambda^1 M\otimes End(E))$. pp.3. Also, one may find how elements in this sheaf act on $\mathcal{A}^0(M)$ on pp.173, inside proof of proposition 4.2.3.
    \item $s\in \mathcal{A}^0(E)$ is Parrallel/flat/constant $\Leftrightarrow \Delta(s) = 0$. pp.4.
    \item $\Delta = d+A$. pp.4.
    \item $\Delta$ be compatible with hermitian structure on $E$. pp.5.
    \item $\Delta$ be compatible with holomorphic vector bundle. pp.6.
    \item $A=\bar{H}^{-1} \partial H$. Chern connection. pp.6.
\end{itemize}

\section{Lecture 19 Holomorphic Connection \& Curvature(20160517)}
\begin{itemize}
    \item Holomorphic Connecction. pp.1.
    \item $At(E)$: Atiyah class of $E$. pp.2.
    \item $\Delta^k$. pp.4.
    \item $F_\Delta$: curvature associated with $\Delta$. pp.5.
    \item $F_\Delta = dA + A\wedge A$: Cartan structure equation. pp.6.
    \item First Chern class of complex line bundle.
\end{itemize}

\section{Lecture 20 Divisors \& (Holomorphic) Line Bundles (20160524)}
\begin{itemize}
    \item Analytic Subvariety. pp.1.
    \item Regular/Smooth Point. pp.1.
    \item Singular Point. pp.2.
    \item Irreducible analytic subvariety. pp.2.
    \item $dim(Y)$: dimension of analytic subvariety. pp.2. Also pp.4.
    \item Affine algebraic varieties. pp.3.
    \item Projective algebraic varieties. pp.3.
    \item Hypersurface. pp.4.
    \item Divisor, $Div(X)$:=group of all divisors. pp.5.
    \item Effective divisor. pp.6.
    \item $Ord_Y(f)$: order of function. pp.6. Also pp.8.
    \item Meromorphic function on complex mnfd.
    \item $(f)$: divisor given by a global meromorphic function.
    \item Principal divisor. pp.8.
\end{itemize}

\section{Lecture 21 Divisors \& (Holomorphic) Line Bundles  (20160530)}
\begin{itemize}
    \item $H^0(X,K^*_X/\mathcal{O}^*_X) \cong Div(X)$. pp.1.
    \item $Pic(X)$: Picard group, all holomorphic line bundles. pp.3.
    \item $Pic(X) \cong \check{H}^1(X,\mathcal{O}^*_X)$. pp.3.
    \item $\mathcal{O}(D)$: line bundle given by divisor $D$. pp.5.
    \item Linear equivalent of divisors.
    \item $*$: used only in this section to denoted the map:
    $$\left( Div(X)/Pic(X) \right) \hookrightarrow Pic(X)$$
    pp.6.
    \item $Z(s)$: divisor constructed from nonzero section $s\in H^0(X,L)$ for a line bundle $L$.
\end{itemize}

\section{Lecture 22 Divisors \& (Holomorphic) Line Bundles (20160606)}
\begin{itemize}
    \item Base point of a line bundle. pp.4.
    \item $Bs(L)$:= set of all base points of line bundle $L$. pp.4.
    \item $\mathcal{O}(1), \mathcal{O}(k)$. pp.6.
\end{itemize}



\part{Indices of Results}

{Theorems, Remarks, etc.}


\chapter{Local Theory}
\label{chap:Local Theory}

\section{1.1 Holomorphic Functions of Several Variables}
\label{sec:book_1.1}

\begin{prop}
    The local ring $\mathcal{O}_{\mathbb{C}^n,0}$ is a UFD.
\end{prop}
(\textbf{pp.14 of \cite{book}})

\begin{prop}{Weierstrass division theorem}
    Let $f\in \mathcal{O}_{\mathbb{C}^n,0}$ and $g\in
    \mathcal{O}_{\mathbb{C}^{n-1},0}[z_1]$ be a Weierstrass
    polynomial of degree $d$. Then there exist $r\in
    \mathcal{O}_{\mathbb{C}^{n-1},0}[z_1]$ of degree $<d$
    and $h\in \mathcal{O}_{\mathbb{C}^n,0}$ such that $f=g\cdot h+r$.
    The functions $h$ and $r$ are uniquely determined.
\end{prop}
(\textbf{pp.15 of \cite{book}})

\begin{prop}
    The local UFT $\mathcal{O}_{\mathbb{C}^n,0}$ is Noetherian.
\end{prop}
(\textbf{pp.16 of \cite{book}})

\begin{coro}
    Let $g\in \mathcal{O}_{\mathbb{C}^n,0}$ be an irreducible function.
    If $f\in\mathcal{O}_{\mathbb{C}^n,0}$ vanishes on $Z(g)$, then
    $g$ divides $f$.
\end{coro}
(\textbf{pp.16 of \cite{book}})

\begin{lemma}
    For any germ $X\subset \mathbb{C}^n$ the set $I(X)\subset
    \mathcal{O}_{\mathbb{C}^n,0}$ is an ideal.
    If $(A)\subset \mathcal{O}_{\mathbb{C}^n,0}$ denotes the ideal
    generated by the subset $A\subset \mathcal{O}_{\mathbb{C}^n,0}$,
    then $Z(A)=Z((A))$ and $Z(A)$ is analytic.
\end{lemma}
(\textbf{pp.18 of \cite{book}})

\begin{lemma}
    If $X_1\subset X_2$, then $I(X_2)\subset I(X_1)$. If $I_1\subset I_2$,
    then $Z(I_2)\subset Z(I_1)4$. For any analytic germ $X$ one has
    $Z(I(X))=X$. For any ideal $I\subset \mathcal{O}_{\mathbb{C}^n,0}$,
    one has $I\subset I(Z(I))$.
\end{lemma}
(\textbf{pp.18 of \cite{book}})
% TODO bookmark this chapter stops at page 18.

\section{1.2 Complex and Hermitian Structures}
\begin{lemma}
    If $I$ is an almost complex structure on a real vector space $V$,
    then $V$ admits in a natural way the structure of a complex vector
    space
\end{lemma}
(\textbf{pp.25 of \cite{book}})

\begin{remark}
    An almost complex structure can only exist on an even dimensional
    real vector space.
\end{remark}

\begin{coro}
    Any almost complex structure on $V$ induces a natural orientation
    on $V$.
\end{coro}
(\textbf{pp.25 of \cite{book}})

\begin{lemma}
    Let $V$ be a real vector space endowed with an almost complex
    structure $I$. Then
    $$ V_{\mathbb{C}} = V^{1,0}\oplus V^{0,1}$$
    Complex conjugation on $V_{\mathbb{C}}$ induces an $\mathbb{R}$-linear
    isomorphism $V^{1,0}\cong V^{0,1}$.
\end{lemma}
(\textbf{pp.26 of \cite{book}})

\begin{remark}
    Two almost complex structures on $V_{\mathbb{C}}$: $I$ and $i$,
    coincide on the subspace $V^{1,0}$ but differ by a sign on
    $V^{0,1}$.
\end{remark}

\begin{lemma}
    Let $V$ be a real vector space endowed with an almost complex
    structure $I$. Then the dual space 
    $V^*=\text{Hom}_{\mathbb{R}}(V,\mathbb{R})$ has a natural almost
    complex structure given by $I(f)(v)=f(I(v))$. The induced 
    decomposition on 
    $(V^*)_{\mathbb{C}}=\text{Hom}_{\mathbb{R}}(V,\mathbb{C})
    = (V_{\mathbb{C}})^*$ is given by
    $$ (V^*)^{1,0} = \{f\in \text{Hom}_{\mathbb{R}}(V,\mathbb{C}) |
        f(I(v)) = i f(v) \} = (V^{1,0})^*$$
    $$ (V^*)^{0,1} = \{f\in \text{Hom}_{\mathbb{R}}(V,\mathbb{C}) |
        f(I(v)) = -i f(v) \} = (V^{0,1})^*$$
    Also note that $(V^*)^{1,0} = 
        \text{Hom}_{\mathbb{C}}((V,I),\mathbb{C})$.
\end{lemma}

\begin{prop}
    For a real vector space $V$ endowed with an almost complex
    structure $I$, one has:
    \begin{enumerate}
        \item $\bigwedge^{p,q}V$ is in a canonical way a subsapce
            of $\bigwedge^{p+q} V_{\mathbb{C}}$.
        \item $\bigwedge^k V_{\mathbb{C}}
            = \bigoplus_{p+q=k}\bigwedge^{p,q}V$.
        \item Complex conjugation on $\bigwedge^* V_{\mathbb{C}}$ defines
            a ($\mathbb{C}$-linear) isomorphism
            $\bigwedge^{p,q}V\cong \bigwedge^{q,p}V$, i.e.
            $\bar{\bigwedge^{p,q}V}=\bigwedge^{q,p}V$.
        \item The exterior prodoct is of bidegree $(0,0)$.
    \end{enumerate}
\end{prop}
(\textbf{pp.27 of \cite{book}})
\begin{remark}{Local calculation of $V^{1,0}$,$(V^*)^{1,0}$}
    $$z_i=\frac{1}{2}(x_i-y_i)\text{, }\bar{z}_i=\frac{1}{2}(x_i+iy_i)$$
    $$z^i=x^i+iy^i\text{, }\bar{z}^i=x^i-iy^i$$
    $$I(z_i)=i z_i\text{, }I(z^i)=i z^i$$
\end{remark}
(\textbf{pp.27 to 28 of \cite{book}})

\begin{lemma}
    For any $m\leq \text{dim}_{\mathbb{C}}V^{1,0}$, one has
    $$(-2i)^m (z_1\wedge\bar{z}_1)\wedge\cdots\wedge(z_m\wedge\bar{z}_m)=
        (x_1\wedge y_1)\wedge\cdots\wedge(x_m\wedge y_m).$$
    For $m=\text{dim}_\mathbb{C} V^{1,0}$, this defines a positive
    oriented volume form for the natural orientation of $V$.

    Also
    $$\left(\frac{i}{2}\right)^m
      (z^1\wedge\bar{z}^1)\wedge\cdots\wedge(z^m\wedge\bar{z}^m)=
      (x^1\wedge y^1)\wedge\cdots\wedge(x^m\wedge y^m).$$
\end{lemma}

\begin{prop}[Lefschetz decomposition]
    There exists a direct sum decomposition of the form:
    \begin{align}
        \bigwedge^kV^* = \bigoplus_{i\geq 0} L^i(P^{k-2i})
    \end{align}
    Also, $P^k={\alpha\in \bigwedge^k V^*|L^{n-k+1}\alpha=0}$, for
    $k\leq n$. Naturally $P^k=0$ for $k>0$.

    We also have several morphisms induced by $L$, which is illustrated
    in the following graph adapted from the book:

    \begin{figure}[H]
        \centering
        \includegraphics[width=0.9\linewidth]{pics/{prop-1.2.30-pp.36}.pdf}
        \caption{Morphisms}
    \end{figure}
\end{prop}
(\textbf{pp.36 of \cite{book}})

As shown in the theorem, the map $\Lambda^{n-k}$ is produce a mirror
effect in $\bigwedge^*V^*$, very similar to the Hodge $*$.
The next proposition relates the two:

\begin{prop}
    For all $\alpha\in P^k$, we have:
    \begin{align}
        *L^j \alpha = (-1)^{\frac{k(k+2)}{2}}
            \frac{j!}{(n-k-j)!}\cdot L^{n-k-j}I(\alpha).
    \end{align}
\end{prop}
Particularly, when $j=k=0$, we have $*1=\text{vol} = \frac{\omega^n}{n!}$,
or,
\begin{align}
    n!\text{vol} = \omega^n
\end{align}
(\textbf{pp.37 of \cite{book}})

\begin{coro}[Hodge—Riemann bilinear relation]
    \begin{align}
        Q(\bigwedge^{p,q}V^*, \bigwedge^{p',q'}V^*)=0
    \end{align}
    for $(p,q)\neq (p',q')$, and
    \begin{align}
        i^{p-q}Q(\alpha,\bar\alpha)=(n-(p+q))! \cdot
        \langle\alpha,\alpha\rangle_{\mathbb{C}}>0
    \end{align}
    for $0\neq\alpha\in P^{p,q}$, with $p+q\leq n$.
\end{coro}
(\textbf{pp.39 of \cite{book}})


\part{Misc}

\chapter{Anchor}
\label{sec:Anchor}

\begin{thebibliography}{1}
	\bibitem{book} Complex Geometry
\end{thebibliography}

\chapter{License}
\label{sec:License}

The entire content of this work (including the source code
for TeX files and the generated PDF documents) by 
Hongxiang Chen (nicknamed we.taper, or just Taper) is
licensed under a 
\href{http://creativecommons.org/licenses/by-nc-sa/4.0/}{Creative 
Commons Attribution-NonCommercial-ShareAlike 4.0 International 
License}. Permissions beyond the scope of this 
license may be available at \url{mailto:we.taper[at]gmail[dot]com}.
\end{document}

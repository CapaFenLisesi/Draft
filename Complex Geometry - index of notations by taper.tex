\documentclass{article}

\usepackage{float}  % For H in figures
\usepackage{amsmath} % For math
\usepackage{amssymb}
\usepackage{mathrsfs}
\numberwithin{equation}{subsection} % have the enumeration go to the subsection level.
									% See:https://en.wikibooks.org/wiki/LaTeX/Advanced_Mathematics
\usepackage{graphicx}   % need for figures
\usepackage{cite} % need for bibligraphy.
\usepackage{hyperref}  % make every cite a link
\usepackage{CJKutf8} % For Chinese characters
\usepackage{fancyref} % For easy adding figure,equation etc in reference. Use \fref or \Fref instead of \ref
\usepackage{braket} %http://tex.stackexchange.com/questions/214728/braket-notation-in-latex

% Following is for theorems etc environments
% http://tex.stackexchange.com/questions/45817/theorem-definition-lemma-problem-numbering && https://en.wikibooks.org/wiki/LaTeX/Theorems
\usepackage{amsthm}
\newtheorem{thm}{Theorem}
\newtheorem{lemma}{Lemma}

\usepackage{xcolor} %For colourful math:http://tex.stackexchange.com/questions/21598/how-to-color-math-symbols

\title{Complex Geometry - Index of Notations and ideas}
\date{\today}
\author{Taper}


\begin{document}


\maketitle
\tableofcontents

\begin{abstract}
Obviously. 

\end{abstract}

\section{Book}
	\subsection{1. Local Theory}
		\subsubsection{1.1 Holomorphic Functions of Several Variables}
		\label{sec:1.1_book}
		\textbf{Note}: the content covered by this seciton is geared for accompanying my personal notes of lecture 1.
		
		\textcolor{blue}{holomorphic}: pp.1. pp4. Def 1.1.1. pp.10(Def.1.1.8).
		
		\textcolor{blue}{Cauchy-Riemann equations}: pp.2
		
		\textcolor{blue}{$\frac{\partial}{\partial z}$,$\frac{\partial}{\partial \bar{z}}$}:
		$\frac{\partial}{\partial z}:= \frac{1}{2}
				(\frac{\partial}{\partial x} 
				-i \frac{\partial}{\partial y})
				$
		,
		$\frac{\partial}{\partial \bar{z}}:= \frac{1}{2}
		(\frac{\partial}{\partial x} 
		+ i \frac{\partial}{\partial y})
		$
		
		
		\textcolor{blue}{Maximum principle}: pp.3.
		
		\textcolor{blue}{Identity theorem}: pp.3.
		
		\textcolor{blue}{Riemann extension theorem}: pp.3. pp.9 (Prop. 1.1.7).
		
		\textcolor{blue}{Riemann mapping theorem}: pp.3.
		
		\textcolor{blue}{Liouville theorem}: pp.4.
		
		\textcolor{blue}{Residue theorem}: pp.4.
		
		\textcolor{blue}{polydiscs $B_\epsilon(\omega)$}: $\{z| |z_i-\omega_i| <\epsilon\} $.  pp.4.
		
		\textcolor{blue}{Hartogs' theorem}: Prop. 1.1.4. pp.6.
		
		\textcolor{blue}{Weierstrass preparation theorem (WPT)}:
		Prop. 1.1.6. pp.8.
		
		\textcolor{blue}{Weierstrass polynomial}: Def. 1.1.5. pp.7.
		
		\textcolor{blue}{$Z(f)$}: zero set of $f$. pp.9.
		
		\textcolor{blue}{biholomorphic}: pp.10.
		
		\textcolor{blue}{(complex) Jacobian, regular, regular value}: Def. 1.1.9. pp.10.
		
		\textcolor{blue}{IFT. Inverse function theorem}: Prop 1.1.10 pp.11.
		
		\textcolor{blue}{IFT. Implicit function theorem}: Prop 1.1.11. pp.10.
		
		\textcolor{blue}{$\mathcal{O}_{\mathbb{C}^n}$}:
		sheaf of holomorphic functions on $\mathbb{C}^n$. Def. 1.1.14. pp.14.
		
		\textcolor{blue}{$\mathcal{O}_{\mathbb{C}^n,z}$}: Def. 1.1.14. pp.14.
		
		\textcolor{blue}{$\mathcal{O}^*_{\mathbb{C}^n,0}$}:
		units of $\mathcal{O}_{\mathbb{C}^n,0}$. pp.14.
		
		\textcolor{blue}{UFD, unique factorization domain, irreducible}: 
		Def. 1.1.16. pp.14.
		
		\textcolor{blue}{Gauss Lemma}: pp.14.
		
		\textcolor{blue}{Weierstrass division theorem}: Prop. 1.1.17. pp.15.
		
		\textcolor{blue}{1}: 
		\textcolor{blue}{1}:
		\textcolor{blue}{1}:
		\textcolor{blue}{1}:
		\textcolor{blue}{1}: 
		\textcolor{blue}{1}:
		\textcolor{blue}{1}:
		\textcolor{blue}{1}:

	\subsection{2.Complex Manifolds}

		\subsubsection{2.2 Holomorphic Vector Bundles}

		\textcolor{blue}{$\tau_X$}: holomorphic tangent bundle of a complex manifold $X$ (Def 2.2.14 at pp. 71).
		
		\textcolor{blue}{$\varOmega_X$, $\varOmega^p_X$}: holomorphic cotangent bundle and holomorphic $p$-forms. (Def 2.2.14 at pp. 71)
		
		\textcolor{blue}{$K_X$}:=det($\varOmega_X$) = $\varOmega_X^n$, the canonical bundle of $X$. (Def 2.2.14 at pp. 71)

		\subsubsection{2.6 Differential Calculus on Complex Manifolds}
	
		\textcolor{blue}{$\wedge^k_{\mathbb{C}}X$}:=$\wedge^k(T_{\mathbb{C}}X)^*$. (Def 2.6.7 at pp. 105)
		
		\textcolor{blue}{$\wedge^{p,q}X$}:=$\wedge^p(T^{1,0}X)^*
							\bigotimes_{\mathbb{C}}\wedge^q(T^{0,1}X)^*$. (Def 2.6.7 at pp. 105)
	
		\textcolor{blue}{$\mathcal{A}^k_{X,\mathbb{C}}$,$\mathcal{A}^{p,q}_X$}: sheaves of section of the above correspond items. (Def 2.6.7 at pp. 105)
		
		\textcolor{blue}{$\mathcal{A}^{p,q}(E)$}: the sheaf of $p,q$-forms with values in $E$, a complex vector bundle. (Def 2.6.22 at pp.109). Note that in particular, $\mathcal{A}^0(E)$ is the sheaf of sections of $E$.

	\subsection{Appendix B: Sheaf Cohomology}
	
	\begin{itemize}
		\item
		\textcolor{blue}{pre-sheaf}: Def B.0.19, pp. 287.
		
		\item
		\textcolor{blue}{$\mathcal{C^0_M}$}: the pre-sheaf of continuous functions on $M$. Example B.0.20, pp. 287.
		
		\item
		\textcolor{blue}{sheaf}: Def B.0.21, at pp.288.
		
		\item
		\textcolor{blue}{\underline{$\mathbb{R}$},\underline{$\mathbb{Z}$}}: constant sheaves, Sometimes written simply as \textcolor{blue}{$\mathbb{R}$}, \textcolor{blue}{$\mathbb{Z}$} respectively. pp. 288.
		\item
		\textcolor{blue}{$\mathcal{E}$}: actually a $\mathcal{C}^0_M$-modules. Sometimes identified as $E$. pp.288.
		\item
		\textcolor{blue}{(pre)-sheaf homomorphism}: Def B.0.23.  pp.288.
		\item
		\textcolor{blue}{Ker($\phi$),Im($\phi$),Coker($\phi$)}: as pre-sheaves in pp.288. sheaves in pp.289, Def B.0.26.
		\item
		\textcolor{blue}{injective, surjective of sheaf-homomorphism}:pp.289.
		\item
		\textcolor{blue}{complex, exact complex}: Def B.0.27. pp.289
		\item
		\textcolor{blue}{text}:
		\item
		\textcolor{blue}{text}:
		\item
		\textcolor{blue}{text}:
		\item
		\textcolor{blue}{text}:
		\item
		\textcolor{blue}{text}:
		\item
		\textcolor{blue}{text}:
		\item
		\textcolor{blue}{text}:
		\item
		\textcolor{blue}{text}:
		\item
		\textcolor{blue}{text}:
		\item
		\textcolor{blue}{text}:
		\item
		\textcolor{blue}{text}:
		\item
		\textcolor{blue}{text}:
		\item
		\textcolor{blue}{text}:
		\item
		\textcolor{blue}{text}:
		\item
		\textcolor{blue}{text}:
		\item
		\textcolor{blue}{text}:
		\item
		\textcolor{blue}{text}:	
	\end{itemize}


\section{My lecture Notes}

	\subsection{Lecture 2016 Lecture 1}
	The first few lectures are not well noted, hence I delegate the task of recording the theorems and notations to the book's correspoding section:\fref{sec:1.1_book}.
	
	\subsection{Lecture 4 (20160307) Complex Manifold}
	
	\textbf{Note}: we use abbrevation \textit{mnfd} for \textit{manifold}.
	
	pp. A:
		\begin{itemize}
			\item \textcolor{blue}{Holomorphic Atlas}
			\item \textcolor{blue}{Holmorphic chart}
			\item \textcolor{blue}{Complex mnfd}
		\end{itemize}
	
	pp. B:
		\begin{itemize}
			\item \textcolor{blue}{Holomorphic function}
			\item \textcolor{blue}{$\mathcal{O}_X$}: sheaf of holomorhic functions on a complex mnfd $X$.
		\end{itemize}
	pp. C:
		\begin{itemize}
			\item \textcolor{blue}{Hartdogs' theorem}: on complex mnfd.
			\item \textcolor{blue}{Holomorphic functions on complex mnfd}:
		\end{itemize}
	pp. D:
		\begin{itemize}
			\item \textcolor{blue}{Complex Lie group}
			\item \textcolor{blue}{Complex Projective Space, $\mathbb{CP}^n$, or just $\mathbb{P}^n$}.
		\end{itemize}
	pp. E:
		\begin{itemize}
			\item \textcolor{blue}{Topology in $\mathbb{P}^n$}
			\item \textcolor{blue}{Mnfd structure on $\mathbb{P}^n$, atlas, and the \textbf{canonical covering}}
		\end{itemize}
	pp. F
		\begin{itemize}
			\item Grassmannian mnfd.
		\end{itemize}
	
	\subsection{Lecture 5 Submanifolds (20160308)}
	
	pp. A:
		\begin{itemize}
			\item Affine Hypersurface (actually this is not quite different from the usual $\mathbb{C}^n$.)
		\end{itemize}
	
	Part 2. \textbf{Sheaf Theory}
	
	pp. A:
		\begin{itemize}
			\item pre-sheaf
			\item $\mathcal{O}_X(U)$
			\item $\mathcal{O}^*_X(U)$
		\end{itemize}
	pp. B:
		\begin{itemize}
			\item $C^{\infty}$
			\item $\mathbb{\underline{Z}}$, sometimes simply denoted as $\mathbb{Z}$: sheaf of localy constant $\mathbb{Z}$-valued functions.
			\item Sheaf
		\end{itemize}
	pp. D:
		\begin{itemize}
			\item sheaf-morphisms
		\end{itemize}
	pp. E:
		\begin{itemize}
			\item Section
			\item Ker($\phi$) - sheaf of kernals.
		\end{itemize}
	pp. F:
		\begin{itemize}
			\item Im($\phi$) is a presheaf, but not a sheaf.
			\item Im($\phi$): the sheafification of Im($\phi$) above. Note that we use the same notation to denote both.
		\end{itemize}
	
	\subsection{Lecture 6 Sheaf \& Cohomology (20160315)}
	pp. A:
		\begin{itemize}
			\item Stalk $\mathcal{F}_x$.
			\item germ
			\item Directed partial order set
			\item Directed System
		\end{itemize}
	pp. B:
		\begin{itemize}
			\item Directed limit
		\end{itemize}
	pp. C:
	\begin{itemize}
		\item Exact Complex/ Exact Sequence.
		\item Exponential sequence \textit{(mentioned under the definition of exact sequence)}.
		\item
	\end{itemize}
	pp. D:
	\begin{itemize}
		\item Čech cohomology
	\end{itemize}
	pp. E:
	\begin{itemize}
		\item q-cochain
		\item coboundary operator. $\delta$.
		\item $Z^p(U,\mathcal{F})$ = Ker.
		\item $B^p(U,\mathcal{F})$ = Im.
		\item $\check{H}^p(U,\mathcal{F})$ = $\frac{Ker}{Im}$.
	\end{itemize}
		\subsubsection{Notes of Čech Cohomology with Coeficients in a Sheaf}
		pp.1:
		\begin{itemize}
			\item q-simplex $\sigma$.
			\item support $|\sigma|$.
			\item q-cocain
			\item $C^q(U,\mathcal{F})$
			\item Coboundary Operator $\delta$.
		\end{itemize}
		pp.2,3,4:
		\begin{itemize}
			\item Cochain Complex
			\item Čech cohomology
			\item cocycle
			\item cochain
			\item $\check{H}^p(U,\mathcal{F}),Z^p(U,\mathcal{F}),B^p(U,\mathbb{F})$.
			\item $\check{H}^0(\{u_i\},\mathcal{F})$ = $\mathcal{F}(X)$.
		\end{itemize}
	
	\subsection{Lecture 7 Vector Bundle (20160321)}
	pp.1,2:
	\begin{itemize}
		\item Vector Bundle
		\item Trivializing covering, $\{(U_i,\tau_i)\}$.
		\item trivializing maps, trivializes.
		\item VB-equivalent of trivializing maps.
		\item E: total space, X: base space.
	\end{itemize}
	pp. 3,5:
	\begin{itemize}
		\item transition maps.
		\item fibre.
		\item $\mathcal{O}(-1)$
		\item cocycle condition.
		\item $\mathcal{T}_X$, Holomorphic tangent bundle.
	\end{itemize}
	pp. 8:
	\begin{itemize}
		\item s: section of a holomorphic vector bundle.
		\item $\mathcal{E}$: sheaf of sections of holomorphic vector bundle. $\mathcal{E}(U)$.
	\end{itemize}
	
	\subsection{Lecture 8 Almost Complex Structures (20160322)}
	pp. 1,2:
	\begin{itemize}
		\item $I$: Almost Complex Structure. $I^2=-1$. Sometime $J$ is used in place of $I$.
		\item $V_{\mathbb{C}}$ := $V\otimes \mathbb{C}$.
		\item $I_\mathbb{C}$: $I$ extending to $V_{\mathbb{C}}$. Usually abbreviated simply as $I$.
		\item $V^{1,0}$:= ker$(I+i)$.
		\item $V^{0,1}$:= ker$(I-i)$.
	\end{itemize}
	
	\subsection{Lecture 9 Exterior Algebra on Complex Manifold (20160329)}
	
	pp.1,2:
	\begin{itemize}
		\item $V^*$: dual of $V$.
		\item $\{dx^i,dy^i\}$.
		\item $J^*$: $J$ extending to dual space.
		\item $dz^i,d\bar{z}^i$.
	\end{itemize}
	pp. 3:
	\begin{itemize}
		\item $S^k(V)$, $\Lambda^k(V)$.
		\item $s$ and $a$, symmetrization and anti-symmetrization of a tensor.
		\item $\Lambda^* V$.
	\end{itemize}
	pp. 4:
	\begin{itemize}
		\item $\Lambda^n T^*_{\mathcal{C}}X$.
		\item $\Lambda^* T^*_{\mathcal{C}}X$.
		\item $\Lambda^{p,q} T^*_{\mathcal{C}}X$.
	\end{itemize}
	pp. 5,6:
	\begin{itemize}
		\item $\mathcal{A}$: sheaf of section of cotangent bundle.
		\item $\mathcal{A}^n(U)$, $\mathcal{A}^{p,q}(U)$.
		\item $\Lambda$ on $\mathcal{A}$.
		\item $d$: de Rham differential.
		\item $\partial, \bar{\partial}$.
	\end{itemize}
	
	\subsection{Lecture 10 Debeault Cohomology (20160406)}
	
	pp. 1:
	\begin{itemize}
		\item $\mathcal{H}^{p,q}(X)$.
		\item $f^*$: pull-back. Various defintion from pp.1 to pp.4.
	\end{itemize}
	pp. 5,6,7:
	\begin{itemize}
		\item $\mathcal{A}^{p,q}(U,E):=\Gamma(U,\Lambda^{p,q}T_{\mathbb{C}}^* X \otimes E)$.
		\item $\bar{\partial}_E$
		\item $\mathcal{H}^{p,q}(X,E)$.
		\item $\bar{\partial}$-Poincaré lemma in one variable.
	\end{itemize}
	
	\subsection{Lecture 11 (20160412)}
	pp.1,2,3: 
	\begin{itemize}
		\item $\bar{\partial}$-Poincaré lemma in n-dimension
		\item $\Omega_X^p$: holomorphic p-forms. On pp.2.
		\item $\check{H}^q(X,\Omega^p)\text{(Čech)}\cong \mathcal{H}_{\bar{\partial}}^{p,q}(X)$(Dolbeault). On pp.3.
	\end{itemize}
	pp. 6,7:
	\begin{itemize}
		\item Analytic Subvarity.
		\item Analytic Hybersurface.
		\item Cousin's Problem.
	\end{itemize}
	
	\subsection{Lecture 12 Hermitan Structure on Manifold  Manifold (20160418)}
	pp. 1,2,3:
	\begin{itemize}
		\item $I$ compatible with $<-,->$.
		\item $\omega$: Fundamental form associated with $<,>$ and $I$. $\omega(v,w):= <I(v),w>$.
		\item Conformal Equivalence.
		\item $<,>$: Hermitian Inner Product.
	\end{itemize}
	pp. 4:
	\begin{itemize}
		\item $( , )$: s.t. $(v,w):=<v,w>-i\omega(v,w) = <v,w> - i <I(v),w>$
	\end{itemize}
	pp. 5:
	\begin{itemize}
		\item $<,>_{\mathbb{C}}$ be s.t.$<v\otimes \alpha, w\otimes \beta>:= \alpha \bar{\beta} <v,w>$.
	\end{itemize}
	pp. 6:
	\begin{itemize}
		\item $\frac{1}{2}(,) = <,>_{\mathbb{C}} \arrowvert_{V^{1,0}}$
	\end{itemize}
	pp. 7,8:
	\begin{itemize}
		\item Local computations: $z_i$,$h_{ij}$,
		\item $\omega = (...dx^i...dy^i)$
		\item $\omega$, Fundamental form on Riemannian Mnfd.
		\item Kähler mnfd: $d\omega \equiv 0$.
	\end{itemize}
	
	\subsection{Lecture 13 Kähler Manifold (20160419)}
	pp.1:
	\begin{itemize}
		\item Local computation: $\omega = (...dz^i...d\bar{z}^i)$
	\end{itemize}
	pp.4:
	\begin{itemize}
		\item Fubini-Study Metric on $\mathbb{CP}^n$.
	\end{itemize}
	
	\subsection{Lecture 14 Hodge Theory (20160425)}
	pp.1:
	\begin{itemize}
		\item $<,>$ on $\Lambda^k V$
		\item vol: volumn element.
		\item $*$: Hodge Star Operator.
	\end{itemize}
	pp.4:
	\begin{itemize}
		\item $L$: Lefschetz Operator
		\item $\Lambda$: adjoint of $L$. $\Lambda = *^{-1}\circ L\circ *$.
	\end{itemize}
	pp.5 :
	\begin{itemize}
		\item $*,L,\Lambda$ on Kähler mnfd.
		\item $d^*:= (-1)^{m*(k+1)+1}*\circ d\circ*$, adjoint of $d$. On a Kähler mnfd, $d^*= -*\circ d\circ*$
		\item $\Delta:=d^*\circ d + d\circ d^*$.
	\end{itemize}
	pp. 6:
	\begin{itemize}
		\item $\bar{\partial}^*, \partial^*$: Similar to the above for $d$.
		\item $\Delta_\partial, \Delta_{\bar{\partial}}$: Similar to the above for $d$.
	\end{itemize}
	
	\subsection{Lecture 15 Hodge Theory on Manifold (20160426)}
	pp.1:
	\begin{itemize}
		\item $(,)$ on $\mathcal{A}^*(X)$. $(\alpha,\beta):=\int_X g_\mathbb{C}(\alpha,\beta) vol$
	\end{itemize}
	pp.3:
	\begin{itemize}
		\item $\mathcal{H}^k(X,g)$: d-harmonic forms. Sometimes we replace $\mathcal{H}$ with $\mathscr{H}$ for harmonic forms, so is for symbols below.
		\item $\mathcal{H}^k_{\bar{\partial}}(X,g)$: $\bar{\partial}$-harmonic forms. 
		(Be careful to distinguish this with Dolbeault Cohomology groups).
		\item $\mathcal{H}^k_{\partial}(X,g)$: $\partial$-harmonic forms.
	\end{itemize}
	pp. 5:
	\begin{itemize}
		\item $\mathcal{H}^k_d(X,g) \cong \mathcal{H}^{2n-k}_d(X,g)$, Poincar\'{e} duality
		\item $\mathcal{H}^{p,q}_{\bar{\partial}}(X,g) \cong 
		\left( \mathcal{H}^{n-p,n-q}_{\bar{\partial}}(X,g)\right)^*$, both are harmonic forms, called Serre Duality. 
	\end{itemize}
	pp. 6,7:
	\begin{itemize}
		\item 
		$\mathcal{A}^{p,q} = 
		\bar{\partial}\mathcal{A}^{p,q-1}(X)\oplus
		\bar{\partial}^* \mathcal{A}^{p,q+1}(X) \oplus
		\mathcal{H}^{p,q}_{\bar{\partial}}(X,g)$:
		
		 Hodge decomposition
	
		\item $\mathcal{H}^{p,q}_{\bar{\partial}}
			\text{(harmonic forms)}
			\cong
			\mathcal{H}^{p,q}_{\bar{\partial}}(X)
			\text{(Dolbeault Cohomology group)}$
		\item $\mathcal{H}^{p,q}_{d}
			\text{(harmonic forms)}
			\cong
			\mathcal{H}^{p,q}_{dR}(X)
			\text{(de Rham Cohomology group)}$
	\end{itemize}
	pp. 8:
	\begin{itemize}
		\item A lot of isomorphisms between \textit{de Rham}, \textit{Dolbeault} and \textit{harmonic forms}.
	\end{itemize}
	
	\subsection{Lecture 16 Harmonic forms on K\"{a}hler Manifold (20160503)}
	
	pp.1:
	\begin{itemize}
		\item $\Delta_\partial = \Delta_{\bar{\partial}} = 
			\frac{1}{2}\Delta_d$, for K\"{a}hler mnfd.
	\end{itemize}
	
	\subsection{Lecture 17 Hermitian Vector Bundle (20160510)}
	pp.1,3:
	\begin{itemize}
		\item Hermitian Vector Bundle. pp.1
		\item Antilinear map. pp.3.
		\item Hermitian Inner Product on $\mathcal{A}^{p,q}(X,E)$. pp.4
		\item $\bar{*}_E$ Hodge Operator on Hermitian vector bundle. pp.5.
		\item $\bar{\partial}_E^*$
	\end{itemize}
	pp. 8:
	\begin{itemize}
		\item Kadaira-Serre Duality.
	\end{itemize}
	
	\subsection{Lecture 18 Connection (20160516)}
	\begin{itemize}
		\item $\bigtriangledown$: connection. pp.1.
		\item Trivial connections. pp.2
		\item $\mathcal{A}^1(M,End(E)):=\Gamma(M,\Lambda^1 M\otimes End(E))$. pp.3. Also, one may find how elements in this sheaf act on $\mathcal{A}^0(M)$ on pp.173, inside proof of proposition 4.2.3.
		\item $s\in \mathcal{A}^0(E)$ is Parrallel/flat/constant $\Leftrightarrow \Delta(s) = 0$. pp.4.
		\item $\Delta = d+A$. pp.4.
		\item $\Delta$ be compatible with hermitian structure on $E$. pp.5.
		\item $\Delta$ be compatible with holomorphic vector bundle. pp.6.
		\item $A=\bar{H}^{-1} \partial H$. Chern connection. pp.6.
	\end{itemize}
	
	\subsection{Lecture 19 Holomorphic Connection \& Curvature(20160517)}
	\begin{itemize}
		\item Holomorphic Connecction. pp.1.
		\item $At(E)$: Atiyah class of $E$. pp.2.
		\item $\Delta^k$. pp.4.
		\item $F_\Delta$: curvature associated with $\Delta$. pp.5.
		\item $F_\Delta = dA + A\wedge A$: Cartan structure equation. pp.6.
		\item First Chern class of complex line bundle.
	\end{itemize}
	
	\subsection{Lecture 20 Divisors \& (Holomorphic) Line Bundles (20160524)}
	\begin{itemize}
		\item Analytic Subvariety. pp.1.
		\item Regular/Smooth Point. pp.1.
		\item Singular Point. pp.2.
		\item Irreducible analytic subvariety. pp.2.
		\item $dim(Y)$: dimension of analytic subvariety. pp.2. Also pp.4.
		\item Affine algebraic varieties. pp.3.
		\item Projective algebraic varieties. pp.3.
		\item Hypersurface. pp.4.
		\item Divisor, $Div(X)$:=group of all divisors. pp.5.
		\item Effective divisor. pp.6.
		\item $Ord_Y(f)$: order of function. pp.6. Also pp.8.
		\item Meromorphic function on complex mnfd.
		\item $(f)$: divisor given by a global meromorphic function.
		\item Principal divisor. pp.8.
	\end{itemize}
	
	\subsection{Lecture 21 Divisors \& (Holomorphic) Line Bundles  (20160530)}
	\begin{itemize}
		\item $H^0(X,K^*_X/\mathcal{O}^*_X) \cong Div(X)$. pp.1.
		\item $Pic(X)$: Picard group, all holomorphic line bundles. pp.3.
		\item $Pic(X) \cong \check{H}^1(X,\mathcal{O}^*_X)$. pp.3.
		\item $\mathcal{O}(D)$: line bundle given by divisor $D$. pp.5.
		\item Linear equivalent of divisors.
		\item $*$: used only in this section to denoted the map:
		$$\left( Div(X)/Pic(X) \right) \hookrightarrow Pic(X)$$
		pp.6.
		\item $Z(s)$: divisor constructed from nonzero section $s\in H^0(X,L)$ for a line bundle $L$.
	\end{itemize}
	
	\subsection{Lecture 22 Divisors \& (Holomorphic) Line Bundles (20160606)}
	\begin{itemize}
		\item Base point of a line bundle. pp.4.
		\item $Bs(L)$:= set of all base points of line bundle $L$. pp.4.
		\item $\mathcal{O}(1), \mathcal{O}(k)$. pp.6.
	\end{itemize}

\begin{thebibliography}{1}
	
	
	\bibitem{blog} Complex Geometry
	
	
\end{thebibliography}
\end{document}

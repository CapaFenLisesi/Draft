\section{Classification by Dirac Mass Terms}
\label{sec:The Minimal Hamiltonian Approach}

\paragraph{Dirac Hamiltonian and Topological Property of the Boundary}
We first note that different topological phases are distinguished by the closing
and then opening of the gap, i.e. a quantum phase transition. If we adiabatically
flatten the energy spectrum to two bands of energy $\pm E$, the transition
(closing of gap) will happen at some point where locally the energy is of a
Dirac cone type. Therefore, the Dirac Hamiltonian captures this transition
behavior. Then, a class of topological phases, or more specifically, a class of
Hamiltonian matrix that shares the same topological invariant, can be
\textit{represented} by the Dirac Hamiltonian in this class.

\subsection{Crossing Spectrum and Edge Mode}
\label{sec:Crossing Spectrum and Edge Mode}
Here we show that the crossing spectrum characteristic of topological insulators
can be characterized by a Dirac type Hamiltonian, and with this Dirac
Hamiltonian we could find an Edge mode solution built into it.

\paragraph{Crossing Spectrum and Dirac Hamiltonian}
In some local region around the crossing point, the energy spectrum should look
like:
\begin{equation}
    E = \pm \sqrt{\vb{k}^2 + M^2}
\end{equation}
where $\vb{k}$ is the crystal momentum and $M$ is some constant
responsible for the opening and closing of the gap, or in some cases, a parameter
in which we can control to open or close the gap.

As has been analysed in most Quantum Mechanics textbook\footnote{See for
    example, p.99 of \cite{Greiner1997}. Although their discussion is in
dimension $d=3$, it is easy to generalize it to any dimension.}, this energy-momentum
relationship requires a Dirac type Hamiltonian. Then, the Dirac Hamiltonian we
use is:
\begin{equation}
    \label{eq:dirac-H}
    H_\text{Dirac} = M\gamma_0 + \sum_{i=1}^d k_i\gamma_i
\end{equation}
where $d$ is the spatial dimension of the physical system, $\gamma_i$ obeys the
Clifford Algebra anti-commutation relation:
\begin{equation}
    \{\gamma_i,\gamma_j\} = 2\delta_{i,j}\id
\end{equation}

\paragraph{Edge Mode and \texorpdfstring{$M$}{}}
Now we explain why the Dirac Hamiltonian eq.~\ref{eq:dirac-H}  is useful for capturing the topological
phase. First, it has a gapless state. Due to the anti-commutation relation of
$\gamma_i$, we can show\footnote{Note that $H_\text{Dirac}^2= E^2\id$, this
tells us it can only have eigenvalues $E_\pm$. Note that this also explains why
we choose not $\gamma_i^2=-1$, which would give non-sensible result
$H_\text{Dirac}^2= -E^2\id$.}
that the energies of Dirac Hamiltonian are:
\begin{equation}
    E_\pm = \pm\sqrt{M^2+\sum_{i=1}^d k^2_i}
\end{equation}
Therefore, as $M$ varies from negative to $0$ to positive, the system goes from
gaped to gapless and again to gaped state. This shows that the $M$ is parameter
controlling the quantum phase transition, and the system with $M<0$ or $M>0$
possesses two different quantum phase.

In addition, we can analyse its edge mode. Assuming the two material of
different quantum phases touches each other in the $d$th dimension. Then we
should replace $k_d$ by $-i\partial_{x_d}$, since the translational invariance is
broken in that direction. Now the eigenvalue equation is:
\begin{equation}
    ( M\gamma_0 + \sum_{i=1}^{d-1} k_i\gamma_i - i\partial_{x_d}\gamma_d ) \Phi
    = E \Phi
\end{equation}

Past experience teaches us the ansatz (assuming
$M(x_d)$ goes to $\pm\infty$ as $x_d\to\pm\infty$, so that the leading term below
does not blow up the wavefunction):
\begin{equation}
    \Phi = e^{-\int_0^{x_d} M(x'_d)\dd{x'_d}} \phi(x_1,\cdots,x_{d-1})
\end{equation}
which after plugging inside the eigenvalue equation, leads to
\begin{equation}
    \left( M(x_d)(\gamma_0 + i\gamma_d) + \sum_{i=1}^{d-1} k_i\gamma_i \right) \phi
    = E \phi
\end{equation}
After left multiplication of $\gamma_0$, we have:
\begin{equation}
    \left( M(\id + i\gamma_0\gamma_d) + \sum_{i=1}^{d-1} k_i\gamma_i  \right) \phi
    = E \gamma_0 \phi
\end{equation}
Since $i\gamma_0\gamma_d$ squares to $1$, it has eigenvalues $\pm1$. Also,
$i\gamma_0\gamma_d$ and $\gamma_0$ anticommute, so they share the same
eigenspaces with $\gamma_0$ mapping all the $+1$ eigenvectors of
$i\gamma_0\gamma_d$ to the $-1$ eigenvectors and vice versa. Now we want this
state to be a surface state, so we choose the $-1$ eigenvectors (denoted
$\phi_-$) of $i\gamma_0\gamma_d$ to kill the bulk term. This leads to:
\begin{equation}
    \gamma_0 \sum_{i=1}^{d-1} k_i\gamma_i \phi_- = \gamma_0 E\phi_-
\end{equation}
As mentioned, $\gamma_0$ switches the two eigenspaces, so we must necessarily
have a surface Dirac Hamiltonian, by projecting all those matrices to the $-1$
eigenspaces:
\begin{equation}
    \label{eq:H-Dirac-surf}
    H_\mathrm{surf} \phi_- = \sum_{i=1}^{d-1} \Gamma_{i-} \phi_- = E \phi_-
\end{equation}

% \section{Collection of Calculation}
% \label{sec:Collection of Calculation}
% Here collects the calculations done throughout the whole notes.

\subsection{From Homotopy Classification to Minimal Dirac Hamiltonian Method}
\label{sec:Homotopy to Minimal Dirac}

\paragraph{Spectral Flattening} The classification of topological insulators
concerns different classes of Hamiltonian that cannot be adiabatically changed
to each without closing or opening of a gap. Hence, we could in general
adiabatically change our parameters such that the spectrum of Hamiltonian is
simplified into two bands of energy $\pm1$. Under this condition, we have
$H^2=1$, which already hints that the Hamiltonian itself is a viable candidate
for generating Clifford Algebras. 

\paragraph{Extension Problem with General Examples}
The general theme of homotopy classification of classifying topological
insulators are done by the considering the \textit{extension problem}, i.e.
different ways to extend from a Clifford containing only symmetry operators, to
a Clifford Algebra containing Hamiltonians. For example, for class $\mathrm{A}$
in $0$-dimension, the empty space is nothing, hence
$\mathrm{C\ell}_{0}(\C)$\footnote{The reason for using different Clifford
Algebra ($\mathrm{C\ell}(\R)$ or $\mathrm{C\ell}(\C)$ for different symmetry
classes will be explained in section~\ref{sec:Isomorphism}.}. And $H^2=1$ as
mentioned earlier, hence the extension problem is from
$\mathrm{C\ell}_{0}(\C)\to \mathrm{C\ell}_{1}(\C)$, whose possible
representations form some mathematical structure called \textit{classifying
space} $C_0$ \cite{Morimoto2013}. And since we are in $0$-dimension, all
operators are maps starting from $T^0=S^0$. So our classified objects are
different classes of maps from $S^0\to C_0$, which is mathematically captured by
the $0$-th homotopy group $\pi_0(C_0)=\Z$. For class $\mathrm{AIII}$ which has
only chiral symmetry operator $S$, we choose a phase such that $S^2=1$, i.e.
making it a candidate for generators of Clifford Algebra. The symmetry condition
$\{H,S\}=0$ tells us now we have a extension problem of
$\mathrm{C\ell}_{1}(\C)\to \mathrm{C\ell}_{2}(\C)$, whose classifying space is
$C_1$, and has $\pi_0(C_1)=0$. Therefore, in $0$-dimension, the class
$\mathrm{A}$ has $\Z$ topological insulators and class $\mathrm{AIII}$ has
trivial topological insulators. The case for real symmetry classes are not too
complicated and is concisely mentioned in \cite{Morimoto2013}.

However, the method to obtain such classifying spaces are mathematical daunting
and requires $K$-theory. We try to simplify it by looking at Dirac Hamiltonians,
since they should capture the essence of topological states.

\paragraph{Extension Problem with Dirac Hamiltonian}
The Dirac Hamiltonian
\begin{equation}
    H = \vec{k}\vdot \vec{\gamma} + m\tilde{\gamma}_0
\end{equation}
has a gap closing and opening mass term $m\tilde{\gamma}_0(r)$, depending on some
parameter $r$. Suppose that we have a domain wall squeezed by two bulk regions
$A$ and $B$. Now as the parameter $r$ changes freely from region $A$ to region
$B$, the symmetry condition will force the matrix $\gamma_0$ to explorer some
space having the same homotopy type of some classifying space (again!). For
example, for class $\mathrm{A}$ in $2$-dimension, the Hamiltonian without mass
term ($k_1\gamma_1 + k_2\gamma_2$) consists of two gamma matrices, generating a
$\mathrm{C\ell}_{2}(\C)$, whereas adding the mass term, we have
$\mathrm{C\ell}_{3}(\C)$. Since there is no symmetry in class $\mathrm{A}$, the
extension problem concerns different ways to extend the algebra
$\mathrm{C\ell}_{2}(\C)\to \mathrm{C\ell}_{3}(\C)$, which tells us the number of
unitarily non-equivalent mass terms. Detailed examples can be found in section
III.C.1 of \cite{Chiu2016}.













\paragraph{The View from SPEMT} Let us consider a modified Dirac Hamiltonian
with extra mass term:
\begin{equation}
    \label{eq:H-spemt}
    H = M\tilde{\gamma}_0 + \sum_{i=1}^{d} k_i \gamma_i 
    + \sum_{j=1}^D m_j\tilde{\gamma}_j
\end{equation}
Here, the parameter $M$ characterized the domain wall between different phases
of topological insulators. The extra mass term $m_j$ represents a perturbation
caused by disorder. I argue that we can do our classification in the following
way. First, we consider the case when $D=0$, i.e. without perturbation, and we
consider how many different mass terms we could have. Second, we consider whether we can add
an extra mass term which respect the symmetry of the Hamiltonian ($D=1$) or not.
Tthis symmetry preserving extra mass term is denoted as SPEMT. If
a SPEMT can be added, then the system is not robust against perturbation and gap
may be opened by disorder. So it is not protected by symmetry and is trivial topological insulator. If a SPEMT cannot be added, then this system is topological non-trivial. 


One best thing about this classification is that it can be done by considering
only the minimal matrix dimension of those gamma matrices in Hamiltonian. The
argument is that, complex gamma matrices are of even dimension (except the
trivial cases of $\mathrm{C\ell}_{0/1}(\C)$)\footnote{See this post
\cite{PhysicsStackExchange}, or the p.12 and Theorem in p.5 of this
\cite{West1998}.}. Therefore, gamma matrices of different
matrix dimension are built by tensor products of Pauli matrices.
\footnote{Or, the non-trivial irreducible representations (turns out that there are
at most 2) are built by tensoing Pauli matrices.}
Now, increasing the dimension of the matrix, is equivalent to
tensoring them. One can add more bands of the same type, or add more bands of a
different type in the smaller dimension, or add trivial bands. In all cases, the
triviality of topological insulators can be detected in minimal dimension, since
adding trivial bands can be ignored (insensitive to addition of trivial bands),
and the possible different types of matrices are limited.

However, to distinguish between a $\Z$ topological insulator and a $\Z_2$
topological insulator, we need to increase the matrix dimension to consider
multiple copies of the Dirac Hamiltonian. If the topological state is stable for
an arbitrary copies, then this is a $\Z$ topological insulator. If the
topological state is stable only of an odd number of copies, this is a $\Z_2$
topological insulator.

Another way to view this way of classification, is to look at the surface
Dirac Hamiltonian \ref{eq:H-Dirac-surf}. Then the above mentioned approach is to
see if the surface mode can be gapped, so as to detect the topological
properties of the bulk.

\subsection{Isomorphism between Symmetry Classes and Clifford Algebras}
\label{sec:Isomorphism}
As with other classifications, we construct isomorphisms between Dirac matrices
and Clifford Algebras, after which we are faced with an extension problem. The
isomorphism is constructed now. 

A bit of notation note. We use $\mathrm{C\ell}_{p,q}(\R)$ to denote a Clifford
Algebra over $\R$ with $p$ for $+$ signature and $q$ for $-$ signature. We use
$\mathrm{C\ell}_{n}(\C)$ to denote a Clifford Algebra over $\C$. We use
$\mathcal{M}(n,K)$ to denote the algebra of $n\times n$ matrices over $K=\C$ or
$K=\R$. We use $\id_n$ to denote the $n\times n$ identify matrix.

\subsubsection{Symmetry Constraint}
\label{sec:Symmetry Constraint}
The Hamiltonian with extra mass term is
\begin{equation}
    H = M\tilde{\gamma}_0 + \sum_{j=1}^D m_j\tilde{\gamma}_j
    + \sum_{i=1}^{d} k_i \gamma_i
\end{equation}
The symmetry properties \ref{eq:T-sym-Hk},\ref{eq:C-sym-Hk} give the following
conditions on the gamma matrices:
\begin{subequations}
    \label{eq:sym-spemt}
\begin{align}
    & \{T,\gamma_i \} = 0,\, [C,\tilde{\gamma}_j] =0 \\
    & \{C,\tilde{\gamma}_j \} = 0,\, [C,\gamma_i] =0 \\
    & \{S,\tilde{\gamma}_j \} = \{S,\gamma_i\} =0
\end{align}
\end{subequations}
With this, we can establish an isomorphism between "Hamiltonian and Symmetry
Operators" and "Clifford Algebra over $\R$ or $\C$".  But let's first see some
examples in action.

\subsubsection{Examples of Constructing Isomorphism}
\label{sec:iso-Examples}
For class AI ($T^2=1$, $C=0,S=0$) in $d=1,D=0$, we define:
\begin{equation}
	G_{\mathrm{AI}} = \{i, T, \tilde{\gamma_0}, \gamma_1\}
\end{equation}
We have its (anti)commutation relations listed in Table~\ref{tab:generator-AI}.
\begin{table}[htpb]
    \centering
    \caption{Generators in $G_{AI}$ with $d=1,D=0$}
    \label{tab:generator-AI}
    $ \begin{tabu}{ c | *4{c} }
        ~                & i      & T         & \tilde{\gamma}_0       & \gamma_1 \\
        \hline
        i                & i^2=-1 & \{i,T\}=0 & [i,\tilde{\gamma}_0]=0 & [i,\gamma_1]=0 \\
        T                & ~      & T^2=1     & [T,\tilde{\gamma}_0]=0 & \{T,\gamma_1\}=0 \\
        \tilde{\gamma}_0 & ~      & ~         & \tilde{\gamma}_0^2 =1  & \{\tilde{\gamma}_0,\gamma_1\}=0 \\
        \gamma_1         & ~      & ~         & ~                      & \gamma_1^2=1
    \end{tabu} $
\end{table}

Now with $J_4 = T\tilde{\gamma}_0\gamma_1$,
$J_3=iT\tilde{\gamma}_0$, we could verify that all
$\{i,T,J_3,J_4\}$ anticommute with each other, and
$J_3^2=J_4^2=1$. So together they generate the algebra
$\mathrm{C\ell}_{3,1}(\R) \cong \mathrm{C\ell}_{2,2}(\R)$. This is consistent with the
isomorphism $G_{\mathrm{AI}} \cong \mathrm{C\ell}_{2+D,1+d}(\R)$ shown later in
Table~\ref{tab:map-sym-cl}.

Another example. For class AII ($T^2=-1$, $C=0,S=0$) in $d=1,D=0$, we have:
\begin{equation}
    G_{\mathrm{AI}} = \{i, T, \tilde{\gamma_0}, \gamma_1\}
\end{equation}
We have its (anti)commutation relations listed in Table~\ref{tab:generator-AII}.
\begin{table}[htpb]
    \centering
    \caption{Generators in $G_{\mathrm{AII}}$ with $d=1,D=0$}
    \label{tab:generator-AII}
    $ \begin{tabu}{ c | *4{c} }
        ~                & i      & T         & \tilde{\gamma}_0       & \gamma_1 \\
        \hline
        i                & i^2=-1 & \{i,T\}=0 & [i,\tilde{\gamma}_0]=0 & [i,\gamma_1]=0 \\
        T                & ~      & T^2=-1    & [T,\tilde{\gamma}_0]=0 & \{T,\gamma_1\}=0 \\
        \tilde{\gamma}_0 & ~      & ~         & \tilde{\gamma}_0^2 =1  & \{\tilde{\gamma}_0,\gamma_1\}=0 \\
        \gamma_1         & ~      & ~         & ~                      & \gamma_1^2=1
    \end{tabu} $
\end{table}
Now with $J_4 = T\tilde{\gamma}_0\gamma_1$, $J_3=iT\tilde{\gamma}_0$, we could
verify that all $\{i,T,J_3,J_4\}$ anticommute with each other, and
$J_3^2=J_4^2=-1$. So together they generate the algebra $\mathrm{C\ell}_{0,4}(\R)$.
Notice that, assume we have $\mathrm{C\ell}_{1,3}(\R)=\{K_1,K_2,K_3,K'_1\}$, and
$K_i^2=-1$, $(K'_1)^2=1$. The following map
\begin{equation}
    J_1 = K_1,\, J_2 = K_2,\, J_3=K_3,\, J_4=K_1K_2K_3K'_1
\end{equation}
has the property shown in the table~\ref{tab:map-cl13-2-cl04}.
\begin{table}[htpb]
    \centering
    \caption{Map from $\mathrm{C\ell}_{1,3}(\R)$ to $\mathrm{C\ell}_{0,4}(\R)$.}
    \label{tab:map-cl13-2-cl04}
    $ \begin{tabu}{ c | *4{c} }
        ~      & J_1=K_1 & J_2=K_2 & J_3=K_3 & J_4 = K_1K_2K_3K'_1\\
        \hline
        J_1    & -1      & \{\}    & \{\}    & \{\} \\
        J_2    & ~       & -1      & \{\}    & \{\} \\
        J_3    & ~       & ~       & -1      & \{\} \\
        J_4    & ~       & ~       & ~       & -1
    \end{tabu} $

    where $\{\}$ means anticommute.
\end{table}
Therefore, $\mathrm{C\ell}_{0,4}(\R) \cong \mathrm{C\ell}_{1,3}(\R) $, which is
consistent with $G_{\mathrm{AII}} \cong
\mathrm{C\ell}_{d,3+D}(\R)$ in Table~\ref{tab:map-sym-cl}.

\subsubsection{Constructing the Isomorphism (Complex Symmetry Classes)}
\label{sec:Complex Classes-iso}
The general steps for constructing the isomorphism is presented here. First, we
need to treat the situation of complex classes and real classes
differently. The complex classes are isomorphic to complex Clifford Algebras,
while the real classes are isomorphic to real Clifford Algebras. The reason for
this distinction is that, in real classes, there are complex conjugate operator
$K$. First, there is no way to represent complex conjugation simply as
multiplication of complex matrices. Second, complex matrices come with a nature
definition of complex conjugate $\dagger$, whereas $K^\dagger$ is ill-defined
\footnote{Think about this. $\braket{c \phi |K \psi}$ could have different
    values depending on whether we move $K$ to the left as $K^\dagger$ first, or
we move $c$ out of the inner-product first.}. Therefore, we use real Clifford
Algebras when dealing with real classes (which is the reason why they are named
real classes).

For complex classes, let $G_\#$ be the group generated by elements
$\{\gamma_i,\tilde{\gamma}_j,S\}$ ($i=1,\cdots,d$, $j=0,1$) in each symmetry
class $\#$ (so $S$ exists only in class $\mathrm{AIII}$). We may choose a phase
such that $S^2=1$, as mentioned earlier. Then obviously $G_\#$ will be a
Clifford Algebra:
\begin{align}
    G_{A} &= \text{generated by }\{\gamma_i,\tilde{\gamma}_j\} \cong
    \mathrm{C\ell}_{d+D+1}(\C) \\
    G_{\mathrm{AIII}} &= \text{generated by }\{\gamma_i,\tilde{\gamma}_j,S\}
    \cong  \mathrm{C\ell}_{d+D+2}(\C)
\end{align}

\subsubsection{Constructing the Isomorphism (Real Symmetry Classes)}
\label{sec:Real Classes-iso}

For real classes, we need to include $i$, and symmetry operators $T,C$ in our
group $G_\#$. It turns out that we need to pick a phase such that $\{T,C\}=0$,
which is possible as mentioned earlier. Now we demonstrate the proof of
isomorphism in class $\mathrm{AII}$.\footnote{For reader unfamiliar with the
complexfication of real vector space, and the realification of complex vector
space, I recommend a quick reading of section 12 of \cite{suetin1989linear}.}

We first note that, given a real vector space $V$, we can complexify a real space in
two ways. The first is trivially taking the tensor product $V\otimes_\R \C$,
which does not suit our purpose. The second is to find an almost complex
structure $J$, which is a $\R$-linear map that squares to $-\id$, i.e.
$J^2=-\id$. With this almost complex structure $J$, then $V$ admits in a natural
way the structure of a complex vector space $V_\C$\cite{DanielHuybrechts2005}. Also, a
$\R$-linear map $A$ on $V$ is $\C$-linear($\C$-antilinear) if and only if $A$
commutes (anticommutes) with $J$. Therefore, we can model the antiunitary
operators on a real vector space naturally.

Let us denote the generators in Clifford Algebra as $J_j$ and $\tilde{J}_i$,
where $J_j^2=-\tilde{J}_i^2 =-\id$. Since this almost complex structure $J$
squares to $-\id$ and anticommutes with antiunitary symmetry operators, we
naturally take it to be the first generator in our Clifford Algebra, $J_1=J$.

Then we discuss some tips that will be useful for later calculation.
\paragraph{Tips}
\begin{enumerate}
    \item Since all generator anticommute, all matrices either commute or
        anticommute. So the Algebra is pretty simple.
    \item Since $[A,BC] = \pm[A,CB]$ and $\{A,BC\}=\pm\{A,CB\}$, the commutation
        or anticommutation does not depends on the order of the matrices $BC$ or
        $CB$. So one might rearrange them in these orders whichever is convenient.
    \item For $A=\{J_1,\cdots,J_n\},B=\{K_1,\cdots,K_m\}$, we have $AB=(-1)^{mn}
        BA$. However, if $A$ and $B$ has something in common, then the above
        condition breaks. This is like adding an "impurity" in it to change the
        commutation/anticommutation relations.
    \item The restriction given by $\sum_i k_i \gamma_i$, and $\sum_j m_j
        \tilde{\gamma}_j$ (will be shown later) are much restrictive that the
        candidates for symmetry operator are only a small finite set.
        \label{enum:tips-4}
\end{enumerate}

\paragraph{Class $\mathrm{AII}$} Now we prove the isomorphism. First, we try to
construct each gamma matrices in Hamiltonian. All matrices commute with $i$, so
in real vector space, they commute with $J_1=J$. Then, each gamma matrices
should have an even number of Clifford Algebra generators other than $J_1$
itself. Similarly, antiunitary symmetry operators should have an odd number of
Clifford Algebra generators other than $J_1$. We "dope" the gamma matrices with
some $J_1$ to make it commute/anticommute with symmetry operators. More
explicitly, bearing in mind that class $\mathrm{AII}$ has only $T^2=-1$, we
take a quick look into Hamiltonian \ref{eq:H-spemt} and symmetry conditions
\ref{eq:sym-spemt}, and they give us the inspiration to set:
\begin{equation}
    H_{\mathrm{AII}} = m J_1J_2J_3 + \sum_{j=1}^D m_jJ_1J_2J_{3+j} +
    \sum_{i=1}^d k_i J_2\tilde{J}_i
\end{equation}
Here $\tilde{\gamma}_j$ are $J_1J_2J_{3+j}$, which square to $1$. $\gamma_i$ are
$J_2\tilde{J}_i$. The Clifford Algebra is at least $\mathrm{C\ell}_{d,3+D}(\R)$.
And within this algebra, only $J_2$ is a possible candidate for symmetry
operators, which satisfy equations~\ref{eq:sym-spemt} (use
tips~\ref{enum:tips-4} for calculation). $J_2$ is found to be a
time reversal operator, and $J_2^2=-1$ confirms that this Hamiltonian belongs to
class $\mathrm{AII}$. Therefore, the map:
\begin{align}
    f: \mathrm{C\ell}_{d,3+D}(\R) &\to G_{\mathrm{AII}} \\
    J_1J_2J_{3+j} &\to \tilde{\gamma}_j,\, (j=0,1,\cdots,D) \nonumber\\
    J_2 \tilde{J}_i &\to \gamma_i,\, (i=0,1,\cdots,d) \nonumber
\end{align}
is a map from Clifford Algebra $\mathrm{C\ell}_{d,3+D}(\R)$ to symmetry class
$\mathrm{AII}$. The inverse map can be found easily. We first solve some
formality problems. A complex number can be written as a real number by
identifying $i$ with $J=-i\sigma_y$ and $1$ with $\id$, the identity matrix:
\begin{equation}
    a+bi \to a \begin{pmatrix}
        1 & \\ & 1
    \end{pmatrix} + b \begin{pmatrix}
         & -1 \\ 1 & 
    \end{pmatrix} = a\id + b J
\end{equation}
Therefore, all complex matrices $\gamma_i,\tilde{\gamma}_j$ are identified with
real matrices $\Gamma_i,\tilde{\Gamma}_j$, of twice the size of
$\gamma_i,\tilde{\gamma}_j$. More explicitly, $(a_{mn}+i b_{mn})$ is identified
as $A+iB = (a_{mn})+i(b_{mn}) \to \id_2\otimes(a_{mn})+J\otimes(b_{mn})$. On
important matrix is the matrix $J_1$ which is identified as $i$, and is equal to:
\begin{equation}
    J_1 \equiv -i\sigma_y \otimes \id_{n\times n}
\end{equation}
The complex conjugate $K$ is then a matrix anticommuting with $J_1$.  Then $T\to
\Gamma_T$, $C\to \Gamma_C$ for some real matrices anticommute with $i\to J_1$.
We also note that, as mentioned earlier, we make a phase choice of $T$ and $C$
such that:
\begin{equation}
    \{\Gamma_T,\Gamma_C\} = 0
\end{equation}

Now, we write symbolically $J_1J_2J_{3+j} = J_1 \Gamma_T J_{3+j} =
\tilde{\Gamma}_j$, then clearly $J_{3+j} = \Gamma_T J_1\tilde{\Gamma}_j$.
Similarly, $\tilde{J}_i = \Gamma_T\Gamma_i$. So the map:
\begin{align}
    f^{-1}: G_{\mathrm{AII}} &\to \mathrm{C\ell}_{d,3+D}(\R) \\
    i\gamma_y \otimes \id_{n\times n} &\to J_1 \nonumber\\
    \Gamma_T &\to J_2 \nonumber\\
    \Gamma_TJ_1\Gamma_j &\to J_{3+j} \nonumber\\
    \Gamma_T\tilde{\Gamma}_i &\to J_{i} \nonumber
\end{align}
is the desired inverse map.

\paragraph{Class $\mathrm{CII}$} The class $\mathrm{CII}$ has only one $C^2=-1$
more than class $\mathrm{AII}$, therefore, we only need to enlarge the Clifford
Algebra to include one more $J_{4+D}$, and set it as the $C$ symmetry operator.
The rest is exactly the same as in class $\mathrm{AII}$.

All other classes can be treated similarly, so we do not repeat the calculation
and only list the result in Table~\ref{tab:map-sym-cl}.
\begin{table}[htpb]
    \centering
    \caption{Mapping Relations between symmetry classes and Clifford Algebras.
    Here $J_j^2=-\id,\tilde{J}_i^2=\id$, and they generates the Clifford
    Algebra. Also, only one direction of mapping is shown. The inverse map can
    be easily constructed accordingly. The complex class are also added for
    convenience.}
    \label{tab:map-sym-cl}
    \begin{tabular}{c | c c c | l | c }
        Class($\#$) & T & C & S & Mappings  & $G_\#\cong $ \\
        \hline 
        D & $0$ & $+$ & $0$ & 
        $\tilde{J}_1\to\Gamma_C$,
        $\tilde{J}_1J_{2+j}\to \tilde{\Gamma}_j$,
        $J_1\tilde{J}_1\tilde{J}_{1+i}\to \Gamma_i$ &
        $\mathrm{C\ell}_{1+d,2+D}(\R)$ \\
        DIII & $-$ & $+$ & $1$ & 
        $\tilde{J}_1\to\Gamma_C$, $\tilde{J}_{3+D}\to\Gamma_T$,
        $\tilde{J}_1J_{2+j}\to \tilde{\Gamma}_j$,
        $J_1\tilde{J}_1\tilde{J}_{1+i}\to \Gamma_i$ &
        $\mathrm{C\ell}_{1+d,3+D}(\R)$ \\
        AII & $-$ & $0$ & $0$ & 
        $J_2\to \Gamma_T$,
        $J_1J_2J_{3+j}\to \tilde{\Gamma}_j$,
        $J_2\tilde{J}_i \to \Gamma_i$
        & $\mathrm{C\ell}_{d,3+D}(\R)$ \\
        CII & $-$ & $-$ & $1$ & 
        $J_2\to \Gamma_T$, $J_{4+D}\to \Gamma_C$,
        $J_1J_2J_{3+j}\to \tilde{\Gamma}_j$,
        $J_2\tilde{J}_i \to \Gamma_i$
        & $\mathrm{C\ell}_{d,4+D}(\R)$ \\
        C & $0$ & $-$ & $0$ & 
        $J_2\to \Gamma_C$, 
        $J_2\tilde{J}_{1+j} \to \tilde{\Gamma}_j$,
        $J_1J_2J_{i+2}\to \Gamma_i$
        & $\mathrm{C\ell}_{1+D,2+d}(\R)$ \\
        CI & $+$ & $-$ & $1$ & 
        $J_2\to \Gamma_C$, $\tilde{J}_{2+D}\to \Gamma_T$,
        $J_2\tilde{J}_{1+j} \to \tilde{\Gamma}_j$,
        $J_1J_2J_{i+2}\to \Gamma_i$
        & $\mathrm{C\ell}_{2+D,2+d}(\R)$ \\
        AI & $+$ & $0$ & $0$ & 
        $\tilde{J}_1\to \Gamma_T$,
        $J_1\tilde{J}_1\tilde{J}_{2+j} \to \tilde{\Gamma}_j$,
        $\tilde{J}_1J_{i+1} \to \Gamma_i$
        & $\mathrm{C\ell}_{2+D,1+d}(\R)$ \\
        BDI & $+$ & $+$ & $1$ & 
        $\tilde{J}_1\to \Gamma_T$, $\tilde{J}_{3+D}\to \Gamma_C$,
        $J_1\tilde{J}_1\tilde{J}_{2+j} \to \tilde{\Gamma}_j$,
        $\tilde{J}_1J_{i+1} \to \Gamma_i$
        & $\mathrm{C\ell}_{3+D,1+d}(\R)$ \\
        \hline
        A & $0$ & $0$ & $0$ & 
        & $\mathrm{C\ell}_{d+D+1}(\C)$ \\
        AIII & $0$ & $0$ & $1$ & 
        & $\mathrm{C\ell}_{d+D+2}(\C)$ \\
        \hline
    \end{tabular}
\end{table}
\subsection{Classification in 1-dimension}
\label{sec:Classification in 1-dimension}

\subsubsection{Model Systems}
\label{sec:Model Systems}
Classification of $1$ dimension is the simplest, since the minimal matrix
dimension will mostly be $2$, which means we can use the familiar Pauli matrices
directly.

\paragraph{Class $\mathrm{A}$} We have the extension from
$\mathrm{C\ell}_{2}(\C)$ to $\mathrm{C\ell}_{3}(\C)$. We have for example:
\begin{equation}
    H_A = M \sigma_x + k_x \sigma_y
\end{equation}
Obviously, there is an SPEMT $m\sigma_x$. So this phase is trivial.

\paragraph{Class $\mathrm{AIII}$} We have the extension from
$\mathrm{C\ell}_{3}(\C)$ to $\mathrm{C\ell}_{4}(\C)$. We have
for example:
\begin{equation}
    H_A = M \sigma_x + k_x \sigma_y
\end{equation}
However, the symmetry operator $S$ takes the rest of Pauli matrices $\sigma_z$, and
there is no other matrix possible for the extra mass term. So the state is


topologically non-trivial. Now we consider the state with arbitrary copies of it.
\begin{equation}
    H_A = M \sigma_x\otimes \id_n + k_x\sigma_y\otimes\id_n,
    S=\sigma_z\otimes\id_n
\end{equation}
where $n$ is some positive integer. Since gamma matrices are basically tensor
products of Pauli matrices, there is no SPEMT term. Hence this is a $\Z$
topological insulator.

The above are for complex classes. For real classes, we have to be careful,
since one generator $J_1$ is taken up by $i$.

\paragraph{Class $\mathrm{AII}$} We have the extension from
$\mathrm{C\ell}_{1,3}(\R)$ to $\mathrm{C\ell}_{1,4}(\R)$.
Explicitly, we could let
\begin{equation}
    H_{\mathrm{AII}} = M\sigma_x\otimes\id_2 + k_x\sigma_y\otimes\id_2
\end{equation}
with $T=\id_2\otimes\sigma_x K$. The SPEMT is $m\sigma_z\otimes\sigma_x$. So
this is a trivial insulator.

\paragraph{Class $\mathrm{C}$} We have the extension from
$\mathrm{C\ell}_{1,3}(\R)$ to $\mathrm{C\ell}_{2,3}(\R)$. Explicitly, we
could let
\begin{equation}
    H_{\mathrm{C}} = M\sigma_z\otimes\id_2 + k_x\sigma_y\otimes\id_2
\end{equation}
with $C=\sigma_x\otimes\sigma_y K$. The SPEMT is $m\sigma_x\otimes\sigma_x$. So
this is a trivial insulator.

\paragraph{Class $\mathrm{CI}$} We have the extension from
$\mathrm{C\ell}_{2,3}(\R)$, to $\mathrm{C\ell}_{3,3}(\R)$. Explicitly, we
could let
\begin{equation}
    H_{\mathrm{CI}} = M\sigma_z\otimes\id_2 + k_x\sigma_y\otimes\id_2
\end{equation}
with $C=\sigma_x\otimes\sigma_y K$, $T=K$. The SPEMT is again
$m\sigma_x\otimes\sigma_x$. So this is a trivial insulator.

\paragraph{Class $\mathrm{AI}$} We have the extension from
$\mathrm{C\ell}_{2,2}(\R)$ to $\mathrm{C\ell}_{3,2}(\R)$. Explicitly, we could let
\begin{equation}
    H_{\mathrm{AI}} = M\sigma_x + k_x\sigma_y
\end{equation}
with $T=K$. The SPEMT is $m\sigma_z$. So this is a trivial insulator.

\paragraph{Class $\mathrm{CII}$} We have extension from
$\mathrm{C\ell}_{1,4}(\R)$ to $\mathrm{C\ell}_{1,5}(\R)$.
We let:
\begin{equation}
    H_{\mathrm{CII}} = M\sigma_x\otimes\id_2 + k_x\sigma_y\otimes\id_2
\end{equation}
with $T=\id_2\otimes\sigma_y K$, $C=\sigma_z\otimes\sigma_y K$. There is no
SPEMT\footnote{To anticommute with $\sigma_x\otimes\id_2$ and
$\sigma_y\otimes\id_2$, it must be of the form $\sigma_z\otimes?$. But none of
the remaining choice are acceptable, considering $T$ and $C$.}. Consider
arbitrary copies of it:
\begin{equation}
    H_{\mathrm{CII}} = (M\sigma_x\otimes\id_2 +
    k_x\sigma_y\otimes\id_2)\otimes\id_n
\end{equation}
with $T=\id_2\otimes\sigma_y\otimes\id_n K$,
$C=\sigma_z\otimes\sigma_y\otimes\id_n K$. There is still no SPEMT\footnote{The
SPEMT has to be of the form $\sigma_z\otimes\sigma_\alpha\otimes A_{n\times n}$.
Commuting with $T$ makes $A$ completely imaginary. But it cannot anticommute
with $C$.}. Therefore, this is a $\Z$ topological insulator.

\paragraph{Class $\mathrm{BDI}$} We have extension from
$\mathrm{C\ell}_{3,2}(\R)$ to $\mathrm{C\ell}_{4,2}(\R)$. We let:
\begin{equation}
    H_{\mathrm{BDI}} = M\sigma_z + k_x \sigma_y
\end{equation}
with $T=K$, $C=\sigma_x K$. Obviously, there is no SPEMT. For arbitrary copies
of it:
\begin{equation}
    H_{\mathrm{BDI}} = M\sigma_z\otimes\id_n + k_x \sigma_y\otimes\id
\end{equation}
with $T=\id_{n+2}K$, $C=\sigma_x\otimes\id_n K$.  The SPEMT must be of the form
$\sigma_x\otimes\Delta$ to anticommute with $H_{\mathrm{BDI}}$. But to commute with
$T$ leads to $\Delta$ being real, and to anticommute with $C$ leads to $\Delta$ being
complex. Hence there is no SPEMT. This is a $\Z$ topological insulator.

\paragraph{Class $\mathrm{D}$} We let:
\begin{equation}
    H_{\mathrm{D}} = M\sigma_z + k_x \sigma_y
\end{equation}
with $C=\sigma_x K$. Obviously, there is no SPEMT. Now consider two copies of
it:
\begin{equation}
    H_{\mathrm{D}} = (M\sigma_z + k_x \sigma_y)\otimes\id_2
\end{equation}
with $C=\sigma_x\otimes\id_2 K$. There is one SPEMT $m \sigma_z\otimes\sigma_y$.
Hence, this is a $\Z_2$ topological insulator.

To lighten the notation a bit, we introduce $\tau_i=s_i=\sigma_i$ ($i=0,1,2,3$).
But $\tau_i$, $s_i$, and $\sigma_i$ all act on the different spaces.
\paragraph{Class $\mathrm{DIII}$} We let:
\begin{equation}
    H_{\mathrm{DIII}} = M \sigma_z s_0 + k_x \sigma_y s_0
\end{equation}
with $C=\sigma_x K$, $T=s_y K$. Any SPEMT anticommute with Hamiltonian has the
form $\sigma_x s_i$. For it to anticommute with $C$, it is $\sigma_x s_y$. But
this does not commute with $T$. Hence, there is no SPEMT. Now consider two
copies of it:
\begin{equation}
    H_{\mathrm{DIII}} = (M \sigma_z s_0 + k_x \sigma_y s_0)\tau_0
\end{equation}
with $C=\sigma_x K$, $T=s_y K$.  There is one SPEMT $m \sigma_xs_x\tau_y$. So
this is a $\Z_2$ topological insulator.

In summary, we shown examples of different classes of topological insulators in
$d=1$, summarized in Table~\ref{tab:ti-d=1}.
\begin{table}[htpb]
    \centering
    \caption{Topological Insulators in $d=1$}
    \label{tab:ti-d=1}
    \begin{tabular}{c | c c c | c }
        Class($\#$) & T & C & S & $d=1$ \\
        \hline 
        A & $0$ & $0$ & $0$ & $0$ \\
        AIII & $0$ & $0$ & $1$ & $\Z$ \\
        \hline
        D & $0$ & $+$ & $0$ & $\Z_2$ \\
        DIII & $-$ & $+$ & $1$ & $\Z_2$ \\
        AII & $-$ & $0$ & $0$ & $0$ \\
        CII & $-$ & $-$ & $1$ & $\Z$ \\
        C & $0$ & $-$ & $0$ & $0$ \\
        CI & $+$ & $-$ & $1$ & $0$ \\
        AI & $+$ & $0$ & $0$ & $0$ \\
        BDI & $+$ & $+$ & $1$ & $\Z$ \\
        \hline
    \end{tabular}
\end{table}
\subsubsection{Irreducible Representation Arguments for Classification}
\label{sec:Irreducible Representation Arguments for Classification}
The above examples are only valid in special points of the Brillouin Zone.
However, in real materials, one will be faced with Hamiltonians of various form:
\begin{equation}
    H = m\gamma_0 \pm k_1\gamma_1 \pm k_2\gamma_2 \cdots
\end{equation}
Here, we look at this problem from a representation theory point of view. It is
well known that representation of Clifford Algebras are completely reducible,
and there are only 1 or 2 nontrivial (dimension $>1$) inequivalent irreducible
representations of Clifford Algebras (see Appendix~\ref{sec:Various Properties for
Clifford Algebras}).  The dimension of the representation of relevant Clifford
Algebras are listed in Table~\ref{tab:mat-dim-allClass-1d}. There are only three
types of change occurs in that table:

\begin{enumerate}
    \item $\mathbf{n\to n}$ or $\mathbf{n\to n_2}$ Basically, this means that in
        the same dimension we could always have a SPEMT present. So there will
        never be stable edge states.
    \item 
    $\mathbf{n\to 2n}$ or $\mathbf{n_2\to 2n}$ In this case, a single copy of
    Hamiltonian is stable against perturbation. Therefore, we consider two copies of
    the Hamiltonian and see if the double-size Hamiltonian can be gapped by
    perturbation. If one denote the one irreducible representation in dimension $n$
    by $\Gamma$, then we are most importantly faced with two types of double-size
    Hamiltonian:
    \begin{equation}
        H_\mathrm{doubled,1} = (M\tilde{\Gamma}_0 + \sum_i k_i\Gamma_i
        )\otimes\id_{2\times 2}
    \end{equation}
    or
    \begin{equation}
        H_\mathrm{doubled,2} = \begin{pmatrix}
            M\tilde{\Gamma}_0 + \sum_i k_i\Gamma_i & 0 \\
            0 & \pm M\tilde{\Gamma}_0 + \sum_i \pm k_i\Gamma_i \\
        \end{pmatrix}
    \end{equation}
    Where the sign $\pm$ for $H_\mathrm{doubled,2}$ is such that
    $H_\mathrm{doubled,2}\neq H_\mathrm{doubled,1}$.  The first Hamiltonian is
    called two equivalent copies of the original Hamiltonian, whereas the second one
    is called two inequivalent copies of the original Hamiltonian.\footnote{The
    exact reason for "equivalent" and "inequivalent" would be found in
    representation theory or Clifford Algebras, but it is not important for
    our discussion here.} For the second copy, we could always find a SPEMT to gap
    the system. For example, for
    \begin{equation}
        H_\mathrm{doubled,2} = \begin{pmatrix}
            M\tilde{\Gamma}_0 + \sum_i k_i\Gamma_i & 0 \\
            0 & - M\tilde{\Gamma}_0 + \sum_i k_i\Gamma_i \\
        \end{pmatrix} =
        M\tilde{\Gamma}_0\otimes\sigma_z + 
            (\sum_i k_i\Gamma_i)\otimes\id_{2\times 2}
    \end{equation}
    It can be gapped by a SPEMT such as $\Gamma_1\otimes \sigma_x$. 

    For the first type, the situation is different and it distinguishes the $\Z$
    and $\Z_2$ topological insulator. We first not that, a SPEMT in this case
    must be of the form $\Gamma\otimes A_{2\times 2}$, where $\Gamma$
    anticommutes with $\Gamma_i$ and $\tilde{\Gamma}_0$, $A$ is some $2\times 2$
    matrix.
    \begin{enumerate}
        \item $\mathbf{n\to 2n}$: In this case, there will be a SPEMT to gap
            the doubled Hamiltonian. So this is a $\Z_2$ topological insulator.
            To be more specific, assume that we have the extension problem of
            $\mathrm{C\ell}_{p,q}(\R)\to\mathrm{C\ell}_{p,q+1}(\R)$, where
            $\mathrm{dim}(\mathrm{C\ell}_{p,q}(\R))=n$,
            $\mathrm{dim}(\mathrm{C\ell}_{p,q+1}(\R))=2n$. It can be checked in
            Table~\ref{tab:mat-dim-allClass-1d} that
            $\mathrm{dim}(\mathrm{C\ell}_{p+1,q}(\R))=n_2$, i.e. an extra
            $\Gamma_{p+1}$ is present in the same dimension. This extra term
            form a SPEMT in the doubled Hamiltonian as $\Gamma_{p+1}\otimes
            i\sigma_y$.

            Similarly, if we have $\mathrm{C\ell}_{p,q}(\R)\to
            \mathrm{C\ell}_{p+1,q}(\R)$, an extra term $\Gamma_{q+1}$ from
            $\mathrm{C\ell}_{p,q+1}(\R)$ is present in the same dimension, and
            it forms a SPEMT in the doubled Hamiltonian as $\Gamma_{q+1}\otimes
            i\sigma_y$.
        \item $\mathbf{n_2\to 2n}$: In this case, unfortunately no SPEMT can be
            found from either $\mathrm{C\ell}_{p+1,q}(\R)$ or
            $\mathrm{C\ell}_{p,q+1}(\R)$, as can be checked in
            Table~\ref{tab:mat-dim-allClass-1d} that they all double the
            dimension.
    \end{enumerate}
\end{enumerate}

In this way we confirmed the classification in Table~\ref{tab:ti-d=1}. And we
note that, as long as $p-q \text{mod 8}$ is fixed, all the above discussion will
be the same. This come as a property of the representation of Clifford Algebras
(see Appendix~\ref{sec:Various Properties for Clifford Algebras} for details).
Also, if one is interested in how those model system mentioned before can be
reflected in the representation of Clifford Algebras, one could see
Appendix~\ref{sec:a case study} for one example.

\begin{table}[htpb]
    \centering
    \caption{Matrix dimensions related to the extension problem for different
        classes in $d=1$. Note that a subscript $2$ as in $2_2$, means that
        there are two inequivalent representations in the same matrix dimension.
        Note also that since the complexification of a real matrix reduce the
        dimension by $\frac{1}{2}$, we write only half of the dimension as shown
    in Appendix~\ref{sec:Various Properties for Clifford Algebras}. }
    \label{tab:mat-dim-allClass-1d}
    \begin{tabular}{c | c c c}
        Name of Class & Extension of $\mathrm{C\ell}$ & Dimension Change &
        Topological Invariant \\
        \hline
        A & $\mathrm{C\ell}_{2}(\C)\to\mathrm{C\ell}_{3}(\C)$ & $2\to2_2$ & $0$
        \\
        AIII & $\mathrm{C\ell}_{3}(\C)\to\mathrm{C\ell}_{4}(\C)$ & $2_2\to4$ &
        $\Z$
        \\
        AII & $\mathrm{C\ell}_{1,3}(\R)\to\mathrm{C\ell}_{1,4}(\R)$ & $4\to 4_2$ & $0$
        \\
        C & $\mathrm{C\ell}_{1,3}(\R)\to\mathrm{C\ell}_{2,3}(\R)$ & $4\to 4$ & $0$
        \\
        CI & $\mathrm{C\ell}_{2,3}(\R)\to\mathrm{C\ell}_{3,3}(\R)$ & $4\to 4$ & $0$
        \\
        AI & $\mathrm{C\ell}_{2,2}(\R)\to\mathrm{C\ell}_{3,2}(\R)$ & $2\to 2_2$ & $0$
        \\
        CII & $\mathrm{C\ell}_{1,4}(\R)\to\mathrm{C\ell}_{1,5}(\R)$ & $4_2\to 8$
        with $\mathrm{C\ell}_{2,4}(\R)\sim 8$.
        & $\Z$
        \\
        BDI & $\mathrm{C\ell}_{3,2}(\R)\to\mathrm{C\ell}_{4,2}(\R)$ & $2_2\to 4$ 
        with $\mathrm{C\ell}_{3,3}(\R)\sim 4$.
        & $\Z$
        \\
        D & $\mathrm{C\ell}_{2,2}(\R)\to\mathrm{C\ell}_{2,3}(\R)$ & $2\to 4$
        with $\mathrm{C\ell}_{3,2}(\R)\sim2_2$.
        & $\Z_2$
        \\
        DIII & $\mathrm{C\ell}_{2,3}(\R)\to\mathrm{C\ell}_{2,4}(\R)$ & $4\to 8$ 
        with $\mathrm{C\ell}_{3,3}(\R)\sim 4$.
        & $\Z_2$
        \\
        \hline
    \end{tabular}
\end{table}
\subsection{Classification in Arbitrary Dimension}
\label{sec:Classification in Arbitrary Dimension}
We now show that, when combined with some properties, the classification in $d=1$ can be
generalized to classification in arbitrary dimensions. In $K$-theoretic
classification, the extension problem is not affected when tensored with some
other algebras. Specifically, for complex classes, the extension problem of
$\mathrm{C\ell}_{n}(\C)\to\mathrm{C\ell}_{m}(\C)$ is the same as the extension
problem of $\mathrm{C\ell}_{n}(\C)\otimes\mathrm{C\ell}_{2}(\C)\to
\mathrm{C\ell}_{m}(\C)\otimes\mathrm{C\ell}_{2}(\C)$ (See for example, sec
III.C.1 of \cite{Chiu2016}). For real classes, similar property holds.
This insensitivity to tensoring a new algebra, can be understood in SPEMT
background. Because $\mathrm{C\ell}_{2}(\C)\cong\mathcal{M}(2,\C)$, which only
multiply the matrix dimension of both sides by a constant $2$. So the pattern of
$n\to n, n\to n_2$, $n\to 2n$, and $n_2\to 2n$, mentioned in previous section,
remains unaltered. For real classes, instead of $\mathrm{C\ell}_{2}(\C)$, we
have $\mathrm{C\ell}_{1,1}(\R)\cong\mathcal{M}(2,\R)$, or
$\mathrm{C\ell}_{2,2}(\R)\cong\mathrm{C\ell}_{1,1}(\R)\otimes\mathrm{C\ell}_{1,1}(\R)\cong\mathcal{M}(4,\R)$.
So the reasoning is the same.

Now, let us write $\sim$ to represent the equivalence between two extension
problems.  The Complex Clifford Algebra has a periodicity of $2$:
\begin{equation}
    \mathrm{C\ell}_{n+2}(\C) \cong \mathrm{C\ell}_{n}(\C)\otimes
    \mathrm{C\ell}_{2}(\C)
\end{equation}
The Real Clifford Algebra has a periodicity of $8$:
\begin{equation}
    \mathrm{C\ell}_{p+8,q}(\R) = \mathrm{C\ell}_{p,q+8}(\R) =
    \mathrm{C\ell}_{p,q}(\R)\otimes\mathcal{M}(16,\R)
\end{equation}
This two relation means that, the extension problem will be the same with respect to
a periodicity of $2$ for complex symmetry classes, and $8$ for real symmetry
classes. Hence the classification of topological insulators need only be done
with $d=1,2$ for complex symmetry classes, and $d=0,1,\cdots,7$ for real
symmetry classes, i.e.
\begin{subequations}
    \label{eq:cli-periodic}
\begin{align}
    G_{\#}(d=d_0) &\sim G_{\#}(d=(d_0 \text{mod} 2)) \quad\text{(complex class)} \\
    G_{\#}(d=d_0) &\sim G_{\#}(d=(d_0 \text{mod} 8)) \quad\text{(real class)}
\end{align}
\end{subequations}
Another useful property is:
\begin{align}
    \mathrm{C\ell}_{p+1,q+1}(\R) &\cong
    \mathrm{C\ell}_{p,q}(\R)\otimes\mathcal{M}(2,\R)
\end{align}
This property tells us that Clifford Algebra basically depends only on the
difference $p-q$, or combined with previous periodicity, $p-q$ (mod $8$). For
example, we have (let $n$ be an arbitrary integer):
$\mathrm{C\ell}_{p+1,q+1}(\R)\sim\mathrm{C\ell}_{p,q}(\R)$. With this, one can
derive two chains of relations:
\begin{align}
    &\mathrm{C\ell}_{1+(D+n),2+D}(\R) \sim \mathrm{C\ell}_{1+(D+1+n),3+D}(\R)
    \nonumber\\
    \sim & \mathrm{C\ell}_{(D+2+n),3+D}(\R) \sim \mathrm{C\ell}_{(D+3+n),4+D}(\R)
\end{align}
and
\begin{align}
    &\mathrm{C\ell}_{1+D,2+(D+4+n)}(\R)\sim \mathrm{C\ell}_{2+D,2+(D+5+n)}(\R)
    \nonumber\\
    \sim& \mathrm{C\ell}_{2+D,1+(D+6+n)}(\R) \sim
    \mathrm{C\ell}_{3+D,1+(D+7+n)}(\R)
\end{align}
Now we try to connect the two chains together. We note further that the
Clifford Algebra has the following properties:
\begin{align}
    \mathrm{C\ell}_{p+1,q}(\R) &\cong \mathrm{C\ell}_{q+1,p}(\R) \\
    \mathrm{C\ell}_{q,p+2}(\R) &\cong \mathrm{C\ell}_{p,q}(\R) \otimes\mathcal{M}(2,\R)
\end{align}
Then:
\begin{align}
    &\mathrm{C\ell}_{1+D,2+(D+4+n)}(\R) \cong
    \mathrm{C\ell}_{D+4+n,1+D}(\R)\otimes\mathcal{M}(2,\R) \nonumber\\
    \cong& \mathrm{C\ell}_{2+D,D+3+n}(\R)\otimes\mathcal{M}(2,\R)
    \cong \mathrm{C\ell}_{D+1+n,2+D}(\R)\otimes\mathcal{M}(4,\R)
\end{align}
Therefore,
$\mathrm{C\ell}_{1+D,2+(D+4+n)}(\R)\sim\mathrm{C\ell}_{1+(D+n),2+D}(\R)$,
connecting the two chains. If we compare this carefully with
Table~\ref{tab:map-sym-cl}, then we would realize that we actually managed to
prove the dimension-shift feature of classification:
\begin{align}
    \label{eq:cli-Chain1}
    G_{\mathrm{D}}(d=d_0) &\sim G_{\mathrm{DIII}}(d=d_0+1) \nonumber\\
    \sim G_{\mathrm{AII}}(d=d_0+2) &\sim G_{\mathrm{CII}}(d=d_0+3) \nonumber\\
    \sim G_{\mathrm{C}}(d=d_0+4) &\sim G_{\mathrm{CI}}(d=d_0+5) \nonumber\\
    \sim G_{\mathrm{AI}}(d=d_0+6) &\sim G_{\mathrm{BDI}}(d=d_0+7)
\end{align}
We need one more property:
\begin{equation}
    \mathrm{C\ell}_{1+d,2+D}(\R)\cong
    \mathrm{C\ell}_{3+D,d}(\R)=\mathrm{C\ell}_{3+D,1+(d-1)}(\R)
\end{equation}
Then
\begin{equation}
    \label{eq:cli-Chain2}
    G_{\mathrm{D}}(d=d_0) \sim G_{\mathrm{BDI}}(d=d_0-1)
\end{equation}
With equivalences~\ref{eq:cli-periodic},\ref{eq:cli-Chain1},\ref{eq:cli-Chain2},
we see that we need only the result in a dimension, to obtain the whole table of
classification of all real classes. Similar fact holds for complex classes:
\begin{equation}
    G_{\mathrm{AIII}}(d=d_0) \sim G_{\mathrm{A}}(d=d_0-1)
\end{equation}
which is obvious. 

In a word, our classification for topological insulators in $1$ spatial
dimension is sufficient to generate the whole Table~\ref{tab:master-table2}.
This is a remarkable consequence of $K$-theory.
\begin{table}
\begin{center}
\begin{tabular}{|c|cccccccc|ccc|}
\hline
\multicolumn{12}{|c|}{
The original classification table} \\ \hline
   $\mbox{AZ class} \backslash d$  & 0 & 1 & 2 & 3 & 4 & 5 & 6 & 7 & T & C & S  \\
\hline\hline
  A & $\mathbb{Z}$ & 0 & $\mathbb{Z}$ & 0 & $\mathbb{Z}$ & 0 & $\mathbb{Z}$ & 0             & 0 & 0 & 0    \\
  AIII & 0 & $\mathbb{Z}$ & 0 & $\mathbb{Z}$ & 0 & $\mathbb{Z}$ & 0 & $\mathbb{Z}$          & 0 & 0 & 1    \\  \hline

  AI & $\mathbb{Z}$ & 0 & 0 & 0 & $2\mathbb{Z}$ & 0 & $\mathbb{Z}_2$ & $\mathbb{Z}_2$    & $+$ & 0 & 0     \\
  BDI & $\mathbb{Z}_2$ & $\mathbb{Z}$ & 0 & 0 & 0 & $2\mathbb{Z}$ & 0 & $\mathbb{Z}_2$     & $+$ & $+$ & 1    \\
  D & $\mathbb{Z}_2$ & $\mathbb{Z}_2$ & $\mathbb{Z}$ & 0 & 0 & 0 & $2\mathbb{Z}$ & 0     & 0 & $+$ & 0     \\
  DIII & 0 & $\mathbb{Z}_2$ & $\mathbb{Z}_2$ & $\mathbb{Z}$ & 0 & 0 & 0 & $2\mathbb{Z}$  & $-$ & $+$ & 1     \\
  AII & $2\mathbb{Z}$ & 0 & $\mathbb{Z}_2$ & $\mathbb{Z}_2$ & $\mathbb{Z}$ & 0 & 0 & 0   & $-$ & 0 & 0     \\
  CII & 0 & $2\mathbb{Z}$ & 0 & $\mathbb{Z}_2$ & $\mathbb{Z}_2$ & $\mathbb{Z}$ & 0 & 0   & $-$ & $-$ & 1     \\
  C & 0 & 0 & $2\mathbb{Z}$ & 0 & $\mathbb{Z}_2$ & $\mathbb{Z}_2$ & $\mathbb{Z}$ & 0     & 0 & $-$ & 0     \\
  CI & 0 & 0 & 0 & $2\mathbb{Z}$ & 0 & $\mathbb{Z}_2$ & $\mathbb{Z}_2$ & $\mathbb{Z}$    & $+$ & $-$ & 1
   \\
\hline
\end{tabular}
\caption{The original classification table of topological insulators and
superconductors. Note that in this document we could not yet differentiate
between $\Z$ and $2\Z$ topological invariants.}
\label{tab:master-table2}
\end{center}
\end{table}

\section{Ten-fold Way}

We begin our discussion with a background before this classification.  The
classification of topological phases with different non-unitary symmetries in
non-interacting picture, are in essence, the classification of $N\times N$
matrices by its response to three discrete symmetries: time reversal symmetry
($T$), charge conjugation/particle hole symmetry ($C$), and chiral/sub-lattice
symmetry ($S$). As for those unitary symmetries, one can in principle, block
diagonalize the Hamiltonian
$H=\mathrm{diag}(H^{\lambda_1},H^{\lambda_2},\cdots)$, such that each block
$H^{\lambda_i}$ is labeled by irreducible representations $\lambda_i$, and it
has no memory of the unitary symmetries. Therefore, our classification is
applied to those blocks and still holds.

It will be find that, those $T,C$ are the only two meaningful non-unitary
symmetries of the Single-particle system, and there are in total 10 different
ways in which the Single-particle Hamiltonian could respect the three symmetries
$T,C,S$. Now we explain in detail, about the reason why we care about, and only
care about these three symmetries $T,C,$ and $S$.

\subsection{The symmetries of Single-particle Hamiltonian}
\label{sec:The symmetries of Single-particle Hamiltonian}

According to Wigner's theorem, symmetries of physical system can either be
unitary represented, or anti-unitarily represented. Here we show that in the
case of anti-unitary symmetries, there are only 2 different kinds of them that a
Single-particle Hamiltonian in first-quantized space can have. Actually, there
can be more anti-unitary symmetries than this. But under reasonable assumptions,
we can limit our discussion in only this two types.

The Single-particle Hamiltonian acting on Fock space is
\begin{equation}
\label{eq:H-2nd}
\hat{H} = \sum_{A,B} \hat\psi^\dagger_A H_{AB} \hat\psi_{B}
\end{equation}
where $H_{AB}$ are just complex numbers, $\hat\psi^\dagger_A$ and $\hat\psi_B$
are creation and annihilation operators acting on Fock space.

Here and henceforth, we will add a hat $\hat{}$ to all operators in Fock space
to stress that it acts on Second-quantized Fock space.  A symmetry of the
Hamiltonian, represented as an operator $\hat{U}$, must have:
\begin{equation}
\label{eq:sym-in-2nd-1}
\hat{U}\hat{H}\hat{U}^{-1} = \hat{H}
\end{equation}

We expect $\hat{U}$ to change the creation/annihilation operators in two different
ways. First, it may only permute the creation/annihilation operators:
\begin{subequations}
	\label{eq:sym-in-2nd-permute}
	\begin{align}
	\label{eq:sym-in-2nd-permute-1}
	\hat{\psi}'_A = 
	\hat{U} \hat\psi_A \hat{U}^{-1} &=
	\sum_B (u^\dagger)_{AB} \hat\psi_B \\
	\label{eq:sym-in-2nd-permute-2}
	\hat\psi'^\dagger_A =
	\hat{U} \hat\psi^\dagger_A \hat{U}^{-1} &= 
	\sum_B  \hat\psi^\dagger_B u_{AB}
	\end{align}
\end{subequations}
Where $u$ is some matrix implementing the permutation.
Or, it may interchange the role of creation/annihilation operators:
\begin{subequations}
	\label{eq:sym-cc}
	\begin{align}
	\label{eq:sym-cc-1}
	\hat{\psi}'_A = 
	\hat{U} \hat\psi_A \hat{U}^{-1} &=
	\sum_B (u^*)^\dagger_{AB} \hat\psi^\dagger_B \\
	\label{eq:sym-cc-2}
	\hat\psi'^\dagger_A =
	\hat{U} \hat\psi^\dagger_A \hat{U}^{-1} &= 
	\sum_B  \hat\psi_B u^*_{BA}
	\end{align}
\end{subequations}
Where $u^*$ is some other matrix implementing the interchange. We write complex
conjugation $u^*$ instead of $u$ for convenience. In both cases (permute or
interchange), to conserve the anticommutation relation between
$\hat\psi^\dagger_A$ and $\hat\psi_B$ operators, one can easily show that $u$
should be a unitary matrix.

Also, $\hat{U}$ may be linear or anti-linear. The different combinations of these
conditions give us 1 unitary symmetry, 2 anti-unitary symmetries, and 1 special
symmetry, to be explained below.

\paragraph{Case 1: Unitary Symmetry}

Assume the symmetry just permutes the creation/annihilation operators, as in
equation~\ref{eq:sym-in-2nd-permute}, and assume it is linear in
Second-quantized Hamiltonian, i.e. $\hat{U}i\hat{U}^{-1}=i$. The permutation
relation~\ref{eq:sym-in-2nd-permute} plugged into
equation~\ref{eq:sym-in-2nd-1}, gives
\begin{equation}
u H u^{-1} = H
\end{equation}
Therefore, in this case the symmetry is unitarily realized in First-quantized
Hamiltonian $H_{AB}$.

\paragraph{Case 2: Anti-unitary T Symmetry}
Assume the symmetry just permutes the creation/annihilation operators, as in
equation~\ref{eq:sym-in-2nd-permute}, but assume it is anti-linear in
Second-quantized Hamiltonian, i.e. $\hat{U}i\hat{U}^{-1}=-i$. The permutation
relation~\ref{eq:sym-in-2nd-permute} plugged into
equation~\ref{eq:sym-in-2nd-1}, gives a different result, since
$\hat{U}H_{AB}\hat{U}^{-1} = H_{AB}^*$ now.
\begin{equation}
u H^* u^\dagger = H
\end{equation}
Or:
\begin{equation}
uK H u^t K = uK H (uK)^{-1} = H
\end{equation}
where $K$ is complex conjugation. This symmetry is called Time-reversal
symmetry, and is realized in First-quantized Hamiltonian as an anti-unitary
operator $T= uK$.

\paragraph{Case 3: Anti-unitary C Symmetry}
Assume the symmetry just interchange the creation/annihilation operators, as in
equation~\ref{eq:sym-cc}, but assume it is linear in Second-quantized
Hamiltonian, i.e. $\hat{U}i\hat{U}^{-1}=i$. The interchange
relation~\ref{eq:sym-cc} plugged into equation~\ref{eq:sym-in-2nd-1}, gives:
\footnote{
	Note that in calculating this, $\sum_i\hat\psi^\dagger_i \hat\psi_i =
	\id$ on 1st-quantized single-particle Hilbert space.
}
\begin{equation}
\label{eq:sym-C-cond}
u (H-\frac{1}{2}\tr(H))^t u^\dagger = - (H-\frac{1}{2}\tr(H))
\end{equation}

Taking the trace of above equality will give $2\tr(H)=N\tr(H)$, since
in solids $N>>2$, we must have $\tr(H)=0$. Then the above equality
simplifies into \footnote{note that $H^t = H^*$ for Hermitian $H$}:
\begin{equation}
u H^* u^\dagger = uK H (uK)^{-1} = -H
\end{equation}
This type of symmetry is called charge-conjugation symmetry. It is
also called particle-hole symmetry in condensed matter physics. It is realized
as $C=uK$ with $C H C^{-1} = -H$ for 1st-quantized single-particle Hamiltonian.

\paragraph{Case 4: Unitary S Symmetry}
Assume the symmetry now interchange the creation/annihilation operators, as in
equation~\ref{eq:sym-cc}, and assume it is anti-linear in Second-quantized
Hamiltonian, i.e. $\hat{U}i\hat{U}^{-1}=-i$. The interchange
relation~\ref{eq:sym-cc} plugged into equation~\ref{eq:sym-in-2nd-1}, gives:
\begin{equation}
\label{eq:sym-S-cond}
u (H-\frac{1}{2}\tr(H)) u^\dagger = - (H-\frac{1}{2}\tr(H))
\end{equation}
and since $N>>2$, we have $\tr(H)=0$. Then
\begin{equation}
u H u^\dagger = -H
\end{equation}
This symmetry will be called the chiral symmetry, denoted $\hat{S}=u$. It is
unitarily realized in First-quantized Hamiltonian, but since $\{S,H\}=0$ instead
of $[S,H]=0$, it is not a traditional symmetry that we are used to. Also, it is
easy to see that $S$ is a combination of $T$ and $C$,\footnote{In fact, we could have
	defined $S=TC$ or $S=CT$, and obtained the same result.} and the symmetry
property of Hamiltonian under $T$ or $C$ uniquely defined the symmetry property
of Hamiltonian under $S$. But there is one exception. When the
Hamiltonian does not obey $T$ and $C$, it may or may not obey $S=TC$ as a
whole.

\paragraph{Squaring of $T,C,S$} 
Squaring of $T/C$ symmetry operators should be proportional to identity $\id$,
hence it is only a phase as they are unitary.
This can be viewed from two perspectives. First, we expect the system, after
applying twice of symmetry operation $T/C$, should come back to the same state, except
a possible phase difference. Second, $T^2/C^2$ commutes with all
$T/C$ symmetric Hamiltonians (easy to derive) in all irreducible representations
of unitary symmetries (to be explained later), therefore by Schur's lemma, they
must be proportional to a constant, which is a phase.

It is easy to find that this phase $e^{i\delta}$should be $\pm1$. For example, let $T=uK$,
then $T^2=uu^*$, and $(uu^*)u = u(u^*u)$ gives us $e^{2i\delta}=1$, hence
$e^{i\delta}=\pm1$.

But the square of $S$ is tricky. Since $S=TC=u_T u_C^*$, and each unitary matrix
$u_T$ and $u_C$ has a phase freedom (as they acts on creation/annihilation
operators), we can always pick a phase such that $S^2$ is some phase we want.
Sometimes we pick $S^2=1$. Sometime we want $\{T,C\}=0$, and pick some specific
phase for $S^2$.\footnote{This is a bit troublesome. First we denote
	$T^2=\varepsilon_T$, $C^2=\varepsilon_C$, then it can be found that
	$(u_T)^t=\varepsilon_T u_T$, $(u_C)^t = \varepsilon_C u_C$. Then, with
	$S=u_Tu_C^*$, we have $S^\dagger=\varepsilon_C\varepsilon_Tu_Cu_T^* =
	\varepsilon_C\varepsilon_T CT$, or $CT= \varepsilon_C\varepsilon_TS^\dagger$.
	Since $TC+CT=0$, we have $S+\varepsilon_C\varepsilon_T S^\dagger=0$, or
	$S^2=-\varepsilon_C\varepsilon_T$.}

\paragraph{Why Ten Classes} For the three unitary symmetries under
consideration, as will be mentioned, $S$ symmetry is a combination of $T$ and
$C$, $S=TC$. It will be obtained that $T^2=\pm 1$, $C^2=\pm 1$, and $S^2$ is
undetermined (depending on the phase choice\footnote{To be explained later.} we
give for $T$ and $C$). Therefore, there are in total 10 different possible ways
that this 3 symmetries could be combined together: We denote $T^2=0$ ($C^2=0$)
to symbolize that the system does not follow $T$($C$) symmetry. Then
$T^2=0,\pm1$ and $C^2=0,\pm1$ gives us $3\times 3=9$ different ways. But if
$T^2=C^2=0$, $S$ may or may not conserve. Therefore, we have $9-1+2=10$
different possible ways. In each way the topological invariants of the
Hamiltonian respecting these symmetries are tabulated in Figure.1
in\cite{Ludwig2016}. Each class is given a name ($\mathrm{A}$, $\mathrm{AII}$,
etc.), and their properties in different spatial dimensions respect a periodic
structure. For the class $A$ and class $AII$, they are named complex classes and
respect a $2$-fold period. For the other eight classes are named real classes
and respect a $8$-fold period, which is why we only listed the first $0\to 7$
spatial dimensions in Table~\ref{tab:master-table2}.

\subsection{Dealing with Unitary Symmetries}
\label{sec:Dealing with Unitary Symmetries}

As mentioned, we could module out those unitary symmetries. This statement is
made precise by the following theorem about unitary symmetries of the Hamiltonian.
\begin{thm}[Diagonalization of Hamiltonian in unitary representation]
	This space $\mathcal{V}$ decomposes into a direct sum of vector spaces
	$\mathcal{V}_\lambda$ associated with the irrep (irreducible
	representations, labeled by $\lambda$) of $G_0$.
	\begin{equation}
	\mathcal{V} = \oplus_\lambda m_\lambda \mathcal{V}_\lambda
	\end{equation}
	where $m_\lambda$ denotes the multiplicity of $\lambda$th irrep.
	Denote the dimension of each irrep as $d_\lambda$.
	
	In each vector space $\mathcal{V}_\lambda$, one can choose a
	(orthogonal) basis of the form:
	\begin{equation}
	\ket{v^{(\lambda)}_\alpha} \otimes \ket{w^{(\lambda)}_k}
	\end{equation}
	where
	\begin{itemize}
		\item $G_0$ acts only only $\ket{w^{(\lambda)}}_k$,
		$k=1,\cdots,d_\lambda$,
		\item $H$ acts only on $\ket{v^{(\lambda)}}_\alpha$,
		$\alpha=1,\cdots,m_\lambda$.
	\end{itemize}
\end{thm}
Therefore, with all unitary symmetries ignored, we are classifying how
Hamiltonian will be like when it respects, in 10 different ways, the
combinations of $T,C,S$ symmetries.

\paragraph{Differently Realized Anti-unitary Symmetries}
It should be noted that there is still a freedom of unitary matrix that
implements the $T$ or $C$ symmetry. Therefore, we could have, for example, two
different time-reversal symmetries $T_1 = u_1K$ and $T_2= u_2 K$. They are
obviously related by a unitary matrix, say $T_1 = u_{12} T_2$. And we could
easily get that $u_{12}$ is also a unitary symmetry of the Hamiltonian (if the
Hamiltonian respect both $T_1$ and $T_2$). Therefore, upon enlarging the
symmetry group $G_0$ to include the element $u_{12}$ and repeat the process
described in the theorem above, we only have one time-reversal symmetry. Similar
analysis could be done for the $C$ symmetry as well.

\paragraph{The Tenfold Way}

The Tenfold Way, is originally the ten different ways to write down a
Hamiltonian when there are ten different combinations of symmetries that the 
system respect. The classes of Hamiltonian (or more precisely, the evolution
operators $e^{itH}$) are tabulated in Figure.1 of \cite{Ludwig2016}. Ludwig
classified the topological insulators in each spatial dimension by computing the
homotopy groups for each dimension. On the other hand, this classification can
also be done using $K$-theory.

\paragraph{Translational Symmetry}

Since we are classifying in the solids, it is natural to expect translational
symmetry to present. Also, the addition of translational symmetry not only
produces the same classification, but also make this classification more
convenient to be done.

With translational symmetry added, the classification can be made the will yield
the same result.\footnote{This is still a mystery to me which this is true. The
	exact reason is related to some isomorphism between certain homotopy groups.
	Interested reader could see Ludwig's lecture
	\href{http://boulderschool.yale.edu/2016/boulder-school-2016-lecture-notes}{here}
	for a clue.}

The Hamiltonian in $k$-space that preserve these three discrete symmetries
should follow\footnote{It is easy to anticipate the change $k\to -k$ for
	anti-unitary symmetries, since they change the phase factor when one Fourier
	transform the Hamiltonian.}:
\begin{subequations}
	\begin{align}
	\label{eq:T-sym-Hk}
	TH(k)T^{-1} &= H(-k) \\
	\label{eq:C-sym-Hk}
	CH(k)C^{-1} &= -H(-k) \\
	\label{eq:S-sym-Hk}
	SH(k)S^{-1} &= -S(k)
	\end{align}
\end{subequations}


\section{Looking Further}
\label{sec:Looking Further}
There has be a great deal of progress made in classification of topological
insulator and superconductors. We first comment that our discussion could be
easily generalized to some topological superconductors because they naturally
admits a matrix like Hamiltonian using Nambu spinors.

An unmentioned hypothesis above is that we
are classifying strong topological insulators. They are topological insulators
that are robust against disorder\cite{Fu2007}. There is a consensus that in
classification of strong topological insulators, we replace the Brillouin Zone
$T^d$ by $S^d$\footnote{The exact reason can be found in Kitaev's
 classification\cite{Kitaev2009a} or another article \cite{Kennedy2015} which
discuss it from a different view.}, and we do it in a stable way, i.e., we choose
our classification to be independent of and insensitive to the addition of
irrelevant trivial bands. The first replacement allow us to use homotopy groups,
which classify maps with domain in spheres. The "stable way" allows the use of
$K$-theory (sec.III.C.1 of \cite{Chiu2016}).

However, our discussion falls short in two respects. First, it cannot distinguish
between $\Z$ and $2\Z$ cases. Secondly, we do not give explicit formulae for
calculating topological invariants in different cases. The former problem can be
addressed in Ludwig's \cite{Schnyder2008}, whereas the latter requires
considerable work, and requires different topological invariant in different
cases (see sec. III.B of \cite{Chiu2016}).

There has been progress made both in incorporating defects into classification,
as well as incorporating unitary symmetries into the classification. The
defect's classification has been incorporated beautifully into the same table of
Tenfold Way. On the other hand, introduction of unitary symmetries gives a more
fruitful result. For example, introduction of crystal symmetries gives a new
class of topological insulators, dubbed topological crystalline
insulators.\cite{Ando2015} In the case of reflection symmetry, some new topological
invariants denoted as $M\Z$, called mirror topological invariants, are needed to
classify topological matters\cite{Chiu2013}.

Lastly, the discussion of topological states in interaction picture extends to
the case of symmetry-protected phases (SPTs), with which I am not familiar.
For all these new concepts, the review \cite{Chiu2016} is an excellent source of
information.




% The entire content of this work (including the source code
% for TeX files and the generated PDF documents) by 
% Hongxiang Chen (nicknamed we.taper, or just Taper) is
% licensed under a 
% Creative Commons Attribution-NonCommercial-ShareAlike 4.0 
% International License (Link to the complete license text:
% http://creativecommons.org/licenses/by-nc-sa/4.0/).
\documentclass{article}

% My own physics package
% The following line load the package xparse with additional option to
% prevent the annoying warnings, which are caused by the package
% "physics" loaded in package "physicist-taper".
\usepackage[log-declarations=false]{xparse}
\usepackage{physicist-taper}
\usepackage{fontspec}
\setmainfont{Times New Roman}

\makenomenclature % For an index of symbols.

\title{Notes about the Minimal Dimensional Method to Classify
Topological Phases (preliminary)}
\date{\today}
\author{Taper}


\begin{document}


\maketitle
\abstract{
    The abstract is left blank due of unforeseeable changes that may happen to
    this document.
}
\tableofcontents

\section{Outline}
\label{sec:Outline}
 
The classification of topological insulators in non-interacting system is in
effect a classification of $N\times N$ Hamiltonian matrix. %sym
The simple case considered here
is the classification done by considering how the Hamiltonian matrix responses
to those anti-unitary symmetries, when we effectively ignored all unitary
symmetries. Specifically, when we module out any unitary symmetries of the
Hamiltonian, the number of meaningful anti-unitary symmetries are limited, and
we only classify topological insulators based on how our Hamiltonian matrix
responses to these anti-unitary symmetries.

There are two classical approaches towards this classification. The first
\cite{Schnyder2008} is to examine the existence of Anderson delocalization at
the boundary of the insulator. The existence of robust surface conducting state
is the signature of topological insulators. When some random impurity potentials
could be present in these surfaces, the Anderson localization is the obstruction
that surface conducting states must overcome. This approach uses the Nonlinear
Sigma Models (NL$\sigma$Ms) to describe the surface Hamiltonian and consider the
when can a localization-breaking term can be added to NL$\sigma$Ms.

The second approach\cite{Kitaev2009a} is to use see the classification as an
extension problem of Clifford Algebras. The anti-unitary symmetries already
possess some behavior similar to generators of Clifford Algebras. And sometimes
Hamiltonians could be regarded as yet another generator of Clifford
Algebra\cite{Morimoto2013}. The $K$-theory approach is to put the symmetries in
the Clifford Algebra first, and consider adding the Hamiltonian that preserve
such symmetries to the Clifford Algebra. So the problem of classification of
Hamiltonians becomes a problem of extending the Clifford Algebra.

But the key to understand topological insulators, are not necessarily the
Hamiltonian of the whole bulk, but the physics happening when the bulk gap
closes. For example, the Chern number is the integration on a compact manifold
without boundary. Therefore, it should always be zero because of the Stokes
theorem, unless inside this manifold there is somewhere where the formula for
Chern number got "blown up", and it is just this locally "blown up" point
(caused by the crossing) that is responsible for the nonzero Chern number, and
for the non-triviality of topological insulators. More generally, we believe
that there is a bulk-boundary correspondence that the properties of the bulk is
reflected by the properties in the boundary.

Therefore, we can detect the different types of topological insulators when we
are focused in only the crossing point of the band, or, when we shifted the
energy zero level to the crossing point, in only the low energy physics. We
expect a locally linear crossing energy spectrum around this point, so we expect
$E=\pm\sqrt{\vb{p}^2+m^2}$, with $m$ being the parameter controlling the closing or
opening of this spectrum. One simple model with only first-order derivatives to
describe such a low energy physics is the Dirac Hamiltonian. Also, the
traditional discussion of Klein-Gordon equation all tells us to look at Dirac
Hamiltonian.

Using the Dirac Hamiltonian, it is discovered by Dr. Chiu \cite{Chiu2013a} that
the classification of topological insulators could be brought done into very
simple applications of classification and representation theory of Clifford
Algebras, which have been well studied with excellent resources provided online.
In his classification, he established a isomorphism from $G_{\text{symmetries 
class}}$, the algebra generated by symmetry operators and Dirac Hamiltonians, to
some Clifford Algebras. Consequently, the Hamiltonian and symmetry operators,
all become some specific representations of their corresponding Clifford
Algebras. Also, it will be argued later that the minimal matrix dimension of
such representation, i.e. the minimal dimension of the Dirac Hamiltonian, can
tell the specific type of this topological insulator. Therefore, as a whole all
the classification work is reduced to the consideration of minimal matrix
dimension of the representation of Clifford Algebra, in different spatial
dimension.

This work may be considered as a note to Chiu's dissertation\cite{Chiu2013a}.
However, I do not guarantee that my interpretation is the same as his as I have
modified and amended some parts to make the work more, in my opinion, congruous
and systematic.

\paragraph{Structure of this work} This work are divided into several parts.
\begin{enumerate}
    \item \textbf{Ten-fold Way} Introduces the result of Ten-fold way: why we
        care only about anti-unitary symmetries, why there are ten different
        classes.
    \item \textbf{Classification by Dirac Mass Hamiltonians} gives comprehensive
        account of how the Minimal Dirac Hamiltonian approach works, and how the
        ten-fold way is derived in this approach.
    \item \textbf{Looking Further} outlines the possible direction in classification
        under unitary symmetries, and some other results.
\end{enumerate}

\section{Ten-fold Way}

We begin our discussion with a background before this classification.  The
classification of topological phases with different non-unitary symmetries in
non-interacting picture, are in essence, the classification of $N\times N$
matrices by its response to three discrete symmetries: time reversal symmetry
($T$), charge conjugation/particle hole symmetry ($C$), and chiral/sub-lattice
symmetry ($S$). As for those unitary symmetries, one can in principle, block
diagonalize the Hamiltonian
$H=\mathrm{diag}(H^{\lambda_1},H^{\lambda_2},\cdots)$, such that each block
$H^{\lambda_i}$ is labeled by irreducible representations $\lambda_i$, and it
has no memory of the unitary symmetries. Therefore, our classification is
applied to those blocks and still holds.

It will be find that, those $T,C$ are the only two meaningful non-unitary
symmetries of the Single-particle system, and there are in total 10 different
ways in which the Single-particle Hamiltonian could respect the three symmetries
$T,C,S$. Now we explain in detail, about the reason why we care about, and only
care about these three symmetries $T,C,$ and $S$.

\subsection{The symmetries of Single-particle Hamiltonian}
\label{sec:The symmetries of Single-particle Hamiltonian}

According to Wigner's theorem, symmetries of physical system can either be
unitary represented, or anti-unitarily represented. Here we show that in the
case of anti-unitary symmetries, there are only 2 different kinds of them that a
Single-particle Hamiltonian in first-quantized space can have. Actually, there
can be more anti-unitary symmetries than this. But under reasonable assumptions,
we can limit our discussion in only this two types.

The Single-particle Hamiltonian acting on Fock space is
\begin{equation}
    \label{eq:H-2nd}
    \hat{H} = \sum_{A,B} \hat\psi^\dagger_A H_{AB} \hat\psi_{B}
\end{equation}
where $H_{AB}$ are just complex numbers, $\hat\psi^\dagger_A$ and $\hat\psi_B$
are creation and annihilation operators acting on Fock space.

Here and henceforth, we will add a hat $\hat{}$ to all operators in Fock space
to stress that it acts on Second-quantized Fock space.  A symmetry of the
Hamiltonian, represented as an operator $\hat{U}$, must have:
\begin{equation}
    \label{eq:sym-in-2nd-1}
    \hat{U}\hat{H}\hat{U}^{-1} = \hat{H}
\end{equation}

We expect $\hat{U}$ to change the creation/annihilation operators in two different
ways. First, it may only permute the creation/annihilation operators:
\begin{subequations}
\label{eq:sym-in-2nd-permute}
\begin{align}
    \label{eq:sym-in-2nd-permute-1}
    \hat{\psi}'_A = 
    \hat{U} \hat\psi_A \hat{U}^{-1} &=
    \sum_B (u^\dagger)_{AB} \hat\psi_B \\
    \label{eq:sym-in-2nd-permute-2}
    \hat\psi'^\dagger_A =
    \hat{U} \hat\psi^\dagger_A \hat{U}^{-1} &= 
    \sum_B  \hat\psi^\dagger_B u_{AB}
\end{align}
\end{subequations}
Where $u$ is some matrix implementing the permutation.
Or, it may interchange the role of creation/annihilation operators:
\begin{subequations}
\label{eq:sym-cc}
\begin{align}
    \label{eq:sym-cc-1}
    \hat{\psi}'_A = 
    \hat{U} \hat\psi_A \hat{U}^{-1} &=
    \sum_B (u^*)^\dagger_{AB} \hat\psi^\dagger_B \\
    \label{eq:sym-cc-2}
    \hat\psi'^\dagger_A =
    \hat{U} \hat\psi^\dagger_A \hat{U}^{-1} &= 
    \sum_B  \hat\psi_B u^*_{BA}
\end{align}
\end{subequations}
Where $u^*$ is some other matrix implementing the interchange. We write complex
conjugation $u^*$ instead of $u$ for convenience. In both cases (permute or
interchange), to conserve the anticommutation relation between
$\hat\psi^\dagger_A$ and $\hat\psi_B$ operators, one can easily show that $u$
should be a unitary matrix.

Also, $\hat{U}$ may be linear or anti-linear. The different combinations of these
conditions give us 1 unitary symmetry, 2 anti-unitary symmetries, and 1 special
symmetry, to be explained below.

\paragraph{Case 1: Unitary Symmetry}

Assume the symmetry just permutes the creation/annihilation operators, as in
equation~\ref{eq:sym-in-2nd-permute}, and assume it is linear in
Second-quantized Hamiltonian, i.e. $\hat{U}i\hat{U}^{-1}=i$. The permutation
relation~\ref{eq:sym-in-2nd-permute} plugged into
equation~\ref{eq:sym-in-2nd-1}, gives
\begin{equation}
    u H u^{-1} = H
\end{equation}
Therefore, in this case the symmetry is unitarily realized in First-quantized
Hamiltonian $H_{AB}$.

\paragraph{Case 2: Anti-unitary T Symmetry}
Assume the symmetry just permutes the creation/annihilation operators, as in
equation~\ref{eq:sym-in-2nd-permute}, but assume it is anti-linear in
Second-quantized Hamiltonian, i.e. $\hat{U}i\hat{U}^{-1}=-i$. The permutation
relation~\ref{eq:sym-in-2nd-permute} plugged into
equation~\ref{eq:sym-in-2nd-1}, gives a different result, since
$\hat{U}H_{AB}\hat{U}^{-1} = H_{AB}^*$ now.
\begin{equation}
    u H^* u^\dagger = H
\end{equation}
Or:
\begin{equation}
    uK H u^t K = uK H (uK)^{-1} = H
\end{equation}
where $K$ is complex conjugation. This symmetry is called Time-reversal
symmetry, and is realized in First-quantized Hamiltonian as an anti-unitary
operator $T= uK$.

\paragraph{Case 3: Anti-unitary C Symmetry}
Assume the symmetry just interchange the creation/annihilation operators, as in
equation~\ref{eq:sym-cc}, but assume it is linear in Second-quantized
Hamiltonian, i.e. $\hat{U}i\hat{U}^{-1}=i$. The interchange
relation~\ref{eq:sym-cc} plugged into equation~\ref{eq:sym-in-2nd-1}, gives:
\footnote{
Note that in calculating this, $\sum_i\hat\psi^\dagger_i \hat\psi_i =
\id$ on 1st-quantized single-particle Hilbert space.
}
\begin{equation}
    \label{eq:sym-C-cond}
    u (H-\frac{1}{2}\tr(H))^t u^\dagger = - (H-\frac{1}{2}\tr(H))
\end{equation}

Taking the trace of above equality will give $2\tr(H)=N\tr(H)$, since
in solids $N>>2$, we must have $\tr(H)=0$. Then the above equality
simplifies into \footnote{note that $H^t = H^*$ for Hermitian $H$}:
\begin{equation}
    u H^* u^\dagger = uK H (uK)^{-1} = -H
\end{equation}
This type of symmetry is called charge-conjugation symmetry. It is
also called particle-hole symmetry in condensed matter physics. It is realized
as $C=uK$ with $C H C^{-1} = -H$ for 1st-quantized single-particle Hamiltonian.

\paragraph{Case 4: Unitary S Symmetry}
Assume the symmetry now interchange the creation/annihilation operators, as in
equation~\ref{eq:sym-cc}, and assume it is anti-linear in Second-quantized
Hamiltonian, i.e. $\hat{U}i\hat{U}^{-1}=-i$. The interchange
relation~\ref{eq:sym-cc} plugged into equation~\ref{eq:sym-in-2nd-1}, gives:
\begin{equation}
    \label{eq:sym-S-cond}
    u (H-\frac{1}{2}\tr(H)) u^\dagger = - (H-\frac{1}{2}\tr(H))
\end{equation}
and since $N>>2$, we have $\tr(H)=0$. Then
\begin{equation}
    u H u^\dagger = -H
\end{equation}
This symmetry will be called the chiral symmetry, denoted $\hat{S}=u$. It is
unitarily realized in First-quantized Hamiltonian, but since $\{S,H\}=0$ instead
of $[S,H]=0$, it is not a traditional symmetry that we are used to. Also, it is
easy to see that $S$ is a combination of $T$ and $C$,\footnote{In fact, we could have
defined $S=TC$ or $S=CT$, and obtained the same result.} and the symmetry
property of Hamiltonian under $T$ or $C$ uniquely defined the symmetry property
of Hamiltonian under $S$. But there is one exception. When the
Hamiltonian does not obey $T$ and $C$, it may or may not obey $S=TC$ as a
whole.

\paragraph{Squaring of $T,C,S$} 
Squaring of $T/C$ symmetry operators should be proportional to identity $\id$,
hence it is only a phase as they are unitary.
This can be viewed from two perspectives. First, we expect the system, after
applying twice of symmetry operation $T/C$, should come back to the same state, except
a possible phase difference. Second, $T^2/C^2$ commutes with all
$T/C$ symmetric Hamiltonians (easy to derive) in all irreducible representations
of unitary symmetries (to be explained later), therefore by Schur's lemma, they
must be proportional to a constant, which is a phase.

It is easy to find that this phase $e^{i\delta}$should be $\pm1$. For example, let $T=uK$,
then $T^2=uu^*$, and $(uu^*)u = u(u^*u)$ gives us $e^{2i\delta}=1$, hence
$e^{i\delta}=\pm1$.

But the square of $S$ is tricky. Since $S=TC=u_T u_C^*$, and each unitary matrix
$u_T$ and $u_C$ has a phase freedom (as they acts on creation/annihilation
operators), we can always pick a phase such that $S^2$ is some phase we want.
Sometimes we pick $S^2=1$. Sometime we want $\{T,C\}=0$, and pick some specific
phase for $S^2$.\footnote{This is a bit troublesome. First we denote
$T^2=\varepsilon_T$, $C^2=\varepsilon_C$, then it can be found that
$(u_T)^t=\varepsilon_T u_T$, $(u_C)^t = \varepsilon_C u_C$. Then, with
$S=u_Tu_C^*$, we have $S^\dagger=\varepsilon_C\varepsilon_Tu_Cu_T^* =
\varepsilon_C\varepsilon_T CT$, or $CT= \varepsilon_C\varepsilon_TS^\dagger$.
Since $TC+CT=0$, we have $S+\varepsilon_C\varepsilon_T S^\dagger=0$, or
$S^2=-\varepsilon_C\varepsilon_T$.}

\paragraph{Why Ten Classes} For the three unitary symmetries under
consideration, as will be mentioned, $S$ symmetry is a combination of $T$ and
$C$, $S=TC$. It will be obtained that $T^2=\pm 1$, $C^2=\pm 1$, and $S^2$ is
undetermined (depending on the phase choice\footnote{To be explained later.} we
give for $T$ and $C$). Therefore, there are in total 10 different possible ways
that this 3 symmetries could be combined together: We denote $T^2=0$ ($C^2=0$)
to symbolize that the system does not follow $T$($C$) symmetry. Then
$T^2=0,\pm1$ and $C^2=0,\pm1$ gives us $3\times 3=9$ different ways. But if
$T^2=C^2=0$, $S$ may or may not conserve. Therefore, we have $9-1+2=10$
different possible ways. In each way the topological invariants of the
Hamiltonian respecting these symmetries are tabulated in Figure.1
in\cite{Ludwig2015}. Each class is given a name ($\mathrm{A}$, $\mathrm{AII}$,
etc.), and their properties in different spatial dimensions respect a periodic
structure. For the class $A$ and class $AII$, they are named complex classes and
respect a $2$-fold period. For the other eight classes are named real classes
and respect a $8$-fold period, which is why we only listed the first $0\to 7$
spatial dimensions in Table~\ref{tab:master-table2}.

\subsection{Dealing with Unitary Symmetries}
\label{sec:Dealing with Unitary Symmetries}

As mentioned, we could module out those unitary symmetries. This statement is
made precise by the following theorem about unitary symmetries of the Hamiltonian.
\begin{thm}[Diagonalization of Hamiltonian in unitary representation]
This space $\mathcal{V}$ decomposes into a direct sum of vector spaces
$\mathcal{V}_\lambda$ associated with the irrep (irreducible
representations, labeled by $\lambda$) of $G_0$.
\begin{equation}
    \mathcal{V} = \oplus_\lambda m_\lambda \mathcal{V}_\lambda
\end{equation}
where $m_\lambda$ denotes the multiplicity of $\lambda$th irrep.
Denote the dimension of each irrep as $d_\lambda$.

In each vector space $\mathcal{V}_\lambda$, one can choose a
(orthogonal) basis of the form:
\begin{equation}
    \ket{v^{(\lambda)}_\alpha} \otimes \ket{w^{(\lambda)}_k}
\end{equation}
where
\begin{itemize}
    \item $G_0$ acts only only $\ket{w^{(\lambda)}}_k$,
        $k=1,\cdots,d_\lambda$,
    \item $H$ acts only on $\ket{v^{(\lambda)}}_\alpha$,
        $\alpha=1,\cdots,m_\lambda$.
\end{itemize}
\end{thm}
Therefore, with all unitary symmetries ignored, we are classifying how
Hamiltonian will be like when it respects, in 10 different ways, the
combinations of $T,C,S$ symmetries.

\paragraph{Differently Realized Anti-unitary Symmetries}
It should be noted that there is still a freedom of unitary matrix that
implements the $T$ or $C$ symmetry. Therefore, we could have, for example, two
different time-reversal symmetries $T_1 = u_1K$ and $T_2= u_2 K$. They are
obviously related by a unitary matrix, say $T_1 = u_{12} T_2$. And we could
easily get that $u_{12}$ is also a unitary symmetry of the Hamiltonian (if the
Hamiltonian respect both $T_1$ and $T_2$). Therefore, upon enlarging the
symmetry group $G_0$ to include the element $u_{12}$ and repeat the process
described in the theorem above, we only have one time-reversal symmetry. Similar
analysis could be done for the $C$ symmetry as well.

\paragraph{Translational Symmetry}

Since we are classifying in the solids, it is natural to expect translational
symmetry to present. Also, the addition of translational symmetry do not alter
our classification, which is based on non-unitary symmetries, but also make this
classification more convenient to be done.

With translational symmetry added, the classification can be made the will yield
the same result.\footnote{This is still a mystery to me which this is true. The
    exact reason is related to some isomorphism between certain homotopy groups.
    Interested reader could see Ludwig's lecture
\href{http://boulderschool.yale.edu/2016/boulder-school-2016-lecture-notes}{here}
for a clue.}

The Hamiltonian in $k$-space that preserve these three discrete symmetries
should follow\footnote{It is easy to anticipate the change $k\to -k$ for
    anti-unitary symmetries, since they change the phase factor when one Fourier
transform the Hamiltonian.}:
\begin{subequations}
\begin{align}
    \label{eq:T-sym-Hk}
    TH(k)T^{-1} &= H(-k) \\
    \label{eq:C-sym-Hk}
    CH(k)C^{-1} &= -H(-k) \\
    \label{eq:S-sym-Hk}
    SH(k)S^{-1} &= -S(k)
\end{align}
\end{subequations}

\subsection{The Tenfold Way}
\label{sec:The Tenfold Way}

The Tenfold Way, is originally the ten different ways to write down a
Hamiltonian when there are ten different combinations of symmetries that the 
system respect.
\todo{Add the table, and the meaning of each Hamiltonian, and the class
of each evolutionary operators, if I have time}

\section{Classification by Dirac Mass Terms}
\label{sec:The Minimal Hamiltonian Approach}

\paragraph{Dirac Hamiltonian and Topological Property of the Bulk}
We first note that different topological phases are distinguished by the closing
and then opening of the gap, i.e. a quantum phase transition. If we adiabatically
flatten the energy spectrum to two bands of energy $\pm E$, the transition
(closing of gap) will happen at some point where locally the energy is of a
Dirac cone type. Therefore, the Dirac Hamiltonian captures this transition
behavior. Then, a class of topological phases, or more specifically, a class of
Hamiltonian matrix that shares the same topological invariant, can be
\textit{represented} by the Dirac Hamiltonian in this class.

\subsection{Crossing Spectrum and Edge Mode}
\label{sec:Crossing Spectrum and Edge Mode}
Here we show that the crossing spectrum characteristic of topological insulators
can be characterized by a Dirac type Hamiltonian, and with this Dirac
Hamiltonian we could find an Edge mode solution built into it.

\paragraph{Crossing Spectrum and Dirac Hamiltonian}
In some local region around the crossing point, the energy spectrum should look
like:
\begin{equation}
    E = \pm \sqrt{\vb{k}^2 + M^2}
\end{equation}
where $\vb{k}$ is the crystal momentum and $M$ is some constant
responsible for the opening and closing of the gap, or in some cases, a parameter
in which we can control to open of close the gap.

As has been analysed in most Quantum Mechanics textbook\footnote{See for
    example, p.99 of \cite{Greiner1997}. Although their discussion is in
dimension $d=3$, it is easy to generalize it to any dimension.}, this energy-momentum
relationship requires a Dirac type Hamiltonian. Then, the Dirac Hamiltonian we
use is:
\begin{equation}
    \label{eq:dirac-H}
    H_\text{Dirac} = M\gamma_0 + \sum_{i=1}^d k_i\gamma_i
\end{equation}
where $d$ is the spatial dimension of the physical system, $\gamma_i$ obeys the
Clifford Algebra anti-commutation relation:
\begin{equation}
    \{\gamma_i,\gamma_j\} = 2\delta_{i,j}\id
\end{equation}

\paragraph{Edge Mode and \texorpdfstring{$M$}{}}
Now we explain why the Dirac Hamiltonian\ref{eq:dirac-H}  is useful for capturing the topological
phase. First, it has a gapless state. Due to the anti-commutation relation of
$\gamma_i$, we can show\footnote{Note that $H_\text{Dirac}^2= E^2\id$, this
tells us it can only have eigenvalues $E_\pm$. Note that this also explains why
we choose not $\gamma_i^2=-1$, which would give non-sensible result
$H_\text{Dirac}^2= -E^2\id$.}
that the energies of Dirac Hamiltonian are:
\begin{equation}
    E_\pm = \pm\sqrt{M^2+\sum_{i=1}^d k^2_i}
\end{equation}
Therefore, as $M$ varies from negative to $0$ to positive, the system goes from
gaped to gapless and again to gaped state. This shows that the $M$ is parameter
controlling the quantum phase transition, and the system with $M<0$ or $M>0$
possesses two different quantum phase.

In addition, we can analyse its edge mode. Assuming the two material of
different quantum phases touches each other in the $d$th direction. Then we
should replace $k_d$ by $-i\partial_{x_d}$, since the translational invariance is
broken in that direction. Now the eigenvalue equation is:
\begin{equation}
    ( M\gamma_0 + \sum_{i=1}^{d-1} k_i\gamma_i - i\partial_{x_d}\gamma_d ) \Phi
    = E \Phi
\end{equation}

Past experience teaches us the ansatz (assuming
$M(x_d)$ goes to $\pm\infty$ as $x_d\to\pm\infty$, so that the leading term below
does not blow up the wavefunction):
\begin{equation}
    \Phi = e^{-\int_0^{x_d} M(x'_d)\dd{x'_d}} \phi(x_1,\cdots,x_{d-1})
\end{equation}
which after plugging inside the eigenvalue equation, leads to
\begin{equation}
    \left( M(x_d)(\gamma_0 + i\gamma_d) + \sum_{i=1}^{d-1} k_i\gamma_i \right) \phi
    = E \phi
\end{equation}
After left multiplication of $\gamma_0$, we have:
\begin{equation}
    \left( M(\id + i\gamma_0\gamma_d) + \sum_{i=1}^{d-1} k_i\gamma_i  \right) \phi
    = E \gamma_0 \phi
\end{equation}
Since $i\gamma_0\gamma_d$ squares to $1$, it has eigenvalues $\pm1$. Also,
$i\gamma_0\gamma_d$ and $\gamma_0$ anticommute, so they share the same
eigenspaces with $\gamma_0$ mapping all the $+1$ eigenvectors of
$i\gamma_0\gamma_d$ to the $-1$ eigenvectors and vice versa. Now we want this
state to be a surface state, so we choose the $-1$ eigenvectors (denoted
$\phi_-$) of $i\gamma_0\gamma_d$ to kill the bulk term. This leads to:
\begin{equation}
    \gamma_0 \sum_{i=1}^{d-1} k_i\gamma_i \phi_- = \gamma_0 E\phi_-
\end{equation}
As mentioned, $\gamma_0$ switches the two eigenspaces, so we must necessarily
have a surface Dirac Hamiltonian, by projecting all those matrices to the $-1$
eigenspaces:
\begin{equation}
    \label{eq:H-Dirac-surf}
    H_\mathrm{surf} \phi_- = \sum_{i=1}^{d-1} \Gamma_{i-} \phi_- = E \phi_-
\end{equation}

% \section{Collection of Calculation}
% \label{sec:Collection of Calculation}
% Here collects the calculations done throughout the whole notes.

\subsection{From Homotopy Classification to Minimal Dirac Hamiltonian Method}
\label{sec:Homotopy to Minimal Dirac}

\paragraph{Spectral Flattening} The classification of topological insulators
concerns different classes of Hamiltonian that cannot be adiabatically changed
to each without closing or opening of a gap. Hence, we could in general
adiabatically change our parameters such that the spectrum of Hamiltonian is
simplified into two bands of energy $\pm1$. Under this condition, we have
$H^2=1$, which already hints that the Hamiltonian itself is a viable candidate
for generating Clifford Algebras. 

\paragraph{Extension Problem with General Examples}
The general theme of homotopy classification of classifying topological
insulators are done by the considering the \textit{extension problem}, i.e.
different ways to extend from a Clifford containing only symmetry operators, to
a Clifford Algebra containing Hamiltonians. For example, for class $\mathrm{A}$
in $0$-dimension, the empty space is nothing, hence
$\mathrm{C\ell}_{0}(\C)$\footnote{The reason for using different Clifford
Algebra ($\mathrm{C\ell}(\R)$ or $\mathrm{C\ell}(\C)$ for different symmetry
classes will be explained in section~\ref{sec:Isomorphism}.}. And $H^2=1$ as
mentioned earlier, hence the extension problem is from
$\mathrm{C\ell}_{0}(\C)\to \mathrm{C\ell}_{1}(\C)$, whose possible
representations form some mathematical structure called \textit{classifying
space} $C_0$ \cite{Morimoto2013}. And since we are in $0$-dimension, all
operators are maps starting from $T^0=S^0$. So our classified objects are
different classes of maps from $S^0\to C_0$, which is mathematically captured by
the $0$-th homotopy group $\pi_0(C_0)=\Z$. For class $\mathrm{AIII}$ which has
only chiral symmetry operator $S$, we choose a phase such that $S^2=1$, i.e.
making it a candidate for generators of Clifford Algebra. The symmetry condition
$\{H,S\}=0$ tells us now we have a extension problem of
$\mathrm{C\ell}_{1}(\C)\to \mathrm{C\ell}_{2}(\C)$, whose classifying space is
$C_1$, and has $\pi_0(C_1)=0$. Therefore, in $0$-dimension, the class
$\mathrm{A}$ has $\Z$ topological insulators and class $\mathrm{AIII}$ has
trivial topological insulators. The case for real symmetry classes are not too
complicated and is concisely mentioned in \cite{Morimoto2013}.

However, the method to obtain such classifying spaces are mathematical daunting
and requires $K$-theory. We try to simplify it by looking at Dirac Hamiltonians,
since they should capture the essence of topological states.

\paragraph{Extension Problem with Dirac Hamiltonian}
The Dirac Hamiltonian
\begin{equation}
    H = \vec{k}\vdot \vec{\gamma} + m\tilde{\gamma}_0
\end{equation}
has a gap closing and opening mass term $m\tilde{\gamma}_0(r)$, depending on some
parameter $r$. Suppose that we have a domain wall squeezed by two bulk regions
$A$ and $B$. Now as the parameter $r$ changes freely from region $A$ to region
$B$, the symmetry condition will force the matrix $\gamma_0$ to explorer some
space having the same homotopy type of some classifying space (again!). For
example, for class $\mathrm{A}$ in $2$-dimension, the Hamiltonian without mass
term ($k_1\gamma_1 + k_2\gamma_2$) consists of two gamma matrices, generating a
$\mathrm{C\ell}_{2}(\C)$, whereas adding the mass term, we have
$\mathrm{C\ell}_{3}(\C)$. Since there is no symmetry in class $\mathrm{A}$, the
extension problem concerns different ways to extend the algebra
$\mathrm{C\ell}_{2}(\C)\to \mathrm{C\ell}_{3}(\C)$, which tells us the number of
unitarily non-equivalent mass terms. Detailed examples can be found in section
III.C.1 of \cite{Chiu2016}.

\paragraph{Stable Classification} An unmentioned hypothesis above are that we
are classifying strong topological insulators. They are topological insulators
that robust against disorder\cite{Fu2007}. There is a consensus that in
classification of strong topological insulators, we replace the Brillouin Zone
$T^d$ by $S^d$\footnote{The exact reason can be found in Kitaev's
    classification\cite{Kitaev2009a} or another article \cite{Kennedy2015} which
discuss it from a different view.}, and we do it in a stable way, i.e., we choose
our classification to be independent of and insensitive to the addition of
irrelevant trivial bands. The first replacement allow us to use homotopy groups,
which classify maps with domain in spheres. The "stable way" allows the use of
$K$-theory (sec.III.C.1 of \cite{Chiu2016}).

\paragraph{The View from SPEMT} Let us consider a modified Dirac Hamiltonian
with extra mass term:
\begin{equation}
    \label{eq:H-spemt}
    H = M\tilde{\gamma}_0 + \sum_{i=1}^{d} k_i \gamma_i 
    + \sum_{j=1}^D m_j\tilde{\gamma}_j
\end{equation}
Here, the parameter $M$ characterized the domain wall between different phases
of topological insulators. The extra mass term $m_j$ represents a perturbation
caused by disorder. I argue that we can do our classification in the following
way. First, we consider the case when $D=0$, i.e. without perturbation, and we
consider how many different mass terms we could have. Second, we consider adding
an extra mass term which respect the symmetry of the Hamiltonian ($D=1$).
Whether this symmetry preserving extra mass term (SPEMT) can be added or not. If
a SPEMT can be added, then the system is not robust against perturbation and gap
may be opened by disorder. So it is trivial topological insulator. If a SPEMT
cannot be added, then this system is topological non-trivial. 

One best thing about this classification is that it can be done by considering
only the minimal matrix dimension of those gamma matrices in Hamiltonian. The
argument is that, complex gamma matrices are of even dimension (except the
trivial cases of $\mathrm{C\ell}_{0/1}(\C)$)\footnote{See this post
\cite{PhysicsStackExchange}, or the p.12 and Theorem in p.5 of this
\cite{West1998}.}. Therefore, gamma matrices of different
matrix dimension are built by tensor products of Pauli matrices.
\footnote{Or, the non-trivial irreducible representations (turns out there are
at most 2) are built by tensoing Pauli matrices.}
Now, to increase the dimension of the matrix, is equivalent to
tensor them. One can add more bands of the same type, or add more bands of a
different type in the smaller dimension, or add trivial bands. In all cases, the
triviality of topological insulators can be detected in minimal dimension, since
adding trivial bands can be ignored (insensitive to addition of trivial bands),
and the possible different types of matrices are limited.

However, to distinguish between a $\Z$ topological insulator and a $\Z_2$
topological insulator, we need to increase the matrix dimension to consider
multiple copies of the Dirac Hamiltonian. If the topological state is stable for
an arbitrary copies, then this is a $\Z$ topological insulator. If the
topological state is stable only of an odd number of copies, this is a $\Z_2$
topological insulator.

Another way to view this way of classification, is to look at the surface
Dirac Hamiltonian \ref{eq:H-Dirac-surf}. Then the above mentioned approach is to
see if the surface mode can be gapped, so as to detect the topological
properties of the bulk.

\subsection{Isomorphism between Symmetry Classes and Clifford Algebras}
\label{sec:Isomorphism}
As with other classifications, we construct isomorphisms between Dirac matrices
and Clifford Algebras, after which we are faced with an extension problem. The
isomorphism is constructed now.
\subsubsection{Symmetry Constraint}
\label{sec:Symmetry Constraint}
The Hamiltonian with extra mass term is
\begin{equation}
    H = M\tilde{\gamma}_0 + \sum_{j=1}^D m_j\tilde{\gamma}_j
    + \sum_{i=1}^{d} k_i \gamma_i
\end{equation}
The symmetry properties \ref{eq:T-sym-Hk},\ref{eq:C-sym-Hk} give the following
conditions on the gamma matrices:
\begin{subequations}
    \label{eq:sym-spemt}
\begin{align}
    & \{T,\gamma_i \} = 0,\, [C,\tilde{\gamma}_j] =0 \\
    & \{C,\tilde{\gamma}_j \} = 0,\, [C,\gamma_i] =0 \\
    & \{S,\tilde{\gamma}_j \} = \{S,\gamma_i\} =0
\end{align}
\end{subequations}
With this, we can establish a isomorphism between "Hamiltonian and Symmetry
Operators" and "Clifford Algebra over $\R$ or $\C$".  But let's first see some
examples in action.

\subsubsection{Examples of Constructing Isomorphism}
\label{sec:iso-Examples}
For class AI ($T^2=1$, $C=0,S=0$) in $d=1,D=0$, we define:
$$ G_{\mathrm{AI}} = \{i, T, \tilde{\gamma_0}, \gamma_1\} $$
We have its (anti)commutation relations listed in Table~\ref{tab:generator-AI}.
\begin{table}[htpb]
    \centering
    \caption{Generators in $G_{AI}$ with $d=1,D=0$}
    \label{tab:generator-AI}
    $ \begin{tabu}{ c | *4{c} }
        ~                & i      & T         & \tilde{\gamma}_0       & \gamma_1 \\
        \hline
        i                & i^2=-1 & \{i,T\}=0 & [i,\tilde{\gamma}_0]=0 & [i,\gamma_1]=0 \\
        T                & ~      & T^2=1     & [T,\tilde{\gamma}_0]=0 & \{T,\gamma_1\}=0 \\
        \tilde{\gamma}_0 & ~      & ~         & \tilde{\gamma}_0^2 =1  & \{\tilde{\gamma}_0,\gamma_1\}=0 \\
        \gamma_1         & ~      & ~         & ~                      & \gamma_1^2=1
    \end{tabu} $
\end{table}

Now with $J_4 = T\tilde{\gamma}_0\gamma_1$,
$J_3=iT\tilde{\gamma}_0$, we could verify that all
$\{i,T,J_3,J_4\}$ anticommute with each other, and
$J_3^2=J_4^2=1$. So together they generate the algebra
$\mathrm{C\ell}_{3,1}(\R) \cong \mathrm{C\ell}^g((2,2),\R)$. This will be consistent with the
isomorphism $G_{\mathrm{AI}} \cong \mathrm{C\ell}_{2+D,1+d}(\R)$ in
Table~\ref{tab:map-sym-cl}.

Another example. For class AII ($T^2=-1$, $C=0,S=0$) in $d=1,D=0$, we have:
\begin{equation}
    G_{\mathrm{AI}} = \{i, T, \tilde{\gamma_0}, \gamma_1\}
\end{equation}
We have its (anti)commutation relations listed in Table~\ref{tab:generator-AII}.
\begin{table}[htpb]
    \centering
    \caption{Generators in $G_{\mathrm{AII}}$ with $d=1,D=0$}
    \label{tab:generator-AII}
    $ \begin{tabu}{ c | *4{c} }
        ~                & i      & T         & \tilde{\gamma}_0       & \gamma_1 \\
        \hline
        i                & i^2=-1 & \{i,T\}=0 & [i,\tilde{\gamma}_0]=0 & [i,\gamma_1]=0 \\
        T                & ~      & T^2=-1    & [T,\tilde{\gamma}_0]=0 & \{T,\gamma_1\}=0 \\
        \tilde{\gamma}_0 & ~      & ~         & \tilde{\gamma}_0^2 =1  & \{\tilde{\gamma}_0,\gamma_1\}=0 \\
        \gamma_1         & ~      & ~         & ~                      & \gamma_1^2=1
    \end{tabu} $
\end{table}
Now with $J_4 = T\tilde{\gamma}_0\gamma_1$, $J_3=iT\tilde{\gamma}_0$, we could
verify that all $\{i,T,J_3,J_4\}$ anticommute with each other, and
$J_3^2=J_4^2=-1$. So together they generate the algebra $\mathrm{C\ell}_{0,4}(\R)$.
Notice that, assume we have $\mathrm{C\ell}_{1,3}(\R)=\{K_1,K_2,K_3,K'_1\}$, and
$K_i^2=-1$, $(K'_1)^2=1$. The following map
\begin{equation}
    J_1 = K_1,\, J_2 = K_2,\, J_3=K_3,\, J_4=K_1K_2K_3K'_1
\end{equation}
has the property shown in the table~\ref{tab:map-cl13-2-cl04}.
\begin{table}[htpb]
    \centering
    \caption{Map from $\mathrm{C\ell}_{1,3}(\R)$ to $\mathrm{C\ell}_{0,4}(\R)$.}
    \label{tab:map-cl13-2-cl04}
    $ \begin{tabu}{ c | *4{c} }
        ~      & J_1=K_1 & J_2=K_2 & J_3=K_3 & J_4 = K_1K_2K_3K'_1\\
        \hline
        J_1    & -1      & \{\}    & \{\}    & \{\} \\
        J_2    & ~       & -1      & \{\}    & \{\} \\
        J_3    & ~       & ~       & -1      & \{\} \\
        J_4    & ~       & ~       & ~       & -1
    \end{tabu} $

    where $\{\}$ means anticommute.
\end{table}
Therefore, $\mathrm{C\ell}_{0,4}(\R) \cong \mathrm{C\ell}_{1,3}(\R) $, which is
consistent with $G_{\mathrm{AII}} \cong
\mathrm{C\ell}_{d,3+D}(\R)$ in Table~\ref{tab:map-sym-cl}.

\subsubsection{Constructing the Isomorphism (Complex Symmetry Classes)}
\label{sec:Complex Classes-iso}
The general steps for constructing the isomorphism is presented here. First, we
need to treat the situation of complex classes and real classes
differently. The complex classes are isomorphic to complex Clifford Algebras,
while the real classes are isomorphic to real Clifford Algebras. The reason for
this distinction is that, in real classes, there are complex conjugate operator
$K$. First, there is not way to represent complex conjugation simply as
multiplication of complex matrices. Second, complex matrices come with a nature
definition of complex conjugate $\dagger$, whereas $K^\dagger$ is ill-defined
\footnote{Think about this. $\braket{c \phi |K \psi}$ could have different
    values depending on whether we move $K$ to the left as $K^\dagger$ first, or
we move $c$ out of the inner-product first.}. Therefore, we use real Clifford
Algebras when dealing with real classes (which is the reason why they are named
real classes).

For complex classes, let $G_\#$ be the group generated by elements
$\{\gamma_i,\tilde{\gamma}_j,S\}$ ($i=1,\cdots,d$, $j=0,1$) in each symmetry
class $\#$ (so $S$ exists only in class $\mathrm{AIII}$). We may choose a phase
such that $S^2=1$, as mentioned earlier. Then obviously $G_\#$ will be a
Clifford Algebra:
\begin{align}
    G_{A} &= \text{generated by }\{\gamma_i,\tilde{\gamma}_j\} \cong
    \mathrm{C\ell}_{d+D+1}(\C) \\
    G_{\mathrm{AIII}} &= \text{generated by }\{\gamma_i,\tilde{\gamma}_j,S\}
    \cong  \mathrm{C\ell}_{d+D+2}(\C)
\end{align}

\subsubsection{Constructing the Isomorphism (Real Symmetry Classes)}
\label{sec:Real Classes-iso}

For real classes, we need to include $i$, and symmetry operators $T,C$ in our
group $G_\#$. It turns out that we need to pick a phase such that $\{T,C\}=0$,
which is possible as is mentioned earlier. Now we demonstrate the proof of
isomorphism in class $\mathrm{AII}$.

We first note that, given a real vector space $V$, we can complexify a real space in
two ways. The first is trivially taking the tensor product $V\otimes_\R \C$,
which does not suit our purpose. The second is to find an almost complex
structure $J$, which is a $\R$-linear map that squares to $-\id$, i.e.
$J^2=-\id$. With this almost complex structure $J$, then $V$ admits in a natural
way the structure of a complex vector space $V_\C$\cite{DanielHuybrechts2005}. Also, a
$\R$-linear map $A$ on $V$ is $\C$-linear($\C$-antilinear) if and only if $A$
commutes (anticommutes) with $J$. Therefore, we map model the antiunitary
operators on a real vector space naturally.

Let us denote the generators in Clifford Algebra as $J_j$ and $\tilde{J}_i$,
where $J_j^2=-\tilde{J}_i^2 =-\id$. Since this almost complex structure $J$
squares to $-\id$ and anticommutes with antiunitary symmetry operators, we
naturally take it to be the first generator in our Clifford Algebra, $J_1=J$.

Then we discuss some tips that will useful for later calculation.
\paragraph{Tips}
\begin{enumerate}
    \item Since all generator anticommute, all matrices either commute or
        anticommute. So the Algebra is pretty simple.
    \item Since $[A,BC] = \pm[A,CB]$ and $\{A,BC\}=\pm\{A,CB\}$, the commutation
        or anticommutation does not depends on the order of the matrices $BC$ or
        $CB$. So one might rearrange them in the order whichever is convenient.
    \item For $A=\{J_1,\cdots,J_n\},B=\{K_1,\cdots,K_m\}$, we have $AB=(-1)^{mn}
        BA$. However, if $A$ and $B$ has something in common, then the above
        condition breaks. This is like adding an "impurity" in it to change the
        commutation/anticommutation relations.
    \item The restriction given by $\sum_i k_i \gamma_i$, and $\sum_j m_j
        \tilde{\gamma}_j$ (will be shown later) are much restrictive that the
        candidates for symmetry operator are only a small finite set.
        \label{enum:tips-4}
\end{enumerate}

\paragraph{Class $\mathrm{AII}$} Now we prove the isomorphism. First, we try to
construct each gamma matrices in Hamiltonian. All matrices commute with $i$, so
in real vector space, they commute with $J_1=J$. Then, each gamma matrices
should have an even number of Clifford Algebra generators other than $J_1$
itself. Similarly, antiunitary symmetry operators should have an odd number of
Clifford Algebra generators other than $J_1$. We "dope" the gamma matrices with
some $J_1$ to make it commute/anticommute with symmetry operators. More
explicitly, bearing in mind that in class $\mathrm{AII}$ has only $T^2=-1$, we
take a quick look into Hamiltonian \ref{eq:H-spemt} and symmetry conditions
\ref{eq:sym-spemt}, and they give us the inspiration to set:
\begin{equation}
    H_{\mathrm{AII}} = m J_1J_2J_3 + \sum_{j=1}^D m_jJ_1J_2J_{3+j} +
    \sum_{i=1}^d k_i J_2\tilde{J}_i
\end{equation}
Here $\tilde{\gamma}_j$ are $J_1J_2J_{3+j}$, which square to $1$. $\gamma_i$ are
$J_2\tilde{J}_i$. The Clifford Algebra is at least $\mathrm{C\ell}_{d,3+D}(\R)$.
And within this algebra, only $J_2$ is a possible candidate for symmetry
operators, which satisfy equations~\ref{eq:sym-spemt} (use
tips~\ref{enum:tips-4} for calculation). $J_2$ is found to be a
time reversal operator, and $J_2^2=-1$ confirms that this Hamiltonian belongs to
class $\mathrm{AII}$. Therefore, the map:
\begin{align}
    f: \mathrm{C\ell}_{d,3+D}(\R) &\to G_{\mathrm{AII}} \\
    J_1J_2J_{3+j} &\to \tilde{\gamma}_j,\, (j=0,1,\cdots,D) \nonumber\\
    J_2 \tilde{J}_i &\to \gamma_i,\, (i=0,1,\cdots,d) \nonumber
\end{align}
is a map from Clifford Algebra $\mathrm{C\ell}_{d,3+D}(\R)$ to symmetry class
$\mathrm{AII}$. The inverse map can be found easily. We first solve some
formality problems. A complex number can be written as a real number by
identifying $i$ with $J=i\sigma_y$ and $1$ with $\id$, the identity matrix:
\begin{equation}
    a+bi \to a \begin{pmatrix}
        1 & \\ & 1
    \end{pmatrix} + b \begin{pmatrix}
         & 1 \\ -1 & 
    \end{pmatrix} = a\id + b J
\end{equation}
Therefore, all complex matrices $\gamma_i,\tilde{\gamma}_j$ are identified with
real matrices $\Gamma_i,\tilde{\Gamma}_j$, of twice the size of
$\gamma_i,\tilde{\gamma}_j$. More explicitly, $(a_{mn}+i b_{mn})$ is identified
as $A+iB = (a_{mn})+i(b_{mn}) \to (a_{mn})+J_1(b_{mn})$, where
\begin{equation}
    J_1 \equiv i\sigma_y \otimes \id_{n\times n}
\end{equation}
The complex conjugate $K$ is then a matrix anticommute with $J_1$, we define:
\begin{equation}
    K \equiv \sigma_z \otimes \id_{n\times n}
\end{equation}
Then $T\to \Gamma_T$, $C\to \Gamma_C$ for some real matrices.  We also note
that, as mentioned earlier, we make a phase choice of $T$ and $C$ such that
\begin{equation}
    \{\Gamma_T,\Gamma_C\} = 0
\end{equation}

Now, we write symbolically $J_1J_2J_{3+j} = J_1 \Gamma_T J_{3+j} =
\tilde{\Gamma}_j$, then clearly $J_{3+j} = \Gamma_T J_1\tilde{\Gamma}_j$.
Similarly, $\tilde{J}_i = \Gamma_T\Gamma_i$. So the map:
\begin{align}
    f^{-1}: G_{\mathrm{AII}} &\to \mathrm{C\ell}_{d,3+D}(\R) \\
    i\gamma_y \otimes \id_{n\times n} &\to J_1 \nonumber\\
    \Gamma_T &\to J_2 \nonumber\\
    \Gamma_TJ_1\Gamma_j &\to J_{3+j} \nonumber\\
    \Gamma_T\tilde{\Gamma}_i &\to J_{i} \nonumber
\end{align}
is the desired inverse map.

\paragraph{Class $\mathrm{CII}$} The class $\mathrm{CII}$ has only one $C^2=-1$
more than class $\mathrm{AII}$, therefore, we only need to enlarge the Clifford
Algebra to include one more $J_{4+D}$, and set it as the $C$ symmetry operator.
The rest is exactly the same as in class $\mathrm{AII}$.

All other classes can be treated similarly, so we do not repeat the calculation
and only list the result in Table~\ref{tab:map-sym-cl}.
\begin{table}[htpb]
    \centering
    \caption{Mapping Relations between symmetry classes and Clifford Algebras.
    Here $J_j^2=-\id,\tilde{J}_i^2=\id$, and they generates the Clifford
    Algebra. Also, only one direction of mapping is shown. The inverse map can
    be easily constructed accordingly. The complex class are also added for
    convenience.}
    \label{tab:map-sym-cl}
    \begin{tabular}{c | c c c | l | c }
        Class($\#$) & T & C & S & Mappings  & $G_\#\cong $ \\
        \hline 
        D & $0$ & $+$ & $0$ & 
        $\tilde{J}_1\to\Gamma_C$,
        $\tilde{J}_1J_{2+j}\to \tilde{\Gamma}_j$,
        $J_1\tilde{J}_1\tilde{J}_{1+i}\to \Gamma_i$ &
        $\mathrm{C\ell}_{1+d,2+D}(\R)$ \\
        DIII & $-$ & $+$ & $1$ & 
        $\tilde{J}_1\to\Gamma_C$, $\tilde{J}_{3+D}\to\Gamma_T$,
        $\tilde{J}_1J_{2+j}\to \tilde{\Gamma}_j$,
        $J_1\tilde{J}_1\tilde{J}_{1+i}\to \Gamma_i$ &
        $\mathrm{C\ell}_{1+d,3+D}(\R)$ \\
        AII & $-$ & $0$ & $0$ & 
        $J_2\to \Gamma_T$,
        $J_1J_2J_{3+j}\to \tilde{\Gamma}_j$,
        $J_2\tilde{J}_i \to \Gamma_i$
        & $\mathrm{C\ell}_{d,3+D}(\R)$ \\
        CII & $-$ & $-$ & $1$ & 
        $J_2\to \Gamma_T$, $J_{4+D}\to \Gamma_C$,
        $J_1J_2J_{3+j}\to \tilde{\Gamma}_j$,
        $J_2\tilde{J}_i \to \Gamma_i$
        & $\mathrm{C\ell}_{d,4+D}(\R)$ \\
        C & $0$ & $-$ & $0$ & 
        $J_2\to \Gamma_C$, 
        $J_2\tilde{J}_{1+j} \to \tilde{\Gamma}_j$,
        $J_1J_2J_{i+2}\to \Gamma_i$
        & $\mathrm{C\ell}_{1+D,2+d}(\R)$ \\
        CI & $+$ & $-$ & $1$ & 
        $J_2\to \Gamma_C$, $\tilde{J}_{2+D}\to \Gamma_T$,
        $J_2\tilde{J}_{1+j} \to \tilde{\Gamma}_j$,
        $J_1J_2J_{i+2}\to \Gamma_i$
        & $\mathrm{C\ell}_{2+D,2+d}(\R)$ \\
        AI & $+$ & $0$ & $0$ & 
        $\tilde{J}_1\to \Gamma_T$,
        $J_1\tilde{J}_1\tilde{J}_{2+j} \to \tilde{\Gamma}_j$,
        $\tilde{J}_1J_{i+1} \to \Gamma_i$
        & $\mathrm{C\ell}_{2+D,1+d}(\R)$ \\
        BDI & $+$ & $+$ & $1$ & 
        $\tilde{J}_1\to \Gamma_T$, $\tilde{J}_{3+D}\to \Gamma_C$,
        $J_1\tilde{J}_1\tilde{J}_{2+j} \to \tilde{\Gamma}_j$,
        $\tilde{J}_1J_{i+1} \to \Gamma_i$
        & $\mathrm{C\ell}_{3+D,1+d}(\R)$ \\
        \hline
        A & $0$ & $0$ & $0$ & 
        & $\mathrm{C\ell}_{d+D+1}(\C)$ \\
        AIII & $0$ & $0$ & $1$ & 
        & $\mathrm{C\ell}_{d+D+2}(\C)$ \\
        \hline
    \end{tabular}
\end{table}
\subsection{Classification in 1-dimension}
\label{sec:Classification in 1-dimension}

\subsubsection{Model Systems}
\label{sec:Model Systems}
Classification of $1$ dimension is the simplest, since the minimal matrix
dimension will mostly be $2$, which means we can use the familiar Pauli matrices
directly.

\paragraph{Class $\mathrm{A}$} We have the extension from
$\mathrm{C\ell}_{2}(\C)$ to $\mathrm{C\ell}_{3}(\C)$. Since
$\mathrm{C\ell}_{2}(\C)\cong\mathcal{M}(2,\C)$, we have for example:
\begin{equation}
    H_A = M \sigma_x + k_x \sigma_y
\end{equation}
Obviously, there is an SPEMT $m\sigma_x$. So this phase is trivial.

\paragraph{Class $\mathrm{AIII}$} We have the extension from
$\mathrm{C\ell}_{3}(\C)$ to $\mathrm{C\ell}_{4}(\C)$. Since
$\mathrm{C\ell}_{3}(\C)\cong \mathcal{M}(2,\C)\oplus\mathcal{M}(2,\C)$, we use
Pauli matrices. For example:
\begin{equation}
    H_A = M \sigma_x + k_x \sigma_y
\end{equation}
However, the symmetry operator $S$ takes rest Pauli matrices $\sigma_z$, and
there is not other matrix possible for the extra mass term (this can be viewed
alternatively, from $\mathrm{C\ell}_{4}(\C)\cong \mathcal{M}(4,\C)$, which only
has irreducible representation in dimension $1$ or $4$). So the state is
topologically non-trivial. Now we consider the state with arbitrary copies of it.
\begin{equation}
    H_A = M \sigma_x\otimes \id_n + k_x\sigma_y\otimes\id_n,
    S=\sigma_z\otimes\id_n
\end{equation}
where $n$ is some positive integer. Since gamma matrices are basically tensor
products of Pauli matrices, there is no SPEMT term. Hence this is a $\Z$
topological insulator.

The above are for complex classes. For real classes, we have to be careful,
since one generator $J_1$ is taken up by $i$.

\paragraph{Class $\mathrm{AII}$} We have the extension from
$\mathrm{C\ell}_{1,3}(\R)\cong\mathcal{M}(2,\R)\otimes\mathbb{H}$, to
$\mathrm{C\ell}_{1,4}(\R)\cong\mathcal{M}(2,\R)\otimes(\mathbb{H}\oplus\mathbb{H})$.
This already hints that we will have SPEMT. Explicitly, we could let
\begin{equation}
    H_{\mathrm{AII}} = M\sigma_x\otimes\id_2 + k_x\sigma_y\otimes\id_2
\end{equation}
with $T=\id_2\otimes\sigma_x K$. The SPEMT is $m\sigma_z\otimes\sigma_x$. So
this is a trivial insulator.

\paragraph{Class $\mathrm{C}$} We have the extension from
$\mathrm{C\ell}_{1,3}(\R)\cong\mathcal{M}(2,\R)\otimes\mathrm{H}$, to
$\mathrm{C\ell}_{2,3}(\R)\cong\mathcal{M}(4,\R)\otimes\C$. Explicitly, we
could let
\begin{equation}
    H_{\mathrm{C}} = M\sigma_z\otimes\id_2 + k_x\sigma_y\otimes\id_2
\end{equation}
with $C=\sigma_x\otimes\sigma_y K$. The SPEMT is $m\sigma_x\otimes\sigma_x$. So
this is a trivial insulator.

\paragraph{Class $\mathrm{CI}$} We have the extension from
$\mathrm{C\ell}_{2,3}(\R)\cong\mathcal{M}(4,\R)\otimes\C$, to
$\mathrm{C\ell}_{3,3}(\R)\cong\mathcal{M}(6,\R)$. If one realize that $\C$ is
realized by $2\times 2$ matrices, then this should have SPEMT. Explicitly, we
could let
\begin{equation}
    H_{\mathrm{CI}} = M\sigma_z\otimes\id_2 + k_x\sigma_y\otimes\id_2
\end{equation}
with $C=\sigma_x\otimes\sigma_y K$, $T=K$. The SPEMT is again
$m\sigma_x\otimes\sigma_x$. So this is a trivial insulator.

\paragraph{Class $\mathrm{AI}$} We have the extension from
$\mathrm{C\ell}_{2,2}(\R)\cong\mathcal{M}(4,\R)$, to
$\mathrm{C\ell}_{3,2}(\R)\cong\mathcal{M}(4,\R)\otimes(\R\oplus\R)$. This hints
that we have SPEMT. Explicitly, we could let
\begin{equation}
    H_{\mathrm{AI}} = M\sigma_x + k_x\sigma_y
\end{equation}
with $T=K$. The SPEMT is $m\sigma_z$. So this is a trivial insulator.

\paragraph{Class $\mathrm{CII}$} We have extension from
$\mathrm{C\ell}_{1,4}(\R)\cong\mathcal{M}(2,\R)\otimes(\mathbb{H}\oplus\mathbb{H})$,
to
$\mathrm{C\ell}_{1,5}(\R)\cong\mathcal{M}(2,\R)\otimes\mathcal{M}(2,\mathbb{H})$.
We let:
\begin{equation}
    H_{\mathrm{CII}} = M\sigma_x\otimes\id_2 + k_x\sigma_y\otimes\id_2
\end{equation}
with $T=\id_2\otimes\sigma_y K$, $C=\sigma_z\otimes\sigma_y K$. There is no
SPEMT\footnote{To anticommute with $\sigma_x\otimes\id_2$ and
$\sigma_y\otimes\id_2$, it must be of the form $\sigma_z\otimes?$. But none of
the remaining choice are acceptable, considering $T$ and $C$.}. Consider
arbitrary copies of it:
\begin{equation}
    H_{\mathrm{CII}} = (M\sigma_x\otimes\id_2 +
    k_x\sigma_y\otimes\id_2)\otimes\id_n
\end{equation}
with $T=\id_2\otimes\sigma_y\otimes\id_n K$,
$C=\sigma_z\otimes\sigma_y\otimes\id_n K$. There is still no SPEMT\footnote{The
SPEMT has to be of the form $\sigma_z\otimes\sigma_\alpha\otimes A_{n\times n}$.
Commuting with $T$ makes $A$ completely imaginary. But it cannot anticommute
with $C$.}. Therefore, this is a $\Z$ topological insulator.

\paragraph{Class $\mathrm{BDI}$} We have extension from
$\mathrm{C\ell}_{3,2}(\R)\cong\mathcal{M}(4,\R)\otimes(\R\oplus\R)$, to
$\mathrm{C\ell}_{4,2}(\R)\cong\mathcal{M}(8,\R)$. We let:
\begin{equation}
    H_{\mathrm{BDI}} = M\sigma_z + k_x \sigma_y
\end{equation}
with $T=K$, $C=\sigma_x K$. Obviously, there is no SPEMT. For arbitrary copies
of it:
\begin{equation}
    H_{\mathrm{BDI}} = M\sigma_z\otimes\id_n + k_x \sigma_y\otimes\id
\end{equation}
with $T=\id_{n+2}K$, $C=\sigma_x\otimes\id_n K$.  The SPEMT must be of the form
$\sigma_x\otimes\Delta$ to anticommute with $H_{\mathrm{BDI}}$. But to commute with
$T$ leads to $\Delta$ being real, and to anticommute with $C$ leads to $\Delta$ being
complex. Hence there is no SPEMT. This is a $\Z$ topological insulator.

\paragraph{Class $\mathrm{D}$} We let:
\begin{equation}
    H_{\mathrm{D}} = M\sigma_z + k_x \sigma_y
\end{equation}
with $C=\sigma_x K$. Obviously, there is no SPEMT. Now consider two copies of
it:
\begin{equation}
    H_{\mathrm{D}} = (M\sigma_z + k_x \sigma_y)\otimes\id_2
\end{equation}
with $C=\sigma_x\otimes\id_2 K$. There is one SPEMT $m \sigma_z\otimes\sigma_y$.
Hence, this is a $\Z_2$ topological insulator.

To lighten the notation a bit, we introduce $\tau_i=s_i=\sigma_i$ ($i=0,1,2,3$).
But $\tau_i$, $s_i$, and $\sigma_i$ all act on the different spaces.
\paragraph{Class $\mathrm{DIII}$} We let:
\begin{equation}
    H_{\mathrm{DIII}} = M \sigma_z s_0 + k_x \sigma_y s_0
\end{equation}
with $C=\sigma_x K$, $T=s_y K$. Any SPEMT anticommute with Hamiltonian has the
form $\sigma_x s_i$. For it to anticommute with $C$, it is $\sigma_x s_y$. But
this does not commute with $T$. Hence, there is no SPEMT. Now consider two
copies of it:
\begin{equation}
    H_{\mathrm{DIII}} = (M \sigma_z s_0 + k_x \sigma_y s_0)\tau_0
\end{equation}
with $C=\sigma_x K$, $T=s_y K$.  There is one SPEMT $m \sigma_xs_x\tau_y$. So
this is a $\Z_2$ topological insulator.

In summary, we shown examples of different classes of topological insulators in
$d=1$, summarized in Table~\ref{tab:ti-d=1}.
\begin{table}[htpb]
    \centering
    \caption{Topological Insulators in $d=1$}
    \label{tab:ti-d=1}
    \begin{tabular}{c | c c c | c }
        Class($\#$) & T & C & S & $d=1$ \\
        \hline 
        A & $0$ & $0$ & $0$ & $0$ \\
        AIII & $0$ & $0$ & $1$ & $\Z$ \\
        \hline
        D & $0$ & $+$ & $0$ & $\Z_2$ \\
        DIII & $-$ & $+$ & $1$ & $\Z_2$ \\
        AII & $-$ & $0$ & $0$ & $0$ \\
        CII & $-$ & $-$ & $1$ & $\Z$ \\
        C & $0$ & $-$ & $0$ & $0$ \\
        CI & $+$ & $-$ & $1$ & $0$ \\
        AI & $+$ & $0$ & $0$ & $0$ \\
        BDI & $+$ & $+$ & $1$ & $\Z$ \\
        \hline
    \end{tabular}
\end{table}
\subsubsection{Irreducible Representation Arguments for Classification}
\label{sec:Irreducible Representation Arguments for Classification}
The above examples are only valid in special points of the Brillouin Zone.
However, in real materials, one will be faced with Hamiltonians of various form:
\begin{equation}
    H = m\gamma_0 \pm k_1\gamma_1 \pm k_2\gamma_2 \cdots
\end{equation}
Here, we look at this problem from a representation theory point of view. It is
well known that representation of Clifford Algebras are completely reducible,
and there are only 1 or 2 nontrivial (dimension $>1$) inequivalent irreducible
representations of Clifford Algebras (see \ref{sec:Various Properties for
Clifford Algebras}).  The dimension of the representation of relevant Clifford
Algebras are listed in Table~\ref{tab:mat-dim-allClass-1d}. There are only three
types of change occurs in that table:

\paragraph{$\mathbf{n\to n}$ or $\mathbf{n\to n_2}$} Basically, this means that in the same
dimension we could always have a SPEMT present. So there will never be stable
edge states.

\paragraph{$\mathbf{n\to 2n}$} This means that, although in the same dimension we have
always stable edge mode. When we take two copies of the Hamiltonian, i.e., when
there are two edge modes. There will a new gamma matrix in the doubled-dimension
acting as a SPEMT to gap out the system. Hence in general, an odd copies of
the edge mode is stable whereas an even copies of the edge mode is not. Hence
this is a $\Z_2$ topological insulator.

\paragraph{$\mathbf{n_2\to 2n}$} This is similar to the above $\Z_2$ case. However, since
we have two inequivalent representations in a single edge mode, we could
the doubled system could have came from different representations, which could
not be gapped out by the extra SPEMT in doubled-dimension (otherwise we would
have two inequivalent representations in the doubled-dimension). Hence, this is
a $\Z$ topological insulator.

In this way we confirmed the classification in Table~\ref{tab:ti-d=1}.

\begin{table}[htpb]
    \centering
    \caption{Matrix dimensions related to the extension problem for different
    classes in $d=1$. Note that a subscript $2$ as in $2_2$, means that there
    are two inequivalent representations in the same matrix dimension.}
    \label{tab:mat-dim-allClass-1d}
    \begin{tabular}{c | c c c}
        Name of Class & Extension of $\mathrm{C\ell}$ & Dimension Change &
        Topological Invariant \\
        \hline
        A & $\mathrm{C\ell}_{2}(\C)\to\mathrm{C\ell}_{3}(\C)$ & $2\to2_2$ & $0$
        \\
        AIII & $\mathrm{C\ell}_{3}(\C)\to\mathrm{C\ell}_{4}(\C)$ & $2_2\to4$ &
        $\Z$
        \\
        AII & $\mathrm{C\ell}_{1,3}(\R)\to\mathrm{C\ell}_{1,4}(\R)$ & $4\to 4_2$ & $0$
        \\
        C & $\mathrm{C\ell}_{1,3}(\R)\to\mathrm{C\ell}_{2,3}(\R)$ & $4\to 4$ & $0$
        \\
        CI & $\mathrm{C\ell}_{2,3}(\R)\to\mathrm{C\ell}_{3,3}(\R)$ & $4\to 4$ & $0$
        \\
        AI & $\mathrm{C\ell}_{2,2}(\R)\to\mathrm{C\ell}_{3,2}(\R)$ & $2\to 2_2$ & $0$
        \\
        CII & $\mathrm{C\ell}_{1,4}(\R)\to\mathrm{C\ell}_{1,5}(\R)$ & $4_2\to 8$
        & $\Z$
        \\
        BDI & $\mathrm{C\ell}_{3,2}(\R)\to\mathrm{C\ell}_{4,2}(\R)$ & $2_2\to 4$ 
        & $\Z$
        \\
        D & $\mathrm{C\ell}_{2,2}(\R)\to\mathrm{C\ell}_{2,3}(\R)$ & $2\to 4$ 
        & $\Z_2$
        \\
        DIII & $\mathrm{C\ell}_{2,3}(\R)\to\mathrm{C\ell}_{2,4}(\R)$ & $4\to 8$ 
        & $\Z_2$
        \\
        \hline
    \end{tabular}
\end{table}
\subsection{Classification in Arbitrary Dimension}
\label{sec:Classification in Arbitrary Dimension}
We now show combined with some properties, the classification in $d=1$ can be
generalized to classification in arbitrary dimensions. In $K$-theoretic
classification, the extension problem is not affected when tensored with some
other algebras. Specifically, for complex classes, the extension problem of
$\mathrm{C\ell}_{n}(\C)\to\mathrm{C\ell}_{m}(\C)$ is the same as the extension
problem of $\mathrm{C\ell}_{n}(\C)\otimes\mathrm{C\ell}_{2}(\C)\to
\mathrm{C\ell}_{m}(\C)\otimes\mathrm{C\ell}_{2}(\C)$ (See for example, sec
III.C.1 of \cite{Chiu2016}). For real classes, similar property holds.
This insensitive to tensoring a new algebra, can be understand in SPEMT
background. Because $\mathrm{C\ell}_{2}(\C)\cong\mathcal{M}(2,\C)$, which only
multiply the matrix dimension of both sides by a constant $2$. So the pattern of
$n\to n, n\to n_2$, $n\to 2n$, and $n_2\to 2n$, mentioned in previous section,
remains unaltered. For real classes, instead of $\mathrm{C\ell}_{2}(\C)$, we
have $\mathrm{C\ell}_{1,1}(\R)\cong\mathcal{M}(2,\R)$, or
$\mathrm{C\ell}_{2,2}(\R)\cong\mathrm{C\ell}_{1,1}(\R)\otimes\mathrm{C\ell}_{1,1}(\R)\cong\mathcal{M}(4,\R)$.
So the reasoning is the same.

Now, let us write $\sim$ to represent the equivalence between two extension
problems.  The Complex Clifford Algebra has a periodicity of $2$:
\begin{equation}
    \mathrm{C\ell}_{n+2}(\C) \cong \mathrm{C\ell}_{n}(\C)\otimes
    \mathrm{C\ell}_{2}(\C)
\end{equation}
The Real Clifford Algebra has a periodicity of $8$:
\begin{equation}
    \mathrm{C\ell}_{p+8,q}(\R) = \mathrm{C\ell}_{p,q+8}(\R) =
    \mathrm{C\ell}_{p,q}(\R)\otimes\mathcal{M}(16,\R)
\end{equation}
This two relation means that, the extension problem will be the same with respect to
a periodicity of $2$ for complex symmetry classes, and $8$ for real symmetry
classes. Hence the classification of topological insulators need only be done
with $d=1,2$ for complex symmetry classes, and $d=0,1,\cdots,7$ for real
symmetry classes, i.e.
\begin{subequations}
    \label{eq:cli-periodic}
\begin{align}
    G_{\#}(d=d_0) &\sim G_{\#}(d=(d_0 \text{mod} 2)) \quad\text{(complex class)} \\
    G_{\#}(d=d_0) &\sim G_{\#}(d=(d_0 \text{mod} 8)) \quad\text{(real class)}
\end{align}
\end{subequations}
Another useful property is:
\begin{align}
    \mathrm{C\ell}_{p+1,q+1}(\R) &\cong
    \mathrm{C\ell}_{p,q}(\R)\otimes\mathcal{M}(2,\R)
\end{align}
This property tells us that Clifford Algebra basically depends only on the
difference $p-q$, or combined with previous periodicity, $p-q$ (mod $8$). For
example, we have (let $n$ be an arbitrary integer):
$\mathrm{C\ell}_{p+1,q+1}(\R)\sim\mathrm{C\ell}_{p,q}(\R)$. With this, one can
derive easily two chains of relation:
\begin{align}
    &\mathrm{C\ell}_{1+(D+n),2+D}(\R) \sim \mathrm{C\ell}_{1+(D+1+n),3+D}(\R)
    \nonumber\\
    \sim & \mathrm{C\ell}_{(D+2+n),3+D}(\R) \sim \mathrm{C\ell}_{(D+3+n),4+D}(\R)
\end{align}
and
\begin{align}
    &\mathrm{C\ell}_{1+D,2+(D+4+n)}(\R)\sim \mathrm{C\ell}_{2+D,2+(D+5+n)}(\R)
    \nonumber\\
    \sim& \mathrm{C\ell}_{2+D,1+(D+6+n)}(\R) \sim
    \mathrm{C\ell}_{3+D,1+(D+7+n)}(\R)
\end{align}
Now we tries to connect the two chains together. We note further that the
Clifford Algebra has the following properties:
\begin{align}
    \mathrm{C\ell}_{p+1,q}(\R) &\cong \mathrm{C\ell}_{q+1,p}(\R) \\
    \mathrm{C\ell}_{q,p+2}(\R) &\cong \mathrm{C\ell}_{p,q}(\R) \otimes\mathcal{M}(2,\R)
\end{align}
Then:
\begin{align}
    &\mathrm{C\ell}_{1+D,2+(D+4+n)}(\R) \cong
    \mathrm{C\ell}_{D+4+n,1+D}(\R)\otimes\mathcal{M}(2,\R) \nonumber\\
    \cong& \mathrm{C\ell}_{2+D,D+3+n}(\R)\otimes\mathcal{M}(2,\R)
    \cong \mathrm{C\ell}_{D+1+n,2+D}(\R)\otimes\mathcal{M}(4,\R)
\end{align}
Therefore,
$\mathrm{C\ell}_{1+D,2+(D+4+n)}(\R)\sim\mathrm{C\ell}_{1+(D+n),2+D}(\R)$,
connecting the two chains. If we compare this carefully with
Table~\ref{tab:map-sym-cl}, then we would realize that we actually managed to
prove the dimension-shift feature of classification:
\begin{align}
    \label{eq:cli-Chain1}
    G_{\mathrm{D}}(d=d_0) &\sim G_{\mathrm{DIII}}(d=d_0+1) \nonumber\\
    \sim G_{\mathrm{AII}}(d=d_0+2) &\sim G_{\mathrm{CII}}(d=d_0+3) \nonumber\\
    \sim G_{\mathrm{C}}(d=d_0+4) &\sim G_{\mathrm{CI}}(d=d_0+5) \nonumber\\
    \sim G_{\mathrm{AI}}(d=d_0+6) &\sim G_{\mathrm{BDI}}(d=d_0+7)
\end{align}
We need one more property:
\begin{equation}
    \mathrm{C\ell}_{1+d,2+D}(\R)\cong
    \mathrm{C\ell}_{3+D,d}(\R)=\mathrm{C\ell}_{3+D,1+(d-1)}(\R)
\end{equation}
Then
\begin{equation}
    \label{eq:cli-Chain2}
    G_{\mathrm{D}}(d=d_0) \sim G_{\mathrm{BDI}}(d=d_0-1)
\end{equation}
With equivalences~\ref{eq:cli-periodic},\ref{eq:cli-Chain1},\ref{eq:cli-Chain2},
we see that we need only the result in a dimension, to obtain the whole of
classification of all real classes. Similar fact holds for complex classes:
\begin{equation}
    G_{\mathrm{AIII}}(d=d_0) \sim G_{\mathrm{A}}(d=d_0-1)
\end{equation}
which is too trivial to be mentioned here. 

In a word, our classification for topological insulators in $1$ spatial
dimension is sufficient to generate the whole Table~\ref{tab:master-table2}.
This is a remarkable consequence of $K$-theory.
\begin{table}
\begin{center}
\begin{tabular}{|c|cccccccc|ccc|}
\hline
\multicolumn{12}{|c|}{
The original classification table} \\ \hline
   $\mbox{AZ class} \backslash d$  & 0 & 1 & 2 & 3 & 4 & 5 & 6 & 7 & T & C & S  \\
\hline\hline
  A & $\mathbb{Z}$ & 0 & $\mathbb{Z}$ & 0 & $\mathbb{Z}$ & 0 & $\mathbb{Z}$ & 0             & 0 & 0 & 0    \\
  AIII & 0 & $\mathbb{Z}$ & 0 & $\mathbb{Z}$ & 0 & $\mathbb{Z}$ & 0 & $\mathbb{Z}$          & 0 & 0 & 1    \\  \hline

  AI & $\mathbb{Z}$ & 0 & 0 & 0 & $2\mathbb{Z}$ & 0 & $\mathbb{Z}_2$ & $\mathbb{Z}_2$    & $+$ & 0 & 0     \\
  BDI & $\mathbb{Z}_2$ & $\mathbb{Z}$ & 0 & 0 & 0 & $2\mathbb{Z}$ & 0 & $\mathbb{Z}_2$     & $+$ & $+$ & 1    \\
  D & $\mathbb{Z}_2$ & $\mathbb{Z}_2$ & $\mathbb{Z}$ & 0 & 0 & 0 & $2\mathbb{Z}$ & 0     & 0 & $+$ & 0     \\
  DIII & 0 & $\mathbb{Z}_2$ & $\mathbb{Z}_2$ & $\mathbb{Z}$ & 0 & 0 & 0 & $2\mathbb{Z}$  & $-$ & $+$ & 1     \\
  AII & $2\mathbb{Z}$ & 0 & $\mathbb{Z}_2$ & $\mathbb{Z}_2$ & $\mathbb{Z}$ & 0 & 0 & 0   & $-$ & 0 & 0     \\
  CII & 0 & $2\mathbb{Z}$ & 0 & $\mathbb{Z}_2$ & $\mathbb{Z}_2$ & $\mathbb{Z}$ & 0 & 0   & $-$ & $-$ & 1     \\
  C & 0 & 0 & $2\mathbb{Z}$ & 0 & $\mathbb{Z}_2$ & $\mathbb{Z}_2$ & $\mathbb{Z}$ & 0     & 0 & $-$ & 0     \\
  CI & 0 & 0 & 0 & $2\mathbb{Z}$ & 0 & $\mathbb{Z}_2$ & $\mathbb{Z}_2$ & $\mathbb{Z}$    & $+$ & $-$ & 1
   \\
\hline
\end{tabular}
\caption{The original classification table of topological insulators and
superconductors. Note that in this document we could not yet differentiate
between $\Z$ and $2\Z$ topological invariants.}
\label{tab:master-table2}
\end{center}
\end{table}

\section{Looking Further}
\label{sec:Looking Further}
There has be great deal of progress made in classification of topological
insulator and superconductors. We first comment that our discussion could be
easily generalized to some topological superconductors because they naturally
admits a matrix like Hamiltonian using Nambu spinors.

Also, our discussion falls short in two respects. First, it cannot distinguish
between $\Z$ and $2\Z$ cases. Secondly, we do not give explicit formulae for
calculating topological invariants in different cases. The former problem can be
addressed in Ludwig's \cite{Schnyder2008}, while the latter requires
considerable work, and provides different topological invariant in different
cases (see sec. III.B of \cite{Chiu2016}).

There has been progress made both in incorporating defects into classification,
as well as incorporating unitary symmetries into the classification. The
defect's classification has been incorporated beautifully into the same table of
Tenfold Way. On the other hand, introduction of unitary symmetries gives a more
fruitful result. For example, introduction of crystal symmetries gives a new
class of topological insulators, dubbed topological crystalline
insulators.\cite{Ando2015} the case of reflection symmetry, some new topological
invariants denoted as $M\Z$, called mirror topological invariants, are needed to
classify topological matters\cite{Chiu2013}.

Lastly, the discussion of topological states in interaction picture extends to
the case of symmetry-protected phases (SPTs), of which I am not familiar with.
For all these new concepts, the review \cite{Chiu2016} is an excellent source of
information.

\section{Various Properties for Clifford Algebras}
\label{sec:Various Properties for Clifford Algebras}
We have used various properties of Clifford Algebras from various sources. Here
I summarized those properties for the convenience of readers.

\paragraph{Isomorphism Relations}
The best source of isomorphism between Clifford Algebras is actually the
Wikipedia page about \textit{Classifications of Clifford Algebras}. For a more
authoritative account, I am using the book \cite{VazJayme2016}. The results
used in this document are to be found in chapter 4 of that book.

\paragraph{Representation Theory on Complex Clifford Algebras}
This paper \cite{West1998} has a good discussion on the irreducible
representations of complex Clifford Algebras, in p.5, p.12. They can be
summarized as:
\begin{thm}
    Consider $\mathrm{C\ell}_{N}(\C)$. For both $N = 2n$ and $N = 2n + 1$, the
    dimension d of the irreducible representation is uniquely given by
    \begin{equation}
        d = 2^n \text{($N=2n$ or $N=2n+1$)}
    \end{equation}
    Moreover, for $N = 2n$, the irreducible space is unique, while for $N = 2n +
    1$, we have two inequivalent irreducible representations with the same
    dimension $d$.
\end{thm}

\paragraph{Representation Theory on Real Clifford Algebra}
The two papers \cite{Okubo1991a}\cite{Okubo1991} are great in this respect.
Their result used in this document could be summarized as:
\begin{thm}
    Consider $\mathrm{C\ell}_{p,q}(\R)$. For $N=2n$, we have:
    \begin{align}
        d= \begin{cases}
              2^n_1, & \text{for } p-q=0\text{ or }2\text{, mod }8 \\
              2^{n+1}_1, & \text{for } p-q=4\text{ or }6\text{, mod }8 \\
          \end{cases}
    \end{align}
    For $N=2n+1$, we have:
    \begin{align}
        d= \begin{cases}
              2^n_2, & \text{for } p-q=1\text{, mod }8 \\
              2^{n+1}_1, & \text{for } p-q=3\text{ or }7\text{, mod }8 \\
              2^{n+1}_2, & \text{for } p-q=5\text{, mod }8 \\
          \end{cases}
    \end{align}
    Where we have used a subscript $\phantom{1}_1$ to denote that there is only
    one unique irreducible representation, and a subscript $\phantom{1}_2$ to
    denote that there are two inequivalent irreducible representations.
\end{thm}
Below collects a list of dimensions that I have used in the discussion of
Table~\ref{tab:mat-dim-allClass-1d}:
\begin{align*}
    &
    \mathrm{C\ell}_{1,2}(\R)\sim 2^2_1,\, \mathrm{C\ell}_{1,3}(\R)\sim 4^2_1,\,
    \mathrm{C\ell}_{1,4}(\R)\sim 4^2_2,\, \mathrm{C\ell}_{1,5}(\R)\sim 8^2_1,\,
    \\&
    \mathrm{C\ell}_{2,2}(\R)\sim 2^2_1,\, \mathrm{C\ell}_{2,3}(\R)\sim 4^2_1,\,
    \mathrm{C\ell}_{3,2}(\R)\sim 2^2_2,\, \mathrm{C\ell}_{2,4}(\R)\sim 8^2_1,\,
    \\&
    \mathrm{C\ell}_{3,3}(\R)\sim 4^2_1,\, \mathrm{C\ell}_{4,2}(\R)\sim 4^2_1.
\end{align*}

\section{Online Version}
This document may be updated online at
\href{https://github.com/we-taper/Draft/tree/master/Notes%20of%20various%20papers/Chiu%20Minimal%20Dirac%20Hamiltonian%20Method}{here}.


\bibliography{../../library}{}
\bibliographystyle{alphaurl}
% \begin{thebibliography}{1}
% 	\bibitem{book} 
% \end{thebibliography}
\printnomenclature

\end{document}

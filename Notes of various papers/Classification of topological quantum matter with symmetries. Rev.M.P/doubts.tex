% The entire content of this work (including the source code
% for TeX files and the generated PDF documents) by 
% Hongxiang Chen (nicknamed we.taper, or just Taper) is
% licensed under a 
% Creative Commons Attribution-NonCommercial-ShareAlike 4.0 
% International License (Link to the complete license text:
% http://creativecommons.org/licenses/by-nc-sa/4.0/).
\documentclass{article}

% My own physics package
% The following line load the package xparse with additional option to
% prevent the annoying warnings, which are caused by the package
% "physics" loaded in package "physicist-taper".
\usepackage[log-declarations=false]{xparse}
\usepackage{physicist-taper}

\makenomenclature

\title{Notes for Classification of topological quantum matter with
symmetries}
\date{\today}
\author{Taper}


\begin{document}


\maketitle
\abstract{
As title suggests.
}
\tableofcontents
\paragraph{Asks}
\begin{enumerate}
    \item Page 6. Why the scalar in Schur's lemma becomes of unit lenght.
    \item page 9. What does it mean by:
        \begin{quote}
            Note that unitary symmetries, which commute with the
            Hamiltonian, allow us to bring the Hamiltonian into a
            block diagonal form.
        \end{quote}
    \item page 9, table I. When he talks about \textit{codimension},
        what is the dimension of the whole space? ($3$? $3+1$?).
        Similar problem also exists in page 10.
        Note, he mentions codimension of gaples modes on page 11. He
        asserts that codimension $1$ means $1$ dimension less than the
        bulk. But it should be strange to compare the dimension of
        defects with the dimension of the bulk.

        Possibly related
        resources:\href{http://www.virginia.edu/bohr/mse209/chapter4.htm}
        {Online notes about Imperfection}:
        \begin{itemize}
            \item $0$D (zero dimension) – point defects: vacancies and
                interstitials. Impurities.
            \item 1D – linear defects: dislocations (edge, screw,
                mixed) 
            \item 2D – grain boundaries, surfaces.  
            \item 3D – extended defects: pores, cracks.
        \end{itemize}

        Note: it is finally defined on page 12. that is:
        codimension of defect $d_c:= d_\text{bulk} - d_\text{defect}$.
    \item page 6, what does it mean by saying \textit{"unitary
        symmetry"}.
    \item page 10, what is a \textit{"quantum phase diagram"}.
    \item page 11, what does he says, \textit{"Topological properties of
        adiabatic cycles can also be discussed in a similar manner."}.
        Does this mean that all previous classification does not
        concern the adiabatic cycles? What is \textit{"adiabatic
        cycles"} exactly in his language?

        Note: "adiabatic cycle" may refer to a cycle in phase space
        (most likely argumented by time $t$ parameter). "Adiabatic"
        describes the process to be adiabatical, i.e. vary very
        slowly. The detailed criterion is on page 12, just above
        equation 3.6:
        \begin{equation}
            \xi \abs{\Delta_r H(k,r)}<<\varepsilon_g
        \end{equation}
    \item page 12, "disinclination" is what kind of defect? Any books
        on crystall defects?
    \item page 12, is the \textit{"mass gap"} a massive gap or a gap
        composed of mass?
    \item page 12, about the $D$: if $d_c=1$ (line defect in a
        $2d$-bulk), then $D=0$. So a line is wrapped by a point?
        also, on fig. 2, the ($D=2,d=1$) gives a $d_\text{defect}=-1$!
        Judging from this graph, a $d_\text{defect}=-1$ means a
        temporal defect. Is this true?
\end{enumerate}

\paragraph{Ask friends}
\begin{enumerate}
    \item page 13. What is a homotopy type?
\end{enumerate}
\paragraph{Doubts}
\begin{enumerate}
    \item page 10. Amazingly, he says, \textit{"all TIs and TSCs in
        the ten AZ symmetry classes are stable against disorder, and
        hence the assumption of translation invariance is not at all
        necessary"}.  How can translational invariance be ignored?
    \item page 12, he mentions:
        \begin{quote}
            As in the case of gapped TIs and TSCs, we are interested
            in the highest dimension strong topologies of the defect
            that do not involve lower dimensional cycles
        \end{quote}
        I don't get what "strong topologies" and "lower dimensional
        cycles" mean.
    \item page 13 right column, again he mentioned the strong topology
        and compactify the space into a $S^{d+D}$. I don't get why:
        \begin{quote}
            Physically this means the defect band theory are assumed
            to have trivial winding around those low-dimensional
            cycles.
        \end{quote}
        
    \item page 13, What does this mean:
        \begin{quote}
            It deformation retracts from the defect complement of
            spacetime.
        \end{quote}
        
\end{enumerate}

\paragraph{Revisit}
\begin{enumerate}
    \item page 12, bottom. How this procedure of relating real and
        complex classification is done?
\end{enumerate}
\section{Anchor}
\label{sec:Anchor}

\begin{thebibliography}{1}
    %\bibitem{book} 
\end{thebibliography}
\printnomenclature
\section{License}
The entire content of this work (including the source code
for TeX files and the generated PDF documents) by 
Hongxiang Chen (nicknamed we.taper, or just Taper) is
licensed under a 
\href{http://creativecommons.org/licenses/by-nc-sa/4.0/}{Creative 
Commons Attribution-NonCommercial-ShareAlike 4.0 International 
License}. Permissions beyond the scope of this 
license may be available at \url{mailto:we.taper[at]gmail[dot]com}.
\end{document}
